\documentclass[dvipsnames]{article}
\input{preamble}

\begin{document}

In athletics, a ``hammer'' is not construction tool but a 4-kilogram metal ball connected to a handle by a steel wire. Suppose Anita spins the hammer in a circular path, supplying a centripetal force of \SI{320}{N} and giving the hammer a tangential speed of \SI{12}{m/s}. Calculate the radius of curvature of the hammer's path.

\begin{minipage}{6cm}
    \centering
    \begin{randomizechoices}
        \correctchoice \SI{1.8}{m}
        \choice \SI{26.7}{m}
        \choice \SI{1.1}{m}
        \choice \SI{2.5}{m}
    \end{randomizechoices}
\end{minipage}%
\begin{minipage}{6cm}
    \centering
\begin{center}
    \begin{tikzpicture}
        \begin{axis}[
            width=8cm,height=6cm,
            axis line style={draw=none},
            axis equal image,
            ticks=none,
            axis lines=middle,
            xmin=-1.1,xmax=1.6,
            ymin=-0.2,ymax=1.1,
            clip=true,
        ]
        % \clip (-1.1,-0.2) rectangle ++(2.2,1.4);
        \draw[gray,dashed] (0,0) circle (1);
        \fill (0,0) circle (1pt);
        \fill ({cos(0)},{sin(0)}) circle (3pt) node[above right] {$\SI{4.0}{kg}$};
        \begin{scope}[shift={(axis direction cs: 0,-0.04)}]
            \draw[dashed,<->] (0,0) -- ({cos(0)},{sin(0)}) node[pos=0.5,below] {$r$};
        \end{scope}
        \draw[Green,very thick,->] ({cos(0)},{sin(0)}) -- ++(axis direction cs: -0.4,0) node[black,above=3pt] {$a_{\text{c}}$};
        \draw[red,very thick,->] ({cos(0)},{sin(0)}) -- ++(axis direction cs: 0,0.7) node[black,pos=1.1] {$\SI{12}{m/s}$};
    \end{axis}
    \end{tikzpicture}
    \end{center}
\end{minipage}

    

\end{document}