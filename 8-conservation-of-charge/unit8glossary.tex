\newglossaryentry{electric charge}
{
    name=electric charge,
    description={the property of matter that causes objects to attract or repel each other; it comes in two types: positive ($+$) and negative ($-$)}
}

\newglossaryentry{elementary charge}
{
    name = elementary charge,
    description = {the smallest possible unit of charge, with a magnitude of \SI{1.60e-19}{C}; also known as the fundamental unit of electric charge}
}

\newglossaryentry{electron}{
    name=electron,
    description={subatomic particle that carries one indivisible unit of negative electric charge}
}

\newglossaryentry{proton}{
    name=proton,
    description={subatomic particle that carries the same magnitude charge as the electron, but its charge is positive}
}

\newglossaryentry{coulomb}{
    name=coulomb,
    description={the SI unit for electric charge}
}

\newglossaryentry{Coulomb's law}{
    name={Coulomb's law},
    description={describes the electrostatic force between charged objects, which is proportional to the charge on each object and inversely proportional to the square of the distance between the objects}
}

\newglossaryentry{law of conservation of charge}{
    name=law of conservation of charge,
    description={states that total charge is constant in any process}
}

\newglossaryentry{conductor}{
    name=conductor,
    description={material through which electric charge can easily move, such as metals}
}

\newglossaryentry{insulator}{
    name=insulator,
    description={material through which a charge does not move, such as rubber}
}

\newglossaryentry{factor of change method}{
    name=factor of change method,
    description={a method that allows you to calculate proportional changes in the electrostatic force with proportional changes in distance and/or charge; it involves starting with the form of the equation for Coulomb's law, plugging in a 1 for factors that don't change, and plugging in the factor of change for the quantities that do change}
}

\newglossaryentry{electric current}{
    name=electric current,
    description={electric charge that is moving}
}

\newglossaryentry{ampere}{
    name=ampere,
    description={unit for electric current; one ampere is one coulomb per second}
}

\newglossaryentry{conventional current}{
    name=conventional current,
    description={flows in the direction that a positive charge would flow if it could move}
}

\newglossaryentry{direct current}{
    name=direct current,
    description={electric current that flows in a single direction}
}

\newglossaryentry{alternating current}{
    name=alternating current,
    description={electric current whose direction alternates back and forth at regular intervals}
}

\newglossaryentry{resistance}{
    name=resistance,
    description={how much a circuit element opposes the passage of electric current; it appears as the constant of proportionality in Ohm's law}
}

\newglossaryentry{Ohm's law}{
    name={Ohm's law},
    description={electric current is proportional to the voltage applied across a circuit or other path}
}

\newglossaryentry{circuit diagram}{
    name=circuit diagram,
    description={schematic drawing of an electrical circuit including all circuit elements, such as resistors, lamps, batteries, and so on}
}

\newglossaryentry{in series}{
    name=in series,
    description={when elements in a circuit are connected one after the other in the same branch of the circuit}
}

\newglossaryentry{in parallel}{
    name=in parallel,
    description={when a group of resistors are connected side by side, with the top ends of the resistors connected together by a wire and the bottom ends connected together by a different wire}
}

\newglossaryentry{voltage}{
    name=voltage,
    description={the electric potential energy per unit charge; also known simply as electric potential}
}