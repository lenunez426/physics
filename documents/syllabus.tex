\documentclass{article}
\input{preamble}

\title{Physics Syllabus}
\author{Mr.~Nu\~{n}ez}
\date{\today}


\begin{document}

\maketitle

\section{Introduction}

Hello Panther Scholar! Welcome to Physics. In this class, you may gain the capacity to do great things. You will be challenged to improve your speaking, reading, writing, and math skills, in addition to learning about the fundamental aspects of the universe. I feel grateful that you are here to share this school year with me, and I wish you happiness and success in this and all your classes. The following is a brief outline of how our class will run.
\vspace{1em}

\section{Rules and Procedures}

\subsection{Commandments}

The 3 Commandments are:
\vspace{-1ex}

\begin{enumerate}
\setlength\itemsep{0.1ex}
    \item Be nice.
    \item Pay attention.
    \item Be quiet.
    
\end{enumerate}

\subsection{Rules}

\begin{regla}
    Do not talk while your teacher is talking.
\end{regla}

\begin{regla}
    Be on time every day. When the tardy bell rings, I close the door. If you're not in the room by the time the door is closed, you will be marked absent. This absence will change to a tardy if you show up.
\end{regla}

\begin{regla}
    Sit in your assigned seat. The tardy bell is your notification to immediately take your assigned seat and turn your voice off.
\end{regla}

\begin{regla}
    Do not leave your personal belongings in the room after the end of class.
\end{regla}

\begin{regla}
    Keep cellphones and unauthorized electronics in your backpack and out of sight. Unauthorized electronics include but are not limited to AirPods, headphones, tablets, phones, and video game consoles.
\end{regla}



\begin{itemize}
\setlength\itemsep{0.1ex}
    \item Bring the following classroom materials with you every day: a pencil, notebook, scratch paper, 3-ring binder/folder, laptop and charger, a backpack. Recommended (but optional) materials include pens (of different colors), highlighters, markers, and a daily planner. 
    \item Honor the 10/10 bathroom rule. You are not allowed to use the bathroom within 10 minutes after the tardy bell or prior to the dismissal bell. Take the pass when you go out.
    \item Leave your area as clean as you found it when you entered the classroom. Before you leave, tuck your chair into the desk, pick up any trash from the floor, ensure the tables are aligned, and take all of your belongings with you.
    \item Don't eat or drink in class. Water is allowed, but keep the snacks away.

    \item Wear your ID on your lanyard and abide by the dress code.
\end{itemize}

I reserved the right to write you up in a \textbf{minor incident} or \textbf{office referral} if you violate any of the classroom rules or procedures. 

\subsection{Consequences}

Verbal warning. Step out side. Minor Incident \#1. Minor Incident \#2 with phone call home and possible loss of privileges. Office referral with phone call home and call to AP and counselor.


\section{Grades}
Your grade in this class consists of three parts, with which by now you are no doubt familiar: checking for understanding, relevant applications, and summative assessments.

\subsection{Checking for Understanding}

\texttt[cyan]{Checking for Understanding} grades (also know as daily grades) are worth 20\% of your overall grade. Most of your \texttt{Checking for Understanding} assignments consist of a set of exercises that you must complete on paper. These exercises are carefully designed to test your understanding of the material and to provide opportunities to practice and reinforce your mastery. At the start of every graded assignment, you will be given a grade of 100\%. It is your responsibility to maintain that grade as high as possible by staying on task. A minimum of five points will be deducted each time you commit one of the following \textbf{infractions}:

\begin{enumerate}
\setlength\itemsep{0.1ex}
    \item You are talking too loudly and deviating from a noise level than one would expect inside a library. If the teacher can hear your conversation from across the room, you're being too loud.
    \item You are talking about things unrelated to the assignment.
    \item You are using your phone or an unauthorized electronic.
    \item You are distracting others.
    \item You are using your laptop. (If you must use it, your purpose should be related to physics and the current assignment.)
    \item You are working on something from another class.
    \item You are failing to complete the exercises that are assigned to you on a particular day.
\end{enumerate}

\subsection{Relevant Applications}
\texttt[cyan]{Relevant Applications} grades are worth 30\% of your grade. \texttt[cyan]{Relevant Applications} grades mostly consist of labs, which are conducted about once a week. The same infractions rules described in the section above apply for labs.

\subsection{Summative Assessments}
\texttt[cyan]{Summative Assessments} grades are worth 50\% of your grade. \texttt[cyan]{Summative Assessments} are the tests that you will take at the end of every unit. 



\end{document}