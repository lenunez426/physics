\documentclass[answers]{exam}
\usepackage{marvosym}

%...TikZ & PGF
\usepackage{pgfplots}
\pgfplotsset{compat=1.11}
\tikzset{>=latex}
\usetikzlibrary{calc,math}
\usepackage{tikzsymbols}
\usepgfplotslibrary{fillbetween}
\usetikzlibrary{decorations.markings} 
\usetikzlibrary{arrows.meta} %...APP2 for arrows as objects and images
\usetikzlibrary{backgrounds} %...For shading portions of graphs
\usetikzlibrary{patterns} %...Unit 5 Problems
\usetikzlibrary{shapes.geometric} %...For drawing cylinders in Unit 2
\usepackage{makecell} %...use \thead{} to enable line skip in table headers
\tikzset{
    mark position/.style args={#1(#2)}{
        postaction={
            decorate,
            decoration={
                markings,
                mark=at position #1 with \coordinate (#2);
            }
        }
    }
} %...See https://tex.stackexchange.com/questions/43960/define-node-at-relative-coordinates-of-draw-plot

\tikzset{
    declare function = {trajectoryequation10(\x,\vi,\thetai)= tan(\thetai)*\x - 10*\x^2/(2*(\vi*cos(\thetai))^2);},
    declare function = {trajectoryequation(\x,\vi,\thetai)= tan(\thetai)*\x - 9.8*\x^2/(2*(\vi*cos(\thetai))^2);},
    declare function = {patheq(\x,\yi,\vi,\thetai)= \yi + tan(\thetai)*\x - 9.8*\x^2/(2*(\vi*cos(\thetai))^2);},
    declare function = {patheqten(\x,\yi,\vi,\thetai)= \yi + tan(\thetai)*\x - 10*\x^2/(2*(\vi*cos(\thetai))^2);} %like patheq but with gravity = 10
}

%...siunitx
\usepackage{siunitx}
\DeclareSIUnit{\nothing}{\relax}
\def\mymu{\SI{}{\micro\nothing} }
\DeclareSIUnit\mmHg{mmHg}
\DeclareSIUnit{\mile}{mi}
%...NOTE: "The product symbol between the number and unit is set using the quantity-product option."

%...Other
\usepackage{amsthm}
\usepackage{amsmath}
\usepackage{amssymb}
\usepackage{cancel}
\usepackage{subcaption}
\usepackage{dashrule}
\usepackage{enumitem}
% \usepackage{fontawesome}
\usepackage{fontawesome5}
\usepackage{multicol}
\usepackage{glossaries}
%\numberwithin{equation}{section}
\numberwithin{figure}{section}
\usepackage{float}
\usepackage{twemojis} %...twitter emojis
\usepackage{utfsym}
\usepackage{linearb} %...For \BPwheel in Unit 8
\newcommand{\R}{\mathbb{R}} %...real number symbol
\usepackage{graphicx}
\usepackage{mdframed} %...For FRQ teacher boxes
\graphicspath{ {../Figures/} }
\usepackage{hyperref}
\hypersetup{colorlinks=true,
    linkcolor=blue,
    filecolor=magenta,
    urlcolor=cyan,}
\urlstyle{same}
\newcommand{\hdashline}{{\hdashrule{\textwidth}{0.5pt}{0.8mm}}}
\newcommand{\hgraydashline}{{\color{lightgray} \hdashrule{0.99\textwidth}{1pt}{0.8mm}}}

%...Miscellaneous user-defined symbols
\newcommand{\fnet}{F_{\text{net}}} %...For net force
\newcommand{\bvec}[1]{\vec{\mathbf{#1}}} %...bold vector
\newcommand{\bhat}[1]{\,\hat{\mathbf{#1}}} %...bold hat vector
\newcommand{\que}{\mathord{?}}  %...Question mark symbol in equation env
%...Define thick horizontal rule for examples:
\newcommand{\hhrule}{\hrule\hrule}
\let\oldtexttt\texttt% Store \texttt
\renewcommand{\texttt}[2][black]{\textcolor{#1}{\ttfamily #2}}% 

%...For use in the exam document class
\newif\ifprintmetasolutions


%...Decreases space above and below align and gather enironment
\makeatletter
\g@addto@macro\normalsize{%
  \setlength\abovedisplayskip{-3pt}
  \setlength\belowdisplayskip{6pt} 
}
\makeatother





\usepackage[margin=1in]{geometry}
\usepackage[figurewithin=none]{caption}
\usepackage{exam-randomizechoices}

\CorrectChoiceEmphasis{\color{red}\bfseries}
\renewcommand{\solutiontitle}{\noindent\textbf{\textcolor{red}{Solution:}}\enspace}

\usepackage{OutilsGeomTikz}
\usepackage{utfsym} %...Symbols in Unit 7 Problems
\usepackage{tabu} %...Symbols in Unit 7 Problems

%...For use in Unit 2            %    
\setlength{\columnsep}{2cm}      %
\setlength{\columnseprule}{1pt}  %
\usepackage[none]{hyphenat}      %
%%%%%%%%%%%%%%%%%%%%%%%%%%%%%%%%%

%...For use in Unit 11 on Waves:
\pgfdeclarehorizontalshading{visiblelight}{50bp}{  %
color(0.00000000000000bp)=(red);                   %
color(8.33333333333333bp)=(orange);                %
color(16.66666666666670bp)=(yellow);               %
color(25.00000000000000bp)=(green);                %
color(33.33333333333330bp)=(cyan);                 %
color(41.66666666666670bp)=(blue);                 %
color(50.00000000000000bp)=(violet)                %
}                                                  %

\newcommand{\checkbox}[1]{%
  \ifnum#1=1
    \makebox[0pt][l]{\raisebox{0.15ex}{\hspace{0.1em}\Large$\checkmark$}}%
  \fi
  $\square$%
}
%%%%%%%%%%%%%%%%%%%%%%%%%%%%%%%%%%%%%%%%%%%%%%%%%%%%

%...If using circuitikz package:
% \ctikzset{bipoles/battery1/height=0.5}
% \ctikzset{bipoles/battery1/width=0.25}
% \ctikzset{bipoles/resistor/height=0.15}
% \ctikzset{bipoles/resistor/width=0.4}

\newif\ifKlevel

%\Kleveltrue
\Klevelfalse

\addpoints

\setrandomizerseed{1}
\bracketedpoints

\header{{Name:\enspace\makebox[6cm]{\hrulefill}}}{}{Physics FRQ Test on Unit 3: Changing Motion}
% \header{Physics}{Test on Unit 3}{Changing Motion}

% \firstpageheader{\bfseries DO NOT WRITE ON THIS DOCUMENT}{}{Physics Test on Unit 3: Changing Motion}
% \runningheader{\bfseries DO NOT WRITE ON THIS DOCUMENT}{}{Physics Test Unit 3}

\begin{document}
\begin{questions}

\question 
What is the acceleration of a 90.0\,kg race-car driver while the race car is accelerating from rest to 44.7\,m/s in 4.50\,s?

\begin{randomizeoneparchoices}
    \choice \SI{893.7}{m/s^2}
    \choice \SI{0.10}{m/s^2}
    \choice \SI{9.8}{m/s^2}
    \correctchoice \SI{9.93}{m/s^2}
\end{randomizeoneparchoices}

\question 
A cart is given an initial push up the ramp and is stopped at its highest point.

\begin{center}
\begin{tikzpicture}[rotate=+8,scale=0.9,transform shape]
    \begin{axis}[width=12cm,
        axis lines = left,
        axis y line=none,
        xlabel = {Position (cm)},
        ymin=0, ymax=1, 
        xmin=0, xmax=122,
        xtick={0,10,...,120},
        clip=false,
        minor x tick num=4,
        ]
        \draw[fill=lightgray,opacity=0.5] (0,-1mm) rectangle (122,1mm);
        \draw[fill=gray] (0,1mm) node[above left,align=left] {motion\\detector} rectangle ++(5,4mm);
        \begin{scope}[shift={(10,2.6mm)}]
            \draw (0,0) rectangle ++(15,5mm);
            \draw[fill=black!10] (3,0) circle (4pt);
            \draw[fill=black!10] (12,0) circle (4pt);
            \draw[thick,->] (17,1.3mm) -- ++(6,0) node[above,pos=0.5] {$\vec{v}$};
        \end{scope}
    \end{axis}
\end{tikzpicture}
\end{center}

Which of the following $x$ vs $t$ graphs represents the motion of the cart moving ONLY up the ramp.

\begin{EnvUplevel}
\centering
\begin{tikzpicture}[x=2cm,y=2cm]
    \draw[->] (0,0) node[below left=-2pt] {0} -- (0,1.2) node[pos=0.5,left] {$x$};
    \draw[->] (0,0) -- (1.2,0) node[pos=0.5,below] {$t$};
    \draw[very thick,domain=0:1,smooth] plot(\x,\x^2);
    \node at (0.6,1.2) {Graph A};
    \begin{scope}[shift={(1.7,0)}]
        \draw[->] (0,0) node[below left=-2pt] {0} -- (0,1.2) node[pos=0.5,left] {$x$};
        \draw[->] (0,0) -- (1.2,0) node[pos=0.5,below] {$t$};
        \draw[very thick,domain=0:1,smooth] plot(\x,{-(\x-1)^2+1});
        \node at (0.6,1.2) {Graph B};
    \end{scope}
    \begin{scope}[shift={(3.4,0)}]
        \draw[->] (0,0) node[below left=-2pt] {0} -- (0,1.2) node[pos=0.5,left] {$x$};
        \draw[->] (0,0) -- (1.2,0) node[pos=0.5,below] {$t$};
        \draw[very thick,domain=0:1,smooth] plot(\x,{-\x^2+1});
        \node at (0.6,1.2) {Graph C};
    \end{scope}
    \begin{scope}[shift={(5.1,0)}]
        \draw[->] (0,0) node[below left=-2pt] {0} -- (0,1.2) node[pos=0.5,left] {$x$};
        \draw[->] (0,0) -- (1.2,0) node[pos=0.5,below] {$t$};
        \draw[very thick,domain=0:1,smooth]  plot(\x,{(\x-1)^2});
        \node at (0.6,1.2) {Graph D};
    \end{scope}
\end{tikzpicture}
\end{EnvUplevel}

\ifprintanswers
\textcolor{red}{Graph B}
\fi

\scalebox{0}{
\begin{randomizeoneparchoices}[norandomize] %...DO NOT REMOVE NORANDOMIZE
    \choice Graph A
    \correctchoice Graph B
    \choice Graph C
    \choice Graph D
\end{randomizeoneparchoices}
}

\question 
Consider the table below.

\begin{center}
    \begin{tabular}{|c|c|}
        \hline
         \textbf{Time} (s) & \textbf{Position} (m) \\ \hline
         0 & 0 \\ \hline
         1 & 8 \\ \hline
         2 & 16 \\ \hline
         3 & 24 \\ \hline
    \end{tabular}
\end{center}

What is the acceleration of the object?

\begin{randomizeoneparchoices}[norandomimze]
    \choice \SI{16}{m/s^2}   
    \choice \SI{32}{m/s^2}  
    \correctchoice \SI{0}{m/s^2}  
    \choice \SI{8}{m/s^2}  
\end{randomizeoneparchoices}


\question
Which of the following statements best describes the motion of the object depicted below?

\begin{center}
\begin{tikzpicture}[x=2cm,y=2cm]
        \draw (0,0) -- (0,1.2) node[pos=0.5,above,rotate=90] {Position};
        \draw (0,0) -- (1.2,0) node[pos=0.5,below] {Time};
        \draw[very thick,domain=0:1,smooth] plot(\x,{-(\x-1)^2+1});
\end{tikzpicture}
\end{center}

\begin{randomizechoices}
    \choice The object is moving in the positive direction at a constant speed.
    \correctchoice The object is moving in the positive direction and slowing down.
    \choice The object is moving in the positive direction and speeding up.
    \choice The object is moving in the negative direction at a constant speed.  
\end{randomizechoices}

\clearpage

\question
The arrow in the figure below represents the direction of the object's velocity.

\begin{center}
\begin{tikzpicture}
    \begin{axis}[width=12cm,height=2.5cm,
        axis lines=none,
        xmin=0,xmax=9,
        ymin=0,ymax=1,
        clip=false,
        ]
        \draw[domain=0:3,samples=7,only marks,mark=*] plot({\x^2},{0});
        \draw[ultra thick,->,shift={(0,5mm)}] (3,0) -- ++(2,0);
    \end{axis}
\end{tikzpicture}
\end{center}

 Which of the following statements best describes the motion of the object depicted above?

\begin{randomizechoices}
    \choice The acceleration of the object is zero.
    \choice The acceleration of the object is negative.
    \correctchoice The acceleration of the object is positive.
    \choice There is not enough information given to determine the direction of the object’s acceleration.
\end{randomizechoices}


\question
When a tow rope pulls on a 1200\,kg truck, the truck's acceleration is \SI{3.6}{m/s^2}. What is the net force on the truck?

\begin{randomizeoneparchoices}
    \correctchoice \SI{4320}{N}
    \choice \SI{333}{N}
    \choice \SI{299}{N}
    \choice \SI{8400}{N}
\end{randomizeoneparchoices}

% \begin{solutionorbox}[3cm]
% \begin{equation*}
%     F_\mathrm{net} = ma = (\SI{1100}{kg})(\SI{2.4}{m/s^2}) = \boxed{\SI{2640}{N}}
% \end{equation*}
% \end{solutionorbox}

\begin{EnvUplevel}
    \textbf{Questions \ref{Q15}--\ref{Q16}.} Use the data table below to answer questions \ref{Q15} \& \ref{Q16}.
\end{EnvUplevel}

\begin{center}
    \begin{tabular}{|c|c|}
        \hline
        \textbf{Time} (s) & \textbf{Velocity} (m/s) \\ \hline
        0 & 10\\ \hline
        1 & 5\\ \hline
        2 & 0\\ \hline
        3 & $-5$ \\ \hline
    \end{tabular}
\end{center}

\question \label{Q15}
Given the above velocity and time data above, determine the object’s acceleration.

\begin{randomizeoneparchoices}
    \correctchoice \SI{-5}{m/s^2}
    \choice \SI{5}{m/s^2}
    \choice \SI{10}{m/s^2}
    \choice \SI{-15}{m/s^2}    
\end{randomizeoneparchoices}


\question \label{Q16}
Given the above velocity and time data above, determine what the object’s velocity will be at time $t = \SI{5}{s}$ assuming it continues with constant acceleration.

\begin{randomizeoneparchoices}
    \choice \SI{-5}{m/s}
    \choice \SI{0}{m/s}
    \choice \SI{-10}{m/s}
    \correctchoice \SI{-15}{m/s}   
\end{randomizeoneparchoices}

\bigskip
\hrule

\question
The car pictured below is moving to the right and slowing down (decreasing velocity).

\begin{center}
\begin{tikzpicture}
    \foreach \i in {0,5,7.5,8.75}{
        \node at (\i,0) {\reflectbox{\twemoji[width=8mm]{automobile}}};
    }
\end{tikzpicture}    
\end{center}

Which of the following is TRUE about the acceleration of the object and the net force acting on it.

\begin{randomizechoices}
    \correctchoice The object has a positive velocity and negative acceleration.
    \choice The object has a negative velocity and negative acceleration.
    \choice The object has a negative velocity and positive acceleration.
    \choice The object has a positive velocity and a positive acceleration.
\end{randomizechoices}

\question
Consider the graph below.

\begin{center}
\begin{tikzpicture}
    \begin{axis}[height=5cm,
        width=5cm,
        ymin=0,ymax=8,
        xmin=0,xmax=4,
        ticks=none,
        axis lines=left,
        ylabel={Velocity (m/s)},
        xlabel={Time (s)},
    ]
        \addplot[domain=0:3,thick,black] {-2*x+6};
    \end{axis}
\end{tikzpicture}
\end{center}

Which of the following depictions of a person’s motion matches the velocity graph shown below?


\def\myemoji{\reflectbox{\twemoji[height=5mm]{man running}}}

\begin{center}
    \begin{tikzpicture}
        \draw (0,0) rectangle ++(6,1.5);
        \node at (3,0.5) {\myemoji};
        \node[below] at (3,1.5) {\textbf{Representation A}};
    \end{tikzpicture}
    \hspace{0em}
    \begin{tikzpicture}
        \draw (0,0) rectangle ++(6,1.5);
        \foreach \i in {1,2,3,4,5}{
            \node at (\i,0.5) {\myemoji};
        }
        \node[below] at (3,1.5) {\textbf{Representation B}};
    \end{tikzpicture}
    
    \begin{tikzpicture}
        \draw (0,0) rectangle ++(6,1.5);
        \foreach \i in {1,1.6,2.4,3.3,5}{
            \node at (\i,0.5) {\myemoji};
        }
        \node[below] at (3,1.5) {\textbf{Representation C}};
    \end{tikzpicture}
    \hspace{0em}
    \begin{tikzpicture}
        \draw (0,0) rectangle ++(6,1.5);
        \foreach \i in {1.0,2.7,3.6,4.4,5.0}{
            \node at (\i,0.5) {\myemoji};
        }
        %\draw[domain=1:25,samples=5,only marks,mark=*] plot(\x^0.5,0);
        \node[below] at (3,1.5) {\textbf{Representation D}};
    \end{tikzpicture}
\end{center}

{\color{white}
\begin{randomizeoneparchoices}[norandomize,]
    \choice Representation A
    \choice Representation B
    \choice Representation C
    \correctchoice Representation D
\end{randomizeoneparchoices}
}

\question 
An object that has negative acceleration must be\dots

\begin{randomizechoices}
    \choice slowing down
    \correctchoice accelerating in a direction that is opposite to a stated positive direction
    \choice speeding up 
    \choice maintaining a constant speed
\end{randomizechoices}

\question
A hippo requires more net force to achieve an acceleration of \SI{-2}{m/s^2} than an average sized mouse.

\begin{randomizechoices}[norandomize]
    \correctchoice True
    \choice False
\end{randomizechoices}


% \question
% A car travels from A to B at a constant 100\,km/hr.


% Which of the following describes the car’s acceleration?

% \begin{randomizechoices}[keeplast]
%     \correctchoice The car is accelerating because it changes direction.
%     \choice The car is accelerating because it moves at constant speed.
%     \choice The car is not accelerating because it went away from the reference point.
%     \choice none of the above
% \end{randomizechoices}

\question 
Which of the following pair of graphs demonstrates an object moving away from the reference point with a decreasing velocity?

\bigskip

\begin{minipage}{0.45\textwidth}
\centering
\textbf{Pair A}

\vspace{1ex}

\begin{tikzpicture}[x=2cm,y=2cm]
    \draw[->] (0,0) node[left] {0} -- (0,1.2) node[pos=0.5,left] {$x$};
    \draw[->] (0,0) -- (1.2,0) node[pos=0.5,below] {$t$};
    \draw[very thick,domain=0:1,smooth] plot(\x,\x^2);
    \begin{scope}[shift={(1.5,0.6)}]
        \draw[->] (0,-0.6) -- (0,0.6) node[pos=0.8,left] {$v$};
        \draw[->] (0,0) node[left] {0} -- (1.2,0);
        \node[below] at (0.6,-0.6) {$t$};
        \draw[very thick,domain=0:1,smooth] plot(\x,0.5*\x);
    \end{scope}
\end{tikzpicture}
\end{minipage}%
\begin{minipage}{0.45\textwidth}
\centering
\textbf{Pair B}

\vspace{1ex}

\begin{tikzpicture}[x=2cm,y=2cm]
    \draw[->] (0,0) node[left] {0} -- (0,1.2) node[pos=0.5,left] {$x$};
    \draw[->] (0,0) -- (1.2,0) node[pos=0.5,below] {$t$};
    \draw[very thick,domain=0:1,smooth] plot(\x,{-(\x-1)^2+1});
\begin{scope}[shift={(1.5,0.6)}]
    \draw[->] (0,-0.6) -- (0,0.6) node[pos=0.8,left] {$v$};
    \draw[->] (0,0) node[left] {0} -- (1.2,0);
    \node[below] at (0.6,-0.6) {$t$};
    \draw[very thick,domain=0:1,smooth] plot(\x,-0.45*\x+0.45);
\end{scope}
\end{tikzpicture}
\end{minipage}

\vspace{1em}

\begin{minipage}{0.45\textwidth}
\centering
\textbf{Pair C}

\vspace{1ex}

\begin{tikzpicture}[x=2cm,y=2cm]
    \draw[->] (0,0) node[left] {0} -- (0,1.2) node[pos=0.5,left] {$x$};
    \draw[->] (0,0) -- (1.2,0) node[pos=0.5,below] {$t$};
    \draw[very thick,domain=0:1,smooth] plot(\x,{-\x^2+1});
\begin{scope}[shift={(1.5,0.6)}]
    \draw[->] (0,-0.6) -- (0,0.6) node[pos=0.8,left] {$v$};
    \draw[->] (0,0) node[left] {0} -- (1.2,0);
    \node[below] at (0.6,-0.6) {$t$};
    \draw[very thick,domain=0:1,smooth] plot(\x,-0.5*\x);
\end{scope}
\end{tikzpicture}
\end{minipage}%
\begin{minipage}{0.45\textwidth}
\centering
\textbf{Pair D}

\vspace{1ex}

\begin{tikzpicture}[x=2cm,y=2cm]
    \draw[->] (0,0) node[left] {0} -- (0,1.2) node[pos=0.5,left] {$x$};
    \draw[->] (0,0) -- (1.2,0) node[pos=0.5,below] {$t$};
    \draw[very thick,domain=0:1,smooth]  plot(\x,{(\x-1)^2});
\begin{scope}[shift={(1.5,0.6)}]
    \draw[->] (0,-0.6) -- (0,0.6) node[pos=0.8,left] {$v$};
    \draw[->] (0,0) node[left] {0} -- (1.2,0);
    \node[below] at (0.6,-0.6) {$t$};
    \draw[very thick,domain=0:1,smooth] plot(\x,0.5*\x-0.5);
\end{scope}
\end{tikzpicture}
\end{minipage}

\ifprintanswers
\textcolor{red}{Pair B}
\fi


\scalebox{0}{
\begin{randomizeoneparchoices}[norandomize]
    \choice Pair A
    \correctchoice Pair B
    \choice Pair C
    \choice Pair D
\end{randomizeoneparchoices}
}


\question
What is happening in this graph?

\begin{EnvUplevel}
\centering
\begin{tikzpicture}
    \begin{axis}[width=4cm,
        height=4cm,
        axis y line=left,
        axis x line=center,
        ymin=-2,ymax=2,
        xmin=0,xmax=2,
        ytick={-2,-1,...,2},
        xtick=\empty,
        ylabel={Velocity},
        xlabel={$t$},
        ]
        \draw[very thick] (0,0) -- (1.8,-1.8);
    \end{axis}
\end{tikzpicture}%
\hspace{5mm}
\begin{tikzpicture}[rotate=+8,scale=0.9,transform shape]
    \begin{axis}[width=12cm,
        axis lines = left,
        axis y line=none,
        xlabel = {Position (cm)},
        ymin=0, ymax=1, 
        xmin=0, xmax=122,
        xtick={0,10,...,120},
        clip=false,
        minor x tick num=4,
        ]
        \draw[fill=lightgray,opacity=0.5] (0,-1mm) rectangle (122,1mm);
        \draw[fill=gray] (0,1mm) rectangle ++(5,4mm);
        \begin{scope}[shift={(105,2.6mm)}]
            \draw (0,0) rectangle ++(15,5mm);
            \draw[fill=black!10] (3,0) circle (4pt);
            \draw[fill=black!10] (12,0) circle (4pt);
            \draw[thick,->] (-2,1.3mm) -- ++(-6,0) node[above,pos=0.5] {$\vec{v}$};
        \end{scope}
    \end{axis}
\end{tikzpicture}
% \begin{tikzpicture}[scale=0.8,rotate=10,transform shape]
%         \draw (0,0) rectangle ++(10,-0.5) node[pos=0.12,below=-2pt] {0 position} node[pos=0.98,above] {$+$};
%         \draw[above=9mm] (8,0) rectangle ++(1.5,-0.7) node[pos=0.5,below=-3mm] {cart};
%         \draw[fill=white,above=2mm] (8.4,0) circle (2mm);
%         \draw[fill=white,above=2mm] (9.1,0) circle (2mm);
%         \draw[above=0.5cm,ultra thick,->] (8,0) -- ++(-1.5,0);
% \end{tikzpicture}
\end{EnvUplevel}

\begin{randomizechoices}
    \choice negative velocity, positive acceleration, increasing speed
    \choice positive velocity, positive acceleration, increasing speed
    \choice negative velocity, negative acceleration, decreasing speed
    \correctchoice negative velocity, negative acceleration, increasing speed
\end{randomizechoices}

\question
A runner moves according to the velocity vs time graph shown below. What is the acceleration of the runner at 3 seconds?

\begin{minipage}{0.3\textwidth}
    \centering
\begin{randomizechoices}
    \choice \SI{12}{m/s^2}
    \choice \SI{2}{m/s^2}
    \correctchoice \SI{3}{m/s^2}
    \choice \SI{6}{m/s^2}
\end{randomizechoices}
\end{minipage}%
\hspace{1em}
\begin{minipage}{0.45\textwidth}
\centering
    \begin{tikzpicture}
        \begin{axis}[height=5cm,
            width=7cm,
            ylabel={Velocity (m/s)},
            xlabel={Time (s)},
            ymin=0,ymax=14,
            xmin=0,xmax=12,
            ytick={0,2,...,14},
            xtick={0,2,...,12},
            axis lines=left,
            grid=both,
            % title={Runner Velocity vs. Time}
        ]
        \addplot[very thick,black] coordinates{(0,0)(2,6)(4,12)(6,12)(8,12)(10,12)(12,6)};
        \end{axis}
    \end{tikzpicture}
\end{minipage}

\question 
While driving a car due west, a student brakes to stop for a red traffic light. The direction of the acceleration of the car is to the 

\begin{randomizeoneparchoices}
    \correctchoice east
    \choice north
    \choice south
    \choice west
\end{randomizeoneparchoices}

\question
A toy car is given an initial velocity of \SI{5.0}{m/s} and experiences a constant acceleration of \SI{2.0}{m/s^2}. What is the final velocity after \SI{6.0}{s}?

\begin{randomizeoneparchoices}
    \correctchoice \SI{17}{m/s}
    \choice \SI{10}{m/s}
    \choice \SI{12}{m/s}
    \choice \SI{16}{m/s}
\end{randomizeoneparchoices}

\question
Average acceleration is defined as the

\begin{randomizechoices}
    \correctchoice change in velocity divided by the time interval
    \choice change in position divided by the time interval
    \choice change in distance divided by the time interval
    \choice change in momentum divided by the time interval
\end{randomizechoices}

\question 
A car moving with a velocity of $+\SI{10}{m/s}$ experiences an acceleration to the right. What happens to the car?

\begin{randomizechoices}
    \choice The velocity of the car will be unaffected.
    \correctchoice It will speed up.
    \choice It will slow down.
\end{randomizechoices}

\question
Alyssa pushes object A with a force of \SI{80}{N}. She then pushes object B with an equal force of \SI{80}{N}.

\begin{center}
\begin{tikzpicture}
    \draw[fill=black!10] (0,0) rectangle (2,2) node[pos=0.5,align=center] {A\\ \SI{100}{kg}};
    \begin{scope}[shift={(4,0)}]
        \draw[fill=black!10] (0,0) rectangle (1.5,1.5) node[pos=0.5,align=center] {B\\ \SI{50}{kg}};
    \end{scope}
\end{tikzpicture}
\end{center}

Which object experiences a greater acceleration?

\begin{randomizechoices}
    \correctchoice Object B
    \choice Object A
    \choice They experience equal accelerations.
    \choice Neither object accelerates.
\end{randomizechoices}


\end{questions}


% \question
% The rate at which an object’s velocity changes is called its \fillin[acceleration].


\clearpage

% \thispagestyle{headandfoot}
% \header{Physics\\Test on Unit 3: Changing Motion}{}{{Name:\enspace\makebox[5cm]{\hrulefill}}}

\begin{questions}
\begingradingrange{myrange} 
\question
The following shows a velocity vs. time graph, and a student's \textbf{\underline{incorrect}} calculation of the object's average acceleration using two data points.

\begin{EnvUplevel}
\centering
\begin{minipage}{0.4\textwidth}
\centering
\begin{tikzpicture}
\begin{axis}[width=6cm,height=5cm,
    axis lines=left,
    % axis x line=center,
    xlabel = {Time (s)},
    ylabel = {Velocity (m/s)},
    ymin=0, ymax=14,
    xmin=0, xmax=10,
    ytick={0,2,...,14},
    xtick={0,1,...,10},
    ymajorgrids=true,
    xmajorgrids=true,
]
    \addplot[very thick,black,domain=0:10] {-3/2*\x+18};
    \fill (4,12) circle (2.5pt);
    \fill (8,6) circle (2.5pt);
    \end{axis}
\end{tikzpicture}
\end{minipage}%
\hspace{1cm}
\begin{minipage}{0.25\textwidth}
\begin{align*}
    a = \frac{\Delta v}{\Delta t} &= \frac{v_f - v_i}{t_f - t_i} \\[1ex]
    &=  \frac{8 - 4}{6 - 12} \\[1ex] 
    &= \boxed{\SI{0.67}{m/s^2}}
\end{align*}
\end{minipage}
\end{EnvUplevel}

\smallskip

\begin{parts}
\part[1] List 1 thing that is correct about the student's work.

\begin{solution}
    The student used the right formula for average acceleration.
\end{solution}

\ifprintanswers
\else
\fillwithlines{21mm}
\fi

\part[2] List 2 things that are \textbf{incorrect} about the student's work.

\begin{solution}
Answers may vary:

\begin{itemize}[itemsep=0pt]
    \item The difference $6-12$ is negative so their answer should have been negative.
    \item The difference in the numerator $8 - 4$ is the change in time, which belongs in the denominator. 
    \item The difference in the denominator $6 - 12$ is the change in velocity which belongs in the numerator.
    \item The student should have included units in the fraction.
\end{itemize}
\end{solution}

\ifprintanswers
\else
\fillwithlines{28mm}
\fi

\part[2] Re-write the student's steps correctly, and calculate the object's average acceleration.


\begin{solutionorbox}[4cm]
\phantom{.}

% \textbf{1 point earned}: For identifying that the numbers were plugged into the fraction incorrectly or for indicating that the fraction should be flipped.

% In the numerator, the student calculated change in time, and they calculated change in velocity in the denominator. Therefore, since acceleration is change in velocity vs. change in time, the fraction should be flipped.

% \textbf{1 point earned}: For correctly computing average acceleration:

\begin{align*}
    a = \frac{\Delta v}{\Delta t} &= \frac{v_f - v_i}{t_f - t_i} \\[1ex]
    &=  \frac{6 - 12}{8 - 4} \\[1ex] 
    &= \boxed{\SI{-1.5}{m/s^2}}
\end{align*}

or $a = -\frac{3}{2} \mathrm{m/s^2}$.
\end{solutionorbox}
\end{parts}

\hfill \textit{Continued on next page\ldots}

\vspace*{\fill}

\ifprintanswers
\else
\begin{mdframed}[backgroundcolor=black!8]
    \centering
    \textbf{\texttt{TEACHER USE ONLY}}\\[1em]
    \partialgradetable{myrange}[h][questions]
\end{mdframed}
\fi

\clearpage


\question
When a force $F_\mathrm{A}$ is applied on an 80\,kg cannonball in the $+x$ direction, the cannonball accelerates from rest to \SI{28}{m/s} in 2.0 seconds. 

\begin{parts}
\part[1] Draw a free body diagram for the \textbf{\underline{horizontal}} force(s) on the cannonball.

\begin{solutionorbox}[2cm]
\begin{center}
\begin{tikzpicture}
    \fill (0,0) circle (4pt);
    \draw[thick,->] (0,0) -- (2,0) node[right] {$F_\mathrm{A}$};
\end{tikzpicture}
\end{center}
\end{solutionorbox}

\part[2] Calculate the ball's acceleration.

\begin{solutionorbox}[4cm]
\begin{equation*}
    a = \frac{\Delta v}{\Delta t} = \frac{\SI{28}{m/s}}{\SI{2.0}{s}}  = \boxed{\SI{14}{m/s^2}}
\end{equation*}

\end{solutionorbox}

\part[2] What is the net force on the cannonball?

\begin{solutionorbox}[4cm]
\phantom{.}
\begin{equation*}
    F_\mathrm{net} = ma = (\SI{80}{kg})(\SI{14}{m/s^2}) = \boxed{\SI{1120}{N}}
\end{equation*}
\end{solutionorbox}
\end{parts}

\endgradingrange{myrange} 


\clearpage

\end{questions}

% \header{Physics}{}{Unit 3 Test}

\begin{questions}

\question [2]
Consider the table below.

\begin{center}
    \begin{tabular}{|c|c|c|}
        \hline
        \textbf{Time} (s) & \textbf{Speed} (m/s) & \textbf{Direction} \\ \hline
        8 & 16 & East\\ \hline
        9 & 20 & East\\ \hline
        10 & 24 & East\\ \hline
        11 & 28 & East\\ \hline
    \end{tabular}
\end{center}

Does this table represent an object that is accelerating? How do you know? Explain.

\fillwithlines{3cm}

\begin{solutionorbox}
\phantom{.}

\textbf{1 point earned}: For stating that the object's velocity is increasing with time.

\textbf{1 point earned}: For stating that acceleration is the rate of change in velocity, so that a change in velocity implies acceleration.
\end{solutionorbox}

\question [3]
A cart moves on a straight track and its motion is recorded in the graph below.

\begin{center}
\begin{tikzpicture}
    \begin{axis}[height=6cm,
        width=6cm,
        ymin=0,ymax=5,
        xmin=0,xmax=3,
        ticks=none,
        axis lines=left,
        ylabel={$x$},
        y label style={rotate=-90},
        xlabel={$t$},
    ]
        \addplot[domain=0:2,very thick,black] {4*(x+2) - (x+2)^2};
        \fill ({0.4},{4*(0.4+2) - (0.4+2)^2}) circle (2pt) node[above=2pt] {A};
        \fill ({1.2},{4*(1.2+2) - (1.2+2)^2}) circle (2pt) node[above right] {B};
        \fill ({1.8},{4*(1.8+2) - (1.8+2)^2}) circle (2pt) node[above right] {C};
    \end{axis}
\end{tikzpicture}
\end{center}

Points A, B, and C represent three different instants in time. Rank these points according to the object's speed at those instants, from the fastest point to the slowest.

\begin{solutionorbox}[6cm]
Rank: C, B, A
\end{solutionorbox}

\end{questions}
\end{document}

% \begin{randomizechoices}
%     \choice \SI{2}{m/s/s}
%     \correctchoice \SI{14}{m/s/s}
%     \choice \SI{28}{m/s/s}
%     \choice \SI{56}{m/s/s}
% \end{randomizechoices}



% \question
% What is the net force exerted on a 85.0\,kg runner while they are going from rest to 12.0\,m/s in 9.5\,s?

% \begin{solutionorbox}[6cm]
% The runner's acceleration is

% \begin{equation*}
%     a = \frac{\Delta v}{\Delta t} = \frac{\SI{12.0}{m/s}}{\SI{9.5}{s}} = \SI{1.26}{m/s^2}
% \end{equation*}

% Therefore, the net force is

% \begin{equation*}
%     F_\mathrm{net} = ma = (\SI{85.0}{kg})(\SI{1.26}{m/s^2}) = \boxed{\SI{107}{N}}
% \end{equation*}
% \end{solutionorbox}

\clearpage

\begin{EnvUplevel}
\subsection*{DELETED QUESTIONS}

The following questions have been removed from the test.    
\end{EnvUplevel}

\color{red}

\question 
Which of the following representations of motion, depicts an object with constant momentum?

\begin{center}
    \begin{tikzpicture}
        \begin{axis}[height=4cm,
            width=4cm,
            ymin=0,ymax=5,
            xmin=0,xmax=3,
            ticks=none,
            axis lines=left,
            ylabel={$v$},
            y label style={rotate=-90},
            xlabel={$t$},
            title={Graph A},
        ]
            \addplot[domain=0:2.5,very thick] {3};
        \end{axis}
    \end{tikzpicture}
    \hspace{2em}
    \begin{tikzpicture}
        \begin{axis}[height=4cm,
            width=4cm,
            ymin=0,ymax=5,
            xmin=0,xmax=3,
            ticks=none,
            axis lines=left,
            ylabel={$v$},
            y label style={rotate=-90},
            xlabel={$t$},
            title=Graph B,
        ]
            \addplot[domain=0:2,very thick] {4*x - x^2};
        \end{axis}
    \end{tikzpicture}
    \hspace{2em}
    \begin{tikzpicture}
        \begin{axis}[height=4cm,
            width=4cm,
            ymin=0,ymax=5,
            xmin=0,xmax=3,
            ticks=none,
            axis lines=left,
            ylabel={$v$},
            y label style={rotate=-90},
            xlabel={$t$},
            title={Graph C}
        ]
            \addplot[domain=0:2,very thick] {(x-2)^2};
        \end{axis}
    \end{tikzpicture}
    \hspace{2em}
    \begin{tikzpicture}
        \begin{axis}[height=4cm,
            width=4cm,
            ymin=0,ymax=5,
            xmin=0,xmax=3,
            ticks=none,
            axis lines=left,
            ylabel={$v$},
            y label style={rotate=-90},
            xlabel={$t$},
            title={Graph D}
        ]
            \addplot[domain=0:2.5,very thick] {1.5*x};
        \end{axis}
    \end{tikzpicture}
\end{center}

\question
\phantom{.}

\begin{center}
\begin{tabular}{|c|c|}
    \hline 
    Time (s) & Velocity (m/s) \\ \hline
    0 & 0.55 \\ \hline
    1 & 0.45 \\ \hline
    2 & 0.35 \\ \hline
    3 & 0.25\\ \hline
    4 & 0.15 \\ \hline
\end{tabular}    
\end{center}

The following time and velocity data was collected for a ball that was launched up a ramp. Using the information above, determine the acceleration of the ball.

\begin{randomizechoices}
    \correctchoice \SI{-0.10}{m/s^2}
    \choice \SI{-0.55}{m/s^2}
    \choice \SI{-0.35}{m/s^2}
    \choice \SI{-0.15}{m/s^2}
\end{randomizechoices}

\question
\phantom{.}

\begin{center}
    \begin{tikzpicture}
        \begin{axis}[width=6cm,
            height=6cm,
            axis y line=left,
            axis x line=center,
            ymin=-8,ymax=8,
            xmin=0,xmax=13,
            ytick={-8,-7,...,8},
            xtick={0,1,...,13},
            ticks=none,
            % grid=both,
            ylabel={Velocity (m/min)},
            xlabel={Time (min)},
            clip=false,
            ]
            \draw[very thick] (0,5) -- (12,5) node[right] {\bfseries A};
            \draw[very thick] (0,-3) -- (12,-3) node[right] {\bfseries B};
            \draw[very thick] (0,0) -- (7,7) node[pos=1.1] {\bfseries C};
            \draw[very thick] (0,-5) -- (4,0) node[pos=0.95,below=2pt] {\bfseries D};
            \draw[very thick] (0,4) -- (12,-5) node[below,pos=0.9] {\bfseries E};
            \node[above] at (0,8) {N};
            \node[below] at (0,-8) {S};
        \end{axis}
    \end{tikzpicture}
\end{center}

\begin{parts}
    \part What is the main difference between line A and line B?

    \begin{solution}
        A \& B are traveling in different directions.
    \end{solution}

    \part Describe the motion represented by line E.

    \begin{solution}
        E is slowing down then speeding up towards the reference point.
    \end{solution}
\end{parts}




% \end{questions}
% \end{document}



\question
\phantom{.}

\begin{center}
\begin{tabular}{|c|c|}
    \hline 
    Time (s) & Velocity (m/s) \\ \hline
    0 & 0.55 \\ \hline
    1 & 0.45 \\ \hline
    2 & 0.35 \\ \hline
    3 & 0.25\\ \hline
    4 & 0.15 \\ \hline
\end{tabular}    
\end{center}

The following time and velocity data was collected for a ball that was launched up a ramp. Using the information above, determine the acceleration of the ball.

\begin{randomizechoices}
    \correctchoice \SI{-0.10}{m/s^2}
    \choice \SI{-0.55}{m/s^2}
    \choice \SI{-0.35}{m/s^2}
    \choice \SI{-0.15}{m/s^2}
\end{randomizechoices}