\documentclass[answers]{exam}
\usepackage{marvosym}

%...TikZ & PGF
\usepackage{pgfplots}
\pgfplotsset{compat=1.11}
\tikzset{>=latex}
\usetikzlibrary{calc,math}
\usepackage{tikzsymbols}
\usepgfplotslibrary{fillbetween}
\usetikzlibrary{decorations.markings} 
\usetikzlibrary{arrows.meta} %...APP2 for arrows as objects and images
\usetikzlibrary{backgrounds} %...For shading portions of graphs
\usetikzlibrary{patterns} %...Unit 5 Problems
\usetikzlibrary{shapes.geometric} %...For drawing cylinders in Unit 2
\usepackage{makecell} %...use \thead{} to enable line skip in table headers
\tikzset{
    mark position/.style args={#1(#2)}{
        postaction={
            decorate,
            decoration={
                markings,
                mark=at position #1 with \coordinate (#2);
            }
        }
    }
} %...See https://tex.stackexchange.com/questions/43960/define-node-at-relative-coordinates-of-draw-plot

\tikzset{
    declare function = {trajectoryequation10(\x,\vi,\thetai)= tan(\thetai)*\x - 10*\x^2/(2*(\vi*cos(\thetai))^2);},
    declare function = {trajectoryequation(\x,\vi,\thetai)= tan(\thetai)*\x - 9.8*\x^2/(2*(\vi*cos(\thetai))^2);},
    declare function = {patheq(\x,\yi,\vi,\thetai)= \yi + tan(\thetai)*\x - 9.8*\x^2/(2*(\vi*cos(\thetai))^2);},
    declare function = {patheqten(\x,\yi,\vi,\thetai)= \yi + tan(\thetai)*\x - 10*\x^2/(2*(\vi*cos(\thetai))^2);} %like patheq but with gravity = 10
}

%...siunitx
\usepackage{siunitx}
\DeclareSIUnit{\nothing}{\relax}
\def\mymu{\SI{}{\micro\nothing} }
\DeclareSIUnit\mmHg{mmHg}
\DeclareSIUnit{\mile}{mi}
%...NOTE: "The product symbol between the number and unit is set using the quantity-product option."

%...Other
\usepackage{amsthm}
\usepackage{amsmath}
\usepackage{amssymb}
\usepackage{cancel}
\usepackage{subcaption}
\usepackage{dashrule}
\usepackage{enumitem}
% \usepackage{fontawesome}
\usepackage{fontawesome5}
\usepackage{multicol}
\usepackage{glossaries}
%\numberwithin{equation}{section}
\numberwithin{figure}{section}
\usepackage{float}
\usepackage{twemojis} %...twitter emojis
\usepackage{utfsym}
\usepackage{linearb} %...For \BPwheel in Unit 8
\newcommand{\R}{\mathbb{R}} %...real number symbol
\usepackage{graphicx}
\usepackage{mdframed} %...For FRQ teacher boxes
\graphicspath{ {../Figures/} }
\usepackage{hyperref}
\hypersetup{colorlinks=true,
    linkcolor=blue,
    filecolor=magenta,
    urlcolor=cyan,}
\urlstyle{same}
\newcommand{\hdashline}{{\hdashrule{\textwidth}{0.5pt}{0.8mm}}}
\newcommand{\hgraydashline}{{\color{lightgray} \hdashrule{0.99\textwidth}{1pt}{0.8mm}}}

%...Miscellaneous user-defined symbols
\newcommand{\fnet}{F_{\text{net}}} %...For net force
\newcommand{\bvec}[1]{\vec{\mathbf{#1}}} %...bold vector
\newcommand{\bhat}[1]{\,\hat{\mathbf{#1}}} %...bold hat vector
\newcommand{\que}{\mathord{?}}  %...Question mark symbol in equation env
%...Define thick horizontal rule for examples:
\newcommand{\hhrule}{\hrule\hrule}
\let\oldtexttt\texttt% Store \texttt
\renewcommand{\texttt}[2][black]{\textcolor{#1}{\ttfamily #2}}% 

%...For use in the exam document class
\newif\ifprintmetasolutions


%...Decreases space above and below align and gather enironment
\makeatletter
\g@addto@macro\normalsize{%
  \setlength\abovedisplayskip{-3pt}
  \setlength\belowdisplayskip{6pt} 
}
\makeatother





\usepackage[margin=1in]{geometry}
\usepackage[figurewithin=none]{caption}
\usepackage{exam-randomizechoices}

\CorrectChoiceEmphasis{\color{red}\bfseries}
\renewcommand{\solutiontitle}{\noindent\textbf{\textcolor{red}{Solution:}}\enspace}

\usepackage{OutilsGeomTikz}
\usepackage{utfsym} %...Symbols in Unit 7 Problems
\usepackage{tabu} %...Symbols in Unit 7 Problems

%...For use in Unit 2            %    
\setlength{\columnsep}{2cm}      %
\setlength{\columnseprule}{1pt}  %
\usepackage[none]{hyphenat}      %
%%%%%%%%%%%%%%%%%%%%%%%%%%%%%%%%%

%...For use in Unit 11 on Waves:
\pgfdeclarehorizontalshading{visiblelight}{50bp}{  %
color(0.00000000000000bp)=(red);                   %
color(8.33333333333333bp)=(orange);                %
color(16.66666666666670bp)=(yellow);               %
color(25.00000000000000bp)=(green);                %
color(33.33333333333330bp)=(cyan);                 %
color(41.66666666666670bp)=(blue);                 %
color(50.00000000000000bp)=(violet)                %
}                                                  %

\newcommand{\checkbox}[1]{%
  \ifnum#1=1
    \makebox[0pt][l]{\raisebox{0.15ex}{\hspace{0.1em}\Large$\checkmark$}}%
  \fi
  $\square$%
}
%%%%%%%%%%%%%%%%%%%%%%%%%%%%%%%%%%%%%%%%%%%%%%%%%%%%

%...If using circuitikz package:
% \ctikzset{bipoles/battery1/height=0.5}
% \ctikzset{bipoles/battery1/width=0.25}
% \ctikzset{bipoles/resistor/height=0.15}
% \ctikzset{bipoles/resistor/width=0.4}
\usepackage{mdframed}
\usepackage[none]{hyphenat}

\setrandomizerseed{1}
\bracketedpoints
\addpoints


% \firstpageheader{{Name:\enspace\makebox[5.5cm]{\hrulefill}}\\Physics}{Review\\}{Unit 7A: Two-Dimensional Motion\\ (Projectile Motion)}
% \runningheader{Physics}{Review}{Unit 7A}

\runningheader{{Name:\enspace\makebox[5.5cm]{\hrulefill}}\\Physics}{Review\\}{Unit 7B: Two-Dimensional\\ (Circular Motion)}
% \runningheader{Physics}{Review}{Unit 7B}


\begin{document}

\subsection*{Projectile Motion}

\textbf{Part I: Projectiles Launched at an Angle}

\begin{questions}
\question
A projectile is launched at an angle above the horizon as shown in the figure below. The axes $y$ and $x$ are vertical and horizontal positions, respectively.

\begin{center}
\begin{tikzpicture}
    \begin{axis}[height=6cm,width=9cm,
        axis lines=left,
        ylabel={$y$ (m)},
        y label style={rotate=-90},
        xlabel={$x$ (m)},
        ymin=0,ymax=14,
        xmin=0,xmax=30,
        ytick={0,2,...,14},
        xtick={0,3,...,30},
        grid=both,
        clip=false,
    ]
    \addplot[domain=0:28.06,thick,smooth,densely dashed] {trajectoryequation10(x,18,60)};
    \ifprintanswers
    \draw[red,ultra thick,->] (0,0) -- ({2.5cm*cos(60)},{2.5cm*sin(60)}) node[right=2pt,pos=0.9] {\SI{18}{m/s}};
    \draw[red] (0.7cm,0) arc (0:60:0.7cm) node[red,pos=0.7,right=2pt] {\ang{60}};
    \fill[red] (14.03,12.15) circle (2pt) node[above=2pt] {apex};
    \fi
    \end{axis}
\end{tikzpicture}
\end{center}

Complete the following parts:

\begin{parts}
\part At the origin, draw a velocity vector representing the initial speed. Label it \SI{18}{m/s}.
\vspace{1mm}
\part Estimate the launch angle and label it on the graph.
\vspace{1mm}
\part Find the apex, and label it ``apex.''
\vspace{1mm}
\part The apex height is approximately \fillin[12][2cm] meters.
\vspace{1mm}
\part The maximum range is approximately \fillin[28][2cm] meters.
% \vspace{1mm}
% \part Discuss the changes, if any, to the apex height, apex range, and maximum projectile range when the initial speed is increased.
% \vspace{1mm}

% \ifprintanswers
% \bgroup
% \color{red}
% When speed increases, the trajectory grows taller and wider such that the apex height, apex range, and maximum projectile all increase.
% \egroup
% \else
% \fillwithlines{2cm}
% \fi
\end{parts}

\question %...Effects of Changing One Variable
For each situation below, only one variable is changed. Predict what happens to the

\begin{itemize}[itemsep=0pt,topsep=0pt]
    \item hang time
    \item maximum height (apex)
    \item range
\end{itemize}

for each of the following changes:

\begin{parts}
\part Increasing the initial speed

\ifprintanswers
\textcolor{red}{Increasing the initial speed causes the hang time, maximum height, and range to all increase.}
\else
\fillwithlines{14mm}
\fi

\part Increasing the launch angle above \ang{45}

\ifprintanswers
\textcolor{red}{Increasing the launch angle above \ang{45} causes the hang time and maximum height to increase, but it causes the range decrease.}
\else
\fillwithlines{14mm}
\fi

\part Increasing the launch height

\ifprintanswers
\textcolor{red}{Increasing the initial speed causes the hang time, maximum height, and range to all increase. The horizontal position of the apex remains the same.}
\else
\fillwithlines{14mm}
\fi

\part Increasing the mass of the projectile

\ifprintanswers
\textcolor{red}{Changing mass has no effect on hang time, maximum height, or range.}
\else
\fillwithlines{14mm}
\fi
\end{parts}



\question %...Trajectory Comparisons
Two projectiles are launched with the same speed but at angles of \ang{30} and \ang{60}.

\begin{parts}
    \part Which goes higher? \fillin[The projectile launched at \ang{60}][10cm]
    \part Which stays in the air longer? \fillin[The projectile launched at \ang{60}][10cm]
    \part Which travels farther horizontally? \\
    \fillin[Both travel the same range since both are \ang{15} from \ang{45}.][10cm]
\end{parts}

\question
A projectile is launched above ground. If gravity suddenly turned off mid-flight at the apex, what would happen to a projectile’s motion? Sketch the trajectory below.

\ifprintanswers
\textcolor{red}{At the apex, the vertical velocity is zero and the horizontal velocity is non-zero and constant. Due to inertia, the projectile will continue moving at a constant speed in the horizontal direction.}
\else
\fillwithlines{14mm}
\fi

\begin{center}
\begin{tikzpicture}[x=3.5mm,y=3.5mm]
    \draw[domain=0:11.25,thick,densely dashed] plot(\x,{trajectoryequation10(\x,15,45)});
    \draw[domain=11.25:22.5,black!10,thick,densely dashed] plot(\x,{trajectoryequation10(\x,15,45)});
    \draw[<-,thick] (11.25,5.3) -- ++(0,-5mm) node[below,align=center] {gravity turns\\off here};
    \draw[->,very thick] (0,0) -- ({4*cos(45)},{4*sin(45)}) node[above left=-2pt,pos=0.5,align=center] {initial\\velocity};
    \ifprintanswers
    \draw[red,thick,densely dashed] (11.25,5.62) -- (22.5,5.62);
    \fi
    \fill (11.25,5.62) circle (0.2);
\end{tikzpicture}
\end{center}



\uplevel{\textbf{Part II: Horizontally Launched Projectiles}}

\question %...Horizontally Launched Projectile
A ball rolls off a table horizontally. 

\begin{parts}
\part For this situation, describe the horizontal and vertical components of displacement, velocity, and acceleration, by writing ``changing,'' ``constant,'' or ``zero'' in the table below.

\begin{center}
\def\arraystretch{1.5}
\newcommand{\answertableentry}[3]{%
  \ifprintanswers
    #1 & \textcolor{red}{#2} & \textcolor{red}{#3} \\ \hline
  \else
    #1 & \textcolor{white}{#2} & \textcolor{white}{#3} \\ \hline
  \fi
}

\begin{tabular}{|l|l|l|}
    \hline
    \thead{\textbf{Quantity}} & \thead{\textbf{Horizontal}} & \thead{\textbf{Vertical}}\\ \hline
    \answertableentry{Displacement \hspace{0.5cm}}{changing \hspace{1cm}}{changing \hspace{1cm}}
    \answertableentry{Velocity}{constant}{changing}
    \answertableentry{Acceleration}{zero}{constant}
\end{tabular}
\end{center}


\part
Draw and label the horizontal ($v_x$) and vertical ($v_y$) components of velocity for each of the positions below. The magnitudes of the vectors should reflect the changing or constant nature of the component.

\begin{center}
\begin{tikzpicture}[x=4mm,y=4mm]
    \draw[fill=black!20] (-3,10.8) -- (0,10.8) -- (0,9.8) -- (-3,9.8);
    \draw[fill=black!8] (-3,9.8) -- (-1,9.8) -- (-1,-1) -- (-3,-1);
    \draw[domain=0:14.98,black] plot(\x,{trajectoryequation(\x,10,0)+11});
    \ifprintanswers
        \draw[->,thick,red] (0,11) -- ++(3,0) node[above] {$v_{ix}$};
    \fi
    \fill[black] (0,11) circle (0.2);
    \pgfplotsinvokeforeach{0.3,0.6,0.9,1.2,1.5}{
        \ifprintanswers
            \color{red}
        \else
            \color{white}
        \fi
        \coordinate (P) at (10*#1,{11-0.5*9.8*(#1)^2}); 
        \draw[->,thick] (P) -- ++(3,0) node[right] {$v_x$};
        \draw[->,thick] (P) -- ++(0,{-3*#1}) node[below] {$v_y$};
        \fill[black] (P) circle (0.2);
    }
\end{tikzpicture}
\end{center}


\part
Sketch the horizontal and vertical displacements, $\Delta x$ and $\Delta y$, and the horizontal and vertical velocities, $v_x$ and $v_y$, as functions of time $t$, in the space below.

\begin{EnvUplevel}
\centering
\begin{tikzpicture}[x=2.5cm,y=2.5cm]
    \draw[->] (0,-0.6) node[left,pos=0.5] {0} -- (0,0.6) node[above] {$\Delta x$};
    \draw[->] (0,0) -- (1.2,0) node[pos=0.9,below] {$t$};
    \ifprintanswers
        \draw[red,very thick,domain=0:1,smooth] plot(\x,0.8*\x);
    \fi
\end{tikzpicture}
\hspace{5mm}
\begin{tikzpicture}[x=2.5cm,y=2.5cm]
    \draw[->] (0,-0.6) node[left,pos=0.5] {0} -- (0,0.6) node[above] {$\Delta y$};
    \draw[->] (0,0) -- (1.2,0) node[pos=0.9,below] {$t$};
    \ifprintanswers
        \draw[red,very thick,domain=0:1,smooth] plot(\x,-0.6*\x^2);
    \fi
\end{tikzpicture}
\hspace{5mm}
\begin{tikzpicture}[x=2.5cm,y=2.5cm]
    \draw[->] (0,-0.6) node[left,pos=0.5] {0} -- (0,0.6) node[above] {$v_x$};
    \draw[->] (0,0) -- (1.2,0) node[pos=0.9,below] {$t$};
    \ifprintanswers
        \draw[red,very thick,domain=0:1,smooth] plot(\x,0.3);
    \fi
\end{tikzpicture}
\hspace{5mm}
\begin{tikzpicture}[x=2.5cm,y=2.5cm]
    \draw[->] (0,-0.6) node[left,pos=0.5] {0} -- (0,0.6) node[above] {$v_y$};
    \draw[->] (0,0) -- (1.2,0) node[pos=0.9,below] {$t$};
    \ifprintanswers
        \draw[red,very thick,domain=0:1,smooth] plot(\x,-0.6*\x);
    \fi
\end{tikzpicture}
\end{EnvUplevel}

\part
Using the term ``acceleration,'' explain why each graph in the previous part has its particular shape.

\ifprintanswers
\textcolor{red}{Zero horizontal acceleration ($a_x = 0$) produces a linear $\Delta x$ graph with positive slope and a flat $v_x$ graph. Constant vertical acceleration ($a_y = -g = -\SI{10}{m/s^2}$) produces a quadric $\Delta y$ graph and a linear $v_y$ graph with negative slope.}
\else
\fillwithlines{21mm}
\fi

\end{parts}

\question %...Independence of Motion
A ball is launched horizontally from a cliff while another ball is dropped straight down from the same height at the same time. Which hits the ground first? Explain why.

\ifprintanswers
\textcolor{red}{Both hit the ground at the same time. Horizontal and vertical motions are independent. Since they fall from the same height, they both experience the same gravitational acceleration $g$ simultaneously.}
\else
\fillwithlines{21mm}
\fi

\question
For a horizontally launched projectile, draw and label the horizontal ($a_x$) and vertical ($a_y$) components of acceleration for each of the positions below. The magnitudes of the vectors should reflect the changing or constant nature of the component.

\begin{center}
\begin{tikzpicture}[x=3.5mm,y=3.5mm]
    \draw[fill=black!20] (-3,10.8) -- (0,10.8) -- (0,9.8) -- (-3,9.8);
    \draw[fill=black!8] (-3,9.8) -- (-1,9.8) -- (-1,-1) -- (-3,-1);
    \draw[domain=0:14.98,black] plot(\x,{trajectoryequation(\x,10,0)+11});
    \ifprintanswers
        \draw[->,thick,red] (0,11) -- ++(0,-3) node[below] {$a_y$};
    \fi
    \fill[black] (0,11) circle (0.2);
    \pgfplotsinvokeforeach{0.3,0.6,0.9,1.2,1.5}{
        \ifprintanswers
            \color{red}
        \else
            \color{white}
        \fi
        \coordinate (P) at (10*#1,{11-0.5*9.8*(#1)^2}); 
        \draw[->,thick] (P) -- ++(0,-3) node[below] {$a_y$};
        \fill[black] (P) circle (0.2);
    }
\end{tikzpicture}
\end{center}


\question %...Horizontal Launch Problem
A ball rolls off a \SI{1.3}{m} high table with a horizontal speed of \SI{4.0}{m/s}.

\begin{parts}
\part How long does it take to hit the ground?

\begin{solutionorbox}[4cm]
\begin{align*}
    \Delta y &= -\frac{1}{2} g t^2 \\[1ex]
    -2 \Delta y &= gt^2 \\[1ex]
    -\frac{2 \Delta y}{g} &= t^2 \\[1ex]
    t &= \sqrt{-\frac{2 \Delta y}{g}} \\[1ex]
    &= \sqrt{-\frac{2(-\SI{1.3}{m})}{\SI{10}{m/s^2}}} \\[1ex]
    &= \boxed{\SI{0.51}{s}}
\end{align*}
\end{solutionorbox}

\part
What is the horizontal component of velocity with which the ball strikes the ground?

\ifprintanswers
\textcolor{red}{The same as the launch velocity: \SI{4.0}{m/s}}
\else
\fillwithlines{7mm}
\fi

\part
Calculate the vertical component of the velocity, in the instant right before it strikes ground.

\begin{solutionorbox}[3cm]
\begin{align*}
    v_y &= -gt \\[1ex]
    &= - (\SI{10}{m/s^2})(\SI{0.51}{s}) \\[1ex]
    &= \boxed{-\SI{5.1}{m/s}}
\end{align*}
\end{solutionorbox}


\part How far from the base of the table does it land?

\begin{solutionorbox}[3cm]
\begin{align*}
    \Delta x = v_{ix} t = (\SI{4.0}{m/s})(\SI{0.51}{s}) = \boxed{\SI{2.0}{m}}
\end{align*}
\end{solutionorbox}
% \part What is the magnitude of the final velocity just before hitting the ground?  
\end{parts}




\question
In an experiment, a student launches a bowling ball horizontally at \SI{5.0}{m/s} from the top of a really tall building. Their lab partner uses a stopwatch to record a hang time of \SI{12.9}{s}. How tall is the building? Show all your work.

\medskip

\begin{minipage}{0.25\textwidth}
\begin{tikzpicture}[x=3mm,y=3mm]
    \draw[domain=0:10.5,densely dashed] plot(\x,{trajectoryequation10(\x,6,0)+15});
    \draw[fill=black!15] (-2,15) -- (0,15) -- (0,0) -- (-2,0);
    \node at (-0.5,15.8)  {\Strichmaxerl[2]};
    \draw[fill=black] (0.1,15.9) circle (2pt);
\end{tikzpicture}
\end{minipage}
\hspace{2mm}%
\fbox{
\begin{minipage}{0.65\textwidth}
\ifprintanswers
\color{red}
\else
\color{white}
\fi
\vspace{2mm}
The bowling ball's vertical displacement (change in position) is 

\begin{align*}
    \Delta y &= -\frac{1}{2}g t^2 \\[1ex]
    &= -\frac{1}{2} \left(\SI{10}{m/s^2}\right)\left(\SI{12.9}{s}\right)^2 \\[1ex]
    &= -\SI{832}{m}
\end{align*}

Therefore, the height of the building, which equals the magnitude of the ball's displacement, is $\boxed{\SI{832}{m}}$.
\vspace{2mm}
\end{minipage}
}

\question
Write a short story about an object undergoing projectile motion. You must use the following words: launch angle, initial speed, trajectory, apex, hang time, range, height, horizontal, and vertical.

\ifprintanswers
\else
\fillwithlines{56mm}
\fi

\clearpage

\end{questions}

\subsection*{Circular Motion}

\begin{questions}

\question
The figure below shows a fly undergoing circular motion. Draw and label the tangential velocity vector ($v$) and the centripetal acceleration vector ($a_c$). 

\begin{center}
\begin{tikzpicture}[x=2cm,y=2cm]
    \coordinate (P) at ({cos(135)},{sin(135)});
    \draw[densely dashed] (0,0) circle (1);
    \ifprintanswers
    \draw[thick,->,red] (P) -- ++({0.7*cos(45)},{-0.7*sin(45)}) node[above right=-1pt] {$a_c$};
    \draw[thick,red,->] (P) -- ++({-cos(45)},{-sin(45)}) node[above left=-1pt] {$v$};
    \fi
    \draw (P) node[rotate=135] {\twemoji[width=8mm]{fly}};
\end{tikzpicture}
\end{center}

\question 
Calculate the circumference of the circle below.

\bigskip

\begin{minipage}{0.25\textwidth}
\begin{tikzpicture}[x=1.5cm,y=1.5cm]
    \draw (0,0) circle (1);
    \draw (0,0) -- (1,0) node[above,pos=0.5] {\SI{6.0}{cm}};
\end{tikzpicture}
\end{minipage}
\hspace{1cm}%
\fbox{
\begin{minipage}{0.6\textwidth}
\ifprintanswers
\color{red}
\else
\color{white}
\fi
\begin{align*}
    C &= 2\pi r \\[1ex]
    &= 2\pi (\SI{6.0}{cm}) \\[1ex]
    &= \boxed{\SI{37.7}{cm}}
\end{align*}

\vspace{5mm}
\end{minipage}
}

\question \label{OXyBFH}
A car turns at \SI{20}{m/s} on a curved road and undergoes circular motion, as shown below. Calculate its centripetal acceleration.

\begin{minipage}{0.3\textwidth}
\centering
\begin{tikzpicture}[x=2cm,y=2cm]
    \draw[densely dashed] (1,0) arc (0:180:1);
    \draw[->,thick] ({cos(45)},{sin(45)}) -- ++({-0.8*cos(45)},{0.8*sin(45)}) node[left,black] {\SI{20}{m/s}};
    \ifprintanswers
    \draw[->,very thick] ({cos(45)},{sin(45)}) -- ++({-0.6*cos(45)},{-0.6*sin(45)}) node[above left=-2pt,black,pos=0.8] {$a_c$};
    \fi
    \fill (0,0) circle (1pt);
    \draw[<->] (0,0) -- (-1,0) node[above,pos=0.5] {\SI{50}{m}};
    \fill ({cos(45)},{sin(45)}) circle (3pt) node[right=3pt] {car};
\end{tikzpicture}
\end{minipage}
\hspace{1em}%
\fbox{
\begin{minipage}{0.6\textwidth}
\ifprintanswers
\color{red}
\else
\color{white}
\fi

\smallskip

\begin{align*}
    a_c &= \frac{v^2}{r} \\[1ex]
        &= \frac{(\SI{20}{m/s})^2}{\SI{50}{m}} \\[1ex]
        &= \boxed{\SI{8.0}{m/s^2}}
\end{align*}

\smallskip

\end{minipage}
}

\vspace{3mm}

\question
An object moves in counterclockwise circular motion, constrained by an attached string (not shown). The circular path is enclosed by a larger box, as shown in the figure below.

Using a dashed line, \textbf{\underline{draw}} the path the object would take if the string were cut at the instant the object passes through each of the points O--S. Extend each dashed line until it reaches the surrounding box.

\begin{center}
\begin{tikzpicture}[x=1.5cm,y=1.5cm]
     \draw[-{Stealth[length=2.5mm,width=2.5mm]}] ({cos(15)},{sin(15)}) arc (15:110:1);
     \draw[-{Stealth[length=2.5mm,width=2.5mm]}] ({cos(110)},{sin(110)}) arc (110:240:1);
    \draw[-{Stealth[length=2.5mm,width=2.5mm]}] ({cos(240)},{sin(240)}) arc (240:375:1);

    \fill[gray!30, even odd rule]
        (-2.6,-1.6) rectangle (2.6,1.6)
        (-2.5,-1.5) rectangle (2.5,1.5);
    \draw (-2.5,-1.5) rectangle (2.5,1.5);
    \draw (-2.6,-1.6) rectangle (2.6,1.6);
    \coordinate (O) at ({cos(45)},{sin(45)});
    \coordinate (P) at ({cos(155)},{sin(155)});
    \coordinate (Q) at ({cos(200)},{sin(200});
    \coordinate (R) at ({cos(270)},{sin(270});
    \coordinate (S) at ({cos(330)},{sin(330});
    
    \fill (O) circle (2.5pt) node[below left=-1pt] {O};
    \fill (P) circle (2.5pt) node[below right] {P};
    \fill (Q) circle (2.5pt) node[right=2pt] {Q};
    \fill (R) circle (2.5pt) node[above=2pt] {R};
    \fill (S) circle (2.5pt) node[above left] {S};

    \bgroup
    \ifprintanswers
    \begin{pgfonlayer}{background}
        \color{red}
        \draw[very thick,densely dashed,shift={(O)},rotate around={45:(0,0)}] (0,0) -- (0,1.1);
        \draw[very thick,densely dashed,shift={(P)},rotate around={155:(0,0)}] (0,0) -- (0,2.1);
        \draw[very thick,densely dashed,shift={(Q)},rotate around={200:(0,0)}] (0,0) -- (0,1.2);
        \draw[very thick,densely dashed,shift={(R)},rotate around={270:(0,0)}] (0,0) -- (0,2.5);
        \draw[very thick,densely dashed,shift={(S)},rotate around={330:(0,0)}] (0,0) -- (0,2.3);
    \end{pgfonlayer}
    \fi
    \egroup
\end{tikzpicture}
\end{center}

\ifprintanswers
\else
\clearpage
\fi



\question
A remote control toy airplane is undergoing uniform circular motion, and its path is shown below. It takes 45 seconds complete one circular trip. Calculate the plane's linear speed.

\bigskip

\begin{minipage}{0.25\textwidth}
\begin{tikzpicture}[x=2cm,y=2cm]
    \draw[densely dashed] (0,0) circle (1);
    \draw (0,0) -- (1,0) node[above,pos=0.5] {\SI{76}{m}};
\end{tikzpicture}
\end{minipage}
\hspace{1cm}%
\fbox{
\begin{minipage}{0.6\textwidth}
\ifprintanswers
\color{red}
\else
\color{white}
\fi
Distance traveled by the plane is

\begin{equation*}
    s = 2\pi r = 2\pi(\SI{76}{m}) = \SI{477.5}{m}
\end{equation*}

Linear speed is distance over time:

\begin{equation*}
    v = \frac{2\pi r}{t} = \frac{2\pi(\SI{76}{m})}{\SI{45}{s}} = \boxed{\SI{10.6}{m/s}}
\end{equation*}

\vspace{5mm}
\end{minipage}
}




\begin{EnvUplevel}
\textbf{Questions \ref{2nF3qF}--\ref{oSRba9}: Refer to the prompt and figure below.}

A volleyball is tethered to a pole by a 2.0-meter rope and is moving with circular motion. The figure below shows the bird's eye view of the system.

\begin{center}
\begin{tikzpicture}[x=2cm,y=2cm]
    \coordinate (P) at ({cos(45)},{-sin(45)});
    \draw[densely dashed] (0,0) circle (1);
    \draw[thick] (0,0) -- (P);
    \ifprintanswers
    \draw[very thick,->,red] (P) -- ++({-0.7*cos(45)},{0.7*sin(45)}) node[right=2mm] {$F_c$};
    \draw[red,dashed] (P) -- ++({2*cos(45)},{2*sin(45)});
    \draw[thick,red,->] (P) -- ++({cos(45)},{sin(45)}) node[above left=-1pt] {$v$};
    \fi
    \node at (P) {\twemoji[width=8mm]{volleyball}};
\end{tikzpicture}
\end{center}    
\end{EnvUplevel}

\question \label{2nF3qF}
In the figure above\ldots 

\begin{parts}
    \part draw and label the centripetal force vector on the ball.
    \part draw the path the volleyball would take if the rope suddenly snapped off at the instant shown, assuming the ball is moving counterclockwise.
\end{parts}


\question
The volleyball stays in circular motion due to the centripetal force $F_c$. What type of force is $F_c$?

\begin{randomizeoneparchoices}
    \correctchoice tension force
    \choice gravitational force
    \choice friction force
    \choice magnetic force
\end{randomizeoneparchoices}


\question \label{oSRba9}
The volleyball has a mass of 260 grams. If it takes 3.0 seconds to complete one trip around the circle, what is the centripetal force on the ball?

\begin{solutionorbox}[4cm]
The linear speed is

\begin{equation*}
    v = \frac{\text{distance}}{\text{time}} = \frac{2\pi r}{t} = \frac{2\pi(\SI{2.0}{m})}{\SI{3.0}{s}} = \SI{4.19}{m/s}
\end{equation*}

The centripetal force is

\begin{equation*}
    F_c = \frac{mv^2}{r} = \frac{(\SI{0.26}{kg})\left(\SI{4.19}{m/s}\right)^2}{\SI{2.0}{m}} = \boxed{\SI{2.28}{N}}
\end{equation*}
\end{solutionorbox}

\bigskip

\hrule


\clearpage


\question \label{}
A ball of mass $m$ tethered to a pole by a 6.0-meter rope is moving with circular motion. The centripetal force on the ball is $F$, and the ball moves with linear speed $v$. If the centripetal force increases to $\frac{3}{2}F$, what new length of rope must be used to keep the ball's speed at $v$?

\bigskip

\begin{minipage}{0.3\textwidth}
\centering
\begin{tikzpicture}[x=2cm,y=2cm]
    \draw[densely dashed] (1,0) arc (0:180:1);
    \fill (0,0) circle (1pt);
    \draw[thick] (0,0) -- ({cos(45)},{sin(45)}) node {\twemoji[width=8mm]{basketball}} node[pos=0.5,below right] {$r$};
\end{tikzpicture}
\end{minipage}
\hspace{1em}%
\fbox{
\begin{minipage}{0.6\textwidth}
\ifprintanswers
\color{red}
\else
\color{white}
\fi

The centripetal force formula

\begin{equation*}
    F_c = \frac{mv^2}{r}
\end{equation*}

states that centripetal force is inversely proportional to radius. So, if the force increases by a factor for $3/2$, the radius should decrease by the inverse factor of $2/3$. That is, the new radius should be

\begin{equation*}
    r_\mathrm{new} = \frac{2}{3} r_\mathrm{old} = \frac{2}{3} (\SI{6.0}{m}) = \boxed{\SI{4.0}{m}} 
\end{equation*}
\end{minipage}
}

\question
An object moves in circular motion around a path of radius $r$ with tangential speed $v$ and centripetal acceleration $a$. How does the centripetal acceleration change if\ldots

\begin{parts}
\part 
the speed is doubled and the radius stays constant?

\ifprintanswers
\textcolor{red}{Centripetal acceleration increases to $4a$.}
\else
\fillwithlines{7mm}
\fi

\part 
the radius is doubled and the speed stays constant?

\ifprintanswers
\textcolor{red}{Centripetal acceleration decreases to $\frac{1}{2} a$.}
\else
\fillwithlines{7mm}
\fi

\part 
both the speed and radius are doubled?

\ifprintanswers
\textcolor{red}{Centripetal acceleration increases to $2a$.}
\else
\fillwithlines{7mm}
\fi

\part 
the radius decreases to $\frac{1}{2}r$ and the speed stays constant?

\ifprintanswers
\textcolor{red}{Centripetal acceleration increases to $2a$.}
\else
\fillwithlines{7mm}
\fi

\part
the speed decreases to $\frac{1}{2}v$ and the radius stays constant?

\ifprintanswers
\textcolor{red}{Centripetal acceleration decreases to $\frac{1}{4} a$.}

\else
\fillwithlines{7mm}
\fi
\end{parts}

\question
An object of mass $m$ moves in circular motion around a path of radius $r$ with tangential speed $v$ due to a centripetal force $F$. How does the centripetal force change if\ldots

\begin{parts}
\part 
the mass is doubled and the speed and radius stay constant?

\ifprintanswers
\textcolor{red}{Centripetal force increases to $2F$.}
\else
\fillwithlines{7mm}
\fi

\part 
both the mass and radius are doubled and the speed stays constant?

\ifprintanswers
\textcolor{red}{Centripetal force stays constant.}
\else
\fillwithlines{7mm}
\fi

\part 
mass, speed, and radius are all doubled?

\ifprintanswers
\textcolor{red}{Centripetal force increases to $4F$.}
\else
\fillwithlines{7mm}
\fi

\part 
the mass decreases to $\frac{1}{2}m$ and the speed and radius stay constant?

\ifprintanswers
\textcolor{red}{Centripetal force decreases to $\frac{1}{2}F$.}
\else
\fillwithlines{7mm}
\fi

\part 
the radius decreases to $\frac{1}{2}r$ and the speed and mass stay constant?

\ifprintanswers
\textcolor{red}{Centripetal force increases to $2F$.}
\else
\fillwithlines{7mm}
\fi


\end{parts}

\end{questions}


\end{document}



% \clearpage

% \question
% The combined mass of Cody and his car, from Question \ref{OXyBFH}, is \SI{740}{kg}. Calculate the centripetal force on the car.

% \begin{randomizechoices}
%     \correctchoice \SI{2770}{N}
%     \choice \SI{178}{N}
%     \choice \SI{3083}{N}
%     \choice \SI{200400}{N}
% \end{randomizechoices}

% \question
% The JP Morgan Chase Tower is the tallest building in Houston. If you drop a bowling ball from rest from the top of the building, it takes 7.81 seconds for it to strike the ground below. How tall is the building?

% \begin{solutionorbox}[4cm]
% We know time is $t = \SI{7.81}{s}$ and the magnitude of gravitational acceleration is $g = \SI{10}{m/s^2}$. Since the object is dropped from rest, initial velocity is $v_i = 0$. The ball's vertical displacement is given by the one-dimensional motion equation

% \vspace{-1em}
% \begin{align*}
%     \Delta y &= v_i t - \frac{1}{2}g t^2 \\[1ex]
%     &= (\SI{0}{m/s})(\SI{7.81}{s}) - \frac{1}{2} (\SI{10}{m/s^2})(\SI{7.81}{s})^2 \\[1ex]
%     &= -\SI{305}{m}
% \end{align*}

% This vertical displacement must be equal in magnitude to the height of the building. Therefore, the building's height is \SI{305}{m}.    
% \end{solutionorbox}