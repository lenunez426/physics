\documentclass[answers]{exam}
\usepackage{marvosym}

%...TikZ & PGF
\usepackage{pgfplots}
\pgfplotsset{compat=1.11}
\tikzset{>=latex}
\usetikzlibrary{calc,math}
\usepackage{tikzsymbols}
\usepgfplotslibrary{fillbetween}
\usetikzlibrary{decorations.markings} 
\usetikzlibrary{arrows.meta} %...APP2 for arrows as objects and images
\usetikzlibrary{backgrounds} %...For shading portions of graphs
\usetikzlibrary{patterns} %...Unit 5 Problems
\usetikzlibrary{shapes.geometric} %...For drawing cylinders in Unit 2
\usepackage{makecell} %...use \thead{} to enable line skip in table headers
\tikzset{
    mark position/.style args={#1(#2)}{
        postaction={
            decorate,
            decoration={
                markings,
                mark=at position #1 with \coordinate (#2);
            }
        }
    }
} %...See https://tex.stackexchange.com/questions/43960/define-node-at-relative-coordinates-of-draw-plot

\tikzset{
    declare function = {trajectoryequation10(\x,\vi,\thetai)= tan(\thetai)*\x - 10*\x^2/(2*(\vi*cos(\thetai))^2);},
    declare function = {trajectoryequation(\x,\vi,\thetai)= tan(\thetai)*\x - 9.8*\x^2/(2*(\vi*cos(\thetai))^2);},
    declare function = {patheq(\x,\yi,\vi,\thetai)= \yi + tan(\thetai)*\x - 9.8*\x^2/(2*(\vi*cos(\thetai))^2);},
    declare function = {patheqten(\x,\yi,\vi,\thetai)= \yi + tan(\thetai)*\x - 10*\x^2/(2*(\vi*cos(\thetai))^2);} %like patheq but with gravity = 10
}

%...siunitx
\usepackage{siunitx}
\DeclareSIUnit{\nothing}{\relax}
\def\mymu{\SI{}{\micro\nothing} }
\DeclareSIUnit\mmHg{mmHg}
\DeclareSIUnit{\mile}{mi}
%...NOTE: "The product symbol between the number and unit is set using the quantity-product option."

%...Other
\usepackage{amsthm}
\usepackage{amsmath}
\usepackage{amssymb}
\usepackage{cancel}
\usepackage{subcaption}
\usepackage{dashrule}
\usepackage{enumitem}
% \usepackage{fontawesome}
\usepackage{fontawesome5}
\usepackage{multicol}
\usepackage{glossaries}
%\numberwithin{equation}{section}
\numberwithin{figure}{section}
\usepackage{float}
\usepackage{twemojis} %...twitter emojis
\usepackage{utfsym}
\usepackage{linearb} %...For \BPwheel in Unit 8
\newcommand{\R}{\mathbb{R}} %...real number symbol
\usepackage{graphicx}
\usepackage{mdframed} %...For FRQ teacher boxes
\graphicspath{ {../Figures/} }
\usepackage{hyperref}
\hypersetup{colorlinks=true,
    linkcolor=blue,
    filecolor=magenta,
    urlcolor=cyan,}
\urlstyle{same}
\newcommand{\hdashline}{{\hdashrule{\textwidth}{0.5pt}{0.8mm}}}
\newcommand{\hgraydashline}{{\color{lightgray} \hdashrule{0.99\textwidth}{1pt}{0.8mm}}}

%...Miscellaneous user-defined symbols
\newcommand{\fnet}{F_{\text{net}}} %...For net force
\newcommand{\bvec}[1]{\vec{\mathbf{#1}}} %...bold vector
\newcommand{\bhat}[1]{\,\hat{\mathbf{#1}}} %...bold hat vector
\newcommand{\que}{\mathord{?}}  %...Question mark symbol in equation env
%...Define thick horizontal rule for examples:
\newcommand{\hhrule}{\hrule\hrule}
\let\oldtexttt\texttt% Store \texttt
\renewcommand{\texttt}[2][black]{\textcolor{#1}{\ttfamily #2}}% 

%...For use in the exam document class
\newif\ifprintmetasolutions


%...Decreases space above and below align and gather enironment
\makeatletter
\g@addto@macro\normalsize{%
  \setlength\abovedisplayskip{-3pt}
  \setlength\belowdisplayskip{6pt} 
}
\makeatother





\usepackage[margin=1in]{geometry}
\usepackage[figurewithin=none]{caption}
\usepackage{exam-randomizechoices}

\CorrectChoiceEmphasis{\color{red}\bfseries}
\renewcommand{\solutiontitle}{\noindent\textbf{\textcolor{red}{Solution:}}\enspace}

\usepackage{OutilsGeomTikz}
\usepackage{utfsym} %...Symbols in Unit 7 Problems
\usepackage{tabu} %...Symbols in Unit 7 Problems

%...For use in Unit 2            %    
\setlength{\columnsep}{2cm}      %
\setlength{\columnseprule}{1pt}  %
\usepackage[none]{hyphenat}      %
%%%%%%%%%%%%%%%%%%%%%%%%%%%%%%%%%

%...For use in Unit 11 on Waves:
\pgfdeclarehorizontalshading{visiblelight}{50bp}{  %
color(0.00000000000000bp)=(red);                   %
color(8.33333333333333bp)=(orange);                %
color(16.66666666666670bp)=(yellow);               %
color(25.00000000000000bp)=(green);                %
color(33.33333333333330bp)=(cyan);                 %
color(41.66666666666670bp)=(blue);                 %
color(50.00000000000000bp)=(violet)                %
}                                                  %

\newcommand{\checkbox}[1]{%
  \ifnum#1=1
    \makebox[0pt][l]{\raisebox{0.15ex}{\hspace{0.1em}\Large$\checkmark$}}%
  \fi
  $\square$%
}
%%%%%%%%%%%%%%%%%%%%%%%%%%%%%%%%%%%%%%%%%%%%%%%%%%%%

%...If using circuitikz package:
% \ctikzset{bipoles/battery1/height=0.5}
% \ctikzset{bipoles/battery1/width=0.25}
% \ctikzset{bipoles/resistor/height=0.15}
% \ctikzset{bipoles/resistor/width=0.4}

\setrandomizerseed{1}

\firstpageheader{Physics}{Unit 2: Force Interactions}{Test Review}
\runningheader{}{}{}

\begin{document}
\section*{Know}

\begin{multicols}{3}
\begin{enumerate}[itemsep=0pt]
    \item force
    % \item external force
    \item net force
    \item gravity
    \item friction
    \item tension
    \item normal force
    % \item spring force
    \item free body diagram
\end{enumerate}
\end{multicols}

\section*{Understand}

\begin{itemize}[itemsep=0pt]
    \item Identify the type of forces acting in a scenario, including gravitational, normal, frictional, applied, 
    %spring,
    and tension.
    \item Identify and represent the force pairs of each interaction in a scenario .
    \item Recognize that force pairs consist of exactly two forces (same type) resulting from one interaction between two objects.
    \item Recognize that force pairs must be equal in magnitude and opposite in direction.
    % \item Identify instances in which equal and opposite forces are not a force pair (e.g. $F_\text{g}$ and $F_\text{n}$ are not force pairs even when they are equal in magnitude and opposite in direction).
    \item Diagram the forces acting on an object using a free-body diagram.
    \item Explain the effects of balanced forces on the motion of an object.
    \item Explain the effects of unbalanced forces on the motion of an object.
    \item Connect by matching, interpreting, and generating Multiple Representations, the concepts of constant velocity, constant momentum, constant KE, and balanced forces.
    \item Connect by matching, interpreting, and generating Multiple Representations, the concepts of changing velocity, changing momentum, changing KE, and unbalanced forces.
\end{itemize}

\section*{Do}

\begin{questions}

\question %6
A tennis ball strikes a racket. Which has a greater magnitude, the force of the racket on the ball or the force of the ball on the racket?

\question %5
A boxing glove exerts a force by striking someone's face. What is the force pair to the glove's push?

\question %10
The motion map for a shopping cart is shown below. What additional information do you need to know whether the object's speed is increasing or decreasing?

\begin{center}
    \begin{tikzpicture}[scale=0.7]
        \draw[domain=-10:0,mark=*,only marks, samples=6,mark size=3pt] plot({0.1*\x^2},0);
    \end{tikzpicture}
\end{center}


\question \label{PbScd} %1
A book is at rest is shown in the figure.

\begin{center}
\begin{tikzpicture}
    \draw[fill=black!20] (0,0) rectangle ++(4,0.4);
    \draw[fill=black!20] (0.5,0) rectangle ++(0.2,-2);
    \draw[fill=black!20] (3.3,0) rectangle ++(0.2,-2);
    \node[above=2.1mm] at (2,0) {\twemoji[height=1cm]{blue book}};
\end{tikzpicture}
\end{center}

What are the two forces exerted on the object? Draw the forces in a free-body diagram.

\clearpage

\question %4
A box is placed on top of a skateboard, and the system is allowed to move down a straight track. The motion of the box-skateboard system is shown below.

\begin{center}
\begin{tikzpicture}
    \begin{axis}[height=6cm,width=6cm,
        axis lines=left,
        ymin=0,ymax=4,
        xmin=0,xmax=4,
        ylabel={Position (m)},
        xlabel={Time (s)},
        ytick={0,1,...,4},
        xtick={0,1,...,4},
        grid=both,
        clip=false,
    ]
        \addplot[thick,domain=0:1] {2*x};
        \addplot[thick,domain=1:4] {0.2*(x-1)^2+2};
        \fill (0,0) circle (2pt) node[right=8pt,above=-1pt] {a};
        \fill (1,2) circle (2pt) node[right=5pt,above=0pt] {b};
        \fill (4,3.8) circle (2pt) node[left=2pt] {c};
    \end{axis}
\end{tikzpicture}
\end{center}

(a) From point a to point b, are the forces balanced or unbalanced? (b) Is the object speeding up, slowing down, or moving at constant speed, from a to b? (c) From point b to point c, are the forces balanced or unbalanced? (d) Is the object speeding up, slowing down, or moving at constant speed, from b to c?

\question %2
A cable is used to hang a traffic light from a metal bar. What type of force is the cable exerting on the traffic light?

\question %8
A runner is running to the right and is increasing in speed over time. Draw a motion map for their motion. 

\question %9
When the all forces on an object are balanced, what can you conclude about the object's velocity, momentum, and kinetic energy?


\question %3
Draw the force schema for the three force pairs in the scenario shown in Question \ref{PbScd}.


\question %7
True or False? The horizontal forces on the truck below are balanced, so the truck's velocity will increase.

\begin{center}
    \begin{tikzpicture}
        \draw (0,0) node {\reflectbox{\twemoji[height=3cm]{delivery truck}}};
        \draw[thick,<-,left=1.5cm] (0,0) -- (-4,0) node[above,pos=0.5] {400\,N};  
        \draw[thick,<-,right=1.5cm] (0,0) -- (+3,0) node[above,pos=0.5] {300\,N};  
    \end{tikzpicture}
\end{center}





\end{questions}
\end{document}




