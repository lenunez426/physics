\documentclass[]{exam}
\usepackage{marvosym}

%...TikZ & PGF
\usepackage{pgfplots}
\pgfplotsset{compat=1.11}
\tikzset{>=latex}
\usetikzlibrary{calc,math}
\usepackage{tikzsymbols}
\usepgfplotslibrary{fillbetween}
\usetikzlibrary{decorations.markings} 
\usetikzlibrary{arrows.meta} %...APP2 for arrows as objects and images
\usetikzlibrary{backgrounds} %...For shading portions of graphs
\usetikzlibrary{patterns} %...Unit 5 Problems
\usetikzlibrary{shapes.geometric} %...For drawing cylinders in Unit 2
\usepackage{makecell} %...use \thead{} to enable line skip in table headers
\tikzset{
    mark position/.style args={#1(#2)}{
        postaction={
            decorate,
            decoration={
                markings,
                mark=at position #1 with \coordinate (#2);
            }
        }
    }
} %...See https://tex.stackexchange.com/questions/43960/define-node-at-relative-coordinates-of-draw-plot

\tikzset{
    declare function = {trajectoryequation10(\x,\vi,\thetai)= tan(\thetai)*\x - 10*\x^2/(2*(\vi*cos(\thetai))^2);},
    declare function = {trajectoryequation(\x,\vi,\thetai)= tan(\thetai)*\x - 9.8*\x^2/(2*(\vi*cos(\thetai))^2);},
    declare function = {patheq(\x,\yi,\vi,\thetai)= \yi + tan(\thetai)*\x - 9.8*\x^2/(2*(\vi*cos(\thetai))^2);},
    declare function = {patheqten(\x,\yi,\vi,\thetai)= \yi + tan(\thetai)*\x - 10*\x^2/(2*(\vi*cos(\thetai))^2);} %like patheq but with gravity = 10
}

%...siunitx
\usepackage{siunitx}
\DeclareSIUnit{\nothing}{\relax}
\def\mymu{\SI{}{\micro\nothing} }
\DeclareSIUnit\mmHg{mmHg}
\DeclareSIUnit{\mile}{mi}
%...NOTE: "The product symbol between the number and unit is set using the quantity-product option."

%...Other
\usepackage{amsthm}
\usepackage{amsmath}
\usepackage{amssymb}
\usepackage{cancel}
\usepackage{subcaption}
\usepackage{dashrule}
\usepackage{enumitem}
% \usepackage{fontawesome}
\usepackage{fontawesome5}
\usepackage{multicol}
\usepackage{glossaries}
%\numberwithin{equation}{section}
\numberwithin{figure}{section}
\usepackage{float}
\usepackage{twemojis} %...twitter emojis
\usepackage{utfsym}
\usepackage{linearb} %...For \BPwheel in Unit 8
\newcommand{\R}{\mathbb{R}} %...real number symbol
\usepackage{graphicx}
\usepackage{mdframed} %...For FRQ teacher boxes
\graphicspath{ {../Figures/} }
\usepackage{hyperref}
\hypersetup{colorlinks=true,
    linkcolor=blue,
    filecolor=magenta,
    urlcolor=cyan,}
\urlstyle{same}
\newcommand{\hdashline}{{\hdashrule{\textwidth}{0.5pt}{0.8mm}}}
\newcommand{\hgraydashline}{{\color{lightgray} \hdashrule{0.99\textwidth}{1pt}{0.8mm}}}

%...Miscellaneous user-defined symbols
\newcommand{\fnet}{F_{\text{net}}} %...For net force
\newcommand{\bvec}[1]{\vec{\mathbf{#1}}} %...bold vector
\newcommand{\bhat}[1]{\,\hat{\mathbf{#1}}} %...bold hat vector
\newcommand{\que}{\mathord{?}}  %...Question mark symbol in equation env
%...Define thick horizontal rule for examples:
\newcommand{\hhrule}{\hrule\hrule}
\let\oldtexttt\texttt% Store \texttt
\renewcommand{\texttt}[2][black]{\textcolor{#1}{\ttfamily #2}}% 

%...For use in the exam document class
\newif\ifprintmetasolutions


%...Decreases space above and below align and gather enironment
\makeatletter
\g@addto@macro\normalsize{%
  \setlength\abovedisplayskip{-3pt}
  \setlength\belowdisplayskip{6pt} 
}
\makeatother





\usepackage[margin=1in]{geometry}
\usepackage[figurewithin=none]{caption}
\usepackage{exam-randomizechoices}

\CorrectChoiceEmphasis{\color{red}\bfseries}
\renewcommand{\solutiontitle}{\noindent\textbf{\textcolor{red}{Solution:}}\enspace}

\usepackage{OutilsGeomTikz}
\usepackage{utfsym} %...Symbols in Unit 7 Problems
\usepackage{tabu} %...Symbols in Unit 7 Problems

%...For use in Unit 2            %    
\setlength{\columnsep}{2cm}      %
\setlength{\columnseprule}{1pt}  %
\usepackage[none]{hyphenat}      %
%%%%%%%%%%%%%%%%%%%%%%%%%%%%%%%%%

%...For use in Unit 11 on Waves:
\pgfdeclarehorizontalshading{visiblelight}{50bp}{  %
color(0.00000000000000bp)=(red);                   %
color(8.33333333333333bp)=(orange);                %
color(16.66666666666670bp)=(yellow);               %
color(25.00000000000000bp)=(green);                %
color(33.33333333333330bp)=(cyan);                 %
color(41.66666666666670bp)=(blue);                 %
color(50.00000000000000bp)=(violet)                %
}                                                  %

\newcommand{\checkbox}[1]{%
  \ifnum#1=1
    \makebox[0pt][l]{\raisebox{0.15ex}{\hspace{0.1em}\Large$\checkmark$}}%
  \fi
  $\square$%
}
%%%%%%%%%%%%%%%%%%%%%%%%%%%%%%%%%%%%%%%%%%%%%%%%%%%%

%...If using circuitikz package:
% \ctikzset{bipoles/battery1/height=0.5}
% \ctikzset{bipoles/battery1/width=0.25}
% \ctikzset{bipoles/resistor/height=0.15}
% \ctikzset{bipoles/resistor/width=0.4}
\usepackage[normalem]{ulem}

\setrandomizerseed{1}

\firstpageheader{Physics}{Unit 1: Constant Motion}{Test}
\runningheader{}{}{}

\begin{document}
\begin{questions}

\question 
The position vs. time graph for obj 1 and obj 2 are shown below.

\begin{center}
    \begin{tikzpicture}
        \draw[->] (0,0) -- (0,3) node[rotate=90,pos=0.5,above] {$x$ (m)};
        \draw[->] (0,0) -- (3,0) node[pos=0.5,below] {$t$ (s)};
        \draw (0,0) -- (3,2) node[pos=1.1] {obj 1};
        \draw (0,0) -- (2.5,2.9) node[left=2pt,pos=0.9] {obj 2};
    \end{tikzpicture}
\end{center}

Based upon this graph, which object has the greater velocity and which explanation demonstrates the most correct reasoning?

\begin{randomizechoices}[norandomize]
    \choice Obj 1 because it traveled for a longer time.
    \choice Obj 2 because it had a greater displacement.
    \choice Obj 1 because it took longer time to travel the same displacement as obj 2.
    \correctchoice Obj 2 because it had a greater displacement in the same time interval as obj 1.
\end{randomizechoices}

\begin{EnvUplevel}
    \textbf{Question \ref{Q2}--\ref{Q3}}. Below is a quantitative motion map for a moving object. The dots indicate the position of the object each second.
\end{EnvUplevel}

\begin{center}
    \begin{tikzpicture}
    \begin{axis}[width=12cm,
        axis lines = left,
        axis y line=none,
        xlabel = {Position (m)},
        ymin=0, ymax=12, 
        xmin=0, xmax=10,
        xtick={0,5,10},
        minor x tick num=4,
        clip=false,
        ]
        \draw[->] (10,1) --++ (-1,0) node[above,pos=0.5] {$v$};
        \draw[->] (8,1) --++ (-1,0) node[above,pos=0.5] {$v$};
        \draw[->] (6,1) --++ (-1,0) node[above,pos=0.5] {$v$};
        \fill (10,1) circle (3pt) node[below=3pt] {start};
        \fill (8,1) circle (3pt);
        \fill (6,1) circle (3pt);
        \fill (4,1) circle (3pt);
    \end{axis}
    \end{tikzpicture}
\end{center}

\question \label{Q2}
What is the dog's displacement from $t=\SI{0}{s}$ to $t=\SI{1}{s}$ in the above motion map?

\begin{randomizechoices}[norandomize]
    \choice \SI{2}{m}
    \correctchoice \SI{-2}{m}
    \choice \SI{6}{m}
    \choice \SI{-6}{m}
\end{randomizechoices}

\question \label{Q3}
What is the dog's velocity from $t=\SI{0}{s}$ to $t=\SI{2}{s}$ in the above motion map?

\begin{randomizechoices}[norandomize]
    \correctchoice \SI{-2}{m/s}
    \choice $-3/2\,\mathrm{m/s}$
    \choice \SI{-3}{m/s}
    \choice \SI{-6}{m/s}
\end{randomizechoices}


\begin{EnvUplevel}
    \textbf{Questions \ref{Q4}--\ref{Q5}}. Below is a quantitative motion map for a moving object. 
\end{EnvUplevel}

\begin{center}
    \begin{tikzpicture}
        \begin{axis}[width=6cm,height=6cm,
            axis y line=left,
            axis x line=center,
            ylabel={Velocity (m/s)},
            xlabel={Time (s)},
            xmin=0,xmax=6,
            ymin=-3,ymax=3,
            xtick={0,1,...,6},
            ytick={-3,-2,...,3},
            grid=both,
            x label style={at={(axis description cs:1,0.5)},anchor=west},
            ]
            % \addplot[color=black,mark=*]
            %     coordinates {
            %     (0,1)(1,1)(2,1)(3,1)(4,1)(5,1)
            %     };
            \addplot[color=black,ultra thick,domain=0:6] {2};
        \end{axis}
    \end{tikzpicture}
\end{center}

\question \label{Q4}
What is the velocity of the object depicted in the graph below at time $t=\SI{2}{s}$?

\begin{randomizechoices}[norandomize]
    \choice \SI{0}{m/s} (at rest)
    \choice \SI{1}{m/s}
    \correctchoice \SI{2}{m/s}
    \choice \SI{5}{m/s}
\end{randomizechoices}

\question \label{Q5}
What is the displacement of the object depicted (shown) in the graph above from time $t=\SI{0}{s}$ to $t=\SI{5}{s}$?

\begin{randomizechoices}[norandomize]
    \choice \SI{0}{m}
    \choice \SI{5}{m}
    \choice \SI{2}{m}
    \correctchoice \SI{10}{m}
\end{randomizechoices}

\bigskip
\hrule

% \begin{EnvUplevel}
%     \textbf{Question \ref{Q6}--\ref{Q9}}. Below are the position vs time data tables for three runners---Runner A, Runner B, and Runner C. The starting line is the reference point in each case. The positive direction is considered forward on the track.
% \end{EnvUplevel}

\question
Below are the position vs time data tables for three runners---Runner A, Runner B, and Runner C. The starting line is the reference point in each case. The positive direction is considered forward on the track.

    
    \begin{table}[h!]
        \centering
        \begin{tabular}{|c|c||c|c||c|c|}
            \hline
             \textbf{Time} (s) & \textbf{Position} A (m) & \textbf{Time} (s) & \textbf{Position} B (m) & \textbf{Time} (s) & \textbf{Position} C (m)\\ \hline
             0 & 0 & 0 & 0 & 0 & $-10$\\ \hline
             1 & 4 & 1 & 2 & 1 & $-5$ \\ \hline
             2 & 8 & 2 & 4 & 2 & 0\\ \hline
             3 & 12 & 3 & 6 & 3 & 5\\ \hline
        \end{tabular}
    \end{table}

% \question \label{Q6}
% Which runner started behind the starting line?

% \begin{randomizechoices}[norandomize]
%     \choice Runner A
%     \choice Runner B
%     \correctchoice Runner C
%     \choice None of the runners
% \end{randomizechoices}

% \clearpage

% \question 
% Which runner is running in the negative direction for at least part of the race?

% \begin{randomizechoices}[norandomize]
%     \choice Runner A
%     \choice Runner B
%     \choice Runner C
%     \correctchoice None of the runners  
% \end{randomizechoices}

% \question
% Assuming the runners each continue with the same constant motion, in what order will they cross the finish line which is at the \SI{15}{m} position?

% \begin{randomizechoices}
%     \choice Order A, B, C 
%     \correctchoice Order A, C, B
%     \choice Order C, B, A
%     \choice Order C, A, B
% \end{randomizechoices}

% \question \label{Q9}
Rank the runners from slowest (smallest speed) to fastest (greatest speed).

\begin{randomizechoices}
    \choice Rank A, B, C 
    \correctchoice Rank B, A, C
    \choice Rank C, B, A
    \choice Rank C, A, B
\end{randomizechoices}

\bigskip
\hrule


\question
What is the velocity of the object depicted in the position vs. time graph below?

\begin{center}
    \begin{tikzpicture}
        \begin{axis}[width=6cm,height=6cm,
            axis y line=left,
            xmin=0,xmax=7,
            ymin=0,ymax=70,
            xtick={0,1,...,7},
            ytick={0,10,...,70},
            xlabel={$t$ (s)},
            ylabel={$x$ (m)},
            grid=both,
            ylabel style={rotate=-90},
            ]
            \draw[domain=0:7,ultra thick] plot (\x,{20+(10/3)*\x});
        \end{axis}
    \end{tikzpicture}
\end{center}

\begin{randomizeoneparchoices}[norandomize]
    \correctchoice \SI{3.33}{m/s}
    \choice \SI{6.67}{m/s}
    \choice \SI{0.30}{m/s}
    \choice \SI{0.15}{m/s}
\end{randomizeoneparchoices}    

% \question 
% A 65\,kg person is walking at a speed of 5\,m/s. Which of the following representations of motion CANNOT represent the person's motion?

% \begin{minipage}{0.45\textwidth}
% {\Large \textbf{A}}
%     \begin{center}
%     \begin{tikzpicture}
%         \draw[->] (0,0) -- (4.5,0) node[right] {$x$ (m)};
%         \foreach \i in {0,0.5,...,4} \fill (\i,0.5) circle (2pt);
%     \end{tikzpicture}
%     \end{center}
% \end{minipage}%
% \begin{minipage}{0.45\textwidth}
% {\Large \textbf{B}}
%     \begin{center}
%     \begin{tikzpicture}
%         \begin{axis}[width=5cm,height=5cm,
%             ylabel={$v$ (m/s)},
%             xlabel={$t$ (s)},
%             axis lines = left,
%             xmin=0,xmax=5,
%             ymin=0,ymax=10,
%             xtick={0,1,...,5},
%             ytick={0,2,...,10},
%             ]
%             \draw[thick,blue] (0,5) -- (5,5); 
%         \end{axis}
%     \end{tikzpicture}
%     \end{center}
% \end{minipage}

% \begin{minipage}{0.45\textwidth}
% {\Large \textbf{C}}
%     \begin{center}
%     \begin{tabular}{|c|c|}
%         \hline
%         \textbf{Time}: $t$ (s) & \textbf{Momentum}: $p$ (\SI{}{kg\cdot m/s}) \\ \hline
%          0 & 5\\ \hline
%          1 & 5\\ \hline
%          2 & 5\\ \hline
%          3 & 5\\ \hline
%     \end{tabular}
%     \end{center}
% \end{minipage}%
% \begin{minipage}{0.45\textwidth}
% {\Large \textbf{D}}
%     \begin{center}
%     \begin{tikzpicture}
%         \begin{axis}[width=5cm,height=5cm,
%             ylabel={KE (\SI{}{kg\cdot m^2/s^2})},
%             xlabel={$t$ (s)},
%             axis lines = left,
%             xmin=0,xmax=10,
%             ymin=0,ymax=1000,
%             xtick={0,2,...,10},
%             ytick={0,200,...,1000},
%             ]
%             \draw[thick,blue] (0,812.5) -- (10,812.5); 
%         \end{axis}
%     \end{tikzpicture}
%     \end{center}
% \end{minipage}

% \begin{randomizeoneparchoices}[norandomize]
%     \choice Representation A
%     \choice Representation B
%     \correctchoice Representation C
%     \choice Representation D
% \end{randomizeoneparchoices}

\clearpage
\question %63
The motion of a 4-kg object is described by the graph below. What is the kinetic energy of the object at 3.5 seconds?

\begin{center}
    \begin{tikzpicture}
        \begin{axis}[height=7cm,
            width=7cm,
            axis y line=left,
            axis x line = center,
            ylabel={Position (m)},
            xlabel={Time (s)},
            ymin=-30,ymax=30,
            xmin=0,xmax=10,
            grid=both,
            ytick={-30,-25,...,30},
            xtick={0,1,...,10},
            x label style={at={(axis description cs:1,0.5)},anchor=west},
        ]
        \addplot[ultra thick,black] coordinates{(0,10)(1,20)(3,20)(7,-20)(8,-20)(10,-25)};
        \end{axis}
    \end{tikzpicture}
\end{center}

\begin{randomizechoices}
    \correctchoice 200\,J
    \choice 400\,J
    \choice 40\,J
    \choice 20\,J
\end{randomizechoices}

\begin{solution}
    From the graph, the object's velocity from 3 to 7 seconds is

    \begin{equation*}
        v = \frac{\Delta x}{\Delta t} = \frac{-\SI{40}{m}}{\SI{4}{s}} = -\SI{10}{m/s}
    \end{equation*}

    Its kinetic energy is

    \begin{equation*}
        \mathrm{KE} = \frac{1}{2}mv^2 = \frac{1}{2}\left(\SI{4}{kg}\right)\left(-\SI{10}{m/s}\right)^2 = \boxed{\SI{200}{J}}
    \end{equation*}
\end{solution}

\question 
Mary is on a bus that is passing a traffic light. Which of the following statements best describes her motion?

\begin{randomizechoices}[norandomize]
    \choice She is moving relative to both the traffic light and the bus driver.
    \choice She is not moving relative to either the traffic light or the bus driver.
    \correctchoice She is moving relative to the traffic light, but she is not moving relative to the bus driver.
    \choice She is moving relative to the bus driver, but she is not moving relative to the traffic light.
\end{randomizechoices}

\question
Two cars are moving in opposite directions as pictured below:

\begin{center}
    \begin{tikzpicture}
        \draw[->] (0,0) -- (1,0) node[right] {\SI{30}{km/hr}};
        \draw (0,0) node[left=1em] {\textbf{A}} node[above=-2.7mm] {\reflectbox{\twemoji[width=6mm]{automobile}}};
        \begin{scope}[xshift=4cm,yshift=1cm]
            \draw[->] (0,0) -- (-1.5,0) node[left] {\SI{50}{km/hr}};
            \draw (0,0) node[right=1em] {\textbf{B}} node[above=-3mm] {\twemoji[width=7mm]{articulated lorry}};
        \end{scope}
    \end{tikzpicture}
\end{center}

Using car A as a reference, how fast is car B moving?

\begin{randomizeoneparchoices}[norandomize]
    \correctchoice \SI{80}{km/hr}
    \choice \SI{50}{km/hr}
    \choice \SI{30}{km/hr}
    \choice \SI{20}{km/hr}
\end{randomizeoneparchoices}

\begin{EnvUplevel}
    \textbf{Questions \ref{Q14}--\ref{Q16}} Imagine two astronauts playing catch in outer space in a location where both frictional and gravitational forces are negligible (do not make a difference). They are throwing and catching the three objects listed in the table below. 
\end{EnvUplevel}

\begin{center}
    \begin{tabular}{|l|l|l|}
        \hline
        \textbf{object} & \textbf{mass} & \textbf{velocity} \\ \hline
        wrench & \SI{2}{kg} & \SI{5}{m/s} \\ \hline
        $\mathrm{O}_2$ tank & \SI{20}{kg} & \SI{0.5}{m/s}\\ \hline
        Ball & \SI{0.2}{kg} & \SI{50}{m/s}\\ \hline
    \end{tabular}
\end{center}

\clearpage
\question \label{Q14}
Which object would resist a change in motion the most and why?

\begin{randomizechoices}[norandomize]
    \correctchoice The $\mathrm{O}_2$ Tank because its has the largest mass and therefore the greatest inertia
    \choice The $\mathrm{O}_2$ Tank because its has the largest mass and therefore the greatest momentum
    \choice The Ball because it has the largest speed and therefore the greatest inertia
    \choice The Ball because it has the largest speed and therefore the greatest momentum
\end{randomizechoices}

\question 
Which object has the greatest momentum once thrown based on the table above?

\begin{randomizechoices}[norandomize]
    \choice Wrench
    \choice $\mathrm{O}_2$ Tank
    \choice Ball
    \correctchoice All momenta are equal.
\end{randomizechoices}

\question \label{Q16}
Which object has the greatest kinetic energy once thrown based on the table above?

\begin{randomizechoices}[norandomize]
    \choice Wrench
    \choice $\mathrm{O}_2$ Tank
    \correctchoice Ball
    \choice All momenta are equal.
\end{randomizechoices}

\bigskip
\hrule


\question 
What is the kinetic energy of a \SI{2}{kg} object moving with a velocity of \SI{5}{m/s}?

\begin{randomizechoices}[norandomize]
    \choice \SI{7}{J}
    \choice \SI{10}{J}
    \correctchoice \SI{25}{J}
    \choice \SI{50}{J}
\end{randomizechoices}

\question
What is the momentum of a \SI{4}{kg} object moving with a velocity of \SI{7}{m/s}?

\begin{randomizechoices}[norandomize]
    \choice \SI{98}{kg\cdot m/s}
    \correctchoice \SI{28}{kg\cdot m/s}
    \choice \SI{11}{kg\cdot m/s}
    \choice \SI{196}{kg\cdot m/s}
\end{randomizechoices}

\question 
A person is standing at $+\SI{6}{m}$ from the reference (zero) point. If this person moves with a constant velocity of \SI{5}{m/s} in the positive direction, what is their final position after 3 seconds?

\begin{randomizechoices}[norandomize]
    \correctchoice 21\,m
    \choice 15\,m
    \choice \SI{-9}{m}
    \choice \SI{30}{m}
\end{randomizechoices}

% \question 
% A dog runs with a constant velocity of \SI{10}{m/s} for \SI{9}{s}. What is the dog's displacement?

% \begin{solution}
%     \begin{equation*}
%         \bar{v} = \frac{\Delta x}{\Delta t} \quad \Rightarrow \quad
%         \Delta x = \bar{v} t = \boxed{\SI{90}{m}}
%     \end{equation*}
% \end{solution}

\clearpage

\Large
\printkeytable



\end{questions}
\end{document}