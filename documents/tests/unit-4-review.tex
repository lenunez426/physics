\documentclass[answers]{exam}
\usepackage{marvosym}

%...TikZ & PGF
\usepackage{pgfplots}
\pgfplotsset{compat=1.11}
\tikzset{>=latex}
\usetikzlibrary{calc,math}
\usepackage{tikzsymbols}
\usepgfplotslibrary{fillbetween}
\usetikzlibrary{decorations.markings} 
\usetikzlibrary{arrows.meta} %...APP2 for arrows as objects and images
\usetikzlibrary{backgrounds} %...For shading portions of graphs
\usetikzlibrary{patterns} %...Unit 5 Problems
\usetikzlibrary{shapes.geometric} %...For drawing cylinders in Unit 2
\usepackage{makecell} %...use \thead{} to enable line skip in table headers
\tikzset{
    mark position/.style args={#1(#2)}{
        postaction={
            decorate,
            decoration={
                markings,
                mark=at position #1 with \coordinate (#2);
            }
        }
    }
} %...See https://tex.stackexchange.com/questions/43960/define-node-at-relative-coordinates-of-draw-plot

\tikzset{
    declare function = {trajectoryequation10(\x,\vi,\thetai)= tan(\thetai)*\x - 10*\x^2/(2*(\vi*cos(\thetai))^2);},
    declare function = {trajectoryequation(\x,\vi,\thetai)= tan(\thetai)*\x - 9.8*\x^2/(2*(\vi*cos(\thetai))^2);},
    declare function = {patheq(\x,\yi,\vi,\thetai)= \yi + tan(\thetai)*\x - 9.8*\x^2/(2*(\vi*cos(\thetai))^2);},
    declare function = {patheqten(\x,\yi,\vi,\thetai)= \yi + tan(\thetai)*\x - 10*\x^2/(2*(\vi*cos(\thetai))^2);} %like patheq but with gravity = 10
}

%...siunitx
\usepackage{siunitx}
\DeclareSIUnit{\nothing}{\relax}
\def\mymu{\SI{}{\micro\nothing} }
\DeclareSIUnit\mmHg{mmHg}
\DeclareSIUnit{\mile}{mi}
%...NOTE: "The product symbol between the number and unit is set using the quantity-product option."

%...Other
\usepackage{amsthm}
\usepackage{amsmath}
\usepackage{amssymb}
\usepackage{cancel}
\usepackage{subcaption}
\usepackage{dashrule}
\usepackage{enumitem}
% \usepackage{fontawesome}
\usepackage{fontawesome5}
\usepackage{multicol}
\usepackage{glossaries}
%\numberwithin{equation}{section}
\numberwithin{figure}{section}
\usepackage{float}
\usepackage{twemojis} %...twitter emojis
\usepackage{utfsym}
\usepackage{linearb} %...For \BPwheel in Unit 8
\newcommand{\R}{\mathbb{R}} %...real number symbol
\usepackage{graphicx}
\usepackage{mdframed} %...For FRQ teacher boxes
\graphicspath{ {../Figures/} }
\usepackage{hyperref}
\hypersetup{colorlinks=true,
    linkcolor=blue,
    filecolor=magenta,
    urlcolor=cyan,}
\urlstyle{same}
\newcommand{\hdashline}{{\hdashrule{\textwidth}{0.5pt}{0.8mm}}}
\newcommand{\hgraydashline}{{\color{lightgray} \hdashrule{0.99\textwidth}{1pt}{0.8mm}}}

%...Miscellaneous user-defined symbols
\newcommand{\fnet}{F_{\text{net}}} %...For net force
\newcommand{\bvec}[1]{\vec{\mathbf{#1}}} %...bold vector
\newcommand{\bhat}[1]{\,\hat{\mathbf{#1}}} %...bold hat vector
\newcommand{\que}{\mathord{?}}  %...Question mark symbol in equation env
%...Define thick horizontal rule for examples:
\newcommand{\hhrule}{\hrule\hrule}
\let\oldtexttt\texttt% Store \texttt
\renewcommand{\texttt}[2][black]{\textcolor{#1}{\ttfamily #2}}% 

%...For use in the exam document class
\newif\ifprintmetasolutions


%...Decreases space above and below align and gather enironment
\makeatletter
\g@addto@macro\normalsize{%
  \setlength\abovedisplayskip{-3pt}
  \setlength\belowdisplayskip{6pt} 
}
\makeatother





\usepackage[margin=1in]{geometry}
\usepackage[figurewithin=none]{caption}
\usepackage{exam-randomizechoices}

\CorrectChoiceEmphasis{\color{red}\bfseries}
\renewcommand{\solutiontitle}{\noindent\textbf{\textcolor{red}{Solution:}}\enspace}

\usepackage{OutilsGeomTikz}
\usepackage{utfsym} %...Symbols in Unit 7 Problems
\usepackage{tabu} %...Symbols in Unit 7 Problems

%...For use in Unit 2            %    
\setlength{\columnsep}{2cm}      %
\setlength{\columnseprule}{1pt}  %
\usepackage[none]{hyphenat}      %
%%%%%%%%%%%%%%%%%%%%%%%%%%%%%%%%%

%...For use in Unit 11 on Waves:
\pgfdeclarehorizontalshading{visiblelight}{50bp}{  %
color(0.00000000000000bp)=(red);                   %
color(8.33333333333333bp)=(orange);                %
color(16.66666666666670bp)=(yellow);               %
color(25.00000000000000bp)=(green);                %
color(33.33333333333330bp)=(cyan);                 %
color(41.66666666666670bp)=(blue);                 %
color(50.00000000000000bp)=(violet)                %
}                                                  %

\newcommand{\checkbox}[1]{%
  \ifnum#1=1
    \makebox[0pt][l]{\raisebox{0.15ex}{\hspace{0.1em}\Large$\checkmark$}}%
  \fi
  $\square$%
}
%%%%%%%%%%%%%%%%%%%%%%%%%%%%%%%%%%%%%%%%%%%%%%%%%%%%

%...If using circuitikz package:
% \ctikzset{bipoles/battery1/height=0.5}
% \ctikzset{bipoles/battery1/width=0.25}
% \ctikzset{bipoles/resistor/height=0.15}
% \ctikzset{bipoles/resistor/width=0.4}
\usepackage[none]{hyphenat}

\setrandomizerseed{1}

\newif\ifversionKlevel

\versionKleveltrue

% \firstpageheader{Physics L\\Review on Unit 4: Impulse and Work}{}{{Name:\enspace\makebox[5cm]{\hrulefill}}}
% \runningheader{Physics L}{}{Unit 4 Review}

% \ifversionKlevel
%     \firstpageheader{Physics K\\Review on Unit 4: Impulse and Work}{}{{Name:\enspace\makebox[5cm]{\hrulefill}}}
%     \runningheader{Physics K}{}{Unit 4 Review}
% \fi

\firstpageheader{{Name:\enspace\makebox[6cm]{\hrulefill}}}{}{Physics Review on Unit 4: Work and Impulse}

\begin{document}
\section*{Know}

\begin{multicols}{3}
\begin{enumerate}[itemsep=0pt]
    \item momentum
    \item kinetic energy (KE)
    \item impulse
    \item work
    \item net force
    \item change in momentum
    \item change in KE
    \item impulse-momentum theorem
    \item net force
    \item work-energy theorem
    \item power
\end{enumerate}
\end{multicols}

%\section*{Understand}

% How do you\dots

% \begin{itemize}[itemsep=0pt,topsep=2pt]
%     \item calculate an object's change in momentum from its change in velocity?
%     \item describe the relationship between the size of a net force on an object to the object’s change in momentum?
%     \item describe the relationship between how long the force is applied and an object’s change in momentum?
%     \item explain the real-life applications of the relationship between $F_\text{net}$ and $t$ in producing a given change in momentum (i.e. safety equipment, catapult plane launch)?
%     \item solve problems utilizing the Impulse-Momentum Theorem, $F_\text{net} \Delta t = \Delta p$?
%     \item identify if a net force is doing work on an object based on there being a component of the net force parallel to displacement?
%     \item identify if a net force is doing work on an object based on a change in kinetic energy?
%     \item calculate the net work done on an object?
%     \item describe the relationship between the size of the net force and an object's changing kinetic energy?
%     \item describe the relationship between the displacement over which the net force is applied and the changing kinetic energy of an object?
%     \item solve problems using the work-energy theorem, $W_\text{net} = \Delta \mathrm{KE}$?
%     \item compare the power of various situations with differing time, work, and $\Delta \mathrm{KE}$? (Note: only change one variable at a time)
%     \item calculate the power based on the work and time?
%     \item calculate the power based on change in kinetic energy?
% \end{itemize}

\section*{Do}

\begin{questions}

\question 
As the length of time 
%a given 
that a net force acts on an object increases, what other quantity must also increase? List as many physical quantities as you can. 

\begin{solution}
    velocity, momentum, change in velocity, change in momentum, position, displacement
\end{solution}

\ifprintanswers
\else
\fillwithlines{2cm}
\fi

\question
Describe a scenario in which a force is applied to an object while the work done on the object is zero.

\begin{solution}
    Answers will vary.

    No work is done when the displacement of the object is zero. For example, a person pushes on a large truck that remains stationary, applying a strong force but doing no work on the truck.
\end{solution}

\ifprintanswers
\else
\fillwithlines{3cm}
\fi

\question
The object depicted below has a mass of \SI{3}{kg}.

\begin{center}
    \begin{tikzpicture}[x=0.3cm,y=0.3cm]
        \draw[->,thick] (1cm,0.5cm) -- ++(10,0) node[right] {\SI{10}{N}};
        \draw[->,thick] (0,0.5cm) -- ++(-6,0) node[left] {\SI{6}{N}};
        \draw[fill=lightgray] (0,0) rectangle ++(1cm,1cm);
    \end{tikzpicture}
\end{center}

The forces shown last for 6 seconds. What is the object's change in velocity during that time?

\begin{solutionorbox}[3cm]
The net force is

\begin{equation*}
    F_\text{net} = \SI{10}{N} - \SI{6}{N} = \SI{4}{N}
\end{equation*}

The impulse-momentum theorem states

\begin{equation*}
    F_\text{net} \Delta t = \Delta p
\end{equation*}

Plugging in force and time, the change in momentum is

\begin{align*}
    \Delta p &= F_\text{net} \Delta t \\[1ex]
    &= (\SI{4}{N})(\SI{6}{s}) \\[1ex]
    &= \SI{24}{N\cdot s} \\[1ex]
    &= \SI{24}{kg\cdot m/s}
\end{align*}

But change in momentum is

\begin{equation*}
    \Delta p = m \Delta v
\end{equation*}

Therefore, the change in velocity is

\begin{equation*}
    \Delta v = \frac{\Delta p}{m} = \frac{\SI{24}{kg\cdot m/s}}{\SI{3}{kg}} = \boxed{\SI{8}{m/s}}
\end{equation*}
\end{solutionorbox}

\question
Draw a velocity vs time graph for an object traveling with a changing momentum. Explain how the graph shows that momentum is changing.

\begin{solutionorbox}[4cm]
Answers will vary.

\begin{center}
    \begin{tikzpicture}
        \begin{axis}[height=4cm,
            width=4cm,
            ymin=0,ymax=5,
            xmin=0,xmax=3,
            ticks=none,
            axis lines=left,
            ylabel={$v$},
            y label style={rotate=-90},
            xlabel={$t$},
        ]
            \addplot[domain=0:2.5,very thick,black] {1.5*x};
        \end{axis}
    \end{tikzpicture}
\end{center}

Momentum is mass multiplied by velocity. So, if velocity is changing, so is momentum.
\end{solutionorbox}


\question
The average net force shown in the graph below acts on a \SI{12}{kg} object for 9 seconds.

\begin{center}
    \begin{tikzpicture}
        \begin{axis}[width=6cm,height=4cm,
            xmin=0,xmax=10,
            ymin=0,ymax=5,
            axis lines=left,
            xlabel={Time (s)},
            ylabel={Force (N)},
            y label style={rotate=-90},
            xtick={0,1,...,10},
            ytick={0,1,...,5},
            grid=both,
        ]
            \draw[very thick] (0,4) -- ++(10,0);
        \end{axis}
    \end{tikzpicture}
\end{center}

What is the impulse on the object during this time period?

\begin{solutionorbox}[4cm]
The net force is $F_\text{net} = \SI{4}{N}$ and the time interval is $\Delta t$. Therefore impulse is


\begin{equation*}
    F_\mathrm{net} \Delta t = (\SI{4}{N})(\SI{10}{s}) = \boxed{\SI{40}{N \cdot s}}
\end{equation*}

\end{solutionorbox}

\question
Two identical coffee mugs are dropped from the exact same height. The first mug lands on a concrete floor and breaks. The second mug lands on a pillow and does not break. Using principles learned in this unit on impulse and work, explain why the first coffee mug broke while the second one didn't.

\begin{solution}
    The mugs experience the same change in momentum ($\Delta p$), since they are identical and are dropped from the same height. By the impulse momentum theorem,

    \begin{equation*}
        F_{\mathrm{net}} \Delta t = \Delta p
    \end{equation*}

    each mug experiences a different force and impact time with the surface on which it falls. The mug landing on the pillow took a longer time $\Delta t$ and thus experienced a smaller force $F_\text{net}$. The mug that broke experiences a smaller impact time and thus felt a larger force, which was sufficient to break it.
\end{solution}

\ifprintanswers
\else
\fillwithlines{3cm}
\fi


\ifversionKlevel
\else
\question
The motion map above shows a ball's motion during an 8 second time period. 

\begin{center}
    \begin{tikzpicture}[scale=0.6]
        \draw[domain=0:10,mark=*,only marks, samples=7,mark size=3.5pt] plot({0.1*\x^2},0);
        \draw[thick, <-] (6,0.8) -- ++(-3,0);
    \end{tikzpicture}
\end{center}

During that time, what is the direction of the net force? Is the net work done on the ball positive or negative? Explain.

\ifprintanswers
\else
\fillwithlines{3cm}
\fi

\begin{solution}
The force is to the right (positive), since the object is accelerating to the right. Defining the rightward direction as positive, there is positive net work on the object.
\end{solution}
\fi



\question
The velocity vs time graph below shows the motion of a \SI{12}{kg} object.

\begin{center}
\begin{tikzpicture}
    \begin{axis}[height=6cm,width=7cm,
        axis y line=left,
        axis x line=center,
        ylabel={Velocity (m/s)},
        xlabel={Time (s)},
        ymin=-20,ymax=20,
        xmin=0,xmax=10,
        grid=both,
        ytick={-20,-15,...,20},
        xtick={0,1,...,10},
        x label style={at={(axis description cs: 1,0.5)},anchor=west},
    ]
        \addplot[very thick] coordinates{(0,-10) (6,-10) (6,15) (10,15)};
    \end{axis}
\end{tikzpicture}
\end{center}

\ifprintanswers
\else
\clearpage
\fi

What is the object's change in momentum during this time?

\begin{solutionorbox}[5cm]
According to the graph the object's initial and final velocities are \SI{-10}{m/s} and \SI{15}{m/s}. Therefore, it's change in momentum is

\begin{align*}
    \Delta p &= m \Delta v \\[1ex]
    &= m(v_f - v_i) \\[1ex]
    &= (\SI{12}{kg}) \left(\SI{15}{m/s} - \left(\SI{-10}{m/s}\right)\right) \\[1ex]
    &= \boxed{\SI{300}{kg\cdot m/s}}
\end{align*}
\end{solutionorbox}

\question
A student drops a \SI{3}{kg} textbook from rest from the second floor of the school. If the book reaches a velocity of \SI{11}{m/s} as it hits the ground, what is its change in momentum from the fall?

\begin{solutionorbox}[4cm]
\begin{align*}
    \Delta p &= m \Delta v \\[1ex]
    &= m\left(v_f - v_i\right) \\[1ex]
    &= (\SI{3}{kg})(\SI{11}{m/s} - \SI{0}{m/s}) \\[1ex]
    &= \boxed{\SI{33}{kg\cdot m/s}}
\end{align*}
\end{solutionorbox}

\question
What average net force would need to be applied to an object for \SI{2.0}{s}, in order to change its momentum by \SI{300}{kg\cdot m/s}?

\begin{solutionorbox}[3cm]
By the impulse momentum theorem,

\begin{equation*}
    \Delta p = F_\mathrm{net} \Delta t
\end{equation*}

Therefore,

\begin{equation*}
    F_\mathrm{net} = \frac{\Delta p}{\Delta t} = \frac{\SI{300}{kg\cdot m/s}}{\SI{2.0}{s}} = \boxed{\SI{150}{N}}
\end{equation*}
\end{solutionorbox}

\ifversionKlevel
\question
Chase, the 15-kg dog, slides across a slippery floor at \SI{4}{m/s}. He gets bumped by Rubble, his dog buddy, increasing his speed to \SI{12}{m/s}. Find the work done on Chase.

\begin{solutionorbox}[3cm]
By the work-energy theorem,

\begin{align*}
    W_\text{net} = \Delta \mathrm{KE} &= \frac{1}{2}mv_f^2 - \frac{1}{2}m v_i^2 \\[1ex]
    &= \frac{1}{2} (\SI{15}{kg}) (\SI{12}{m/s})^2 - \frac{1}{2} (\SI{15}{kg}) (\SI{4}{m/s})^2 \\[1ex]
    &= \boxed{\SI{960}{J}}
\end{align*}   
\end{solutionorbox}

\ifprintanswers
\else
\clearpage
\fi

\question
Justin Thyme is traveling down Lake Avenue at 32.9\,m/s in his 1722-kg 1992 Camaro. He spots a police car with a radar gun and quickly slows down to a legal speed of 19.4\,m/s.

\begin{parts}
\part 
Determine the initial kinetic energy of the Camaro.

\begin{solutionorbox}[1.5cm]
\begin{equation*}
    K_i = \frac{1}{2}mv_i^2 = \frac{1}{2}(\SI{1722}{kg})(\SI{32.9}{m/s})^2 = \boxed{\SI{9.32e5}{J}}
\end{equation*}
\end{solutionorbox}

\part 
Determine the kinetic energy of the Camaro after slowing down.

\begin{solutionorbox}[1.5cm]
\begin{equation*}
    K_f = \frac{1}{2}mv_f^2 = \frac{1}{2}(\SI{1722}{kg})(\SI{19.4}{m/s})^2 = \boxed{\SI{3.24e5}{J}}
\end{equation*}
\end{solutionorbox}

\part Determine the amount of work done on the Camaro during the deceleration.

\begin{solutionorbox}[1.5cm]
\begin{align*}
    W &= \Delta K \\[1ex]
    &= K_f - K_i \\[1ex]
    &= \SI{3.24e5}{J} - \SI{9.32e5}{J} \\[1ex]
    &= \boxed{\SI{-6.08e5}{J}}
\end{align*}
\end{solutionorbox}

\end{parts}



\question
The graph below shows the net force on a scooter as a function of time.

\begin{center}
\begin{tikzpicture}
    \begin{axis}[height=5cm,width=6cm,
        axis lines=left,
        ylabel={Net Force (N)},
        xlabel={Time (s)},
        ymin=0,ymax=1000,
        xmin=0,xmax=6,
        ytick={0,200,...,1000},
        xtick={0,1,...,6},
        grid=both,
        minor y tick num=1,
        grid=both
    ]
        \addplot[very thick] (0,0) -- (3,900);   
    \end{axis}
\end{tikzpicture}
\end{center}

At time $t=0$, the velocity of the scooter is \SI{10}{m/s}. The scooter's mass is \SI{90}{kg}. 

\begin{parts}
\part What is the average force on the scooter during the time interval?

\begin{solutionorbox}[1.5cm]
The average force is

\begin{equation*}
    F_\mathrm{ave} = \frac{\SI{0}{N} + \SI{900}{N}}{2} = \boxed{\SI{450}{N}}
\end{equation*}
\end{solutionorbox}

\part Using this average force, calculate the scooter's final velocity.

\begin{solutionorbox}[4.5cm]
The impulse-momentum theorem states

\begin{equation*}
    F_\text{ave} \Delta t = \Delta p =  m (v_f - v_i)
\end{equation*}

where the known values in this problem are $F_\mathrm{ave} = \SI{450}{N}$, $\Delta t = \SI{3.0}{s}$, $m = \SI{90}{kg}$, and \\$v_i = \SI{10}{m/s}$. Dividing both sides by mass leads to

\begin{equation*}
    v_f - v_i = \frac{F_\mathrm{ave} \Delta t}{m}
\end{equation*}

Finally, adding initial velocity to both sides results in 

\begin{align*}
    v_f &= \frac{F_\text{ave} \Delta t}{m} + v_i \\[1ex]
        &= \frac{(\SI{450}{N})(\SI{3}{s})}{\SI{90}{kg}} + \SI{10}{m/s} \\[1ex]
        &= \boxed{\SI{25}{m/s}}
\end{align*}
\end{solutionorbox}
\end{parts}

\fi



\end{questions}
\end{document}