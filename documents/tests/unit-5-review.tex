\documentclass[answers]{exam}
\usepackage{marvosym}

%...TikZ & PGF
\usepackage{pgfplots}
\pgfplotsset{compat=1.11}
\tikzset{>=latex}
\usetikzlibrary{calc,math}
\usepackage{tikzsymbols}
\usepgfplotslibrary{fillbetween}
\usetikzlibrary{decorations.markings} 
\usetikzlibrary{arrows.meta} %...APP2 for arrows as objects and images
\usetikzlibrary{backgrounds} %...For shading portions of graphs
\usetikzlibrary{patterns} %...Unit 5 Problems
\usetikzlibrary{shapes.geometric} %...For drawing cylinders in Unit 2
\usepackage{makecell} %...use \thead{} to enable line skip in table headers
\tikzset{
    mark position/.style args={#1(#2)}{
        postaction={
            decorate,
            decoration={
                markings,
                mark=at position #1 with \coordinate (#2);
            }
        }
    }
} %...See https://tex.stackexchange.com/questions/43960/define-node-at-relative-coordinates-of-draw-plot

\tikzset{
    declare function = {trajectoryequation10(\x,\vi,\thetai)= tan(\thetai)*\x - 10*\x^2/(2*(\vi*cos(\thetai))^2);},
    declare function = {trajectoryequation(\x,\vi,\thetai)= tan(\thetai)*\x - 9.8*\x^2/(2*(\vi*cos(\thetai))^2);},
    declare function = {patheq(\x,\yi,\vi,\thetai)= \yi + tan(\thetai)*\x - 9.8*\x^2/(2*(\vi*cos(\thetai))^2);},
    declare function = {patheqten(\x,\yi,\vi,\thetai)= \yi + tan(\thetai)*\x - 10*\x^2/(2*(\vi*cos(\thetai))^2);} %like patheq but with gravity = 10
}

%...siunitx
\usepackage{siunitx}
\DeclareSIUnit{\nothing}{\relax}
\def\mymu{\SI{}{\micro\nothing} }
\DeclareSIUnit\mmHg{mmHg}
\DeclareSIUnit{\mile}{mi}
%...NOTE: "The product symbol between the number and unit is set using the quantity-product option."

%...Other
\usepackage{amsthm}
\usepackage{amsmath}
\usepackage{amssymb}
\usepackage{cancel}
\usepackage{subcaption}
\usepackage{dashrule}
\usepackage{enumitem}
% \usepackage{fontawesome}
\usepackage{fontawesome5}
\usepackage{multicol}
\usepackage{glossaries}
%\numberwithin{equation}{section}
\numberwithin{figure}{section}
\usepackage{float}
\usepackage{twemojis} %...twitter emojis
\usepackage{utfsym}
\usepackage{linearb} %...For \BPwheel in Unit 8
\newcommand{\R}{\mathbb{R}} %...real number symbol
\usepackage{graphicx}
\usepackage{mdframed} %...For FRQ teacher boxes
\graphicspath{ {../Figures/} }
\usepackage{hyperref}
\hypersetup{colorlinks=true,
    linkcolor=blue,
    filecolor=magenta,
    urlcolor=cyan,}
\urlstyle{same}
\newcommand{\hdashline}{{\hdashrule{\textwidth}{0.5pt}{0.8mm}}}
\newcommand{\hgraydashline}{{\color{lightgray} \hdashrule{0.99\textwidth}{1pt}{0.8mm}}}

%...Miscellaneous user-defined symbols
\newcommand{\fnet}{F_{\text{net}}} %...For net force
\newcommand{\bvec}[1]{\vec{\mathbf{#1}}} %...bold vector
\newcommand{\bhat}[1]{\,\hat{\mathbf{#1}}} %...bold hat vector
\newcommand{\que}{\mathord{?}}  %...Question mark symbol in equation env
%...Define thick horizontal rule for examples:
\newcommand{\hhrule}{\hrule\hrule}
\let\oldtexttt\texttt% Store \texttt
\renewcommand{\texttt}[2][black]{\textcolor{#1}{\ttfamily #2}}% 

%...For use in the exam document class
\newif\ifprintmetasolutions


%...Decreases space above and below align and gather enironment
\makeatletter
\g@addto@macro\normalsize{%
  \setlength\abovedisplayskip{-3pt}
  \setlength\belowdisplayskip{6pt} 
}
\makeatother





\usepackage[margin=1in]{geometry}
\usepackage[figurewithin=none]{caption}
\usepackage{exam-randomizechoices}

\CorrectChoiceEmphasis{\color{red}\bfseries}
\renewcommand{\solutiontitle}{\noindent\textbf{\textcolor{red}{Solution:}}\enspace}

\usepackage{OutilsGeomTikz}
\usepackage{utfsym} %...Symbols in Unit 7 Problems
\usepackage{tabu} %...Symbols in Unit 7 Problems

%...For use in Unit 2            %    
\setlength{\columnsep}{2cm}      %
\setlength{\columnseprule}{1pt}  %
\usepackage[none]{hyphenat}      %
%%%%%%%%%%%%%%%%%%%%%%%%%%%%%%%%%

%...For use in Unit 11 on Waves:
\pgfdeclarehorizontalshading{visiblelight}{50bp}{  %
color(0.00000000000000bp)=(red);                   %
color(8.33333333333333bp)=(orange);                %
color(16.66666666666670bp)=(yellow);               %
color(25.00000000000000bp)=(green);                %
color(33.33333333333330bp)=(cyan);                 %
color(41.66666666666670bp)=(blue);                 %
color(50.00000000000000bp)=(violet)                %
}                                                  %

\newcommand{\checkbox}[1]{%
  \ifnum#1=1
    \makebox[0pt][l]{\raisebox{0.15ex}{\hspace{0.1em}\Large$\checkmark$}}%
  \fi
  $\square$%
}
%%%%%%%%%%%%%%%%%%%%%%%%%%%%%%%%%%%%%%%%%%%%%%%%%%%%

%...If using circuitikz package:
% \ctikzset{bipoles/battery1/height=0.5}
% \ctikzset{bipoles/battery1/width=0.25}
% \ctikzset{bipoles/resistor/height=0.15}
% \ctikzset{bipoles/resistor/width=0.4}
\usepackage[none]{hyphenat}

\setrandomizerseed{1}

\newif\ifversionKlevel

\versionKlevelfalse

% \firstpageheader{Physics\\Review on Unit 5: Force Analysis}{}{{Name:\enspace\makebox[5cm]{\hrulefill}}}
% \runningheader{Physics}{}{Unit 4 Review}

% \ifversionKlevel
%     \firstpageheader{Physics K\\Review on Unit 5: Force Analysis}{}{{Name:\enspace\makebox[5cm]{\hrulefill}}}
%     \runningheader{Physics K}{}{Unit 4 Review}
% \fi

\firstpageheader{{Name:\enspace\makebox[6cm]{\hrulefill}}}{}{Physics Review on Unit 5: Force Analysis}

\begin{document}
\subsection*{Know}

\begin{multicols}{3}
\begin{enumerate}[itemsep=0pt]
    \item Newton's law of universal gravitation
    \item gravitational force
    \item weight
    \item mass
    \item gravitational field
    \item normal force
    \item frictional force
    \item tension force
    \item free body diagram
    \item magnitude
    \item direction
    \item net force equation
\end{enumerate}
\end{multicols}

% \subsection*{Understand}

% How do you\dots

% \begin{itemize}[itemsep=0pt,topsep=2pt]
%     \item 
% \end{itemize}

\subsection*{Do}

\begin{questions}
\question 
A physics textbook that weighs \SI{16.0}{N} rests on a table. What is the normal force the table exerts on the textbook?

\begin{solutionorbox}[1cm]
It's the same as the weight of the book:

\begin{equation*}
    F_\mathrm{N} = \SI{16}{N}
\end{equation*}
\end{solutionorbox}



\question 
What is the weight of an 12 kilogram object on the surface of Earth?

\begin{solutionorbox}[1cm]
\begin{equation*}
    F_g = mg = (\SI{12}{kg})(\SI{10}{N/kg}) = \boxed{\SI{120}{N}}
\end{equation*}
\end{solutionorbox}


\question
An object is on a rough, horizontal surface is pushed to the right at a constant speed. Write net force equations that correctly describe the forces acting in the horizontal and vertical directions. 

\begin{center}
    \begin{tikzpicture}
        \fill (0,0) circle (3pt);
        \draw [<->] (-1,0) node[left] {$F_\mathrm{f}$} -- (+1,0) node[right] {$F_\mathrm{A}$};
        \draw [<->] (0,-1) node[below] {$F_\mathrm{g}$} -- (0,+1) node[above] {$F_\mathrm{N}$};
    \end{tikzpicture}
\end{center}

\begin{solutionorbox}[2cm]
\begin{align*}
    F_{\mathrm{net},x} &= F_\mathrm{A} - F_f = 0 \\[1ex]
    F_{\mathrm{net},y} &= F_\mathrm{N} - F_g = 0
\end{align*}
\end{solutionorbox}

\question 
Peter Parker pulls an empty sled to the right at a constant speed. MJ then jumps onto the sled while Peter is still pulling the sled. How does the magnitude of the friction force after MJ jumped onto the sled compare to the friction force before?

\ifprintanswers
\else
\fillwithlines{1.5cm}
\fi

\begin{solution}
    The friction force is proportional to the normal force, so friction increases.
\end{solution}

\question
Draw a free-body diagram of a ball falling with air resistance and speeding up.

\begin{center}
\begin{minipage}{4cm}
\centering
\begin{tikzpicture}[scale=0.75,transform shape]
    \node at (0,0) {\twemoji[height=1cm]{tennis}};
    \draw[black!80] (-0.6,0.2) -- ++(0,1)
                    (+0.6,0.2) -- ++(0,1)
                    (+0.75,0.4) -- ++(0,0.5)
                    (-0.75,0.4) -- ++(0,0.5);
    \draw[opacity=0] (0,0) -- (0,-2.2);
\end{tikzpicture}
\end{minipage}%
\begin{minipage}{4cm}
\centering
\begin{tikzpicture}[x=1.5cm,y=1.5cm]
    \ifprintanswers \color{red} \else \color{white} \fi
    \draw[thick,->] (0,0) -- (0,-1) node[right] {$F_g$};
    \draw[thick,->] (0,0) -- (0,0.5) node[right] {$F_d$};
    \fill[black] (0,0) circle (3pt);
\end{tikzpicture}
\end{minipage}
\end{center}

\ifprintanswers
\else
\clearpage
\fi
\question 
Three objects with the same mass are sliding across a rough surface with no applied forces acting on any of them. The velocity vs time graph shown below depicts their motion. Which object experiences the greatest friction force?

\begin{center}
    \begin{tikzpicture}
        \begin{axis}[width=7cm,
            height=5cm,
            xmin=0,xmax=30,
            ymin=0,ymax=10,
            ylabel={Velocity (m/s)},
            xlabel={Time (s)},
            grid=both,
            axis lines=left,
            xtick={0,5,...,30},
            % ytick={0,2,...,10},
            % y label style={at={(axis description cs:-0.15,.5)},rotate=90,anchor=south},
            % x label style={at={(axis description cs:1,.5)},anchor=west},
            % minor tick num=4,
            ]
            \draw[thick] (0,8) -- (10,0) node[above=3pt,pos=0.9] {\bfseries C};
            \draw[thick] (0,8) -- (28,0) node[above=2pt,pos=0.9] {\bfseries A};
            \draw[thick] (0,8) -- (17,0) node[above=2pt,pos=0.9] {\bfseries B};
        \end{axis}
    \end{tikzpicture}
\end{center}

\begin{solution}
    The object that comes to rest in the shortest time interval experiences the greatest friction: Object C.
\end{solution}




\question
(a) Describe what happens to an object's mass when it is transported from Earth to the Moon. (b)~Then describe what happens to its weight.

\ifprintanswers
\else
\fillwithlines{21mm}
\fi

\begin{solution}
    (a) The object's mass stays constant. (b) The object's weight decreases under the smaller gravitational field strength on the Moon.
\end{solution}


\question 
An experiment is conducted on Planet X in which mass on a spring scale is varied and the weight is recorded. The results are graphed below. 

\begin{center}
    \begin{tikzpicture}
        \begin{axis}[width=8cm,
            height=6cm,
            xmin=0,xmax=16,
            ymin=0,ymax=26,
            xlabel={Mass (kg)},
            ylabel={Weight (N)},
            xtick={0,2,...,16},
            ytick={0,4,...,26},
            minor tick num=1,
            grid=both,
            axis lines=left,
        ]
            \addplot[color=black,thick,domain=0:15,mark=*,samples=5] {1.5*x};
        \end{axis}
    \end{tikzpicture}
\end{center}

What is the acceleration due to gravity on Planet X? (In other words, what is the strength of the gravitational field?)

\begin{solutionorbox}[4cm]
    The gravitational field strength is the slope of the graph:

\begin{equation*}
    g = \frac{\SI{21}{N} - 0}{\SI{15.0}{kg} - 0} = \boxed{\SI{1.4}{N/kg}}
\end{equation*}
\end{solutionorbox}



\ifprintanswers
\else
\clearpage
\fi

% \begin{solution}
% \begin{center}
%     \begin{tikzpicture}
%         \fill (0,0) circle (3pt);
%         \draw[->] (0,0) -- (0,0.6) node[above] {$F_d$};
%         \draw[->] (0,0) -- (0,-1.2) node[below] {$F_g$};
%     \end{tikzpicture}
% \end{center}
% \end{solution}

% \question
% Which diagram best represents the gravitational forces, $F_g$, between a satellite, S, and Earth?

% \begin{center}
%     \begin{tikzpicture}
%         \draw[->,white] (0,0) -- (0,-0.6-0.6) node[right] {$F_g$}; %...for scaling; don't erase
%         \draw[->] (0,0) -- (0,+0.6+0.6) node[right] {$F_g$};
%         \draw[fill=white] (0,0) circle (6mm) node {Earth};
%         \begin{scope}[yshift=2.8cm]
%             \draw[->,white] (0,0) -- (0,0.3+0.6) node[right] {$F_g$}; %...for scaling; don't erase
%             \draw[->] (0,0) -- (0,-0.3-0.6) node[right] {$F_g$};
%             \draw[fill=white] (0,0) circle (3mm) node {S};
%             \node at (0,1.5) {\bfseries Diagram A};
%         \end{scope}
%     \end{tikzpicture}
%     \hspace{2em}
%     \begin{tikzpicture}
%         \draw[->] (0,0) -- (0,-0.6-0.6) node[right] {$F_g$};
%         \draw[fill=white] (0,0) circle (6mm) node {Earth};
%         \begin{scope}[yshift=2.8cm]
%             \draw[->,white] (0,0) -- (0,0.3+0.6) node[right] {$F_g$}; %...for scaling; don't erase
%             \draw[->] (0,0) -- (0,-0.3-0.6) node[right] {$F_g$};
%             \draw[fill=white] (0,0) circle (3mm) node {S};
%             \node at (0,1.5) {\bfseries Diagram B};
%         \end{scope}
%     \end{tikzpicture}
%     \hspace{2em}
%     \begin{tikzpicture}
%         \draw[->] (0,0) -- (0,-0.6-0.6) node[right] {$F_g$};
%         \draw[fill=white] (0,0) circle (6mm) node {Earth};
%         \begin{scope}[yshift=2.8cm]
%             \draw[->] (0,0) -- (0,0.3+0.6) node[right] {$F_g$};
%             \draw[fill=white] (0,0) circle (3mm) node {S};
%             \node at (0,1.5) {\bfseries Diagram C};
%         \end{scope}
%     \end{tikzpicture}
%     \hspace{2em}
%     \begin{tikzpicture}
%         \draw[->,white] (0,0) -- (0,-0.6-0.6) node[right] {$F_g$}; %...for scaling; don't erase
%         \draw[->] (0,0) -- (0,+0.6+0.6) node[right] {$F_g$};
%         \draw[fill=white] (0,0) circle (6mm) node {Earth};
%         \begin{scope}[yshift=2.8cm]
%             \draw[->] (0,0) -- (0,0.3+0.6) node[right] {$F_g$};
%             \draw[fill=white] (0,0) circle (3mm) node {S};
%             \node at (0,1.5) {\bfseries Diagram D};
%         \end{scope}
%     \end{tikzpicture}
% \end{center}

% \begin{solution}
%     Diagram A
% \end{solution}


\question
According to the Newton's universal law of gravitation, when the distance between any two objects decreases, how does the magnitude of the force of gravity between the two objects change?

\ifprintanswers
\else
\fillwithlines{2cm}
\fi

\begin{solution}
    The gravitational force increases in proportion to the distance squared.
\end{solution}



\begin{EnvUplevel}
    \textbf{Questions \ref{Q24}--\ref{Q25}.}  A box with a mass of \SI{20}{kg} is pulled across a smooth floor with a force of \SI{90}{N} at an angle above the horizontal, as shown below. The horizontal component of the applied force is \SI{72}{N}.
\end{EnvUplevel}

\begin{center}
\begin{tikzpicture}[x=1.1cm,y=1.1cm]
    \draw (0,0) -- (5.5,0);
    \draw[thick,fill=black!10] (2,0) rectangle ++(1.5,1) node[pos=0.5] {\SI{20}{kg}};
    \begin{scope}[shift={(3.5,0.5)}]
        \draw[dashed] (0,0) -- ++(1.5,0);
        \draw[very thick,->] (0,0) -- ++({1.5*cos(37)},{1.5*sin(37)}) node[right] {$\SI{90}{N}$};
        \draw (0.5,0) arc (0:37:0.5) node[right=2pt,pos=0.7] {$\theta$};
    \end{scope}
\end{tikzpicture}
\end{center}

\question \label{Q24}
What is the magnitude of the vertical component of the tension force?

\begin{solutionorbox}[1cm]

\begin{center}
\begin{tikzpicture}[x=4cm,y=4cm]
    \draw[thick,->] (0,0) -- ({cos(37)},{sin(37)}) node[above,pos=0.5,rotate=37] {$F = \SI{90}{N}$};
    \draw[thick,->] ({cos(37)},0) -- ++(0,{sin(37)}) node[right,pos=0.5] {$F_y$};
    \draw[thick,->] (0,0) -- ({cos(37)},0) node[below,pos=0.5] {$F_x = \SI{72}{N}$};
\end{tikzpicture}
\end{center}

By the Pythagorean theorem, the force relates to its components as

\begin{equation*}
    F_x^2 + F_y^2 = F^2
\end{equation*}

Plugging in values and solving for the vertical component leads to

\begin{equation*}
    F_y = \sqrt{90^2 - 72^2} = \boxed{\SI{54}{N}}
\end{equation*}
\end{solutionorbox}


\question \label{Q25}
What is the magnitude of the normal force acting on the box?

\begin{solutionorbox}[1cm]

\begin{center}
\begin{tikzpicture}[x=2cm,y=2cm]
    \fill (0,0) circle (4pt);
    \draw[thick,->] (-1.5pt,0) -- ++(0,1) node[left] {$F_\mathrm{N}$};
    \draw[thick,->,densely dashed] (+1.5pt,0) -- ++(0,0.37) node[right] {$F_y$};
    \draw[thick,->] (0,0) -- ++(0,-1.37) node[right] {$F_g$};
    \draw[thick,->,densely dashed] (0,0) -- (0.5,0) node[right] {$F_x$};
\end{tikzpicture}
\end{center}
The net force equation in the vertical direction states

\begin{equation*}
    F_y + F_\mathrm{N} = F_g
\end{equation*}

where the weight of the box is

\begin{equation*}
    F_g = mg = (\SI{20}{kg})(\SI{10}{N/kg}) = \SI{200}{N}
\end{equation*}

Therefore, the normal force on the box is

\begin{align*}
    F_\mathrm{N} &= F_g - F_y \\[1ex]
    &= \SI{200}{N} - \SI{54}{N} \\[1ex]
    &= \boxed{\SI{146}{N}}
\end{align*}
\end{solutionorbox}

\question
The diagram shows two bowling balls, of different masses, separated by half a meter. 

\begin{EnvUplevel}
\centering
\begin{minipage}{7cm}
\centering
\begin{tikzpicture}[x=8mm,y=8mm]
    \draw (0,0) circle (1) node {\SI{6}{kg}} node[above=8mm] {\textbf{A}};
    \draw (5,0) circle (1) node {\SI{8}{kg}} node[above=8mm] {\textbf{B}};
    \begin{scope}[shift={(0,-1.2)}]
        \draw[|<->|] (0,0) -- (5,0) node[below,pos=0.5] {\SI{0.5}{m}};
    \end{scope}
\end{tikzpicture}
\end{minipage}%
\fbox{
\begin{minipage}{7cm}
\centering
\vspace{0.5cm}
\begin{tikzpicture}[x=8mm,y=8mm]
    \ifprintanswers
        \draw[red,thick,->] (0,0) -- (1.5,0) node[right] {$F_g$};
        \draw[red,thick,->] (5,0) -- ++(-1.5,0) node[left] {$F_g$};
    \fi
    \fill (0,0) circle (3pt) node[above=3pt] {A};
    \fill (5,0) circle (3pt) node[above=3pt] {B};
\end{tikzpicture}
\vspace{0.5cm}
\end{minipage}
}
\end{EnvUplevel}

\begin{parts}
\part Draw free-body diagrams consisting only of the gravitational forces the balls exert on each other, in the space provided above. 


\part What is the magnitude of the gravitational force exerted by ball A on ball B?

\begin{solutionorbox}[2cm]
\begin{equation*}
    F_g = \frac{G m_1 m_2}{r^2} = \frac{\left(\SI{6.67e-11}{N\cdot m^2/kg^2}\right)(\SI{6}{kg})(\SI{8}{kg})}{(\SI{0.5}{m})^2} = \boxed{\SI{1.28e-8}{N}}
\end{equation*}
\end{solutionorbox}
\end{parts}

\question
In the figure below, static friction prevents a box from sliding. Draw the free-body diagram.

% \begin{center}
% \begin{tikzpicture}[x=7mm,y=7mm,rotate=-37]
%     \draw (0,0) -- (5,0);
%     \draw[fill=black!10] (2,0) rectangle ++(1,1);
%     \end{tikzpicture}
% \end{center}

\begin{center}
\begin{minipage}{3.5cm}
\centering
\begin{tikzpicture}[x=7mm,y=7mm,rotate=-37]
    \draw (0,0) -- (5,0);
    \draw[fill=black!10] (2,0) rectangle ++(1,1);
\end{tikzpicture}
\end{minipage}%
\begin{minipage}{3.5cm}
\centering
\begin{tikzpicture}[x=1.5cm,y=1.5cm,rotate=-37]
    \ifprintanswers \color{red} \else \color{white} \fi
        \draw[thick,->,rotate=37] (0,0) -- (0,-1) node[below] {$F_g$};
        \draw[thick,->] (0,0) -- (0,{cos(37)}) node[above] {$F_\mathrm{N}$};
        \draw[thick,->] (0,0) -- ({-sin(37)},0) node[left] {$F_f$};
    \fill[black] (0,0) circle (3pt);
\end{tikzpicture}
\end{minipage}
\end{center}





% \question
% The object is pulled at constant velocity by a force at an angle to the surface with friction.

% \begin{center}
% \begin{tikzpicture}[x=7mm,y=7mm]
%     \draw (-1.5,0) -- ++(4,0);
%     \draw[fill=black!10] (0,0) rectangle ++(1,1);
%     \draw[thick,->] (1,1) -- ++({1.4*cos(45)},{1.4*sin(45)}) node[right] {$F$};
%     \draw[dashed] (1,1) -- ++({1.4*cos(45)},0);
%     \node[xshift=4.4mm,yshift=2mm] at (1,1) {\small $\theta$};
%     \ifprintanswers
%     \color{red}
%     \else
%     \color{white}
%     \fi
%     \begin{scope}[xshift=4cm,x=1.5cm,y=1.5cm]
%         \draw[thick,->] (0,0) -- (0,{1-0.8*sin(45)}) node[left] {$F_\mathrm{N}$};
%         \draw[thick,->] (0,0) -- (0,-1) node[below] {$F_g$};
%         \draw[thick,->,opacity=0.3] (0,0) -- ({0.8*cos(45)},{0.8*sin(45)}) node[right] {$F$};
%         \draw[thick,->] (0,{1-0.8*sin(45)}) -- ++(0,{0.8*sin(45)}) node[above] {$F_y$};
%         \draw[thick,->] (0,0) -- ({0.8*cos(45)},0) node[right] {$F_x$};
%         \draw[thick,->] (0,0) -- ({-0.8*cos(45)},0) node[left] {$F_f$};
%         \fill[black] (0,0) circle (3pt);
%     \end{scope}
% \end{tikzpicture}
% \end{center}

% Write the net force equation in the horizontal direction.


\question
A rocket \twemoji[width=3.5mm]{rocket} is moving across various positions near the moon. When the rocket is located at the position shown in the figure below, the gravitational force on the rocket by the moon is \SI{1440}{N}.

\begin{center}
\begin{tikzpicture}
    \draw[step=1cm,gray] (0,0) grid (15,4);
    \node[opacity=0.9] at (1,2) {\twemoji[width=2cm]{full moon}};
    \draw (5,2) node[rotate=135] {\twemoji[width=5mm]{rocket}} node[below=3pt,align=center] {\small original\\ \small position};
    \fill (3,2) circle (2pt) node[above left] {A};
    \fill (9,2) circle (2pt) node[above left] {B};
    \fill (13,2) circle (2pt) node[above left] {C};
    \fill (7,2) circle (2pt) node[above left] {D};
    \fill (15,2) circle (2pt) node[above left] {E};
    \fill (11,2) circle (2pt) node[above left] {F};    
    \fill (7,4) circle (2pt) node[above left] {G};  
\end{tikzpicture}
\end{center}

Find the gravitational force on the rocket, in newtons, when the rocket is located at each of the Positions A through G. Show your work in the space below.

\begin{solutionorbox}[9cm]

Let $F_0 = \SI{1440}{N}$ be the original force of gravity on the rocket, and $r_0$ be the distance corresponding to the rocket's original position. Let $F$ be the new gravitational force at a different position, and $r$ be the new position. 

Newton's universal law of gravitation states

\begin{equation*}
    F_g = \frac{G m_1 m_2}{r^2}
\end{equation*}

and shows that gravitational force is inversely proportional to the distance squared:

\begin{equation*}
    F_g \propto \frac{1}{r^2}
\end{equation*}

Therefore, the new force as a proportion of the original force is given by

\begin{equation*}
    \boxed{F = \frac{F_0}{\left(r/r_0\right)^2}}
\end{equation*}

At Position A, counting squares, we see that the rocket's distance is cut in half compared to the old distance. The ratio of the old distance so the ratio is

\begin{equation*}
    \frac{r}{r_0} = \frac{2}{4} = \frac{1}{2}
\end{equation*}

Therefore, the gravitational force at Position A is
\vspace{-1ex}

\begin{align*}
    F_\mathrm{A} &= \frac{F_0}{\left(r/r_0\right)^2} \\[1ex]
    &= \frac{F_0}{(1/2)^2} \\[1ex]
    &= \frac{F_0}{0.25} \\[1ex]
    &= \frac{\SI{1440}{N}}{0.25} \\[1ex]
    &= \boxed{\SI{5760}{N}}
\end{align*}


At Position B, the rocket's distance is doubled compared to the old distance:

\begin{equation*}
    \frac{r}{r_0} = \frac{8}{4} = \frac{2}{1}
\end{equation*}

So,

\begin{align*}
    F_\mathrm{B} &= \frac{F_0}{\left(r/r_0\right)^2} \\[1ex]
    &= \frac{F_0}{(2/1)^2} \\[1ex]
    &= \frac{\SI{1440}{N}}{4} \\[1ex]
    &= \boxed{\SI{360}{N}}
\end{align*}

At Position C, counting squares, we see that the rocket's distance is tripled compared to the old distance:

\begin{equation*}
    \frac{r}{r_0} = \frac{12}{4} = \frac{3}{1}
\end{equation*}

Therefore,

\begin{align*}
    F_\mathrm{C} &= \frac{F_0}{\left(r/r_0\right)^2} \\[1ex]
    &= \frac{F_0}{(3/1)^2} \\[1ex]
    &= \frac{\SI{1440}{N}}{9} \\[1ex]
    &= \boxed{\SI{160}{N}}
\end{align*}


At Position D,

\begin{equation*}
    \frac{r}{r_0} = \frac{6}{4} = \frac{3}{2}
\end{equation*}

So,

\begin{align*}
    F_\mathrm{D} &= \frac{F_0}{\left(r/r_0\right)^2} \\[1ex]
    &= \frac{F_0}{(3/2)^2} \\[1ex]
    &= \frac{\SI{1440}{N}}{2.25} \\[1ex]
    &= \boxed{\SI{640}{N}}
\end{align*}

At Position E,

\begin{equation*}
    \frac{r}{r_0} = \frac{14}{4} 
\end{equation*}

Thus,

\begin{align*}
    F_\mathrm{E} &= \frac{F_0}{\left(r/r_0\right)^2} \\[1ex]
    &= \frac{F_0}{(14/4)^2} \\[1ex]
    &= \frac{\SI{1440}{N}}{12.25} \\[1ex]
    &= \boxed{\SI{118}{N}}
\end{align*}


At Position F, 

\begin{equation*}
    \frac{r_0}{r} = \frac{10}{4} 
\end{equation*}

So,

\begin{align*}
    F_\mathrm{F} &= \frac{F_0}{\left(r/r_0\right)^2} \\[1ex]
    &= \frac{F_0}{(10/4)^2} \\[1ex]
    &= \frac{\SI{1440}{N}}{6.25} \\[1ex]
    &= \boxed{\SI{230}{N}}
\end{align*}


At Position G, the new distance has a magnitude of $\sqrt{6^2 + 2^2}$, as shown below.

\begin{center}
\begin{tikzpicture}[x=6mm,y=6mm]
    \draw (0,0) -- (6,0) node[below,pos=0.5] {$6$};
    \draw (6,0) -- (6,2) node[right,pos=0.5] {$2$};
    \draw (0,0) -- (6,2) node[above,pos=0.5,rotate=18] {$\sqrt{6^2 + 2^2}$};
    \node[opacity=0.7] at (0,0) {\twemoji[width=8mm]{full moon}};
    \fill (6,2) circle (2pt) node[above left] {G};
\end{tikzpicture}
\end{center}

The ratio of distances is

\begin{equation*}
    \frac{r}{r_0} = \frac{\sqrt{6^2 + 2^2}}{4} 
\end{equation*}

Therefore,

\begin{align*}
    F_\mathrm{G} &= \frac{F_0}{\left(r/r_0\right)^2} \\[1ex]
    &= \frac{F_0}{\left(\displaystyle \frac{\sqrt{6^2 + 2^2}}{4}\right)^2} \\[1ex]
    &= \frac{F_0}{\left(6^2 + 2^2\right)/4^2} \\[1ex]
    &= \frac{F_0}{2.5} \\[1ex]
    &= \frac{\SI{1440}{N}}{2.5} \\[1ex]
    &= \boxed{\SI{576}{N}}
\end{align*}
\end{solutionorbox}


\question
The object is accelerated rightward by an applied force parallel to the rough surface. Draw the free-body diagram, and write the net force equations in horizontal and vertical directions.

\begin{center}
\begin{minipage}[c][4cm][c]{0.45\textwidth}
\begin{tikzpicture}[x=7mm,y=7mm,remember picture]
    \draw (-1.5,0) -- ++(4,0);
    \draw[fill=black!10] (0,0) rectangle ++(1,1);
    \draw[thick,->] (1,0.5) -- ++(1.2,0) node[right] {$F$};
    \ifprintanswers
    \color{red}
    \else
    \color{white}
    \fi
    \begin{scope}[xshift=4cm,x=1cm,y=1cm]
        \draw[thick,->] (0,0) -- (0,+1) node[above] {$F_\mathrm{N}$};
        \draw[thick,->] (0,0) -- (0,-1) node[below] {$F_g$};
        \draw[thick,->] (0,0) -- (1.5,0) node[right] {$F$};
        \draw[thick,->] (0,0) -- (-0.7,0) node[left] {$F_f$};
        \fill[black] (0,0) circle (3pt);  
    \end{scope}
\end{tikzpicture}
\end{minipage}%
\begin{minipage}[c][4cm][c]{0.35\textwidth}
    \large
    \begin{align*}
        F_{\mathrm{net},x} &= {\ifprintanswers \color{red} F - F_f = m a_x \fi} \\[1ex]
        F_{\mathrm{net},y} &= {\ifprintanswers \color{red} F_\mathrm{N}  - F_g = 0 \fi}
    \end{align*}
\end{minipage}
\end{center}
\end{questions}
\end{document}