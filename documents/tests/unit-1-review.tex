\documentclass[answers]{exam}
\usepackage{marvosym}

%...TikZ & PGF
\usepackage{pgfplots}
\pgfplotsset{compat=1.11}
\tikzset{>=latex}
\usetikzlibrary{calc,math}
\usepackage{tikzsymbols}
\usepgfplotslibrary{fillbetween}
\usetikzlibrary{decorations.markings} 
\usetikzlibrary{arrows.meta} %...APP2 for arrows as objects and images
\usetikzlibrary{backgrounds} %...For shading portions of graphs
\usetikzlibrary{patterns} %...Unit 5 Problems
\usetikzlibrary{shapes.geometric} %...For drawing cylinders in Unit 2
\usepackage{makecell} %...use \thead{} to enable line skip in table headers
\tikzset{
    mark position/.style args={#1(#2)}{
        postaction={
            decorate,
            decoration={
                markings,
                mark=at position #1 with \coordinate (#2);
            }
        }
    }
} %...See https://tex.stackexchange.com/questions/43960/define-node-at-relative-coordinates-of-draw-plot

\tikzset{
    declare function = {trajectoryequation10(\x,\vi,\thetai)= tan(\thetai)*\x - 10*\x^2/(2*(\vi*cos(\thetai))^2);},
    declare function = {trajectoryequation(\x,\vi,\thetai)= tan(\thetai)*\x - 9.8*\x^2/(2*(\vi*cos(\thetai))^2);},
    declare function = {patheq(\x,\yi,\vi,\thetai)= \yi + tan(\thetai)*\x - 9.8*\x^2/(2*(\vi*cos(\thetai))^2);},
    declare function = {patheqten(\x,\yi,\vi,\thetai)= \yi + tan(\thetai)*\x - 10*\x^2/(2*(\vi*cos(\thetai))^2);} %like patheq but with gravity = 10
}

%...siunitx
\usepackage{siunitx}
\DeclareSIUnit{\nothing}{\relax}
\def\mymu{\SI{}{\micro\nothing} }
\DeclareSIUnit\mmHg{mmHg}
\DeclareSIUnit{\mile}{mi}
%...NOTE: "The product symbol between the number and unit is set using the quantity-product option."

%...Other
\usepackage{amsthm}
\usepackage{amsmath}
\usepackage{amssymb}
\usepackage{cancel}
\usepackage{subcaption}
\usepackage{dashrule}
\usepackage{enumitem}
% \usepackage{fontawesome}
\usepackage{fontawesome5}
\usepackage{multicol}
\usepackage{glossaries}
%\numberwithin{equation}{section}
\numberwithin{figure}{section}
\usepackage{float}
\usepackage{twemojis} %...twitter emojis
\usepackage{utfsym}
\usepackage{linearb} %...For \BPwheel in Unit 8
\newcommand{\R}{\mathbb{R}} %...real number symbol
\usepackage{graphicx}
\usepackage{mdframed} %...For FRQ teacher boxes
\graphicspath{ {../Figures/} }
\usepackage{hyperref}
\hypersetup{colorlinks=true,
    linkcolor=blue,
    filecolor=magenta,
    urlcolor=cyan,}
\urlstyle{same}
\newcommand{\hdashline}{{\hdashrule{\textwidth}{0.5pt}{0.8mm}}}
\newcommand{\hgraydashline}{{\color{lightgray} \hdashrule{0.99\textwidth}{1pt}{0.8mm}}}

%...Miscellaneous user-defined symbols
\newcommand{\fnet}{F_{\text{net}}} %...For net force
\newcommand{\bvec}[1]{\vec{\mathbf{#1}}} %...bold vector
\newcommand{\bhat}[1]{\,\hat{\mathbf{#1}}} %...bold hat vector
\newcommand{\que}{\mathord{?}}  %...Question mark symbol in equation env
%...Define thick horizontal rule for examples:
\newcommand{\hhrule}{\hrule\hrule}
\let\oldtexttt\texttt% Store \texttt
\renewcommand{\texttt}[2][black]{\textcolor{#1}{\ttfamily #2}}% 

%...For use in the exam document class
\newif\ifprintmetasolutions


%...Decreases space above and below align and gather enironment
\makeatletter
\g@addto@macro\normalsize{%
  \setlength\abovedisplayskip{-3pt}
  \setlength\belowdisplayskip{6pt} 
}
\makeatother





\usepackage[margin=1in]{geometry}
\usepackage[figurewithin=none]{caption}
\usepackage{exam-randomizechoices}

\CorrectChoiceEmphasis{\color{red}\bfseries}
\renewcommand{\solutiontitle}{\noindent\textbf{\textcolor{red}{Solution:}}\enspace}

\usepackage{OutilsGeomTikz}
\usepackage{utfsym} %...Symbols in Unit 7 Problems
\usepackage{tabu} %...Symbols in Unit 7 Problems

%...For use in Unit 2            %    
\setlength{\columnsep}{2cm}      %
\setlength{\columnseprule}{1pt}  %
\usepackage[none]{hyphenat}      %
%%%%%%%%%%%%%%%%%%%%%%%%%%%%%%%%%

%...For use in Unit 11 on Waves:
\pgfdeclarehorizontalshading{visiblelight}{50bp}{  %
color(0.00000000000000bp)=(red);                   %
color(8.33333333333333bp)=(orange);                %
color(16.66666666666670bp)=(yellow);               %
color(25.00000000000000bp)=(green);                %
color(33.33333333333330bp)=(cyan);                 %
color(41.66666666666670bp)=(blue);                 %
color(50.00000000000000bp)=(violet)                %
}                                                  %

\newcommand{\checkbox}[1]{%
  \ifnum#1=1
    \makebox[0pt][l]{\raisebox{0.15ex}{\hspace{0.1em}\Large$\checkmark$}}%
  \fi
  $\square$%
}
%%%%%%%%%%%%%%%%%%%%%%%%%%%%%%%%%%%%%%%%%%%%%%%%%%%%

%...If using circuitikz package:
% \ctikzset{bipoles/battery1/height=0.5}
% \ctikzset{bipoles/battery1/width=0.25}
% \ctikzset{bipoles/resistor/height=0.15}
% \ctikzset{bipoles/resistor/width=0.4}
\usepackage[none]{hyphenat}
\setlength{\columnseprule}{0pt}

\setrandomizerseed{1}

% \newif\ifversionKlevel

% \versionKlevelfalse

\firstpageheader{{Name:\enspace\makebox[5cm]{\hrulefill}}}{}{Physics Review on Unit 1: Constant Motion}
% \runningheader{Physics}{}{Unit 1 Review}


% \firstpageheader{Physics\\Review on Unit 1: Constant Motion}{}{{Name:\enspace\makebox[5cm]{\hrulefill}}}
% \runningheader{Physics}{}{Unit 1 Review}

% \ifversionKlevel
%     \firstpageheader{Physics K\\Review on Unit 1: Constant Motion}{}{{Name:\enspace\makebox[5cm]{\hrulefill}}}
%     \runningheader{Physics K}{}{Unit 1 Review}
% \fi

\begin{document}
\subsubsection*{Know}

\begin{multicols}{3}
\begin{enumerate}[itemsep=0pt]
    \item position
    \item distance
    \item displacement
    \item magnitude
    \item speed
    \item velocity
    \item motion map
    \item position vs.\,time
    \item velocity vs.\,time
    \item relative velocity
    \item momentum
    \item kinetic energy
\end{enumerate}
\end{multicols}

% \section*{Understand}

% How do you\dots

% \begin{itemize}[itemsep=0pt,topsep=2pt]
%     \item 
% \end{itemize}

\subsubsection*{Do}

\begin{questions}
\question
Below is the position versus time data tables for 3 Dune Buggy Cart runs. The starting point is the reference point for each cart.

\begin{EnvUplevel}
\def\arraystretch{1.3}
\centering
\begin{minipage}{5cm}
    \centering
    \textbf{Cart 1} 
    \medskip
    
    \begin{tabular}{|c|c|}
        \hline
        \textbf{Time} (s) & \textbf{Position} (m) \\ \hline
        0.0 & 0.0 \\ \hline
        1.0 & 5.0 \\ \hline
        2.0 & 10.0 \\ \hline
        3.0 & 15.0 \\ \hline
        4.0 & $x$ \\ \hline
    \end{tabular}
\end{minipage}%
\hspace{2mm}%
\begin{minipage}{5cm}
    \centering
    \textbf{Cart 2} 
    \medskip
    
    \begin{tabular}{|c|c|}
        \hline
        \textbf{Time} (s) & \textbf{Position} (m) \\ \hline
        0.0 & $-8.0$ \\ \hline
        1.0 & $-4.0$ \\ \hline
        2.0 & 0.0 \\ \hline
        3.0 & 4.0 \\ \hline
        4.0 & 8.0 \\ \hline
    \end{tabular}
\end{minipage}%
\hspace{2mm}%
\begin{minipage}{5cm}
    \centering
    \textbf{Cart 3} 
    \medskip
    
    \begin{tabular}{|c|c|}
        \hline
        \textbf{Time} (s) & \textbf{Position} (m) \\ \hline
        0.0 & 0.0 \\ \hline
        1.0 & 2.0 \\ \hline
        2.0 & 4.0 \\ \hline
        3.0 & 6.0 \\ \hline
        4.0 & 8.0 \\ \hline
    \end{tabular}
\end{minipage}%
\end{EnvUplevel}

\medskip

\begin{parts}
    \part Based on the data, determine the value of $x$ as shown for Cart 1. \fillin[20.0][2cm]
    \part Which cart is running in the negative direction for at least, part of its motion? \fillin[none][2cm]
    \part Rank the carts from fastest to slowest. \fillin[Cart 1][2cm]\,, \fillin[Cart 2][2cm]\,, and \fillin[Cart 3][2cm]
    \part Assume all three dune buggies started at the same time and each maintained its constant speed. In which order will they arrive at a position of \SI{25}{m}? \fillin[Cart 1][2cm]\,, \fillin[Cart 2][2cm]\,, and \fillin[Cart 3][2cm]
    \part Which cart started behind the starting line? \fillin[Cart 2][2cm]
\end{parts}

\question
Determine the mathematical relationship (equation) that describes the data in the graph below from 0 to 2 Seconds. (\textit{Hint:} Convert the algebraic equation $y = mx + b$ into a physics equation. In the graph, what is plotted on the $y$-axis?  What is the slope ($m$)? What is plotted on the $x$-axis? What is the $y$-intercept ($b$)?)

\begin{EnvUplevel}
\centering
\begin{minipage}{7cm}
\centering
\begin{tikzpicture}
    \begin{axis}[width=7cm,height=4.5cm,
        axis lines=left,
        ylabel={Position (m)},
        xlabel={Time (s)},
        ymin=0,ymax=10,
        xmin=0,xmax=6,
        ytick={0,2,...,10},
        xtick={0,1,...,6},
        minor x tick num = 1,
        grid=both,
    ]
        \addplot[mark=*,black,thick] coordinates{(0,8)(2,2)(3,6)(5,6)(6,2)};
    \end{axis}
\end{tikzpicture}
\end{minipage}%
\hspace{2mm}
\fbox{
\begin{minipage}{7cm}
    \hfill
    
    \ifprintanswers
    {\color{red}
    \begin{equation*}
        \bar{v} = \frac{\Delta x}{\Delta t} = \frac{\SI{2}{m} - \SI{8}{m}}{\SI{2}{s} - \SI{0}{s}} = -\SI[per-mode=fraction]{3}{\meter\per\second}
    \end{equation*}

    \medskip

    \begin{equation*}
        x(t) = \left(-\SI[per-mode=fraction]{3}{\meter\per\second}\right) + \SI{8}{m}
    \end{equation*}
    
    \vspace{1cm}
    }
    \else
    \vspace{4cm}
    \fi
\end{minipage}}
\end{EnvUplevel}

\question
A man rides in a car traveling north on a highway going \SI{25}{m/s}. A truck traveling south approaches the car going \SI{20}{m/s}. From the perspective of the man in the car, at what speed is the approaching truck traveling?

\begin{solutionorbox}[2.5cm]
Let $v_{cg} = \SI{25}{m/s}$ be the velocity of the car with relative to the ground, and $v_{tg} = -\SI{20}{m/s}$ be the velocity of the truck relative to the ground. Let $v_{tc}$ be the velocity of the truck relative to the car. For inertial frames with relative motion,

\begin{equation*}
    v_{tg} = v_{cg} + v_{tc}    
\end{equation*}

Substituting knowns,

\begin{equation*}
    \SI{-20}{m/s} = \SI{25}{m/s} + v_{tc}
\end{equation*}

Therefore,

\begin{equation*}
    v_{tc} = \SI{-20}{m/s} - \SI{25}{m/s} = \SI{-45}{m/s}
\end{equation*}

From the car's perspective, the truck's speed is the magnitude of the relative velocity: 

\begin{equation*}
    \text{relative speed} = \left|-\SI{45}{m/s}\right| = \boxed{\SI{45}{m/s}}
\end{equation*}
\end{solutionorbox}

\question
For a \SI{45}{kg} boy riding his bicycle at a constant speed of \SI{25}{m/s}, show all possible ways we can represent their motion in the space below.

\begin{solutionorbox}[5cm]

\centering
\begin{minipage}{4.3cm}
\centering
\begin{tikzpicture}
    \begin{axis}[width=4cm,
        height=4cm,
        axis lines=left,
        ylabel={Position (m)},
        xlabel={Time (s)},
        ymin=0,ymax=100,
        xmin=0,xmax=4,
        grid=both,
    ]
        \addplot[very thick,mark=*,samples=5,domain=0:4] {25*x}; 
    \end{axis}
\end{tikzpicture}
\end{minipage}%
\begin{minipage}{10.2cm}
\centering
\begin{tikzpicture}
    \begin{axis}[width=10cm,
        axis lines = left,
        axis y line=none,
        xlabel = {Position (m)},
        ymin=0, ymax=25, 
        xmin=0, xmax=100,
        xtick={0,10,...,100},
        minor x tick num=1,
        clip=false,
        ]
            \pgfplotsinvokeforeach{0,25,50,75,100}{
                \draw[thick,->] (#1,1) -- ++(6mm,0);
                \draw[fill=black] (#1,1) circle (2pt);
            }
    \end{axis}
\end{tikzpicture}
\end{minipage}%

\begin{tikzpicture}
    \begin{axis}[width=4cm,
        height=4cm,
        axis lines=left,
        ylabel={Velocity (m/s)},
        xlabel={Time (s)},
        ymin=0,ymax=50,
        xmin=0,xmax=4,
        grid=both,
        ytick={0,10,...,50}
    ]
        \addplot[very thick,domain=0:4] {25}; 
    \end{axis}
\end{tikzpicture}
\hspace{5mm}
\begin{tikzpicture}
    \begin{axis}[width=4cm,
        height=4cm,
        axis lines=left,
        ylabel={Momentum (\SI{}{kg\cdot m/s})},
        xlabel={Time (s)},
        ymin=0,ymax=2000,
        xmin=0,xmax=4,
        grid=both,
        ytick={0,500,...,2000}
    ]
        \addplot[very thick,domain=0:4] {1125}; 
    \end{axis}
\end{tikzpicture}
\hspace{5mm}
\begin{tikzpicture}
    \begin{axis}[width=4cm,
        height=4cm,
        axis lines=left,
        ylabel={Kinetic Energy (J)},
        xlabel={Time (s)},
        ymin=0,ymax=20000,
        xmin=0,xmax=4,
        grid=both,
        ytick={0,5000,...,20000}
    ]
        \addplot[very thick,domain=0:4] {14062.5}; 
    \end{axis}
\end{tikzpicture}
\end{solutionorbox}

\question
Study the position ($x$) vs.\,time($t$) graphs for cars A, B, C below.

\begin{center}
\begin{tikzpicture}
    \begin{axis}[width=6cm,height=4cm,
        axis lines=left,
        ylabel={Position (m)},
        xlabel={Time (s)},
        ymin=0,ymax=4,
        xmin=0,xmax=5,
        ytick={0,1,...,4},
        xtick={0,1,...,5},
        grid=both,
        clip=false,
    ]
        \draw[very thick] (0,0) -- (4.5,1.5) node[right] {A};
        \draw[very thick] (0,0) -- (4,3.5) node[right] {B};
        \draw[very thick] (0,0) -- (2,3.5) node[right] {C};
    \end{axis}
\end{tikzpicture}
\end{center}

\begin{parts}
\part Which car has the lowest velocity? \fillin[Car A][3cm]
\part What explanation(s) can you offer to justify your answer in part (a) above? 

\begin{solution}
In a position vs.\,time graph, the slope is the velocity. Therefore, the steepest line represents the greatest velocity.
\end{solution}

\ifprintanswers
\else
\fillwithlines{1.4cm}
\fi
\end{parts}

\question
On the diagram below, the dots represent 1-second intervals. The motion maps for a dog and a cat are indicated. The owner's home is located at $x = \SI{0}{m}$.

\begin{EnvUplevel}
\centering
\begin{tikzpicture}
    \begin{axis}[width=14cm,
        axis lines = left,
        axis y line=none,
        xlabel = {Position (m)},
        ymin=0, ymax=25, 
        xmin=0, xmax=10,
        xtick={0,1,...,10},
        minor x tick num=1,
        clip=false,
        ]
            \pgfplotsinvokeforeach{0,1,2,3,4}{
                \draw[thick,->] (#1,1) -- ++(6mm,0);
                \draw[fill=black] (#1,1) circle (2pt);
            }
            \pgfplotsinvokeforeach{10,7.75,5.5}{
                \draw[thick,->] (#1,2) -- ++(-1.2cm,0);
                \draw[fill=black] (#1,2) circle (2pt);
            }
            \node[left=2pt] at (0,1) {dog};
            \node[right=2pt] at (10,2) {cat};
    \end{axis}
\end{tikzpicture}
\end{EnvUplevel}

\begin{parts}
\part Which has a greater magnitude of velocity, the dog or the cat? How do you know?

\begin{solution}
    The cat, because it undergoes a greater displacement each second.
\end{solution}

\ifprintanswers
\else
\fillwithlines{14mm}
\fi

\part Which is moving towards the owner’s home? \fillin[cat][3cm]
\part What is the displacement of the dog after 2 seconds of motion? \fillin[\SI{2}{m}][3cm]
\part What is the displacement of the cat after 2 seconds of motion? \fillin[\SI{-4.5}{m}][3cm]
\part What is the dog’s velocity from $t = \SI{0}{s}$ to $t = \SI{2}{s}$ in the above motion map? \fillin[\SI{1}{m/s}][3cm]
\part What is the cat’s velocity from $t = \SI{0}{s}$ to $t = \SI{2}{s}$ in the above motion map? \fillin[\SI{-2.25}{m/s}][3cm]
\end{parts}

\question
Consider the graph below.

\begin{center}
\begin{tikzpicture}
    \begin{axis}[width=7cm,height=4.5cm,
        axis lines=left,
        ylabel={Velocity (m/s)},
        xlabel={Time (s)},
        ymin=0,ymax=50,
        xmin=0,xmax=22,
        ytick={0,10,...,50},
        xtick={0,2,...,22},
        grid=both
    ]
        \addplot[domain=0:20,samples=5,mark=*] {2*x};
        \ifprintanswers
            \begin{pgfonlayer}{background}
                \draw[fill=red!15] (0,0) -- (20,40) -- (20,0) -- cycle;
            \end{pgfonlayer}
        \fi
    \end{axis}
\end{tikzpicture}
\end{center}

\begin{parts}
\part What is the instantaneous velocity at 10 seconds? 

\begin{solution}
    $v = \SI{20}{m/s}$
\end{solution}

\ifprintanswers
\else
\fillwithlines{7mm}
\fi

\part What is the displacement of the object depicted above from $t = \SI{0}{s}$ to $t = \SI{20}{s}$?

\begin{solutionorbox}[2.5cm]
    In a velocity vs.\,time graph, displacement is the area bounded by the curve (see figure above). Displacement is the area of a triangle, given by

    \begin{equation*}
        \Delta x = \frac{1}{2} (\SI{40}{m/s})(\SI{20}{s}) = \boxed{\SI{400}{m}}
    \end{equation*}
\end{solutionorbox}
\end{parts}


\question
The toy trucks and toy cars with given masses move at the speeds shown by the arrows. 

\begin{EnvUplevel}
\centering
\begin{tikzpicture}
    \draw[very thick,->] (0.3,0) -- ++(1.5,0) node[above,pos=0.7] {\SI{2}{m/s}};
    \node at (0,0) {\reflectbox{\usymH{1F69B}{16pt}}} node[above=8pt] {\SI{5}{kg}} node[below=10pt] {A};
    \begin{scope}[shift={(4,0)}]
        \draw[very thick,->] (0.4,0) -- ++(1.5,0) node[above,pos=0.7] {\SI{12.5}{m/s}};
        \node at (0,0) {\reflectbox{\usymH{1F699}{16pt}}} node[above=8pt] {\SI{0.8}{kg}} node[below=10pt] {B};
    \end{scope}
    \begin{scope}[shift={(8,0)}]
        \draw[very thick,->] (0.3,0) -- ++(1.5,0) node[above,pos=0.7] {\SI{5}{m/s}};
        \node at (0,0) {\reflectbox{\usymH{1F69A}{16pt}}} node[above=8pt] {\SI{2}{kg}} node[below=10pt] {C};
    \end{scope}
    \begin{scope}[shift={(12,0)}]
        \draw[very thick,->] (0.4,0) -- ++(1.5,0) node[above,pos=0.7] {\SI{20}{m/s}};
        \node at (0,0) {\reflectbox{\usymH{1F697}{16pt}}} node[above=8pt] {\SI{0.5}{kg}} node[below=10pt] {D};
    \end{scope}
\end{tikzpicture}
\end{EnvUplevel}

\begin{parts}
\part Which toy vehicle has the greatest kinetic energy? (Show your work)

\begin{solutionorbox}[3.5cm]
The kinetic energy of each vehicle is

\begin{align*}
    K_A &= \frac{1}{2}mv^2 = \frac{1}{2}(\SI{5}{kg})(\SI{2}{m/s})^2 = \SI{10}{J} \\[1em]
    K_B &= \frac{1}{2}mv^2 = \frac{1}{2}(\SI{0.8}{kg})(\SI{12.5}{m/s})^2 = \SI{62.5}{J} \\[1em]
    K_C &= \frac{1}{2}mv^2 = \frac{1}{2}(\SI{2}{kg})(\SI{5}{m/s})^2 = \SI{25}{J} \\[1em]
    K_D &= \frac{1}{2}mv^2 = \frac{1}{2}(\SI{0.5}{kg})(\SI{20}{m/s})^2 = \boxed{\SI{100}{J}}
\end{align*}
\end{solutionorbox}
\part Which toy vehicle has the greatest momentum? (Show your work)

\begin{solutionorbox}[3.5cm]
The momentum of each vehicle is

\begin{align*}
    p_A &= ma = (\SI{5}{kg})(\SI{2}{m/s}) = \SI{10}{kg\cdot m/s} \\[1em]
    p_B &= ma = (\SI{0.8}{kg})(\SI{12.5}{m/s}) = \SI{10}{kg\cdot m/s} \\[1em]
    p_C &= ma = (\SI{2}{kg})(\SI{5}{m/s}) = \SI{10}{kg\cdot m/s} \\[1em]
    p_D &= ma = (\SI{0.5}{kg})(\SI{20}{m/s}) = \SI{10}{kg\cdot m/s}
\end{align*}

So they all have the same momentum.

\end{solutionorbox}
\part Which toy vehicle has the greatest inertia?

\begin{solution}
    Truck A has the greatest inertia because it has the greatest mass.
\end{solution}
\ifprintanswers
\else
\fillwithlines{7mm}
\fi
\end{parts}

\question
A skateboarder mounted their skateboard at a position \SI{5}{m} from the reference (zero) point and then moves with a constant velocity of \SI{1.5}{m/s} in the positive direction. Find their

\begin{parts}
\part displacement ($\Delta x$) after 8 seconds 

\begin{solutionorbox}[1.5cm]

\begin{equation*}
    \Delta x = \bar{v}t = (\SI{1.5}{m/s})(\SI{8}{s}) = \boxed{\SI{12}{m}}
\end{equation*}
\end{solutionorbox}

\part position ($x$) after 8 seconds

\begin{solutionorbox}[1.5cm]
By $\Delta x = x_f - x_0$,

\begin{equation*}
    x_f = \Delta x + x_0 = \SI{12}{m} + \SI{5}{m} = \boxed{\SI{17}{m}}
\end{equation*}
\end{solutionorbox}

\end{parts}

\question
How would the momentum of an object change if its velocity was doubled?

\begin{solution}
    Because momentum is proportional to velocity, the momentum would double.
\end{solution}

\ifprintanswers
\else
\fillwithlines{7mm}
\fi

\question
How would the kinetic energy of an object change if its velocity was doubled?

\begin{solution}
    Because kinetic energy is proportional to velocity squared, the kinetic energy would quadruple.
\end{solution}

\ifprintanswers
\else
\fillwithlines{14mm}
\fi

\question
An object moves \SI{7}{cm} east from the origin, then \SI{3}{cm} south, and finally \SI{6}{cm} west.

\begin{parts}
\part Draw a vector diagram to illustrate the motion.

\begin{solutionorbox}[2cm]

\begin{center}
\begin{tikzpicture}[x=5mm,y=5mm]
    \draw[thick,->] (0,0) -- (7,0) node[above,pos=0.5] {7\,cm east};
    \draw[thick,->] (7,0) -- (7,-3) node[pos=0.5,right] {3\,cm south};
    \draw[thick,->] (7,-3) -- (1,-3) node[below,pos=0.5] {6\,cm west};
    \draw[densely dashed,->,thick,gray] (0,0) -- (1,-3) node[left,pos=0.5] {3.16\,cm};
\end{tikzpicture}
\end{center}
\end{solutionorbox}

\part Use your diagram to determine the object’s total distance traveled. \fillin[\SI{16}{cm}][3cm]

\part Determine the object's displacement. \fillin[3.16\,cm][3cm]
\end{parts}

\question
Use the graph to answer the questions below.

\begin{center}
\begin{tikzpicture}
    \begin{axis}[width=6.5cm,height=5.5cm,
        axis lines=left,
        ylabel={Position (m)},
        xlabel={Time (s)},
        ymin=0,ymax=50,
        xmin=0,xmax=12,
        ytick={0,10,...,50},
        xtick={0,2,...,12},
        minor y tick num=1,
        minor x tick num=1,
        grid=both,
        clip=false,
    ]
        \draw[very thick] (0,0)  -- (2,20) -- (4,20) node[below,pos=0.15] {B} node[below] {C} -- (5,40) -- (8,40) node[below,pos=0.15] {D} node[below,pos=0.85] {E} -- (10,20) -- (11,20) node[below,pos=0] {F} node[below] {G} -- (12,30) node[left] {H};
        \node at (0.7,2.8) {A};
    \end{axis}
\end{tikzpicture}

\begin{parts}
    \part What is the displacement from A -- F (0 -- 10 seconds)? \fillin[\SI{20}{m}][4cm]
    \part What is the velocity from A -- B (0 -- 2 seconds)? \fillin[\SI{10}{m/s}][4cm]
    \part During which time interval is the object stationary? \fillin[B--C, D--E, F--G][4cm]
    \part During which time interval is the velocity greatest? \fillin[C--D][4cm]
    \part When is the object moving in the negative direction? \fillin[E--F][4cm]
\end{parts}
\end{center}
\vspace{-1em}


\end{questions}
\end{document}

        % \draw[->] (10,1) --++ (-1,0) node[above,pos=0.5] {$v$};
        % \draw[->] (8,1) --++ (-1,0) node[above,pos=0.5] {$v$};
        % \draw[->] (6,1) --++ (-1,0) node[above,pos=0.5] {$v$};
        % \fill (10,1) circle (3pt) node[right=3pt] {start};
        % \fill (8,1) circle (3pt);
        % \fill (6,1) circle (3pt);
        % \fill (4,1) circle (3pt);