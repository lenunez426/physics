\makenoidxglossaries

%...UNIT 1: CONSTANT MOTION

\newglossaryentry{scalar}{
    name=scalar,
    description={a quantity that has magnitude (and possibly sign) but no direction}
}

\newglossaryentry{magnitude}{
    name={magnitude},
    description={size or amount}
}

\newglossaryentry{vector}{
    name={vector},
    description={a quantity that has both magnitude and direction}
}

\newglossaryentry{tail}{
    name={tail},
    description={the starting point of a vector; the point opposite to the head or tip of the arrow}
}

\newglossaryentry{head}{
    name={head},
    description={the end point of a vector; the location of the vector's arrow; also referred to as the tip}
}

\newglossaryentry{head-to-tail method}{
    name={head-to-tail method},
    description={a method of adding vectors in which the tail of each vector is placed at the head of the previous vector}
}

\newglossaryentry{position}{
    name={position},
    description={the location of an object at any particular time}
}

\newglossaryentry{reference frame}{
    name={reference frame},
    description={a coordinate system from which the positions of objects are described}
}


\newglossaryentry{displacement}{
    name={displacement},
    description={the change in position of an object against a fixed axis}
}

\newglossaryentry{distance}{
    name={distance},
    description={the length of the path actually traveled between an initial and a final position}
}

\newglossaryentry{position vs. time graph}{
    name={position vs. time graph},
    description={a graph in which position is plotted on the vertical axis and time is plotted on the horizontal axis}
}

\newglossaryentry{speed}{
    name={speed},
    description={rate at which an object changes its location}
}

\newglossaryentry{average speed}{
    name={average speed},
    description={distance traveled divided by the time during which the motion occurs}
}

\newglossaryentry{velocity}{
    name={velocity},
    description={the speed and direction of an object}
}

\newglossaryentry{average velocity}{
    name={average velocity},
    description={displacement divided by the time during which the displacement occurs}
}

\newglossaryentry{velocity vs. time graph}{
    name={velocity vs. time graph},
    description={a graph in which velocity is plotted on the vertical axis and time is plotted on the horizontal axis}
}

\newglossaryentry{mass}{
    name=mass,
    description={the quantity of matter in a substance; the SI unit of mass is the kilogram}
}

\newglossaryentry{inertia}{
    name=inertia,
    description={the tendency of an object at rest to remain at rest, or for a moving object to remain in motion in a straight line and at a constant speed}
}

\newglossaryentry{Newton's first law of motion}{
    name={Newton's first law of motion},
    description={a body at rest remains at rest or, if in motion, remains in motion at a constant speed in a straight line, unless acted on by a net external force; also known as the law of inertia}
}

\newglossaryentry{momentum}{
    name={momentum},
    description={the product of a system's mass and velocity}
}

\newglossaryentry{momentum vs. time graph}{
    name={momentum vs. time graph},
    description={a graph in which momentum is plotted on the vertical axis and time is plotted on the horizontal axis}
}

\newglossaryentry{kinetic energy}{
    name={kinetic energy},
    description={energy of motion}
}

\newglossaryentry{joule}{
    name=joule,
    description={the metric unit for work and energy; equal to 1 newton meter ($\text{N}\cdot\text{m}$)}
}

\newglossaryentry{relative speed}{
    name={relative speed},
    description={how fast or slow an object appears to be moving to another object}
}

\newglossaryentry{relative velocity}{
    name={relative velocity},
    description={the rate at which an object changes position relative to another object}
}

%...UNIT 2: FORCE INTERACTIONS

\newglossaryentry{force}{
    name=force,
    description={a push or pull on an object with a specific magnitude and direction; can be represented by vectors; can be expressed as a multiple of a standard force; the SI unit of force is the Newton (N)}
}

\newglossaryentry{external force}{
    name=external force,
    description={a force acting on an object or system that originates outside of the object or system}
}

\newglossaryentry{free body diagram}{
    name=free body diagram,
    description={a diagram showing all external forces acting on a body}
}

\newglossaryentry{frictional force}{
    name=frictional force,
    description={an external force that acts opposite to the direction of motion or, for when there is no relative motion, in the direction needed to prevent slipping}
}

\newglossaryentry{applied force}{
    name={applied force},
    description={a contact force intentionally implied by a person on an object}
}


\newglossaryentry{gravitational force}{
    name=gravitational force, %...MY DEFINITION
    description={the downward force on an object due to the attraction by the Earth or other massive body}
}

\newglossaryentry{net force}{
    name=net force,
    description={the sum of all forces acting on an object or system}
}

\newglossaryentry{normal force}{
    name=normal force,
    description={that component of the contact force between two objects, which acts perpendicularly to and away from their plane of contact}
}

\newglossaryentry{tension}{
    name=tension,
    description={a pulling force that acts along a connecting medium, especially a stretched flexible connector, such as a rope or cable; when a rope supports the weight of an object, the force exerted on the object by the rope is called tension}
}

\newglossaryentry{spring force}{
    name=spring force, %...From district slides
    description={a force applied from a spring when it is either compressed or stretched}
}

%...UNIT 3: ACCELERATION

\newglossaryentry{acceleration}{
    name={acceleration},
    description={a change in velocity over time}
}

\newglossaryentry{average acceleration}{
    name={average acceleration},
    description={change in velocity divided by the time interval over which it changed}
}

%...UNIT 4:


\newglossaryentry{impulse}{
    name={impulse},
    description={average net external force multiplied by the time the force acts; equal to the change in momentum}
}

\newglossaryentry{impulse-momentum theorem}{
    name={impulse-momentum theorem},
    description={the impulse equals change in momentum}
}

\newglossaryentry{work}{
    name={work},
    description={force multiplied by distance}
}

%...UNIT 5: FORCE ANALYSIS

\newglossaryentry{Newton's universal law of gravitation}{
    name={Newton's universal law of gravitation},
    description={states that gravitational force between two objects is directly proportional to the product of their masses and inversely proportional to the square of the distance between them}
}

\newglossaryentry{gravitational constant}{
    name={gravitational constant},
    description={the proportionality constant in Newton's law of universal gravitation}
}

\newglossaryentry{weight}{
    name={weight},
    description={the force of gravity, $W$, acting on an object of mass $m$; defined mathematically as $W = mg$, where $g$ is the acceleration due to gravity}
}

\newglossaryentry{contact force}{
    name={contact force},
    description={a type of force that occurs when objects are physically in contact with each other}
}

%...UNIT 6: ONE-DIMENSIONAL MOTION

\newglossaryentry{free fall}{
    name=free fall,
    description={a situation in which the only force acting on an object is the force of gravity}
}

\newglossaryentry{kinematic equations}{
    name={kinematic equations},
    description={the 
    %five 
    equations that describe constant acceleration motion in terms of time, displacement, velocity, and acceleration}
}

%...UNIT 7: MOTION IN TWO DIMENSIONS

\newglossaryentry{projectile}{
    name={projectile},
    description={an object that travels through the air and experiences only acceleration due to gravity}
}

\newglossaryentry{projectile motion}{
    name={projectile motion},
    description={the motion of an object that is subject only to the acceleration of gravity}
}

\newglossaryentry{trajectory}{
    name={trajectory},
    description={the path of a projectile through the air}
}

\newglossaryentry{apex}{
    name={apex},
    description={the location on the trajectory at which the projectile reaches maximum height}
}

\newglossaryentry{hang time}{
    name={hang time},
    description={the amount of time that a projectile is in the air during projectile motion}
}

\newglossaryentry{horizontally launched projectile}{
    name={horizontally launched projectile},
    description={a projectile whose initial velocity is entirely in the horizontal direction}
}

\newglossaryentry{impact speed}{
    name={impact speed},
    description={the speed at which a projectile strikes the ground after being launched}
}

%...UNIT 8: CONSERVATION IN MECHANICAL SYSTEMS

\newglossaryentry{system}{
    name={system},
    description={one or more objects of interest for which only the forces acting on them from the outside are considered, but not the forces acting between them or inside them}
}

\newglossaryentry{energy}{
    name={energy},
    description={the ability to do work}
}

\newglossaryentry{potential energy}{
    name={potential energy},
    description={stored energy}
}

\newglossaryentry{gravitational potential energy}{
    name={gravitational potential energy},
    description={energy acquired by doing work against gravity}
}

\newglossaryentry{law of conservation of energy}{
    name={law of conservation of energy},
    description={states that energy is neither created nor destroyed}
}

\newglossaryentry{mechanical energy}{
    name={mechanical energy},
    description={kinetic plus potential energy}
}

\newglossaryentry{elastic collision}{
    name={elastic collision},
    description={a collision in which objects separate after impact and kinetic energy is conserved}
}

\newglossaryentry{inelastic collision}{
    name={inelastic collision},
    description={a collision in which kinetic energy is not conserved}
}

\newglossaryentry{isolated system}{
    name={isolated system},
    description={system in which the net external force is zero}
}

\newglossaryentry{law of conservation of momentum}{
    name={law of conservation of momentum},
    description={when the net external force is zero, the total momentum of the system is conserved or constant}
}

\newglossaryentry{perfectly inelastic collision}{
    name={perfectly inelastic collision},
    description={collision in which objects stick together after impact and kinetic energy is not conserved}
}

\newglossaryentry{recoil}{
    name={recoil},
    description={backward movement of an object caused by the transfer of momentum from another object in a collision}
}

%...UNIT 9: CONSERVATION OF CHARGE

\newglossaryentry{electric charge}{
    name={electric charge},
    description={a physical property of an object that causes it to be attracted toward or repelled from another charged object; each charged object generates and is influenced by a force called an electromagnetic force}
}

\newglossaryentry{elementary charge}{
    name={elementary charge},
    description={the smallest observed unit of charge that can be isolated in nature; also, the magnitude of charge on 1 proton or 1 electron}
}

\newglossaryentry{electron}{
    name={electron},
    description={subatomic particle that carries one indivisible unit of negative electric charge}
}

\newglossaryentry{proton}{
    name={proton},
    description={subatomic particle that carries the same magnitude charge as the electron, but its charge is positive}
}

\newglossaryentry{electric field}{
    name={electric field},
    description={defines the force per unit charge at all locations in space around a charge distribution}
}

\newglossaryentry{law of conservation of charge}{
    name={law of conservation of charge},
    description={states that total charge is constant in any process}
}

\newglossaryentry{polarization}{
    name={polarization},
    description={separation of charge induced by nearby excess charge}
}

\newglossaryentry{Coulomb's law}{
    name={Coulomb's law},
    description={describes the electrostatic force between charged objects, which is proportional to the charge on each object and inversely proportional to the square of the distance between the objects}
}

\newglossaryentry{electric circuit}{
    name={electric circuit},
    description={physical network of paths through which electric current can flow}
}

\newglossaryentry{simple circuit}{
    name={simple circuit},
    description={a circuit with a single voltage source and a single resistor}
}

\newglossaryentry{electric current}{
    name={electric current},
    description={electric charge that is moving}
}


\newglossaryentry{Ohm's law}{
    name={Ohm's law},
    description={electric current is proportional to the voltage applied across a circuit or other path}
}

\newglossaryentry{resistance}{
    name={resistance},
    description={how much a circuit element opposes the passage of electric current; it appears as the constant of proportionality in Ohm’s law}
}

\newglossaryentry{resistor}{
    name={resistor},
    description={circuit element that provides a known resistance}
}

% \newglossaryentry{potential difference (or voltage)}{
%     name={potential difference (or voltage)},
%     description={change in potential energy of a charge moved from one point to another, divided by the charge; units of potential difference are joules per coulomb, known as volt}
% }

\newglossaryentry{voltage}{
    name={voltage},
    description={the electrical potential energy per unit charge; electric pressure created by a power source, such as a battery}
}

\newglossaryentry{electric power}{
    name={electric power},
    description={rate at which electric energy is transferred in a circuit}
}

\newglossaryentry{equivalent resistor}{
    name={equivalent resistor},
    description={resistance of a single resistor that is the same as the combined resistance of a group of resistors}
}

\newglossaryentry{in series}{
    name={in series},
    description={when elements in a circuit are connected one after the other in the same branch of the circuit}
}

\newglossaryentry{in parallel}{
    name={in parallel},
    description={when a group of resistors are connected side by side, with the top ends of the resistors connected together by a wire and the bottom ends connected together by a different wire}
}

\newglossaryentry{induction}{
    name={induction},
    description={creating an unbalanced charge distribution in an object by moving a charged object toward it (but without touching)}
}

%...UNIT 10: ELECTROMAGNETIC INDUCTION

\newglossaryentry{magnetic dipole}{
    name={magnetic dipole},
    description={term that describes magnets because they always have two poles: north and south}
}

\newglossaryentry{magnetic field}{
    name={magnetic field},
    description={directional lines around a magnetic material that indicates the direction and magnitude of the magnetic force}
}

\newglossaryentry{magnetic pole}{
    name={magnetic pole},
    description={part of a magnet that exerts the strongest force on other magnets or magnetic material}
}

\newglossaryentry{electromagnetic induction}{
    name={electromagnetic induction},
    description={rate at which energy is drawn from a source per unit current flowing through a circuit}
}


\newglossaryentry{Faraday's law}{
    name={Faraday's law},
    description={the means of calculating the emf in a coil due to changing magnetic flux}
}

\newglossaryentry{electromagnet}{
    name={electromagnet},
    description={device that uses electric current to make a magnetic field}
}

\newglossaryentry{transformer}{
    name={transformer},
    description={device that transforms voltages from one value to another}
}

\newglossaryentry{electric motor}{
    name={electric motor},
    description={device that transforms electrical energy into mechanical energy}
}

\newglossaryentry{generator}{
    name={generator},
    description={device that transforms mechanical energy into electrical energy}
}

%...UNIT 11: SIMPLE HARMONIC MOTION & WAVES

\newglossaryentry{wave}{
    name={wave},
    description={a disturbance that moves from its source and carries energy}
}

\newglossaryentry{wave velocity}{
    name={wave velocity},
    description={speed at which the disturbance moves; also called the propagation velocity or propagation speed}
}

\newglossaryentry{wavelength}{
    name={wavelength},
    description={distance between adjacent identical parts of a wave}
}

\newglossaryentry{wave cycle}{
    name={wave cycle},
    description={any portion of a wave encompassed by 1 wavelength}
}

\newglossaryentry{transverse wave}{
    name={transverse wave},
    description={a wave in which the disturbance is perpendicular to the direction of propagation}
}

\newglossaryentry{medium}{
    name={medium},
    description={the solid, liquid, or gas material through which a wave propagates}
}

\newglossaryentry{mechanical wave}{
    name={mechanical wave},
    description={wave that requires a medium through which it can travel}
}

\newglossaryentry{longitudinal wave}{
    name={longitudinal wave},
    description={wave in which the disturbance is parallel to the direction of propagation}
}

\newglossaryentry{constructive interference}{
    name={constructive interference},
    description={when two waves arrive at the same point exactly in phase; that is, the crests of the two waves are precisely aligned, as are the troughs}
}

\newglossaryentry{destructive interference}{
    name={destructive interference},
    description={when two identical waves arrive at the same point exactly out of phase that is precisely aligned crest to trough}
}

\newglossaryentry{oscillate}{
    name={oscillate},
    description={to move back and forth regularly between two points}
}

\newglossaryentry{amplitude}{
    name={amplitude},
    description={the maximum displacement from the equilibrium position of an object oscillating around the equilibrium position}
}

\newglossaryentry{frequency}{
    name={frequency},
    description={number of wave cycles per unit of time}
}

\newglossaryentry{simple harmonic motion}{
    name={simple harmonic motion},
    description={the oscillatory motion in a system where the net force can be described by Hooke’s law}
}

\newglossaryentry{simple harmonic oscillator}{
    name={simple harmonic oscillator},
    description={a device that oscillates in SHM,  such as a mass that is attached to a spring, where the restoring force is proportional to the displacement and acts in the direction opposite to the displacement}
}

\newglossaryentry{period}{
    name={period},
    description={the time it takes to complete one oscillation}
}

\newglossaryentry{electromagnetic wave}{
    name={electromagnetic wave},
    description={a radiant energy wave that consists of oscillating electric and magnetic fields}
}

\newglossaryentry{electromagnetic radiation}{
    name={electromagnetic radiation},
    description={radiant energy that consists of oscillating electric and magnetic fields}
}









%... Overview and Student Learning Expectations (OSLE)

\newglossaryentry{OSLE 6.1.a}{
    name={OSLE 6.1.a},
    description={compare the gravitational field strength on Earth to the acceleration due to gravity on Earth}
}

\newglossaryentry{OSLE 6.1.b}{
    name={OSLE 6.1.b},
    description={explain using universal gravitation and $F_\mathrm{net}=ma$ why all objects near Earth's surface fall at the same rate when in free fall}
}

\newglossaryentry{OSLE 6.1.c}{
    name={OSLE 6.1.c},
    description={explain the relationship between the mass, initial position, and initial velocity of an object in free fall on its final velocity and/or time in free fall}
}

\newglossaryentry{OSLE 6.1.d}{
    name={OSLE 6.1.d},
    description={describe the displacement, velocity, momentum, kinetic energy, and acceleration of an object in free fall that was dropped, thrown upward, or thrown downward using Multiple Representations}
}

\newglossaryentry{OSLE 6.1.e}{
    name={OSLE 6.1.e},
    description={relate the gravitational force, impulse, and work done on the object by the Earth to the object's change in velocity (acceleration), momentum, and kinetic energy}
}

        
\newglossaryentry{OSLE 6.2.a}{
    name={OSLE 6.2.a},
    description={describe what is known about an object's motion in a constant acceleration word problem using Multiple Representations}
}

\newglossaryentry{OSLE 6.2.b}{
    name={OSLE 6.2.b},
    description={solve for various unknown quantities utilizing kinematic equations when data is given in Multiple Representations for objects moving horizontally with constant acceleration}
}

\newglossaryentry{OSLE 6.3.c}{
    name={OSLE 6.3.c},
    description={solve for various unknown quantities utilizing kinematic equations when data is given in Multiple Representations for objects moving vertically with constant acceleration (free fall)}
}

\newglossaryentry{OSLE 6.4.d}{
    name={OSLE 6.4.d},
    description={solve multi-step problems that connect kinematic equations, the Law of Acceleration, Work-Energy Theorem, and/or the Impulse-Momentum Theorem}
}

\newglossaryentry{OSLE 7.1.a}{
    name={OSLE 7.1.a},
    description={compare the trajectory, hang time, max height, range, and final velocity of various projectiles that have different initial velocities, launch heights, launch angles, and masses, only varying one parameter at a time}
}

\newglossaryentry{OSLE 7.1.b}{
    name={OSLE 7.1.b},
    description={identify if and explain how the  initial velocity, launch height, launch angle, and mass of a projectile influence its motion---hang time, height, range, final velocity}
}

\newglossaryentry{OSLE 7.2.a}{
    name={OSLE 7.2.a},
    description={describe the vertical and horizontal motion of a projectile with a launch angle of zero using Multiple Representations}
}

\newglossaryentry{OSLE 7.2.b}{
    name={OSLE 7.2.b},
    description={illustrate the resultant motion of the projectile at any point in its trajectory as well as the relationship between the horizontal and vertical components using vector addition}
}

\newglossaryentry{OSLE 7.2.c}{
    name={OSLE 7.2.c},
    description={analyze and solve word problems about the motion of  horizontally launched projectiles using kinematic equations, vector addition, and  Multiple Representations}
}

\newglossaryentry{OSLE 7.3.a}{
    name={OSLE 7.3.a},
    description={describe the motion of an object moving with uniform circular motion in terms of centripetal force, centripetal acceleration, momentum, kinetic energy, and tangential velocity using Multiple Representations}
}

\newglossaryentry{OSLE 7.3.b}{
    name={OSLE 7.3.b},
    description={determine the centripetal force, mass, centripetal acceleration, tangential velocity, or radius of an object in circular motion}
}

\newglossaryentry{OSLE 7.4.a}{
    name={OSLE 7.4.a},
    description={predict the effects of changing the radius or mass of objects in orbiting systems using concepts of uniform circular motion and Newton’s law of universal gravitation}
}


\newglossaryentry{OSLE 8.1.a}{
    name={OSLE 8.1.a},
    description={identify multiple choices for a system given a scenario}
}

\newglossaryentry{OSLE 8.1.b}{
    name={OSLE 8.1.b},
    description={recognize that energy can be stored in the arrangement of particles or objects in a system as potential energy}
}

\newglossaryentry{OSLE 8.1.c}{
    name={OSLE 8.1.c},
    description={identify and calculate (i) gravitational potential energy and (ii) elastic potential energy when a system includes energy stored in the arrangement of its particles or objects}
}

\newglossaryentry{OSLE 8.1.d}{
    name={OSLE 8.1.d},
    description={compare the potential energy of a scenario for various choices of system}
}

\newglossaryentry{OSLE 8.1.e}{
    name={OSLE 8.1.e},
    description={identify, represent using multiple representations, and calculate the total mechanical energy present in a physical system}
}

\newglossaryentry{OSLE 8.1.f}{
    name={OSLE 8.1.f},
    description={predict the effects of changing the mass, velocity, height, gravitational field strength, spring constant, compression or stretching distance on the amount of $E_k$, $E_\mathrm{GP}$, and $E_\mathrm{SP}$}
}

\newglossaryentry{OSLE 8.1.g}{
    name={OSLE 8.1.g},
    description={calculate the total mechanical energy of a system}
}

\newglossaryentry{OSLE 8.2.a.i}{
    name={OSLE 8.2.a.i},
    description={identify, represent using multiple representations, and calculate the amount of energy (1) transformed from one storage mode to another within a system (including kinetic energy, potential energy, and thermal energy), (2) transferred from one object in the system to another in the system, and (3) entering/leaving a system due to work, heat, light, or sound}
}

\newglossaryentry{OSLE 8.2.b.i}{
    name={OSLE 8.2.b.i},
    description={explain the meaning of the Law of Conservation of Energy}
}

\newglossaryentry{OSLE 8.2.b.ii}{
    name={OSLE 8.2.b.ii},
    description={develop an energy formula for systems using energy bar charts and the Law of Conservation of Energy}
}

\newglossaryentry{OSLE 8.2.b.iii}{
    name={OSLE 8.2.b.iii},
    description=solve for various unknown quantities using the concept of the conservation of energy{}
}

\newglossaryentry{OSLE 8.2.c.i}{
    name={OSLE 8.2.c.i},
    description={know the definition of work as change in energy of a system}
}

\newglossaryentry{OSLE 8.2.c.ii}{
    name={OSLE 8.2.c.ii},
    description={know that power is work done divided by time}
}

\newglossaryentry{OSLE 8.3.a}{
    name={OSLE 8.3.a},
    description={calculate and compare the momentum, changes in momentum, force applied to and impulse on each object involved in a collision or explosion scenario}
}

\newglossaryentry{OSLE 8.3.b}{
    name={OSLE 8.3.b},
    description={represent using multiple representations, compare, and calculate the total momentum of a system before and after a collision or explosion scenario}
}

\newglossaryentry{OSLE 8.3.c}{
    name={OSLE 8.3.c},
    description={explain the meaning of the Law of Conservation of Momentum}
}

\newglossaryentry{OSLE 8.3.d}{
    name={OSLE 8.3.d},
    description={solve for unknown quantities using the concept of the conservation of momentum}
}


\newglossaryentry{OSLE 9.1.a}{
    name={OSLE 9.1.a},
    description={identify the particles that contribute positive, negative, or no charge in an atom}
}

\newglossaryentry{OSLE 9.1.b}{
    name={OSLE 9.1.b},
    description={recognize that neutral objects have even numbers of positive and negative charges}
}

\newglossaryentry{OSLE 9.1.c}{
    name={OSLE 9.1.c},
    description={determine the charge of an object given the number of protons and electrons}
}

\newglossaryentry{OSLE 9.1.d}{
    name={OSLE 9.1.d},
    description={predict if two objects will attract, repel, or have no interaction based on their charges}
}

\newglossaryentry{OSLE 9.1.e}{
    name={OSLE 9.1.e},
    description={draw the electric field surrounding single charges and pairs of charges}
}

\newglossaryentry{OSLE 9.2.a}{
    name={OSLE 9.2.a},
    description={recognize that charge is conserved: it cannot be created or destroyed, only transferred}
}

\newglossaryentry{OSLE 9.2.b}{
    name={OSLE 9.2.b},
    description={realize that only electrons are transferred during charging}
}

\newglossaryentry{OSLE 9.2.c}{
    name={OSLE 9.2.c},
    description={compare and contrast charging by induction and conduction}
}

\newglossaryentry{OSLE 9.2.d}{
    name={OSLE 9.2.d},
    description={explain how polarization temporarily charges a neutral object}
}

\newglossaryentry{OSLE 9.2.e}{
    name={OSLE 9.2.e},
    description={describe how an electroscope determines if objects are charged}
}

\newglossaryentry{OSLE 9.2.f}{
    name={OSLE 9.2.f},
    description={determine whether an object is negatively charged, positively charged, or neutral when given the charge of one object and a description or diagram representing how the charged object interacts with an object of unknown charge}
}

\newglossaryentry{OSLE 9.2.g}{
    name={OSLE 9.2.g},
    description={draw and describe the resulting distribution of charge for various scenarios of induction, conduction, and polarization}
}


\newglossaryentry{OSLE 9.3.a}{
    name={OSLE 9.3.a},
    description={draw the free body diagram for 2 charged objects showing the direction and relative magnitude of the electrical force acting on each object at various distances from each other}
}

\newglossaryentry{OSLE 9.3.b}{
    name={OSLE 9.3.b},
    description={describe how the electric force depends on the charges and the distance between them}
}

\newglossaryentry{OSLE 9.3.c}{
    name={OSLE 9.3.c},
    description={compare and contrast the electric force to the gravitational force}
}

\newglossaryentry{OSLE 9.3.d}{
    name={OSLE 9.3.d},
    description={predict how changing the charge or distance affects the electric force}
}


\newglossaryentry{OSLE 9.4.a}{
    name={OSLE 9.4.a},
    description={identify the necessary components for a simple circuit and discover different ways to light a bulb}
}

\newglossaryentry{OSLE 9.4.b}{
    name={OSLE 9.4.b},
    description={trace the conducting path through a simple circuit}
}

\newglossaryentry{OSLE 9.4.c}{
    name={OSLE 9.4.c},
    description={explain the concepts of current, resistance, voltage}
}

\newglossaryentry{OSLE 9.4.d}{
    name={OSLE 9.4.d},
    description={measure the current, resistance and voltage in a circuit using a multimeter, ammeter, current probe, etc}
}

\newglossaryentry{OSLE 9.4.e}{
    name={OSLE 9.4.e},
    description={calculate the voltage drop across, current through, or resistance of a circuit component using Ohm’s Law}
}

\newglossaryentry{OSLE 9.4.f}{
    name={OSLE 9.4.f},
    description={determine the change in current as the voltage or resistance is changed}
}

\newglossaryentry{OSLE 9.4.g}{
    name={OSLE 9.4.g},
    description={interpret electrical power as the rate at which electrical energy is being dissipated in the circuit}
}

\newglossaryentry{OSLE 9.4.h}{
    name={OSLE 9.4.h},
    description={relate the power rating/wattage of a light bulb to its brightness}
}


\newglossaryentry{OSLE 9.5.a}{
    name={OSLE 9.5.a},
    description={measure the current, resistance and voltage at various locations in a series circuit using a multimeter, ammeter, current probe, etc}
}

\newglossaryentry{OSLE 9.5.b}{
    name={OSLE 9.5.b},
    description={describe qualitatively and quantitatively the current flow throughout a series circuit}
}

\newglossaryentry{OSLE 9.5.c}{
    name={OSLE 9.5.c},
    description={calculate the equivalent resistance of multiple resistors in series}
}

\newglossaryentry{OSLE 9.5.d}{
    name={OSLE 9.5.d},
    description={calculate the equivalent voltage of batteries in series}
}

\newglossaryentry{OSLE 9.5.e}{
    name={OSLE 9.5.e},
    description={recognize that the sum of the voltage drops across resistors in series equals the total voltage of the power supply}
}

\newglossaryentry{OSLE 9.5.f}{
    name={OSLE 9.5.f},
    description={describe the energy transformations (transfers) occurring in a series circuit}
}

\newglossaryentry{OSLE 9.5.g}{
    name={OSLE 9.5.g},
    description={determine (i) current through, voltage drop across, and power of each component, and (i) total current of circuit, when given a series circuit diagram}
}


\newglossaryentry{OSLE 9.6.a}{
    name={OSLE 9.6.a},
    description={measure the current, resistance and voltage at various locations in a parallel circuit using a multimeter, ammeter, current probe, etc}
}

\newglossaryentry{OSLE 9.6.b}{
    name={OSLE 9.6.b},
    description={recognize that the current going into a junction is equal to the current coming out of it}
}

\newglossaryentry{OSLE 9.6.c}{
    name={OSLE 9.6.c},
    description={describe qualitatively and quantitatively the current flow throughout a parallel circuit}
}

\newglossaryentry{OSLE 9.6.d}{
    name={OSLE 9.6.d},
    description={recognize that the voltage drops across each resistor are equal to the voltage of the power supply}
}

\newglossaryentry{OSLE 9.6.e}{
    name={OSLE 9.6.e},
    description={describe advantages and disadvantages of parallel circuits compared to series circuits}
}

\newglossaryentry{OSLE 9.6.f}{
    name={OSLE 9.6.f},
    description={calculate the equivalent resistance of multiple resistors in parallel}
}

\newglossaryentry{OSLE 9.6.g}{
    name={OSLE 9.6.g},
    description={calculate the equivalent voltage of batteries in parallel}
}

\newglossaryentry{OSLE 9.6.h}{
    name={OSLE 9.6.h},
    description={describe the energy transformations (transfers) occurring in a parallel circuit}
}

\newglossaryentry{OSLE 9.6.i}{
    name={OSLE 9.6.i},
    description={determine the (i) current through, voltage drop across, and power of each component, and (ii) total current of circuit, when given a series circuit diagram}
}


\newglossaryentry{OSLE 9.7.a}{
    name={OSLE 9.7.a},
    description={determine whether elements of a combination circuit have the same current or voltage}
}

\newglossaryentry{OSLE 9.7.b}{
    name={OSLE 9.7.b},
    description={predict which bulbs will light if switches are open or closed}
}






