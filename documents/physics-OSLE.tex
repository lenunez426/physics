\documentclass[dvipsnames]{article}
\usepackage{marvosym}

%...TikZ & PGF
\usepackage{pgfplots}
\pgfplotsset{compat=1.11}
\tikzset{>=latex}
\usetikzlibrary{calc,math}
\usepackage{tikzsymbols}
\usepgfplotslibrary{fillbetween}
\usetikzlibrary{decorations.markings} 
\usetikzlibrary{arrows.meta} %...APP2 for arrows as objects and images
\usetikzlibrary{backgrounds} %...For shading portions of graphs
\usetikzlibrary{patterns} %...Unit 5 Problems
\usetikzlibrary{shapes.geometric} %...For drawing cylinders in Unit 2
\usepackage{makecell} %...use \thead{} to enable line skip in table headers
\tikzset{
    mark position/.style args={#1(#2)}{
        postaction={
            decorate,
            decoration={
                markings,
                mark=at position #1 with \coordinate (#2);
            }
        }
    }
} %...See https://tex.stackexchange.com/questions/43960/define-node-at-relative-coordinates-of-draw-plot

\tikzset{
    declare function = {trajectoryequation10(\x,\vi,\thetai)= tan(\thetai)*\x - 10*\x^2/(2*(\vi*cos(\thetai))^2);},
    declare function = {trajectoryequation(\x,\vi,\thetai)= tan(\thetai)*\x - 9.8*\x^2/(2*(\vi*cos(\thetai))^2);},
    declare function = {patheq(\x,\yi,\vi,\thetai)= \yi + tan(\thetai)*\x - 9.8*\x^2/(2*(\vi*cos(\thetai))^2);},
    declare function = {patheqten(\x,\yi,\vi,\thetai)= \yi + tan(\thetai)*\x - 10*\x^2/(2*(\vi*cos(\thetai))^2);} %like patheq but with gravity = 10
}

%...siunitx
\usepackage{siunitx}
\DeclareSIUnit{\nothing}{\relax}
\def\mymu{\SI{}{\micro\nothing} }
\DeclareSIUnit\mmHg{mmHg}
\DeclareSIUnit{\mile}{mi}
%...NOTE: "The product symbol between the number and unit is set using the quantity-product option."

%...Other
\usepackage{amsthm}
\usepackage{amsmath}
\usepackage{amssymb}
\usepackage{cancel}
\usepackage{subcaption}
\usepackage{dashrule}
\usepackage{enumitem}
\usepackage{fontawesome}
\usepackage{multicol}
\usepackage{glossaries}
%\numberwithin{equation}{section}
\numberwithin{figure}{section}
\usepackage{float}
\usepackage{twemojis} %...twitter emojis
\usepackage{utfsym}
\usepackage{linearb} %...For \BPwheel in Unit 8
\newcommand{\R}{\mathbb{R}} %...real number symbol
\usepackage{graphicx}
\graphicspath{ {../Figures/} }
\usepackage{hyperref}
\hypersetup{colorlinks=true,
    linkcolor=blue,
    filecolor=magenta,
    urlcolor=cyan,}
\urlstyle{same}
\newcommand{\hdashline}{{\hdashrule{\textwidth}{0.5pt}{0.8mm}}}
\newcommand{\hgraydashline}{{\color{lightgray} \hdashrule{0.99\textwidth}{1pt}{0.8mm}}}

%...Miscellaneous user-defined symbols
\newcommand{\fnet}{F_{\text{net}}} %...For net force
\newcommand{\bvec}[1]{\vec{\mathbf{#1}}} %...bold vector
\newcommand{\bhat}[1]{\,\hat{\mathbf{#1}}} %...bold hat vector
\newcommand{\que}{\mathord{?}}  %...Question mark symbol in equation env
%...Define thick horizontal rule for examples:
\newcommand{\hhrule}{\hrule\hrule}
\let\oldtexttt\texttt% Store \texttt
\renewcommand{\texttt}[2][black]{\textcolor{#1}{\ttfamily #2}}% 

%...For use in the exam document class
\newif\ifprintmetasolutions


%...Decreases space above and below align and gather enironment
\makeatletter
\g@addto@macro\normalsize{%
  \setlength\abovedisplayskip{-3pt}
  \setlength\belowdisplayskip{6pt} 
}
\makeatother





\usepackage[margin=1in]{geometry}
\usepackage{OutilsGeomTikz}

%...Theorem for examples
\theoremstyle{definition}
\newtheorem{example}{Example}
\newtheorem{exercise}{Exercise}

%...Define colors for use in Unit 11: SHM and Waves
\pgfdeclarehorizontalshading{visiblelight}{50bp}{
color(0.00000000000000bp)=(red);
color(8.33333333333333bp)=(orange);
color(16.66666666666670bp)=(yellow);
color(25.00000000000000bp)=(green);
color(33.33333333333330bp)=(cyan);
color(41.66666666666670bp)=(blue);
color(50.00000000000000bp)=(violet)
}

\setlength\parindent{0pt}
\setlength{\parskip}{6pt}

\renewcommand{\thesubsubsection}{\thesubsection\alph{subsubsection}}
\setlength{\parskip}{6pt}
\makenoidxglossaries

%...UNIT 1: CONSTANT MOTION

\newglossaryentry{scalar}{
    name=scalar,
    description={a quantity that has magnitude (and possibly sign) but no direction}
}

\newglossaryentry{magnitude}{
    name={magnitude},
    description={size or amount}
}

\newglossaryentry{vector}{
    name={vector},
    description={a quantity that has both magnitude and direction}
}

\newglossaryentry{tail}{
    name={tail},
    description={the starting point of a vector; the point opposite to the head or tip of the arrow}
}

\newglossaryentry{head}{
    name={head},
    description={the end point of a vector; the location of the vector's arrow; also referred to as the tip}
}

\newglossaryentry{head-to-tail method}{
    name={head-to-tail method},
    description={a method of adding vectors in which the tail of each vector is placed at the head of the previous vector}
}

\newglossaryentry{position}{
    name={position},
    description={the location of an object at any particular time}
}

\newglossaryentry{reference frame}{
    name={reference frame},
    description={a coordinate system from which the positions of objects are described}
}


\newglossaryentry{displacement}{
    name={displacement},
    description={the change in position of an object against a fixed axis}
}

\newglossaryentry{distance}{
    name={distance},
    description={the length of the path actually traveled between an initial and a final position}
}

\newglossaryentry{position vs. time graph}{
    name={position vs. time graph},
    description={a graph in which position is plotted on the vertical axis and time is plotted on the horizontal axis}
}

\newglossaryentry{speed}{
    name={speed},
    description={rate at which an object changes its location}
}

\newglossaryentry{average speed}{
    name={average speed},
    description={distance traveled divided by the time during which the motion occurs}
}

\newglossaryentry{velocity}{
    name={velocity},
    description={the speed and direction of an object}
}

\newglossaryentry{average velocity}{
    name={average velocity},
    description={displacement divided by the time during which the displacement occurs}
}

\newglossaryentry{velocity vs. time graph}{
    name={velocity vs. time graph},
    description={a graph in which velocity is plotted on the vertical axis and time is plotted on the horizontal axis}
}

\newglossaryentry{mass}{
    name=mass,
    description={the quantity of matter in a substance; the SI unit of mass is the kilogram}
}

\newglossaryentry{inertia}{
    name=inertia,
    description={the tendency of an object at rest to remain at rest, or for a moving object to remain in motion in a straight line and at a constant speed}
}

\newglossaryentry{Newton's first law of motion}{
    name={Newton's first law of motion},
    description={a body at rest remains at rest or, if in motion, remains in motion at a constant speed in a straight line, unless acted on by a net external force; also known as the law of inertia}
}

\newglossaryentry{momentum}{
    name={momentum},
    description={the product of a system's mass and velocity}
}

\newglossaryentry{momentum vs. time graph}{
    name={momentum vs. time graph},
    description={a graph in which momentum is plotted on the vertical axis and time is plotted on the horizontal axis}
}

\newglossaryentry{kinetic energy}{
    name={kinetic energy},
    description={energy of motion}
}

\newglossaryentry{joule}{
    name=joule,
    description={the metric unit for work and energy; equal to 1 newton meter ($\text{N}\cdot\text{m}$)}
}

\newglossaryentry{relative speed}{
    name={relative speed},
    description={how fast or slow an object appears to be moving to another object}
}

\newglossaryentry{relative velocity}{
    name={relative velocity},
    description={the rate at which an object changes position relative to another object}
}

%...UNIT 2: FORCE INTERACTIONS

\newglossaryentry{force}{
    name=force,
    description={a push or pull on an object with a specific magnitude and direction; can be represented by vectors; can be expressed as a multiple of a standard force; the SI unit of force is the Newton (N)}
}

\newglossaryentry{external force}{
    name=external force,
    description={a force acting on an object or system that originates outside of the object or system}
}

\newglossaryentry{free body diagram}{
    name=free body diagram,
    description={a diagram showing all external forces acting on a body}
}

\newglossaryentry{frictional force}{
    name=frictional force,
    description={an external force that acts opposite to the direction of motion or, for when there is no relative motion, in the direction needed to prevent slipping}
}

\newglossaryentry{applied force}{
    name={applied force},
    description={a contact force intentionally implied by a person on an object}
}


\newglossaryentry{gravitational force}{
    name=gravitational force, %...MY DEFINITION
    description={the downward force on an object due to the attraction by the Earth or other massive body}
}

\newglossaryentry{net force}{
    name=net force,
    description={the sum of all forces acting on an object or system}
}

\newglossaryentry{normal force}{
    name=normal force,
    description={that component of the contact force between two objects, which acts perpendicularly to and away from their plane of contact}
}

\newglossaryentry{tension}{
    name=tension,
    description={a pulling force that acts along a connecting medium, especially a stretched flexible connector, such as a rope or cable; when a rope supports the weight of an object, the force exerted on the object by the rope is called tension}
}

\newglossaryentry{spring force}{
    name=spring force, %...From district slides
    description={a force applied from a spring when it is either compressed or stretched}
}

%...UNIT 3: ACCELERATION

\newglossaryentry{acceleration}{
    name={acceleration},
    description={a change in velocity over time}
}

\newglossaryentry{average acceleration}{
    name={average acceleration},
    description={change in velocity divided by the time interval over which it changed}
}

%...UNIT 4:


\newglossaryentry{impulse}{
    name={impulse},
    description={average net external force multiplied by the time the force acts; equal to the change in momentum}
}

\newglossaryentry{impulse-momentum theorem}{
    name={impulse-momentum theorem},
    description={the impulse equals change in momentum}
}

\newglossaryentry{work}{
    name={work},
    description={force multiplied by distance}
}

%...UNIT 5: FORCE ANALYSIS

\newglossaryentry{Newton's universal law of gravitation}{
    name={Newton's universal law of gravitation},
    description={states that gravitational force between two objects is directly proportional to the product of their masses and inversely proportional to the square of the distance between them}
}

\newglossaryentry{gravitational constant}{
    name={gravitational constant},
    description={the proportionality constant in Newton's law of universal gravitation}
}

\newglossaryentry{weight}{
    name={weight},
    description={the force of gravity, $W$, acting on an object of mass $m$; defined mathematically as $W = mg$, where $g$ is the acceleration due to gravity}
}

\newglossaryentry{contact force}{
    name={contact force},
    description={a type of force that occurs when objects are physically in contact with each other}
}

%...UNIT 6: ONE-DIMENSIONAL MOTION

\newglossaryentry{free fall}{
    name=free fall,
    description={a situation in which the only force acting on an object is the force of gravity}
}

\newglossaryentry{kinematic equations}{
    name={kinematic equations},
    description={the 
    %five 
    equations that describe constant acceleration motion in terms of time, displacement, velocity, and acceleration}
}

%...UNIT 7: MOTION IN TWO DIMENSIONS

\newglossaryentry{projectile}{
    name={projectile},
    description={an object that travels through the air and experiences only acceleration due to gravity}
}

\newglossaryentry{projectile motion}{
    name={projectile motion},
    description={the motion of an object that is subject only to the acceleration of gravity}
}

\newglossaryentry{trajectory}{
    name={trajectory},
    description={the path of a projectile through the air}
}

\newglossaryentry{apex}{
    name={apex},
    description={the location on the trajectory at which the projectile reaches maximum height}
}

\newglossaryentry{hang time}{
    name={hang time},
    description={the amount of time that a projectile is in the air during projectile motion}
}

\newglossaryentry{horizontally launched projectile}{
    name={horizontally launched projectile},
    description={a projectile whose initial velocity is entirely in the horizontal direction}
}

\newglossaryentry{impact speed}{
    name={impact speed},
    description={the speed at which a projectile strikes the ground after being launched}
}

%...UNIT 8: CONSERVATION IN MECHANICAL SYSTEMS

\newglossaryentry{system}{
    name={system},
    description={one or more objects of interest for which only the forces acting on them from the outside are considered, but not the forces acting between them or inside them}
}

\newglossaryentry{energy}{
    name={energy},
    description={the ability to do work}
}

\newglossaryentry{potential energy}{
    name={potential energy},
    description={stored energy}
}

\newglossaryentry{gravitational potential energy}{
    name={gravitational potential energy},
    description={energy acquired by doing work against gravity}
}

\newglossaryentry{law of conservation of energy}{
    name={law of conservation of energy},
    description={states that energy is neither created nor destroyed}
}

\newglossaryentry{mechanical energy}{
    name={mechanical energy},
    description={kinetic plus potential energy}
}

\newglossaryentry{elastic collision}{
    name={elastic collision},
    description={a collision in which objects separate after impact and kinetic energy is conserved}
}

\newglossaryentry{inelastic collision}{
    name={inelastic collision},
    description={a collision in which kinetic energy is not conserved}
}

\newglossaryentry{isolated system}{
    name={isolated system},
    description={system in which the net external force is zero}
}

\newglossaryentry{law of conservation of momentum}{
    name={law of conservation of momentum},
    description={when the net external force is zero, the total momentum of the system is conserved or constant}
}

\newglossaryentry{perfectly inelastic collision}{
    name={perfectly inelastic collision},
    description={collision in which objects stick together after impact and kinetic energy is not conserved}
}

\newglossaryentry{recoil}{
    name={recoil},
    description={backward movement of an object caused by the transfer of momentum from another object in a collision}
}

%...UNIT 9: CONSERVATION OF CHARGE

\newglossaryentry{electric charge}{
    name={electric charge},
    description={a physical property of an object that causes it to be attracted toward or repelled from another charged object; each charged object generates and is influenced by a force called an electromagnetic force}
}

\newglossaryentry{elementary charge}{
    name={elementary charge},
    description={the smallest observed unit of charge that can be isolated in nature; also, the magnitude of charge on 1 proton or 1 electron}
}

\newglossaryentry{electron}{
    name={electron},
    description={subatomic particle that carries one indivisible unit of negative electric charge}
}

\newglossaryentry{proton}{
    name={proton},
    description={subatomic particle that carries the same magnitude charge as the electron, but its charge is positive}
}

\newglossaryentry{electric field}{
    name={electric field},
    description={defines the force per unit charge at all locations in space around a charge distribution}
}

\newglossaryentry{law of conservation of charge}{
    name={law of conservation of charge},
    description={states that total charge is constant in any process}
}

\newglossaryentry{polarization}{
    name={polarization},
    description={separation of charge induced by nearby excess charge}
}

\newglossaryentry{Coulomb's law}{
    name={Coulomb's law},
    description={describes the electrostatic force between charged objects, which is proportional to the charge on each object and inversely proportional to the square of the distance between the objects}
}

\newglossaryentry{electric circuit}{
    name={electric circuit},
    description={physical network of paths through which electric current can flow}
}

\newglossaryentry{simple circuit}{
    name={simple circuit},
    description={a circuit with a single voltage source and a single resistor}
}

\newglossaryentry{electric current}{
    name={electric current},
    description={electric charge that is moving}
}


\newglossaryentry{Ohm's law}{
    name={Ohm's law},
    description={electric current is proportional to the voltage applied across a circuit or other path}
}

\newglossaryentry{resistance}{
    name={resistance},
    description={how much a circuit element opposes the passage of electric current; it appears as the constant of proportionality in Ohm’s law}
}

\newglossaryentry{resistor}{
    name={resistor},
    description={circuit element that provides a known resistance}
}

% \newglossaryentry{potential difference (or voltage)}{
%     name={potential difference (or voltage)},
%     description={change in potential energy of a charge moved from one point to another, divided by the charge; units of potential difference are joules per coulomb, known as volt}
% }

\newglossaryentry{voltage}{
    name={voltage},
    description={the electrical potential energy per unit charge; electric pressure created by a power source, such as a battery}
}

\newglossaryentry{electric power}{
    name={electric power},
    description={rate at which electric energy is transferred in a circuit}
}

\newglossaryentry{equivalent resistor}{
    name={equivalent resistor},
    description={resistance of a single resistor that is the same as the combined resistance of a group of resistors}
}

\newglossaryentry{in series}{
    name={in series},
    description={when elements in a circuit are connected one after the other in the same branch of the circuit}
}

\newglossaryentry{in parallel}{
    name={in parallel},
    description={when a group of resistors are connected side by side, with the top ends of the resistors connected together by a wire and the bottom ends connected together by a different wire}
}

\newglossaryentry{induction}{
    name={induction},
    description={creating an unbalanced charge distribution in an object by moving a charged object toward it (but without touching)}
}

%...UNIT 10: ELECTROMAGNETIC INDUCTION

\newglossaryentry{magnetic dipole}{
    name={magnetic dipole},
    description={term that describes magnets because they always have two poles: north and south}
}

\newglossaryentry{magnetic field}{
    name={magnetic field},
    description={directional lines around a magnetic material that indicates the direction and magnitude of the magnetic force}
}

\newglossaryentry{magnetic pole}{
    name={magnetic pole},
    description={part of a magnet that exerts the strongest force on other magnets or magnetic material}
}

\newglossaryentry{electromagnetic induction}{
    name={electromagnetic induction},
    description={rate at which energy is drawn from a source per unit current flowing through a circuit}
}


\newglossaryentry{Faraday's law}{
    name={Faraday's law},
    description={the means of calculating the emf in a coil due to changing magnetic flux}
}

\newglossaryentry{electromagnet}{
    name={electromagnet},
    description={device that uses electric current to make a magnetic field}
}

\newglossaryentry{transformer}{
    name={transformer},
    description={device that transforms voltages from one value to another}
}

\newglossaryentry{electric motor}{
    name={electric motor},
    description={device that transforms electrical energy into mechanical energy}
}

\newglossaryentry{generator}{
    name={generator},
    description={device that transforms mechanical energy into electrical energy}
}

%...UNIT 11: SIMPLE HARMONIC MOTION & WAVES

\newglossaryentry{wave}{
    name={wave},
    description={a disturbance that moves from its source and carries energy}
}

\newglossaryentry{wave velocity}{
    name={wave velocity},
    description={speed at which the disturbance moves; also called the propagation velocity or propagation speed}
}

\newglossaryentry{wavelength}{
    name={wavelength},
    description={distance between adjacent identical parts of a wave}
}

\newglossaryentry{wave cycle}{
    name={wave cycle},
    description={any portion of a wave encompassed by 1 wavelength}
}

\newglossaryentry{transverse wave}{
    name={transverse wave},
    description={a wave in which the disturbance is perpendicular to the direction of propagation}
}

\newglossaryentry{medium}{
    name={medium},
    description={the solid, liquid, or gas material through which a wave propagates}
}

\newglossaryentry{mechanical wave}{
    name={mechanical wave},
    description={wave that requires a medium through which it can travel}
}

\newglossaryentry{longitudinal wave}{
    name={longitudinal wave},
    description={wave in which the disturbance is parallel to the direction of propagation}
}

\newglossaryentry{constructive interference}{
    name={constructive interference},
    description={when two waves arrive at the same point exactly in phase; that is, the crests of the two waves are precisely aligned, as are the troughs}
}

\newglossaryentry{destructive interference}{
    name={destructive interference},
    description={when two identical waves arrive at the same point exactly out of phase that is precisely aligned crest to trough}
}

\newglossaryentry{oscillate}{
    name={oscillate},
    description={to move back and forth regularly between two points}
}

\newglossaryentry{amplitude}{
    name={amplitude},
    description={the maximum displacement from the equilibrium position of an object oscillating around the equilibrium position}
}

\newglossaryentry{frequency}{
    name={frequency},
    description={number of wave cycles per unit of time}
}

\newglossaryentry{simple harmonic motion}{
    name={simple harmonic motion},
    description={the oscillatory motion in a system where the net force can be described by Hooke’s law}
}

\newglossaryentry{simple harmonic oscillator}{
    name={simple harmonic oscillator},
    description={a device that oscillates in SHM,  such as a mass that is attached to a spring, where the restoring force is proportional to the displacement and acts in the direction opposite to the displacement}
}

\newglossaryentry{period}{
    name={period},
    description={the time it takes to complete one oscillation}
}

\newglossaryentry{electromagnetic wave}{
    name={electromagnetic wave},
    description={a radiant energy wave that consists of oscillating electric and magnetic fields}
}

\newglossaryentry{electromagnetic radiation}{
    name={electromagnetic radiation},
    description={radiant energy that consists of oscillating electric and magnetic fields}
}









%... Overview and Student Learning Expectations (OSLE)

\newglossaryentry{OSLE 6.1.a}{
    name={OSLE 6.1.a},
    description={compare the gravitational field strength on Earth to the acceleration due to gravity on Earth}
}

\newglossaryentry{OSLE 6.1.b}{
    name={OSLE 6.1.b},
    description={explain using universal gravitation and $F_\mathrm{net}=ma$ why all objects near Earth's surface fall at the same rate when in free fall}
}

\newglossaryentry{OSLE 6.1.c}{
    name={OSLE 6.1.c},
    description={explain the relationship between the mass, initial position, and initial velocity of an object in free fall on its final velocity and/or time in free fall}
}

\newglossaryentry{OSLE 6.1.d}{
    name={OSLE 6.1.d},
    description={describe the displacement, velocity, momentum, kinetic energy, and acceleration of an object in free fall that was dropped, thrown upward, or thrown downward using Multiple Representations}
}

\newglossaryentry{OSLE 6.1.e}{
    name={OSLE 6.1.e},
    description={relate the gravitational force, impulse, and work done on the object by the Earth to the object's change in velocity (acceleration), momentum, and kinetic energy}
}

        
\newglossaryentry{OSLE 6.2.a}{
    name={OSLE 6.2.a},
    description={describe what is known about an object's motion in a constant acceleration word problem using Multiple Representations}
}

\newglossaryentry{OSLE 6.2.b}{
    name={OSLE 6.2.b},
    description={solve for various unknown quantities utilizing kinematic equations when data is given in Multiple Representations for objects moving horizontally with constant acceleration}
}

\newglossaryentry{OSLE 6.3.c}{
    name={OSLE 6.3.c},
    description={solve for various unknown quantities utilizing kinematic equations when data is given in Multiple Representations for objects moving vertically with constant acceleration (free fall)}
}

\newglossaryentry{OSLE 6.4.d}{
    name={OSLE 6.4.d},
    description={solve multi-step problems that connect kinematic equations, the Law of Acceleration, Work-Energy Theorem, and/or the Impulse-Momentum Theorem}
}

\newglossaryentry{OSLE 7.1.a}{
    name={OSLE 7.1.a},
    description={compare the trajectory, hang time, max height, range, and final velocity of various projectiles that have different initial velocities, launch heights, launch angles, and masses, only varying one parameter at a time}
}

\newglossaryentry{OSLE 7.1.b}{
    name={OSLE 7.1.b},
    description={identify if and explain how the  initial velocity, launch height, launch angle, and mass of a projectile influence its motion---hang time, height, range, final velocity}
}

\newglossaryentry{OSLE 7.2.a}{
    name={OSLE 7.2.a},
    description={describe the vertical and horizontal motion of a projectile with a launch angle of zero using Multiple Representations}
}

\newglossaryentry{OSLE 7.2.b}{
    name={OSLE 7.2.b},
    description={illustrate the resultant motion of the projectile at any point in its trajectory as well as the relationship between the horizontal and vertical components using vector addition}
}

\newglossaryentry{OSLE 7.2.c}{
    name={OSLE 7.2.c},
    description={analyze and solve word problems about the motion of  horizontally launched projectiles using kinematic equations, vector addition, and  Multiple Representations}
}

\newglossaryentry{OSLE 7.3.a}{
    name={OSLE 7.3.a},
    description={describe the motion of an object moving with uniform circular motion in terms of centripetal force, centripetal acceleration, momentum, kinetic energy, and tangential velocity using Multiple Representations}
}

\newglossaryentry{OSLE 7.3.b}{
    name={OSLE 7.3.b},
    description={determine the centripetal force, mass, centripetal acceleration, tangential velocity, or radius of an object in circular motion}
}

\newglossaryentry{OSLE 7.4.a}{
    name={OSLE 7.4.a},
    description={predict the effects of changing the radius or mass of objects in orbiting systems using concepts of uniform circular motion and Newton’s law of universal gravitation}
}


\newglossaryentry{OSLE 8.1.a}{
    name={OSLE 8.1.a},
    description={identify multiple choices for a system given a scenario}
}

\newglossaryentry{OSLE 8.1.b}{
    name={OSLE 8.1.b},
    description={recognize that energy can be stored in the arrangement of particles or objects in a system as potential energy}
}

\newglossaryentry{OSLE 8.1.c}{
    name={OSLE 8.1.c},
    description={identify and calculate (i) gravitational potential energy and (ii) elastic potential energy when a system includes energy stored in the arrangement of its particles or objects}
}

\newglossaryentry{OSLE 8.1.d}{
    name={OSLE 8.1.d},
    description={compare the potential energy of a scenario for various choices of system}
}

\newglossaryentry{OSLE 8.1.e}{
    name={OSLE 8.1.e},
    description={identify, represent using multiple representations, and calculate the total mechanical energy present in a physical system}
}

\newglossaryentry{OSLE 8.1.f}{
    name={OSLE 8.1.f},
    description={predict the effects of changing the mass, velocity, height, gravitational field strength, spring constant, compression or stretching distance on the amount of $E_k$, $E_\mathrm{GP}$, and $E_\mathrm{SP}$}
}

\newglossaryentry{OSLE 8.1.g}{
    name={OSLE 8.1.g},
    description={calculate the total mechanical energy of a system}
}

\newglossaryentry{OSLE 8.2.a.i}{
    name={OSLE 8.2.a.i},
    description={identify, represent using multiple representations, and calculate the amount of energy (1) transformed from one storage mode to another within a system (including kinetic energy, potential energy, and thermal energy), (2) transferred from one object in the system to another in the system, and (3) entering/leaving a system due to work, heat, light, or sound}
}

\newglossaryentry{OSLE 8.2.b.i}{
    name={OSLE 8.2.b.i},
    description={explain the meaning of the Law of Conservation of Energy}
}

\newglossaryentry{OSLE 8.2.b.ii}{
    name={OSLE 8.2.b.ii},
    description={develop an energy formula for systems using energy bar charts and the Law of Conservation of Energy}
}

\newglossaryentry{OSLE 8.2.b.iii}{
    name={OSLE 8.2.b.iii},
    description=solve for various unknown quantities using the concept of the conservation of energy{}
}

\newglossaryentry{OSLE 8.2.c.i}{
    name={OSLE 8.2.c.i},
    description={know the definition of work as change in energy of a system}
}

\newglossaryentry{OSLE 8.2.c.ii}{
    name={OSLE 8.2.c.ii},
    description={know that power is work done divided by time}
}

\newglossaryentry{OSLE 8.3.a}{
    name={OSLE 8.3.a},
    description={calculate and compare the momentum, changes in momentum, force applied to and impulse on each object involved in a collision or explosion scenario}
}

\newglossaryentry{OSLE 8.3.b}{
    name={OSLE 8.3.b},
    description={represent using multiple representations, compare, and calculate the total momentum of a system before and after a collision or explosion scenario}
}

\newglossaryentry{OSLE 8.3.c}{
    name={OSLE 8.3.c},
    description={explain the meaning of the Law of Conservation of Momentum}
}

\newglossaryentry{OSLE 8.3.d}{
    name={OSLE 8.3.d},
    description={solve for unknown quantities using the concept of the conservation of momentum}
}


\newglossaryentry{OSLE 9.1.a}{
    name={OSLE 9.1.a},
    description={identify the particles that contribute positive, negative, or no charge in an atom}
}

\newglossaryentry{OSLE 9.1.b}{
    name={OSLE 9.1.b},
    description={recognize that neutral objects have even numbers of positive and negative charges}
}

\newglossaryentry{OSLE 9.1.c}{
    name={OSLE 9.1.c},
    description={determine the charge of an object given the number of protons and electrons}
}

\newglossaryentry{OSLE 9.1.d}{
    name={OSLE 9.1.d},
    description={predict if two objects will attract, repel, or have no interaction based on their charges}
}

\newglossaryentry{OSLE 9.1.e}{
    name={OSLE 9.1.e},
    description={draw the electric field surrounding single charges and pairs of charges}
}

\newglossaryentry{OSLE 9.2.a}{
    name={OSLE 9.2.a},
    description={recognize that charge is conserved: it cannot be created or destroyed, only transferred}
}

\newglossaryentry{OSLE 9.2.b}{
    name={OSLE 9.2.b},
    description={realize that only electrons are transferred during charging}
}

\newglossaryentry{OSLE 9.2.c}{
    name={OSLE 9.2.c},
    description={compare and contrast charging by induction and conduction}
}

\newglossaryentry{OSLE 9.2.d}{
    name={OSLE 9.2.d},
    description={explain how polarization temporarily charges a neutral object}
}

\newglossaryentry{OSLE 9.2.e}{
    name={OSLE 9.2.e},
    description={describe how an electroscope determines if objects are charged}
}

\newglossaryentry{OSLE 9.2.f}{
    name={OSLE 9.2.f},
    description={determine whether an object is negatively charged, positively charged, or neutral when given the charge of one object and a description or diagram representing how the charged object interacts with an object of unknown charge}
}

\newglossaryentry{OSLE 9.2.g}{
    name={OSLE 9.2.g},
    description={draw and describe the resulting distribution of charge for various scenarios of induction, conduction, and polarization}
}


\newglossaryentry{OSLE 9.3.a}{
    name={OSLE 9.3.a},
    description={draw the free body diagram for 2 charged objects showing the direction and relative magnitude of the electrical force acting on each object at various distances from each other}
}

\newglossaryentry{OSLE 9.3.b}{
    name={OSLE 9.3.b},
    description={describe how the electric force depends on the charges and the distance between them}
}

\newglossaryentry{OSLE 9.3.c}{
    name={OSLE 9.3.c},
    description={compare and contrast the electric force to the gravitational force}
}

\newglossaryentry{OSLE 9.3.d}{
    name={OSLE 9.3.d},
    description={predict how changing the charge or distance affects the electric force}
}


\newglossaryentry{OSLE 9.4.a}{
    name={OSLE 9.4.a},
    description={identify the necessary components for a simple circuit and discover different ways to light a bulb}
}

\newglossaryentry{OSLE 9.4.b}{
    name={OSLE 9.4.b},
    description={trace the conducting path through a simple circuit}
}

\newglossaryentry{OSLE 9.4.c}{
    name={OSLE 9.4.c},
    description={explain the concepts of current, resistance, voltage}
}

\newglossaryentry{OSLE 9.4.d}{
    name={OSLE 9.4.d},
    description={measure the current, resistance and voltage in a circuit using a multimeter, ammeter, current probe, etc}
}

\newglossaryentry{OSLE 9.4.e}{
    name={OSLE 9.4.e},
    description={calculate the voltage drop across, current through, or resistance of a circuit component using Ohm’s Law}
}

\newglossaryentry{OSLE 9.4.f}{
    name={OSLE 9.4.f},
    description={determine the change in current as the voltage or resistance is changed}
}

\newglossaryentry{OSLE 9.4.g}{
    name={OSLE 9.4.g},
    description={interpret electrical power as the rate at which electrical energy is being dissipated in the circuit}
}

\newglossaryentry{OSLE 9.4.h}{
    name={OSLE 9.4.h},
    description={relate the power rating/wattage of a light bulb to its brightness}
}


\newglossaryentry{OSLE 9.5.a}{
    name={OSLE 9.5.a},
    description={measure the current, resistance and voltage at various locations in a series circuit using a multimeter, ammeter, current probe, etc}
}

\newglossaryentry{OSLE 9.5.b}{
    name={OSLE 9.5.b},
    description={describe qualitatively and quantitatively the current flow throughout a series circuit}
}

\newglossaryentry{OSLE 9.5.c}{
    name={OSLE 9.5.c},
    description={calculate the equivalent resistance of multiple resistors in series}
}

\newglossaryentry{OSLE 9.5.d}{
    name={OSLE 9.5.d},
    description={calculate the equivalent voltage of batteries in series}
}

\newglossaryentry{OSLE 9.5.e}{
    name={OSLE 9.5.e},
    description={recognize that the sum of the voltage drops across resistors in series equals the total voltage of the power supply}
}

\newglossaryentry{OSLE 9.5.f}{
    name={OSLE 9.5.f},
    description={describe the energy transformations (transfers) occurring in a series circuit}
}

\newglossaryentry{OSLE 9.5.g}{
    name={OSLE 9.5.g},
    description={determine (i) current through, voltage drop across, and power of each component, and (i) total current of circuit, when given a series circuit diagram}
}


\newglossaryentry{OSLE 9.6.a}{
    name={OSLE 9.6.a},
    description={measure the current, resistance and voltage at various locations in a parallel circuit using a multimeter, ammeter, current probe, etc}
}

\newglossaryentry{OSLE 9.6.b}{
    name={OSLE 9.6.b},
    description={recognize that the current going into a junction is equal to the current coming out of it}
}

\newglossaryentry{OSLE 9.6.c}{
    name={OSLE 9.6.c},
    description={describe qualitatively and quantitatively the current flow throughout a parallel circuit}
}

\newglossaryentry{OSLE 9.6.d}{
    name={OSLE 9.6.d},
    description={recognize that the voltage drops across each resistor are equal to the voltage of the power supply}
}

\newglossaryentry{OSLE 9.6.e}{
    name={OSLE 9.6.e},
    description={describe advantages and disadvantages of parallel circuits compared to series circuits}
}

\newglossaryentry{OSLE 9.6.f}{
    name={OSLE 9.6.f},
    description={calculate the equivalent resistance of multiple resistors in parallel}
}

\newglossaryentry{OSLE 9.6.g}{
    name={OSLE 9.6.g},
    description={calculate the equivalent voltage of batteries in parallel}
}

\newglossaryentry{OSLE 9.6.h}{
    name={OSLE 9.6.h},
    description={describe the energy transformations (transfers) occurring in a parallel circuit}
}

\newglossaryentry{OSLE 9.6.i}{
    name={OSLE 9.6.i},
    description={determine the (i) current through, voltage drop across, and power of each component, and (ii) total current of circuit, when given a series circuit diagram}
}


\newglossaryentry{OSLE 9.7.a}{
    name={OSLE 9.7.a},
    description={determine whether elements of a combination circuit have the same current or voltage}
}

\newglossaryentry{OSLE 9.7.b}{
    name={OSLE 9.7.b},
    description={predict which bulbs will light if switches are open or closed}
}








\setenumerate{itemsep=0pt,topsep=0pt}
\setitemize{itemsep=0pt,topsep=0pt}

\renewcommand*{\thesection}{Unit~\arabic{section}:\hspace{-1ex}}

\title{Physics}
\author{}
\date{}

\begin{document}

% \maketitle
\vspace{-1em}
\begin{center}
    Physics Overview and Student Learning Expectations (OSLE)
\end{center}

\section{Constant Motion}

%Enduring Understanding: 
\vspace{-3pt}
The student will understand what it means for an object to be in motion in terms of displacement, velocity, momentum, and kinetic energy.
\vspace{3pt}

\textbf{\underline{Constant Motion Model}} (8--10 days)

\textbf{Overview:} The student discovers the meaning of constant motion concepts by collecting and analyzing data in the form of tables, graphs, and motion maps. Then, the student develops a mathematical model for constant motion from these discoveries and representations.

\begin{enumerate}[itemsep=0pt]
    \item[1.1] \textit{Location and How Far.} The student is expected to:
    \begin{enumerate}[itemsep=0pt,topsep=0pt]
        \item Differentiate between position, displacement, and distance
        \item Determine the displacement of an object from a reference point
        \item Compare the positions (or displacements) of objects with different constant motions at various times using appropriate Multiple Representations.
    \end{enumerate}
    \item[1.2] \textit{How Fast.} The student is expected to:
    \begin{enumerate}[itemsep=0pt,topsep=0pt]
        \item Explain the concept of velocity using words and vectors.
        \item Calculate the velocity of an object.
        \item Compare the speeds and velocities of objects with different constant motions using appropriate Multiple Representations.
        \item Predict the displacement and/or velocity of an object using number sense and the concept of velocity.
        \item Translate (interpret and generate) between different representations of constant motion using appropriate Multiple Representations.
        \item Generate a mathematical model for displacement from a position vs. time graph or a velocity vs. time graph.
    \end{enumerate}
\end{enumerate}

\textbf{\underline{Constant Motion Model Refined With Momentum and Kinetic Energy}} (2--3 days)

\textbf{Overview:} The student makes connections between mass, inertia, velocity, momentum and kinetic energy. Then, the connections are used to expand their conceptual model of motion.

\begin{enumerate}[itemsep=0pt]
    \item[1.3] \textit{Mass and inertia.} The student is expected to:
    \begin{enumerate}[itemsep=0pt,topsep=0pt]
        \item Define inertia and mass;
        \item Relate inertia to mass.
    \end{enumerate}
    \item[1.4] \textit{Momentum.} The student is expected to:
    \begin{enumerate}[itemsep=0pt,topsep=0pt]
        \item Differentiate between inertia and momentum;
        \item Compare the momenta of objects with varying masses or velocities using number sense and the meaning of momentum using Multiple Representations;*
        \item Calculate the momentum of an object.
    \end{enumerate}
    \item[1.5] \textit{Kinetic Energy.} The student is expected to:
    \begin{enumerate}[itemsep=0pt,topsep=0pt]
        \item Differentiate between momentum and kinetic energy.
        \item Compare the kinetic energies of objects with varying masses or velocities using number sense.
        \item Calculate the kinetic energy of an object
    \end{enumerate}
\end{enumerate}




\textbf{\underline{Frame of Reference}} (1--2 days)

\textbf{Overview:} The student analyzes constant velocity situations using multiple frames of reference.

\begin{enumerate}[itemsep=0pt]
    \item[1.6] \textit{Frame of Reference.} The student is expected to:
    \begin{enumerate}[itemsep=0pt,topsep=0pt]
        \item Compare the speeds and velocities of objects with constant motions relative to one another along a line (parallel vectors in the same and opposite directions.)
        \item Calculate the velocity of an object relative to multiple frames of reference.
    \end{enumerate}
\end{enumerate}

\textbf{Notes and Limitations:}

\begin{itemize}[topsep=-3pt,itemsep=0pt]
    \item This unit only deals with 1-D motion. Therefore,  each topic within the unit should only cover 1-D vector quantities.
    \item This unit does not consider the change in momentum of an object. Students will compare the momenta of varying objects.
    \item This section should only be taught in 1-D. Velocities should be linear or co-linear to one another.
\end{itemize}



\section{Force Interactions}

\vspace{-3pt}
The student will understand that an object can be at rest or in constant motion when either no forces or when only balanced forces are acting on it.
\vspace{3pt}

\textbf{\underline{Interactions}} (3 days)

\textbf{Overview:} The student is being introduced to the concept of force as an interaction between two objects that always act in pairs.

\begin{enumerate}[itemsep=0pt]
    \item[2.1] \textit{Interacting Objects.} The student is expected to:
    \begin{enumerate}[itemsep=0pt,topsep=0pt]
        \item Identify the type of forces acting in a scenario (i.e. gravitational, normal, frictional, applied, spring, tension).
        \item Identify and represent the force pairs of each interaction in a scenario. 
        \item Recognize that force pairs (i) consist of exactly two forces (same type) resulting from one interaction between two objects, and (ii) must be equal in magnitude and opposite in direction.
        \item Identify instances in which equal and opposite forces are not a force pair (e.g.  $F_g$ and $F_\mathrm{N}$ are not force pairs even when they are equal in magnitude and opposite in direction).
    \end{enumerate}
\end{enumerate}

\textbf{\underline{Balanced vs. Unbalanced}} (4 days)

\textbf{Overview:} The student will now look at just one object and the forces acting on it. They will determine if the constant motion model applies to a situation based upon the presence of balanced or unbalanced forces.

\begin{enumerate}[itemsep=0pt]
    \item[2.2] \textit{When to use constant motion model based on forces.} The student is expected to:
    \begin{enumerate}[itemsep=0pt,topsep=0pt]
        \item Diagram the forces acting on an object using a free body diagram.
        \item Explain the effects of (i) balanced forces on the motion of an object, and (ii) unbalanced forces on the motion of an object.
        \item Connect by matching, interpreting, and generating Multiple Representations, the concepts of (i)~constant velocity, constant momentum, constant kinetic energy, and balanced forces, and (ii)~changing velocity, changing momentum, changing kinetic energy, and unbalanced forces.
    \end{enumerate}
\end{enumerate}

\textbf{Notes and Limitations:}

\begin{itemize}[topsep=-3pt,itemsep=0pt]
    \item This unit will address the concept of changing motion only as a consequence of unbalanced forces.  Concepts of acceleration and Law of acceleration will be covered in future units.
    \item The 2nd Law and the rate of change of motion/velocity (acceleration) is not discussed in this unit. 
    \item Students know that an object with changing speed is accelerating from 8th grade, but they have not learned that it is a vector or that it is the rate of change of velocity. We will teach this in unit 3.
\end{itemize}

\section{The Law of Acceleration}

\vspace{-3pt}
The student will understand that an object experiencing changing motion must be experiencing a net force causing it to accelerate at a rate dependent on the size of that net force and the object's mass. The student will understand acceleration as a rate of change of velocity.
\vspace{3pt}

\textbf{\underline{Acceleration}} (7--9 days)

\textbf{Overview:} The student discovers the meaning of acceleration by collecting and analyzing data in the form of tables, graphs, and motion maps. Then the student develops a mathematical model for acceleration from these discoveries and representations.

\begin{enumerate}[topsep=0pt]
    \item[3.1] \textit{Acceleration.} The student is expected to:
    \begin{enumerate}[topsep=0pt,itemsep=0pt]
        \item Explain the concept of acceleration qualitatively (e.g. change in velocity per unit of time, etc.) and using quantitative data (e.g. the car in my lab went 1 meter in the first second but 2 meters in the second, meaning its speed increased 1 m/s each second).
        \item Describe the possible motions of an object with zero acceleration, positive acceleration, and negative acceleration.
        \item Compare the accelerations of objects when given changes in velocity over the same time interval.
        \item Determine the change in velocity and/or the acceleration of an object when given Multiple Representations.
        \item Interpret a representation of changing motion and create a different representation of that same motion using appropriate Multiple Representations.
        \item Predict the change in velocity, instantaneous velocity, or acceleration of an object by applying their understanding of the definition of acceleration and number sense when given displacement/velocity/acceleration data and time.
    \end{enumerate}
\end{enumerate}

\textbf{\underline{Factors that Affect Acceleration}} (4--6 days)

\textbf{Overview:} The student will investigate and apply Newton's Law of Acceleration.

\begin{enumerate}[topsep=0pt]
    \item[3.2] \textit{Newton's Law of Acceleration.} The student is expected to:
    \begin{enumerate}[topsep=0pt,itemsep=0pt]
        \item Describe the relationship between Fnet and the acceleration experienced by an object.
        \item Describe the relationship between mass and the acceleration experienced by an object.
        \item Qualitatively predict the effects of changing the net force acting on an object or the object's mass on the object's acceleration using logic, number sense, and Newton's Law of Acceleration (i.e. bigger force, more massive, with or counter direction of motion).
        \item Solve problems using Fnet = ma.
    \end{enumerate}
\end{enumerate}

\textbf{Notes and Limitations:}

\begin{itemize}[topsep=-3pt,itemsep=0pt]
    \item The focus of this section is on using the Fnet = ma equation with the net force as a singular quantity. Students will sum individual forces and/or solve for individual forces in unit 5.
\end{itemize}

\section{Impulse and Work}

\vspace{-3pt}
The student will understand that an object experiencing changing motion must be experiencing a net force causing it to accelerate as well as an impulse that changes its momentum and net work that is changing its kinetic energy. 
\vspace{3pt}

\textbf{\underline{Impulse}} (5--7 days)

\textbf{Overview:} The student will go deeper into the changing motion model from Units 2 and 3 by exploring how the size of the net force and the amount of time it acts on an object affect the object's motion.

\begin{enumerate}[topsep=0pt]
    \item[4.1] \textit{Changing momentum.} The student is expected to:
    \begin{enumerate}[itemsep=0pt,topsep=0pt]
        \item Relate and calculate an object's change in momentum to and from its change in velocity.
        \item Describe the relationship between the size of a net force on an object to the object's change in momentum.
        \item Describe the relationship between how long the force is applied and an object's change in momentum.
        \item Explain the real-life applications of the relationship between $F_\mathrm{net}$ and $t$ in producing a given change in momentum (i.e. safety equipment, catapult plane launch).
        \item Solve problems utilizing the Impulse-Momentum Theorem (i.e. $F t = \Delta p$).
    \end{enumerate}
\end{enumerate}

\textbf{\underline{Work}} (5--7 days)

\textbf{Overview:} The student will go deeper into the changing motion model from Units 2 and 3 by exploring how the size of the net Force and the displacement it acts through affect the object's motion.

\begin{enumerate}[topsep=0pt]
    \item[4.2] \textit{Changing kinetic energy.} The student is expected to:
    \begin{enumerate}[topsep=0pt,itemsep=0pt]
        \item Identify if a Net Force is doing work on an object (i) based on their being a component of the net force parallel to displacement, and (ii) based on the object experiencing a change in kinetic energy.
        \item Calculate the net work done on an object.
        \item Describe the relationship between the size of the net force on an object's changing kinetic energy.
        \item Describe the relationship between the displacement over which net force is applied and the changing kinetic energy of an object.
        \item Solve problems using the Work Energy Theorem (i.e. $W_\mathrm{net} = \Delta \mathrm{KE}$).
        \item Compare the power of various situations with differing time, work, and $\Delta \mathrm{KE}$ (only changing one variable at a time).
        \item Calculate the power based on (i) the work and time, and (ii) change in kinetic energy.
    \end{enumerate}
\end{enumerate}

\section{Force Analysis}

\vspace{-3pt}
Students understand that gravitational force is universal and depends on the amount of mass interacting and the distance between masses. Students should be able to analyze a given situation in terms of forces and the motion of an object.
\vspace{3pt}

\textbf{\underline{Individual Forces}} (3--4 days)

\textbf{Overview:} The student is learning which factors affect various forces and learning to classify/label the forces acting on an object.

\begin{enumerate}[topsep=0pt]
    \item[5.1] \textit{Newton's Law of Universal Gravitation.} The student is expected to:
    \begin{enumerate}[topsep=0pt,itemsep=0pt]
        \item Compare the gravitational force on mass A from B to the gravitational force on mass B from A in a variety of physical situations using Newton's Universal Law of Gravitation and Law of Action-Reaction.
        \item Compare the acceleration of mass A to that of mass B qualitatively using Newton's Universal Law of Gravitation and Law of Acceleration.
        \item Rank the magnitude of the gravitational force between objects at various distances using proportional reasoning, number sense, and Law of Universal Gravitation.
        \item Rank the magnitude of the gravitational force between objects of various masses using proportional reasoning, number sense, and the Law of Universal Gravitation.
        \item Describe what a gravitational field is and compare the gravitational field strength of various masses.
        \item Calculate the force of gravity on an object using Newton's Law of Universal Gravitation.
        \item Know that the gravitational field strength on Earth is \SI{9.8}{N/kg} (on average), which we will round to \SI{10}{N/kg} in this course, and has a symbol of $g$. 
        \item Differentiate between mass and weight.
        \item Perform calculations using $F_g = mg$.
    \end{enumerate}
    \item[5.2] \textit{Forces from surfaces.} The student is expected to:
    \begin{enumerate}[topsep=0pt,itemsep=0pt]
        \item Identify in various situations when a surface-object force interaction involves (i) the normal force and (ii) the frictional force.
        \item Compare the normal force to the gravitational force (weight) on an object in various situations including inclined planes and explain why these are not force pairs.
        \item Compare how various types of motions, surfaces, surface areas, and the normal force will affect the magnitude of the frictional force on an object.
    \end{enumerate}
    \item[5.3] \textit{Applied Forces.} The student is expected to:
    \begin{enumerate}[topsep=0pt,itemsep=0pt]
        \item Identify when a tension force is acting on an object and in which direction.
        \item Identify when any other applied force is acting on an object and in which direction.
    \end{enumerate}
\end{enumerate}

\textbf{\underline{Force Analysis}} (4--5 days)

\textbf{Overview:} The student will now look at just one object and the forces acting on it. They will analyze physical situations and/or word problems in terms of individual forces, components of forces, net forces along each axis, and motion along each axis.

\begin{enumerate}[topsep=0pt]
    \item[5.4] \textit{Analyzing Motion using Forces.} The student is expected to:
    \begin{enumerate}[topsep=0pt,itemsep=0pt]
        \item Represent the forces acting on an object in a free body force diagram including relative magnitude and direction from a given scenario (word problem).
        \item Calculate the magnitude and direction of the net resultant force acting on an object using vector addition when given a scenario (word problem).
        \item Solve for an unknown force, mass, or acceleration using $F_\mathrm{net}=ma$ when given a scenario (word problems).
        \item Solve for unknowns incorporating $F_\mathrm{net}=ma$, impulse, momentum, work, KE, velocity, acceleration, and displacement when given data from Multiple Representations.
    \end{enumerate}
\end{enumerate}

\textbf{Notes and Limitations:}

\begin{itemize}[topsep=-3pt,itemsep=0pt]
    \item On-level students will not need to calculate force of friction using $F_f = \mu F_\mathrm{N}$. 
    \item On-level students do not need to label types of friction as kinetic and static friction, but need an understanding that the amount of friction acting on an object does change.
    \item This is the first time students will be adding non-linear vectors in this course. Time will need to be allotted for introducing and practicing this new skill.
    \item Qualitative analysis of forces at an angle should be done by all students.
    \item Quantitative analysis of forces at an angle should be done using Pythagorean theorem and graphical analysis by all students. K students may also use trigonometric identities.
\end{itemize}



\section{One-Dimensional Motion}

\vspace{-3pt}
The student knows that all objects fall at the same rate when in free fall. The student will be able to use multiple representations, with a focus on mathematical models, to describe 1D motion both horizontally and vertically.
\vspace{3pt}

\textbf{\underline{Free Fall}} (3--4 days)

\textbf{Overview:} The student is investigating and examining the motion of an object when only acted upon by Earth's gravitational force.

\begin{enumerate}
    \item[6.1] \textit{Free Fall Concepts}. The student is expected to:
    \begin{enumerate}
        \item Compare the gravitational field strength on Earth to the acceleration due to gravity on Earth
        \item Explain using universal gravitation and $F_\mathrm{net}=ma$ why all objects near Earth's surface fall at the same rate when in free fall.
        \item Explain the relationship between the mass, initial position, and initial velocity of an object in free fall on its final velocity and/or time in free fall.
        \item Describe the displacement, velocity, momentum, kinetic energy, and acceleration of an object in free fall that was dropped, thrown upward, or thrown downward using Multiple Representations
        \item Relate the gravitational force, impulse, and work done on the object by the Earth to the object's change in velocity (acceleration), momentum, and kinetic energy.
    \end{enumerate}
\end{enumerate}

\textbf{\underline{Constant Acceleration Word Problems}}  (2--3 days)

\textbf{Overview:} The student will analyze and solve constant acceleration word problems utilizing multiple representations including kinematic equations and integrating in all concepts learned in previous units.

\begin{enumerate}
    \item[6.2] \textit{Solving Constant Acceleration Problems}. The student is expected to:
    \begin{enumerate}
        \item Describe what is known about an object's motion in a constant acceleration word problem using Multiple Representations
        \item Solve for various unknown quantities utilizing kinematic equations when data is given in Multiple Representations for objects moving horizontally with constant acceleration
        \item Solve for various unknown quantities utilizing kinematic equations when data is given in Multiple Representations for objects moving vertically with constant acceleration (free fall)
        \item Solve multi-step problems that connect kinematic equations, the Law of Acceleration, Work-Energy Theorem, and/or the Impulse-Momentum Theorem
    \end{enumerate}
\end{enumerate}

\textbf{Notes and Limitations:}

\begin{itemize}[itemsep=0pt,topsep=-3pt]
    \item 1D Motion and Acceleration only.
\end{itemize}



\section{Motion in Two Dimensions}

\vspace{-3pt}
The student will understand that 2-D motion is the motion along two axes that are independent of each other. The student will understand the characteristics and causes of projectile motion and circular motion.
\vspace{3pt}

\textbf{\underline{Analysis of Projectile Motion}} (9--11 days)

\textbf{Overview}: The student will now consider an object moving horizontally while also in free fall.  They will explore factors that affect the motion of a projectile.  

\begin{enumerate}
    \item[7.1] \textit{Factors That Affect Projectile Trajectory}. The student is expected to:
    \begin{enumerate}
        \item Compare the trajectory, hang time, max height, range, and final velocity of various projectiles that have different initial velocities, launch heights, launch angles, and masses (only varying one parameter at a time).
        \item Identify if and explain how the  initial velocity, launch height, launch angle, and mass of a projectile influence its motion (hang time, height, range, final velocity).
    \end{enumerate}
    \item[7.2] \textit{Horizontally Launched Projectiles}. The student is expected to:
    \begin{enumerate}
        \item Describe the vertical and horizontal motion of a projectile with a launch angle of zero using Multiple Representations.
        \item Illustrate the resultant motion of the projectile at any point in its trajectory as well as the relationship between the horizontal and vertical components using vector addition;
        \item Analyze and solve word problems about the motion of  horizontally launched projectiles using kinematic equations, vector addition, and  Multiple Representations.
    \end{enumerate}
\end{enumerate}

\textbf{\underline{Circular Motion}} (7--9 days)

\textbf{Overview}: Students will explore uniform circular motion, the factors that affect it, and the net force that causes its acceleration.

\begin{enumerate}
    \item[7.3] \textit{Circular motion}. The student is expected to:
    \begin{enumerate}
        \item Describe the motion of an object moving with uniform circular motion in terms of centripetal force, centripetal acceleration, momentum, kinetic energy, and tangential velocity using Multiple Representations.
        \item Determine the centripetal force, mass, centripetal acceleration, tangential velocity, or radius of an object in circular motion.
    \end{enumerate}
    \item[7.4] \textit{Circular Motion and Orbiting Bodies}. The student is expected to:
    \begin{enumerate}
        \item Predict the effects of changing the radius or mass of objects in orbiting systems using concepts of uniform circular motion and Newton's law of universal gravitation.
    \end{enumerate}
\end{enumerate}

\textbf{Notes and Limitations:}

\begin{itemize}[itemsep=0pt,topsep=-3pt]
    \item All quantitative scenarios occur in a vacuum.
    \item Acceleration due to gravity is equal to the gravitational field strength.  
    \item As in Unit 4, $g = \SI{10}{m/s^2}$ or \SI{10}{N/kg}.
    \item L will use Pythagorean theorem to find resultant vectors. They should not use trig identities to find directions. K should use trig identities to find directions/angles.
    \item Non-horizontally launched projectiles are only addressed when considering the factors that influence a projectile's trajectory. They do not need to be analyzed quantitatively. If time permits, K-only may solve word problems with them.
\end{itemize}



\section{Conservation in Mechanical Systems}

\vspace{-3pt}
The student will understand that while energy and momentum can be exchanged between objects or even between systems through work, heat, and impulse, all energy and momentum can be accounted for because the Law of Conservation of Energy and the Law of Conservation of Momentum are universal.
\vspace{3pt}

\textbf{\underline{Conservation of Energy}} (4--5 days)

\textbf{Overview}: The student will investigate and demonstrate an understanding that energy is always conserved and thus any change in energy in a system can be accounted for through work or heat. 

\begin{enumerate}
    \item[8.1] \textit{Physical Systems}. The student is expected to:
    \begin{enumerate}
        \item Identify multiple choices for a system given a scenario 
        \item Recognize that energy can be stored in the arrangement of particles or objects in a system as potential energy
        \item Identify and calculate when a system includes energy stored in the arrangement of its particles or objects: (i) Gravitational potential energy, (ii) Elastic potential energy
        \item Compare the potential energy of a scenario for various choices of system
        \item Identify, represent using multiple representations*, and calculate the total mechanical energy present in a physical system.
        \item Predict the effects of changing the mass, velocity,  height, gravitational field strength, spring constant, compression/stretching distance on the amount of $E_k$, $E\mathrm{GP}$, and $E_\mathrm{SP}$.
        \item Calculate the total mechanical energy of a system
    \end{enumerate}
    \item[8.2] \textit{Energy is Conserved}. The student is expected to:
    \begin{enumerate}
        \item Analyze the energy of a system before and after an event 
        \begin{enumerate}
            \item Identify, represent using multiple representations, and calculate the amount of energy
            \begin{itemize}
                \item[] (1) transformed from one storage mode to another within a system (including kinetic energy, potential energy, and thermal energy)
                \item[] (2) transferred from one object in the system to another in the system
                \item[] (3) entering/leaving a system due to work, heat, light, or sound
            \end{itemize}
        \end{enumerate}
        \item Apply the Law of Conservation of Energy to systems
        \begin{enumerate}
            \item Explain the meaning of the Law of Conservation of Energy
            \item Develop an energy formula for systems using energy bar charts and the Law of Conservation of Energy
            \item Solve for various unknown quantities using the concept of the conservation of energy
        \end{enumerate}
        \item Determine the power done to or by a system using time intervals and work done
        \begin{enumerate}
            \item Know the definition of work as change in energy of a system
            \item Know that power is work done divided by time
        \end{enumerate}
    \end{enumerate}
\end{enumerate}

\textbf{\underline{Conservation of Momentum}} (4--5 days)

\textbf{Overview}: The student is expected to understand and demonstrate momentum conservation in both elastic and inelastic collisions.

\begin{enumerate}
    \item[8.3] \textit{Momentum is Conserved}. The student is expected to
    \begin{enumerate}
        \item Calculate and compare the momentum, changes in momentum, force applied to and impulse on each object involved in a collision or explosion scenario.
        \item Represent using multiple representations, compare, and calculate the total momentum of a system before and after a collision or explosion scenario.
        \item Explain the meaning of the Law of Conservation of Momentum.
        \item Solve for unknown quantities using the concept of the conservation of momentum.
    \end{enumerate}
\end{enumerate}



\textbf{Notes and Limitations:}
\begin{itemize}[topsep=-3pt,itemsep=0pt]
    \item Energy lost to friction can be represented as negative work or  thermal energy transfer (heat). Students should see examples of both representations.
    \item Heat will only be calculated using conservation of energy. 
    \item Students are expected to use the universal conservation of energy equation and apply it to each scenario as opposed to being given different scenario-specific equations.
    \item Students are expected to use the universal conservation of momentum equation and apply it to each scenario as opposed to being given different scenario-specific equations.
    \item Students will not be given a scenario and asked to determine what type of collision it is based on the scenario alone. 
    \item K's only may solve for unknown quantities using the concept of the conservation of momentum.
\end{itemize}


\clearpage
\section{Conservation of Charge}

\vspace{-3pt}
The student will understand the concept of charge and the force interactions charges produce through electric fields, both while at rest and when moving through circuits. The student will apply conservation of charge and energy to charge transfers and to the flow of current in circuits.
\vspace{3pt}

\textbf{\underline{Electrostatics}} (4--6 days)

\textbf{Overview}: The student will describe the charges in matter, the interactions between them, and how they are transferred.

\begin{enumerate}
    \item[9.1] \textit{Charges and Matter}. The student is expected to:
    \begin{enumerate}
        \item Identify the particles that contribute positive, negative, or no charge in an atom.
        \item Recognize that neutral objects have even numbers of positive and negative charges.
        \item Determine the charge of an object given the number of protons and electrons.
        \item Predict if two objects will attract, repel, or have no interaction based on their charges.
        \item Draw the electric field surrounding single charges and pairs of charges.
    \end{enumerate}
    \item[9.2] \textit{Transfer of Charge}. The student is expected to:
    \begin{enumerate}
        \item Recognize that charge is conserved: it cannot be created or destroyed, only transferred.
        \item Realize that only electrons are transferred during charging.
        \item Compare and contrast charging by induction and conduction.
        \item Explain how polarization temporarily charges a neutral object.
        \item Describe how an electroscope determines if objects are charged.
        \item Determine whether an object is negatively charged, positively charged, or neutral when given the charge of one object and a description or diagram representing how the charged object interacts with an object of unknown charge.
        \item Draw and describe the resulting distribution of charge for various scenarios of induction, conduction, and polarization
    \end{enumerate}
    \item[9.3] \textit{Coulomb's Law}. The student is expected to:
    \begin{enumerate}
        \item Draw the free body diagram for 2 charged objects showing the direction and relative magnitude of the electrical force acting on each object at various distances from each other
        \item Describe how the electric force depends on the charges and the distance between them.
        \item Compare and contrast the electric force to the gravitational force.
        \item Predict how changing the charge or distance affects the electric force.
    \end{enumerate}
\end{enumerate}

\textbf{\underline{Circuits}} (14--16 days)

\textbf{Overview}: The student will apply conservation of charge to currents and conservation of energy to voltages to determine the values read by multimeters across resistances.

\begin{enumerate}
    \item[9.4] \textit{Simple Circuits}. The student is expected to:
    \begin{enumerate}
        \item Identify the necessary components for a simple circuit and discover different ways to light a bulb.
        \item Trace the conducting path through a simple circuit.
        \item Explain the concepts of current, resistance, voltage.
        \item Measure the current, resistance and voltage in a circuit using a multimeter, ammeter, current probe, etc.
        \item Calculate the voltage drop across, current through, or resistance of a circuit component using Ohm's Law.
        \item Determine the change in current as the voltage or resistance is changed.
        \item Interpret electrical power as the rate at which electrical energy is being dissipated in the circuit.
        \item Relate the power rating/wattage of a light bulb to its brightness.
    \end{enumerate}
    \item[9.5] \textit{Series Circuits}. The student is expected to:
    \begin{enumerate}
        \item Measure the current, resistance and voltage at various locations in a series circuit using a multimeter, ammeter, current probe, etc.
        \item Describe qualitatively and quantitatively the current flow throughout a series circuit.
        \item Calculate the equivalent resistance of multiple resistors in series.
        \item Calculate the equivalent voltage of batteries in series.
        \item Recognize that the sum of the voltage drops across resistors in series equals the total voltage of the power supply.
        \item Describe the energy transformations (transfers) occurring in a series circuit.
        \item Determine (i) current through, voltage drop across, and power of each component, and (i) total current of circuit, when given a series circuit diagram
    \end{enumerate}
    \item[9.6] \textit{Parallel Circuits}. The student is expected to:
    \begin{enumerate}
        \item Measure the current, resistance and voltage at various locations in a parallel circuit using a multimeter, ammeter, current probe, etc.
        \item Recognize that the current going into a junction is equal to the current coming out of it.
        \item Describe qualitatively and quantitatively the current flow throughout a parallel circuit.
        \item Recognize that the voltage drops across each resistor are equal to the voltage of the power supply.
        \item Describe advantages and disadvantages of parallel circuits compared to series circuits
        \item Calculate the equivalent resistance of multiple resistors in parallel.
        \item Calculate the equivalent voltage of batteries in parallel.
        \item Describe the energy transformations (transfers) occurring in a parallel circuit.
        \item Determine the (i) current through, voltage drop across, and power of each component, and (ii) total current of circuit, when given a series circuit diagram
    \end{enumerate}
    \item[9.7] \textit{Combination Circuits}. The student is expected to:
    \begin{enumerate}
        \item Determine whether elements of a combination circuit have the same current or voltage.
        \item Predict which bulbs will light if switches are open or closed.
    \end{enumerate}
\end{enumerate}


\textbf{Notes and Limitations:}

\begin{itemize}[topsep=-3pt,itemsep=0pt]
    \item Students should use Coulomb's Law to calculate the electric force. L should not be required to rearrange the formula and solve for other unknowns.
    \item K may calculate the electric force from multiple charges; however, all charges should be in a line (1D calculations).
    \item All voltage sources are DC only.
    \item All circuit components (batteries, wires, resistors) and meters are considered to be ideal.
    \item The expectation is that all students will have multiple hands-on experiences with circuitry. Ohm's Law, the Junction Rule, equivalence resistance all should be investigated and discovered by students. Simulations should also be used for this purpose but they should not replace hands-on opportunities.
    \item No calculations should be performed on combination circuits.
\end{itemize}

\clearpage
\section{Electromagnetic Induction}

\vspace{-3pt}
The student will understand how electricity and magnetism are related to each other, predict how to change the amount of induced current, how to strengthen electromagnets, and explain the operation of generators, motors, and transformers.
\vspace{3pt}

\textbf{\underline{Magnetic Fields}} (1--2 days)

\textbf{Overview:} The student will understand how magnets produce fields and interact with metals.

\begin{enumerate}[topsep=0pt]
    \item[10.1] \textit{Magnet.} The student is expected to:
    \begin{enumerate}[topsep=0pt,itemsep=0pt]
        \item Understand that all magnets have a north and south pole, even if they are broken in half.
        \item Predict whether two magnets will attract or repel based on their poles.
        \item Classify materials as magnetic or nonmagnetic.
    \end{enumerate}
    \item[10.2] \textit{Magnetic Field}. The student is expected to:
    \begin{enumerate}[topsep=0pt,itemsep=0pt]
        \item Realize that magnetic fields fill the space around magnets and are not directly observable.
        \item Visualize magnetic field loops using iron filings.
        \item Predict if two magnets will attract or repel based on their field lines.
        \item Determine where the magnetic field is stronger due to the density of field lines.
    \end{enumerate}
    \item[10.3] \textit{Compass.} The student is expected to:
    \begin{enumerate}[topsep=0pt,itemsep=0pt]
        \item Explain how a compass points in the direction of the magnetic field.
        \item Recognize that the Earth's magnetic poles are opposite of its geographic poles.
    \end{enumerate}
\end{enumerate}

\textbf{\underline{Electromagnetism}} (3--4 days)

\textbf{Overview:} The student will relate magnetic fields to moving charges.

\begin{enumerate}[topsep=0pt]
    \item[10.4] \textit{Electromagnetic Fields.} The student is expected to:
    \begin{enumerate}[topsep=0pt,itemsep=0pt]
        \item Recognize that a magnetic field surrounds all moving charges.
        \item Describe the magnetic fields that circulate around current-carrying wires.
        \item Represent magnetic fields that go into or out of the page with $\times$ and $\bullet$.
    \end{enumerate}
    \item[10.5] \textit{Electromagnetic Induction.} The student is expected to:
    \begin{enumerate}[topsep=0pt,itemsep=0pt]
        \item Relate changing magnetic fields to induced electrical currents.
        \item Predict qualitative ways to change the amount of induced current using Faraday's Law.
    \end{enumerate}
\end{enumerate}

\textbf{\underline{Applications of Electromagnetic Fields}} (2--3 days)

\begin{enumerate}[topsep=0pt]
    \item[10.6] \textit{Electromagnet.} The student is expected to:
    \begin{enumerate}[topsep=0pt,itemsep=0pt]
        \item Explain how electromagnets use current to generate temporary magnetic fields.
        \item Determine ways to make an electromagnet stronger.
    \end{enumerate}
    \item[10.7] \textit{Transformers.} The student is expected to:
    \begin{enumerate}[topsep=0pt,itemsep=0pt]
        \item Draw a diagram of a transformer.
        \item Demonstrate that transformers change voltage while conserving energy.
        \item Determine if a transformer steps voltage up or down based on the number of coils on the input and output sides.
    \end{enumerate}
    \item[10.8] \textit{Generators and Motors.} The student is expected to:
    \begin{enumerate}[topsep=0pt,itemsep=0pt]
        \item Explain how generators convert mechanical to electrical energy.
        \item Identify ways to induce more or less current in a generator.
        \item Recognize that a moving charge in an external magnetic field experiences a perpendicular force.
        \item Compare the construction and operation of a motor to that of a generator.
        \item Identify ways to change the rotational speed of a motor.
    \end{enumerate}
\end{enumerate}

\textbf{Notes and Limitations:}

\begin{itemize}[topsep=-3pt,itemsep=0pt]
    \item Magnetic domains are an advanced topic.
    \item Predicting the direction of a magnetic Lorentz force using a right-hand rule is an advanced topic.
\end{itemize}



\section{Simple Harmonic Motion and Waves}

\vspace{-3pt}
The student will know how to measure and predict a wave's frequency, wavelength, and wave velocity and describe wave behaviors including reflection, refraction, diffraction, and interference.  The students will know that sound is a longitudinal wave and understand how sound wave behavior affects what we hear.  Students will know that electromagnetic radiation is made of transverse waves and will identify examples of electromagnetic radiation in nature and in technology.  The student will understand that visible light is an electromagnetic wave and describe how these waves' behavior affects what we see.
\vspace{3pt}

\textbf{\underline{Mechanical Waves}} (7 days)

\textbf{Overview}: The student will learn the parts and behavior of both transverse and longitudinal waves and understand how conservation of energy and momentum is applied to these wave behaviors.


\begin{enumerate}
    \item[11.1] \textit{Oscillating motion}. The student is expected to measure and predict the causes and effects of simple harmonic motion.
    \begin{enumerate}
        \item Describe changes in an oscillating medium using conservation of energy and conservation of momentum.
        \item Determine using a position vs time graph the amplitude, frequency and period of an object in simple harmonic motion.
        \item Predict how changes to medium and energy input affects the frequency, amplitude, and period of an object in simple harmonic motion.
    \end{enumerate}
    \item[11.2] \textit{Transverse Waves}. The student is expected to recognize the parts, measurements, and behaviors of transverse waves.
    \begin{enumerate}
        \item Be able to measure the amplitude, wavelength, frequency, and wave speed of a transverse wave.
        \item Predict the effect of changing the medium on the wave speed, wavelength, and frequency.
        \item Understand the relationship between frequency, wave velocity, and wavelength.  Be able to calculate one of these quantities when given the other two.
        \item Predict how constructive and deconstructive interference affect the amplitude of transverse waves.
    \end{enumerate}
    \item[11.3] \textit{Longitudinal Waves}. The student is expected to recognize the parts, measurements, and behaviors of longitudinal waves.
    \begin{enumerate}
        \item Be able to recognize and describe the difference between the structure of a transverse and longitudinal wave.
        \item Know that sound is a longitudinal wave.  Describe how the quality of sound (loudness and pitch) relate to measurable quantities of a longitudinal sound wave.  Be able to calculate the frequency or wavelength of a sound wave.
        \item Understand what resonance is and describe how it affects what we hear from vibrating objects.
        \item Describe how the Doppler effect changes the frequency and wavelength of a sound wave.
    \end{enumerate}
\end{enumerate}

\textbf{\underline{Electromagnetic Spectrum}} (8 days)

\textbf{Overview}: The student will be able to identify where electromagnetic radiation is used in nature and in technology; including how it affects what we see.

\begin{enumerate}
    \item[11.4] \textit{Electromagnetic Waves}. The student is expected to describe the nature of electromagnetic waves and identify where electromagnetic waves of various frequencies are found in nature and used in technology.
    \begin{enumerate}
        \item Understand that Electromagnetic waves are transverse waves that can move across empty space at the speed of light. Describe how this is different from mechanical waves.
        \item Be able to rank Electromagnetic radiation by frequency, wavelength, or energy.
        \item Research and compare different applications of electromagnetic radiation.  Know how the frequency and wavelength differ among these applications.
    \end{enumerate}
    \item[11.5] \textit{Visible Light}. The student is expected to understand and describe how the visible range of the electromagnetic spectrum affects what we see when looking at the world around us.
    \begin{enumerate}
        \item Understand that visible light is made up of electromagnetic waves within a certain range of frequencies.
        \item Describe how we are able to perceive various colors based on the frequency of the electromagnetic waves.
        \item Be able to rank the colors of visible light by frequency, wavelengths, or energy.
        \item Describe how electromagnetic radiation is produced by the emission spectrum of various atoms.
    \end{enumerate}
    \item[11.6] \textit{Optics}. The student is expected to understand the behavior of light waves and predict the effect of reflection and refraction on the direction of light waves.
    \begin{enumerate}
        \item Be able to represent the direction of a light wave using a ray diagram.  Using a ray diagram, be able to draw how this direction changes when light waves encounter lenses and mirrors.
        \item Describe the effect a plane mirror has on the direction of incoming light rays and how this affects the image that is seen.
        \item Describe the effect of a thin convex lens on the direction of incoming light rays and how this affects the image that is seen.
    \end{enumerate}
\end{enumerate}


\textbf{Notes and Limitations}:
\begin{itemize}[topsep=-3pt,itemsep=0pt]
    \item Students do not need to calculate the frequency of sound resulting from the Doppler effect.
    \item Students are not required to use sine or cosine to describe the phase of a wave.
    \item Students are not expected to calculate the intensity of a sound wave.
    \item Students are not expected to understand resonance in open or closed pipes.
    \item Students are not expected to understand the effect of curved mirrors or concave lenses.
    \item Students are not expected to calculate the angle of incident, reflected, or refracted light waves.
\end{itemize}

 



\section{Quantum Physics}

\end{document}