% \documentclass[answers]{exam}
% \usepackage{marvosym}

%...TikZ & PGF
\usepackage{pgfplots}
\pgfplotsset{compat=1.11}
\tikzset{>=latex}
\usetikzlibrary{calc,math}
\usepackage{tikzsymbols}
\usepgfplotslibrary{fillbetween}
\usetikzlibrary{decorations.markings} 
\usetikzlibrary{arrows.meta} %...APP2 for arrows as objects and images
\usetikzlibrary{backgrounds} %...For shading portions of graphs
\usetikzlibrary{patterns} %...Unit 5 Problems
\usetikzlibrary{shapes.geometric} %...For drawing cylinders in Unit 2
\usepackage{makecell} %...use \thead{} to enable line skip in table headers
\tikzset{
    mark position/.style args={#1(#2)}{
        postaction={
            decorate,
            decoration={
                markings,
                mark=at position #1 with \coordinate (#2);
            }
        }
    }
} %...See https://tex.stackexchange.com/questions/43960/define-node-at-relative-coordinates-of-draw-plot

\tikzset{
    declare function = {trajectoryequation10(\x,\vi,\thetai)= tan(\thetai)*\x - 10*\x^2/(2*(\vi*cos(\thetai))^2);},
    declare function = {trajectoryequation(\x,\vi,\thetai)= tan(\thetai)*\x - 9.8*\x^2/(2*(\vi*cos(\thetai))^2);},
    declare function = {patheq(\x,\yi,\vi,\thetai)= \yi + tan(\thetai)*\x - 9.8*\x^2/(2*(\vi*cos(\thetai))^2);},
    declare function = {patheqten(\x,\yi,\vi,\thetai)= \yi + tan(\thetai)*\x - 10*\x^2/(2*(\vi*cos(\thetai))^2);} %like patheq but with gravity = 10
}

%...siunitx
\usepackage{siunitx}
\DeclareSIUnit{\nothing}{\relax}
\def\mymu{\SI{}{\micro\nothing} }
\DeclareSIUnit\mmHg{mmHg}
\DeclareSIUnit{\mile}{mi}
%...NOTE: "The product symbol between the number and unit is set using the quantity-product option."

%...Other
\usepackage{amsthm}
\usepackage{amsmath}
\usepackage{amssymb}
\usepackage{cancel}
\usepackage{subcaption}
\usepackage{dashrule}
\usepackage{enumitem}
% \usepackage{fontawesome}
\usepackage{fontawesome5}
\usepackage{multicol}
\usepackage{glossaries}
%\numberwithin{equation}{section}
\numberwithin{figure}{section}
\usepackage{float}
\usepackage{twemojis} %...twitter emojis
\usepackage{utfsym}
\usepackage{linearb} %...For \BPwheel in Unit 8
\newcommand{\R}{\mathbb{R}} %...real number symbol
\usepackage{graphicx}
\usepackage{mdframed} %...For FRQ teacher boxes
\graphicspath{ {../Figures/} }
\usepackage{hyperref}
\hypersetup{colorlinks=true,
    linkcolor=blue,
    filecolor=magenta,
    urlcolor=cyan,}
\urlstyle{same}
\newcommand{\hdashline}{{\hdashrule{\textwidth}{0.5pt}{0.8mm}}}
\newcommand{\hgraydashline}{{\color{lightgray} \hdashrule{0.99\textwidth}{1pt}{0.8mm}}}

%...Miscellaneous user-defined symbols
\newcommand{\fnet}{F_{\text{net}}} %...For net force
\newcommand{\bvec}[1]{\vec{\mathbf{#1}}} %...bold vector
\newcommand{\bhat}[1]{\,\hat{\mathbf{#1}}} %...bold hat vector
\newcommand{\que}{\mathord{?}}  %...Question mark symbol in equation env
%...Define thick horizontal rule for examples:
\newcommand{\hhrule}{\hrule\hrule}
\let\oldtexttt\texttt% Store \texttt
\renewcommand{\texttt}[2][black]{\textcolor{#1}{\ttfamily #2}}% 

%...For use in the exam document class
\newif\ifprintmetasolutions


%...Decreases space above and below align and gather enironment
\makeatletter
\g@addto@macro\normalsize{%
  \setlength\abovedisplayskip{-3pt}
  \setlength\belowdisplayskip{6pt} 
}
\makeatother





% \usepackage[margin=1in]{geometry}
\usepackage[figurewithin=none]{caption}
\usepackage{exam-randomizechoices}

\CorrectChoiceEmphasis{\color{red}\bfseries}
\renewcommand{\solutiontitle}{\noindent\textbf{\textcolor{red}{Solution:}}\enspace}

\usepackage{OutilsGeomTikz}
\usepackage{utfsym} %...Symbols in Unit 7 Problems
\usepackage{tabu} %...Symbols in Unit 7 Problems

%...For use in Unit 2            %    
\setlength{\columnsep}{2cm}      %
\setlength{\columnseprule}{1pt}  %
\usepackage[none]{hyphenat}      %
%%%%%%%%%%%%%%%%%%%%%%%%%%%%%%%%%

%...For use in Unit 11 on Waves:
\pgfdeclarehorizontalshading{visiblelight}{50bp}{  %
color(0.00000000000000bp)=(red);                   %
color(8.33333333333333bp)=(orange);                %
color(16.66666666666670bp)=(yellow);               %
color(25.00000000000000bp)=(green);                %
color(33.33333333333330bp)=(cyan);                 %
color(41.66666666666670bp)=(blue);                 %
color(50.00000000000000bp)=(violet)                %
}                                                  %

\newcommand{\checkbox}[1]{%
  \ifnum#1=1
    \makebox[0pt][l]{\raisebox{0.15ex}{\hspace{0.1em}\Large$\checkmark$}}%
  \fi
  $\square$%
}
%%%%%%%%%%%%%%%%%%%%%%%%%%%%%%%%%%%%%%%%%%%%%%%%%%%%

%...If using circuitikz package:
% \ctikzset{bipoles/battery1/height=0.5}
% \ctikzset{bipoles/battery1/width=0.25}
% \ctikzset{bipoles/resistor/height=0.15}
% \ctikzset{bipoles/resistor/width=0.4}
% %\usetikzlibrary{patterns}
% \usepackage[none]{hyphenat}
% \makenoidxglossaries

%...UNIT 1: CONSTANT MOTION

\newglossaryentry{scalar}{
    name=scalar,
    description={a quantity that has magnitude (and possibly sign) but no direction}
}

\newglossaryentry{magnitude}{
    name={magnitude},
    description={size or amount}
}

\newglossaryentry{vector}{
    name={vector},
    description={a quantity that has both magnitude and direction}
}

\newglossaryentry{tail}{
    name={tail},
    description={the starting point of a vector; the point opposite to the head or tip of the arrow}
}

\newglossaryentry{head}{
    name={head},
    description={the end point of a vector; the location of the vector's arrow; also referred to as the tip}
}

\newglossaryentry{head-to-tail method}{
    name={head-to-tail method},
    description={a method of adding vectors in which the tail of each vector is placed at the head of the previous vector}
}

\newglossaryentry{position}{
    name={position},
    description={the location of an object at any particular time}
}

\newglossaryentry{reference frame}{
    name={reference frame},
    description={a coordinate system from which the positions of objects are described}
}


\newglossaryentry{displacement}{
    name={displacement},
    description={the change in position of an object against a fixed axis}
}

\newglossaryentry{distance}{
    name={distance},
    description={the length of the path actually traveled between an initial and a final position}
}

\newglossaryentry{position vs. time graph}{
    name={position vs. time graph},
    description={a graph in which position is plotted on the vertical axis and time is plotted on the horizontal axis}
}

\newglossaryentry{speed}{
    name={speed},
    description={rate at which an object changes its location}
}

\newglossaryentry{average speed}{
    name={average speed},
    description={distance traveled divided by the time during which the motion occurs}
}

\newglossaryentry{velocity}{
    name={velocity},
    description={the speed and direction of an object}
}

\newglossaryentry{average velocity}{
    name={average velocity},
    description={displacement divided by the time during which the displacement occurs}
}


\newglossaryentry{velocity vs. time graph}{
    name={velocity vs. time graph},
    description={a graph in which velocity is plotted on the vertical axis and time is plotted on the horizontal axis}
}

\newglossaryentry{mass}{
    name=mass,
    description={the quantity of matter in a substance; the SI unit of mass is the kilogram}
}

\newglossaryentry{inertia}{
    name=inertia,
    description={the tendency of an object at rest to remain at rest, or for a moving object to remain in motion in a straight line and at a constant speed}
}

\newglossaryentry{Newton's first law of motion}{
    name={Newton's first law of motion},
    description={a body at rest remains at rest or, if in motion, remains in motion at a constant speed in a straight line, unless acted on by a net external force; also known as the law of inertia}
}

\newglossaryentry{momentum}{
    name={momentum},
    description={the product of a system's mass and velocity}
}

\newglossaryentry{momentum vs. time graph}{
    name={momentum vs. time graph},
    description={a graph in which momentum is plotted on the vertical axis and time is plotted on the horizontal axis}
}

\newglossaryentry{kinetic energy}{
    name={kinetic energy},
    description={energy of motion}
}

\newglossaryentry{joule}{
    name=joule,
    description={the metric unit for work and energy; equal to 1 newton meter ($\text{N}\cdot\text{m}$)}
}

\newglossaryentry{relative speed}{
    name={relative speed},
    description={how fast or slow an object appears to be moving to another object}
}

\newglossaryentry{relative velocity}{
    name={relative velocity},
    description={the rate at which an object changes position relative to another object}
}

%...UNIT 2: FORCE INTERACTIONS

\newglossaryentry{force}{
    name=force,
    description={a push or pull on an object with a specific magnitude and direction; can be represented by vectors; can be expressed as a multiple of a standard force; the SI unit of force is the Newton (N)}
}

\newglossaryentry{external force}{
    name=external force,
    description={a force acting on an object or system that originates outside of the object or system}
}

\newglossaryentry{free body diagram}{
    name=free body diagram,
    description={a diagram showing all external forces acting on a body}
}

\newglossaryentry{frictional force}{
    name=frictional force,
    description={an external force that acts opposite to the direction of motion or, for when there is no relative motion, in the direction needed to prevent slipping}
}

\newglossaryentry{applied force}{
    name={applied force},
    description={a contact force intentionally exerted by a person on an object}
}



\newglossaryentry{gravitational force}{
    name=gravitational force, %...MY DEFINITION
    description={the downward force on an object due to the attraction by the Earth or other massive body}
}

\newglossaryentry{net force}{
    name=net force,
    description={the sum of all forces acting on an object or system}
}

\newglossaryentry{normal force}{
    name=normal force,
    description={that component of the contact force between two objects, which acts perpendicularly to and away from their plane of contact}
}

\newglossaryentry{tension}{
    name=tension,
    description={a pulling force that acts along a connecting medium, especially a stretched flexible connector, such as a rope or cable; when a rope supports the weight of an object, the force exerted on the object by the rope is called tension}
}

\newglossaryentry{spring force}{
    name=spring force, %...From district slides
    description={a force applied from a spring when it is either compressed or stretched}
}

\newglossaryentry{Newton's third law of motion}{
    name={Newton's third law of motion},
    description={whenever one body exerts a force on a second body, the first body experiences a force that is equal in magnitude and opposite in direction to the force that the first body exerts}
}

%...UNIT 3: ACCELERATION

\newglossaryentry{acceleration}{
    name={acceleration},
    description={a change in velocity over time}
}

\newglossaryentry{average acceleration}{
    name={average acceleration},
    description={change in velocity divided by the time interval over which it changed}
}

%...UNIT 4:


\newglossaryentry{impulse}{
    name={impulse},
    description={average net external force multiplied by the time the force acts; equal to the change in momentum}
}

\newglossaryentry{impulse-momentum theorem}{
    name={impulse-momentum theorem},
    description={the impulse equals change in momentum}
}

\newglossaryentry{work}{
    name={work},
    description={force multiplied by distance}
}

%...UNIT 5: FORCE ANALYSIS

\newglossaryentry{Newton's universal law of gravitation}{
    name={Newton's universal law of gravitation},
    description={states that gravitational force between two objects is directly proportional to the product of their masses and inversely proportional to the square of the distance between them}
}

\newglossaryentry{gravitational constant}{
    name={gravitational constant},
    description={the proportionality constant in Newton's law of universal gravitation}
}

\newglossaryentry{weight}{
    name={weight},
    description={the force of gravity, $F_g$, acting on an object of mass $m$}
}

\newglossaryentry{contact force}{
    name={contact force},
    description={a type of force that occurs when objects are physically in contact with each other}
}

%...UNIT 6: ONE-DIMENSIONAL MOTION

\newglossaryentry{free fall}{
    name=free fall,
    description={a situation in which the only force acting on an object is the force of gravity}
}

\newglossaryentry{kinematic equations}{
    name={kinematic equations},
    description={the 
    %five 
    equations that describe constant acceleration motion in terms of time, displacement, velocity, and acceleration}
}

%...UNIT 7: MOTION IN TWO DIMENSIONS

\newglossaryentry{projectile}{
    name={projectile},
    description={an object that travels through the air and experiences only acceleration due to gravity}
}

\newglossaryentry{projectile motion}{
    name={projectile motion},
    description={the motion of an object that is subject only to the acceleration of gravity}
}

\newglossaryentry{trajectory}{
    name={trajectory},
    description={the path of a projectile through the air}
}

\newglossaryentry{apex}{
    name={apex},
    description={the location on the trajectory at which the projectile reaches maximum height}
}

\newglossaryentry{hang time}{
    name={hang time},
    description={the amount of time that a projectile is in the air during projectile motion}
}

\newglossaryentry{horizontally launched projectile}{
    name={horizontally launched projectile},
    description={a projectile whose initial velocity is entirely in the horizontal direction}
}

\newglossaryentry{impact speed}{
    name={impact speed},
    description={the speed at which a projectile strikes the ground after being launched}
}

%...UNIT 8: CONSERVATION IN MECHANICAL SYSTEMS

\newglossaryentry{system}{
    name={system},
    description={one or more objects of interest for which only the forces acting on them from the outside are considered, but not the forces acting between them or inside them}
}

\newglossaryentry{energy}{
    name={energy},
    description={the ability to do work}
}

\newglossaryentry{potential energy}{
    name={potential energy},
    description={stored energy}
}

\newglossaryentry{gravitational potential energy}{
    name={gravitational potential energy},
    description={energy acquired by doing work against gravity}
}

\newglossaryentry{law of conservation of energy}{
    name={law of conservation of energy},
    description={states that energy is neither created nor destroyed}
}

\newglossaryentry{mechanical energy}{
    name={mechanical energy},
    description={kinetic plus potential energy}
}

\newglossaryentry{elastic collision}{
    name={elastic collision},
    description={a collision in which objects separate after impact and kinetic energy is conserved}
}

\newglossaryentry{inelastic collision}{
    name={inelastic collision},
    description={a collision in which kinetic energy is not conserved}
}

\newglossaryentry{isolated system}{
    name={isolated system},
    description={system in which the net external force is zero}
}

\newglossaryentry{law of conservation of momentum}{
    name={law of conservation of momentum},
    description={when the net external force is zero, the total momentum of the system is conserved or constant}
}

\newglossaryentry{perfectly inelastic collision}{
    name={perfectly inelastic collision},
    description={collision in which objects stick together after impact and kinetic energy is not conserved}
}

\newglossaryentry{recoil}{
    name={recoil},
    description={backward movement of an object caused by the transfer of momentum from another object in a collision}
}

%...UNIT 9: CONSERVATION OF CHARGE

\newglossaryentry{electric charge}{
    name={electric charge},
    description={a physical property of an object that causes it to be attracted toward or repelled from another charged object; each charged object generates and is influenced by a force called an electromagnetic force}
}

\newglossaryentry{elementary charge}{
    name={elementary charge},
    description={the smallest observed unit of charge that can be isolated in nature; also, the magnitude of charge on 1 proton or 1 electron}
}

\newglossaryentry{electron}{
    name={electron},
    description={subatomic particle that carries one indivisible unit of negative electric charge}
}

\newglossaryentry{proton}{
    name={proton},
    description={subatomic particle that carries the same magnitude charge as the electron, but its charge is positive}
}

\newglossaryentry{electric field}{
    name={electric field},
    description={defines the force per unit charge at all locations in space around a charge distribution}
}

\newglossaryentry{law of conservation of charge}{
    name={law of conservation of charge},
    description={states that total charge is constant in any process}
}

\newglossaryentry{polarization}{
    name={polarization},
    description={separation of charge induced by nearby excess charge}
}

\newglossaryentry{Coulomb's law}{
    name={Coulomb's law},
    description={describes the electrostatic force between charged objects, which is proportional to the charge on each object and inversely proportional to the square of the distance between the objects}
}

\newglossaryentry{electric circuit}{
    name={electric circuit},
    description={physical network of paths through which electric current can flow}
}

\newglossaryentry{simple circuit}{
    name={simple circuit},
    description={a circuit with a single voltage source and a single resistor}
}

\newglossaryentry{electric current}{
    name={electric current},
    description={electric charge that is moving}
}


\newglossaryentry{Ohm's law}{
    name={Ohm's law},
    description={electric current is proportional to the voltage applied across a circuit or other path}
}

\newglossaryentry{resistance}{
    name={resistance},
    description={how much a circuit element opposes the passage of electric current; it appears as the constant of proportionality in Ohm’s law}
}

\newglossaryentry{resistor}{
    name={resistor},
    description={circuit element that provides a known resistance}
}

% \newglossaryentry{potential difference (or voltage)}{
%     name={potential difference (or voltage)},
%     description={change in potential energy of a charge moved from one point to another, divided by the charge; units of potential difference are joules per coulomb, known as volt}
% }

\newglossaryentry{voltage}{
    name={voltage},
    description={the electrical potential energy per unit charge; electric pressure created by a power source, such as a battery}
}

\newglossaryentry{electric power}{
    name={electric power},
    description={rate at which electric energy is transferred in a circuit}
}

\newglossaryentry{equivalent resistor}{
    name={equivalent resistor},
    description={resistance of a single resistor that is the same as the combined resistance of a group of resistors}
}

\newglossaryentry{in series}{
    name={in series},
    description={when elements in a circuit are connected one after the other in the same branch of the circuit}
}

\newglossaryentry{in parallel}{
    name={in parallel},
    description={when a group of resistors are connected side by side, with the top ends of the resistors connected together by a wire and the bottom ends connected together by a different wire}
}

\newglossaryentry{induction}{
    name={induction},
    description={creating an unbalanced charge distribution in an object by moving a charged object toward it (but without touching)}
}

%...UNIT 10: ELECTROMAGNETIC INDUCTION

\newglossaryentry{magnetic dipole}{
    name={magnetic dipole},
    description={term that describes magnets because they always have two poles: north and south}
}

\newglossaryentry{magnetic field}{
    name={magnetic field},
    description={directional lines around a magnetic material that indicates the direction and magnitude of the magnetic force}
}

\newglossaryentry{magnetic pole}{
    name={magnetic pole},
    description={part of a magnet that exerts the strongest force on other magnets or magnetic material}
}

\newglossaryentry{electromagnetic induction}{
    name={electromagnetic induction},
    description={rate at which energy is drawn from a source per unit current flowing through a circuit}
}


\newglossaryentry{Faraday's law}{
    name={Faraday's law},
    description={the means of calculating the emf in a coil due to changing magnetic flux}
}

\newglossaryentry{electromagnet}{
    name={electromagnet},
    description={device that uses electric current to make a magnetic field}
}

\newglossaryentry{transformer}{
    name={transformer},
    description={device that transforms voltages from one value to another}
}

\newglossaryentry{electric motor}{
    name={electric motor},
    description={device that transforms electrical energy into mechanical energy}
}

\newglossaryentry{generator}{
    name={generator},
    description={device that transforms mechanical energy into electrical energy}
}

%...UNIT 11: SIMPLE HARMONIC MOTION & WAVES

\newglossaryentry{wave}{
    name={wave},
    description={a disturbance that moves from its source and carries energy}
}

\newglossaryentry{wave velocity}{
    name={wave velocity},
    description={speed at which the disturbance moves; also called the propagation velocity or propagation speed}
}

\newglossaryentry{wavelength}{
    name={wavelength},
    description={distance between adjacent identical parts of a wave}
}

\newglossaryentry{wave cycle}{
    name={wave cycle},
    description={any portion of a wave encompassed by 1 wavelength}
}

\newglossaryentry{transverse wave}{
    name={transverse wave},
    description={a wave in which the disturbance is perpendicular to the direction of propagation}
}

\newglossaryentry{medium}{
    name={medium},
    description={the solid, liquid, or gas material through which a wave propagates}
}

\newglossaryentry{mechanical wave}{
    name={mechanical wave},
    description={wave that requires a medium through which it can travel}
}

\newglossaryentry{longitudinal wave}{
    name={longitudinal wave},
    description={wave in which the disturbance is parallel to the direction of propagation}
}

\newglossaryentry{constructive interference}{
    name={constructive interference},
    description={when two waves arrive at the same point exactly in phase; that is, the crests of the two waves are precisely aligned, as are the troughs}
}

\newglossaryentry{destructive interference}{
    name={destructive interference},
    description={when two identical waves arrive at the same point exactly out of phase that is precisely aligned crest to trough}
}

\newglossaryentry{oscillate}{
    name={oscillate},
    description={to move back and forth regularly between two points}
}

\newglossaryentry{amplitude}{
    name={amplitude},
    description={the maximum displacement from the equilibrium position of an object oscillating around the equilibrium position}
}

\newglossaryentry{frequency}{
    name={frequency},
    description={number of wave cycles per unit of time}
}

\newglossaryentry{simple harmonic motion}{
    name={simple harmonic motion},
    description={the oscillatory motion in a system where the net force can be described by Hooke’s law}
}

\newglossaryentry{simple harmonic oscillator}{
    name={simple harmonic oscillator},
    description={a device that oscillates in SHM,  such as a mass that is attached to a spring, where the restoring force is proportional to the displacement and acts in the direction opposite to the displacement}
}

\newglossaryentry{period}{
    name={period},
    description={the time it takes to complete one oscillation}
}

\newglossaryentry{electromagnetic wave}{
    name={electromagnetic wave},
    description={a radiant energy wave that consists of oscillating electric and magnetic fields}
}

\newglossaryentry{electromagnetic radiation}{
    name={electromagnetic radiation},
    description={radiant energy that consists of oscillating electric and magnetic fields}
}









%... Overview and Student Learning Expectations (OSLE)

\newglossaryentry{OSLE 6.1.a}{
    name={OSLE 6.1.a},
    description={compare the gravitational field strength on Earth to the acceleration due to gravity on Earth}
}

\newglossaryentry{OSLE 6.1.b}{
    name={OSLE 6.1.b},
    description={explain using universal gravitation and $F_\mathrm{net}=ma$ why all objects near Earth's surface fall at the same rate when in free fall}
}

\newglossaryentry{OSLE 6.1.c}{
    name={OSLE 6.1.c},
    description={explain the relationship between the mass, initial position, and initial velocity of an object in free fall on its final velocity and/or time in free fall}
}

\newglossaryentry{OSLE 6.1.d}{
    name={OSLE 6.1.d},
    description={describe the displacement, velocity, momentum, kinetic energy, and acceleration of an object in free fall that was dropped, thrown upward, or thrown downward using Multiple Representations}
}

\newglossaryentry{OSLE 6.1.e}{
    name={OSLE 6.1.e},
    description={relate the gravitational force, impulse, and work done on the object by the Earth to the object's change in velocity (acceleration), momentum, and kinetic energy}
}

        
\newglossaryentry{OSLE 6.2.a}{
    name={OSLE 6.2.a},
    description={describe what is known about an object's motion in a constant acceleration word problem using Multiple Representations}
}

\newglossaryentry{OSLE 6.2.b}{
    name={OSLE 6.2.b},
    description={solve for various unknown quantities utilizing kinematic equations when data is given in Multiple Representations for objects moving horizontally with constant acceleration}
}

\newglossaryentry{OSLE 6.3.c}{
    name={OSLE 6.3.c},
    description={solve for various unknown quantities utilizing kinematic equations when data is given in Multiple Representations for objects moving vertically with constant acceleration (free fall)}
}

\newglossaryentry{OSLE 6.4.d}{
    name={OSLE 6.4.d},
    description={solve multi-step problems that connect kinematic equations, the Law of Acceleration, Work-Energy Theorem, and/or the Impulse-Momentum Theorem}
}

\newglossaryentry{OSLE 7.1.a}{
    name={OSLE 7.1.a},
    description={compare the trajectory, hang time, max height, range, and final velocity of various projectiles that have different initial velocities, launch heights, launch angles, and masses, only varying one parameter at a time}
}

\newglossaryentry{OSLE 7.1.b}{
    name={OSLE 7.1.b},
    description={identify if and explain how the  initial velocity, launch height, launch angle, and mass of a projectile influence its motion---hang time, height, range, final velocity}
}

\newglossaryentry{OSLE 7.2.a}{
    name={OSLE 7.2.a},
    description={describe the vertical and horizontal motion of a projectile with a launch angle of zero using Multiple Representations}
}

\newglossaryentry{OSLE 7.2.b}{
    name={OSLE 7.2.b},
    description={illustrate the resultant motion of the projectile at any point in its trajectory as well as the relationship between the horizontal and vertical components using vector addition}
}

\newglossaryentry{OSLE 7.2.c}{
    name={OSLE 7.2.c},
    description={analyze and solve word problems about the motion of  horizontally launched projectiles using kinematic equations, vector addition, and  Multiple Representations}
}

\newglossaryentry{OSLE 7.3.a}{
    name={OSLE 7.3.a},
    description={describe the motion of an object moving with uniform circular motion in terms of centripetal force, centripetal acceleration, momentum, kinetic energy, and tangential velocity using Multiple Representations}
}

\newglossaryentry{OSLE 7.3.b}{
    name={OSLE 7.3.b},
    description={determine the centripetal force, mass, centripetal acceleration, tangential velocity, or radius of an object in circular motion}
}

\newglossaryentry{OSLE 7.4.a}{
    name={OSLE 7.4.a},
    description={predict the effects of changing the radius or mass of objects in orbiting systems using concepts of uniform circular motion and Newton’s law of universal gravitation}
}


\newglossaryentry{OSLE 8.1.a}{
    name={OSLE 8.1.a},
    description={identify multiple choices for a system given a scenario}
}

\newglossaryentry{OSLE 8.1.b}{
    name={OSLE 8.1.b},
    description={recognize that energy can be stored in the arrangement of particles or objects in a system as potential energy}
}

\newglossaryentry{OSLE 8.1.c}{
    name={OSLE 8.1.c},
    description={identify and calculate (i) gravitational potential energy and (ii) elastic potential energy when a system includes energy stored in the arrangement of its particles or objects}
}

\newglossaryentry{OSLE 8.1.d}{
    name={OSLE 8.1.d},
    description={compare the potential energy of a scenario for various choices of system}
}

\newglossaryentry{OSLE 8.1.e}{
    name={OSLE 8.1.e},
    description={identify, represent using multiple representations, and calculate the total mechanical energy present in a physical system}
}

\newglossaryentry{OSLE 8.1.f}{
    name={OSLE 8.1.f},
    description={predict the effects of changing the mass, velocity, height, gravitational field strength, spring constant, compression or stretching distance on the amount of $E_k$, $E_\mathrm{GP}$, and $E_\mathrm{SP}$}
}

\newglossaryentry{OSLE 8.1.g}{
    name={OSLE 8.1.g},
    description={calculate the total mechanical energy of a system}
}

\newglossaryentry{OSLE 8.2.a.i}{
    name={OSLE 8.2.a.i},
    description={identify, represent using multiple representations, and calculate the amount of energy (1) transformed from one storage mode to another within a system (including kinetic energy, potential energy, and thermal energy), (2) transferred from one object in the system to another in the system, and (3) entering/leaving a system due to work, heat, light, or sound}
}

\newglossaryentry{OSLE 8.2.b.i}{
    name={OSLE 8.2.b.i},
    description={explain the meaning of the Law of Conservation of Energy}
}

\newglossaryentry{OSLE 8.2.b.ii}{
    name={OSLE 8.2.b.ii},
    description={develop an energy formula for systems using energy bar charts and the Law of Conservation of Energy}
}

\newglossaryentry{OSLE 8.2.b.iii}{
    name={OSLE 8.2.b.iii},
    description=solve for various unknown quantities using the concept of the conservation of energy{}
}

\newglossaryentry{OSLE 8.2.c.i}{
    name={OSLE 8.2.c.i},
    description={know the definition of work as change in energy of a system}
}

\newglossaryentry{OSLE 8.2.c.ii}{
    name={OSLE 8.2.c.ii},
    description={know that power is work done divided by time}
}

\newglossaryentry{OSLE 8.3.a}{
    name={OSLE 8.3.a},
    description={calculate and compare the momentum, changes in momentum, force applied to and impulse on each object involved in a collision or explosion scenario}
}

\newglossaryentry{OSLE 8.3.b}{
    name={OSLE 8.3.b},
    description={represent using multiple representations, compare, and calculate the total momentum of a system before and after a collision or explosion scenario}
}

\newglossaryentry{OSLE 8.3.c}{
    name={OSLE 8.3.c},
    description={explain the meaning of the Law of Conservation of Momentum}
}

\newglossaryentry{OSLE 8.3.d}{
    name={OSLE 8.3.d},
    description={solve for unknown quantities using the concept of the conservation of momentum}
}


\newglossaryentry{OSLE 9.1.a}{
    name={OSLE 9.1.a},
    description={identify the particles that contribute positive, negative, or no charge in an atom}
}

\newglossaryentry{OSLE 9.1.b}{
    name={OSLE 9.1.b},
    description={recognize that neutral objects have even numbers of positive and negative charges}
}

\newglossaryentry{OSLE 9.1.c}{
    name={OSLE 9.1.c},
    description={determine the charge of an object given the number of protons and electrons}
}

\newglossaryentry{OSLE 9.1.d}{
    name={OSLE 9.1.d},
    description={predict if two objects will attract, repel, or have no interaction based on their charges}
}

\newglossaryentry{OSLE 9.1.e}{
    name={OSLE 9.1.e},
    description={draw the electric field surrounding single charges and pairs of charges}
}

\newglossaryentry{OSLE 9.2.a}{
    name={OSLE 9.2.a},
    description={recognize that charge is conserved: it cannot be created or destroyed, only transferred}
}

\newglossaryentry{OSLE 9.2.b}{
    name={OSLE 9.2.b},
    description={realize that only electrons are transferred during charging}
}

\newglossaryentry{OSLE 9.2.c}{
    name={OSLE 9.2.c},
    description={compare and contrast charging by induction and conduction}
}

\newglossaryentry{OSLE 9.2.d}{
    name={OSLE 9.2.d},
    description={explain how polarization temporarily charges a neutral object}
}

\newglossaryentry{OSLE 9.2.e}{
    name={OSLE 9.2.e},
    description={describe how an electroscope determines if objects are charged}
}

\newglossaryentry{OSLE 9.2.f}{
    name={OSLE 9.2.f},
    description={determine whether an object is negatively charged, positively charged, or neutral when given the charge of one object and a description or diagram representing how the charged object interacts with an object of unknown charge}
}

\newglossaryentry{OSLE 9.2.g}{
    name={OSLE 9.2.g},
    description={draw and describe the resulting distribution of charge for various scenarios of induction, conduction, and polarization}
}


\newglossaryentry{OSLE 9.3.a}{
    name={OSLE 9.3.a},
    description={draw the free body diagram for 2 charged objects showing the direction and relative magnitude of the electrical force acting on each object at various distances from each other}
}

\newglossaryentry{OSLE 9.3.b}{
    name={OSLE 9.3.b},
    description={describe how the electric force depends on the charges and the distance between them}
}

\newglossaryentry{OSLE 9.3.c}{
    name={OSLE 9.3.c},
    description={compare and contrast the electric force to the gravitational force}
}

\newglossaryentry{OSLE 9.3.d}{
    name={OSLE 9.3.d},
    description={predict how changing the charge or distance affects the electric force}
}


\newglossaryentry{OSLE 9.4.a}{
    name={OSLE 9.4.a},
    description={identify the necessary components for a simple circuit and discover different ways to light a bulb}
}

\newglossaryentry{OSLE 9.4.b}{
    name={OSLE 9.4.b},
    description={trace the conducting path through a simple circuit}
}

\newglossaryentry{OSLE 9.4.c}{
    name={OSLE 9.4.c},
    description={explain the concepts of current, resistance, voltage}
}

\newglossaryentry{OSLE 9.4.d}{
    name={OSLE 9.4.d},
    description={measure the current, resistance and voltage in a circuit using a multimeter, ammeter, current probe, etc}
}

\newglossaryentry{OSLE 9.4.e}{
    name={OSLE 9.4.e},
    description={calculate the voltage drop across, current through, or resistance of a circuit component using Ohm’s Law}
}

\newglossaryentry{OSLE 9.4.f}{
    name={OSLE 9.4.f},
    description={determine the change in current as the voltage or resistance is changed}
}

\newglossaryentry{OSLE 9.4.g}{
    name={OSLE 9.4.g},
    description={interpret electrical power as the rate at which electrical energy is being dissipated in the circuit}
}

\newglossaryentry{OSLE 9.4.h}{
    name={OSLE 9.4.h},
    description={relate the power rating/wattage of a light bulb to its brightness}
}


\newglossaryentry{OSLE 9.5.a}{
    name={OSLE 9.5.a},
    description={measure the current, resistance and voltage at various locations in a series circuit using a multimeter, ammeter, current probe, etc}
}

\newglossaryentry{OSLE 9.5.b}{
    name={OSLE 9.5.b},
    description={describe qualitatively and quantitatively the current flow throughout a series circuit}
}

\newglossaryentry{OSLE 9.5.c}{
    name={OSLE 9.5.c},
    description={calculate the equivalent resistance of multiple resistors in series}
}

\newglossaryentry{OSLE 9.5.d}{
    name={OSLE 9.5.d},
    description={calculate the equivalent voltage of batteries in series}
}

\newglossaryentry{OSLE 9.5.e}{
    name={OSLE 9.5.e},
    description={recognize that the sum of the voltage drops across resistors in series equals the total voltage of the power supply}
}

\newglossaryentry{OSLE 9.5.f}{
    name={OSLE 9.5.f},
    description={describe the energy transformations (transfers) occurring in a series circuit}
}

\newglossaryentry{OSLE 9.5.g}{
    name={OSLE 9.5.g},
    description={determine (i) current through, voltage drop across, and power of each component, and (i) total current of circuit, when given a series circuit diagram}
}


\newglossaryentry{OSLE 9.6.a}{
    name={OSLE 9.6.a},
    description={measure the current, resistance and voltage at various locations in a parallel circuit using a multimeter, ammeter, current probe, etc}
}

\newglossaryentry{OSLE 9.6.b}{
    name={OSLE 9.6.b},
    description={recognize that the current going into a junction is equal to the current coming out of it}
}

\newglossaryentry{OSLE 9.6.c}{
    name={OSLE 9.6.c},
    description={describe qualitatively and quantitatively the current flow throughout a parallel circuit}
}

\newglossaryentry{OSLE 9.6.d}{
    name={OSLE 9.6.d},
    description={recognize that the voltage drops across each resistor are equal to the voltage of the power supply}
}

\newglossaryentry{OSLE 9.6.e}{
    name={OSLE 9.6.e},
    description={describe advantages and disadvantages of parallel circuits compared to series circuits}
}

\newglossaryentry{OSLE 9.6.f}{
    name={OSLE 9.6.f},
    description={calculate the equivalent resistance of multiple resistors in parallel}
}

\newglossaryentry{OSLE 9.6.g}{
    name={OSLE 9.6.g},
    description={calculate the equivalent voltage of batteries in parallel}
}

\newglossaryentry{OSLE 9.6.h}{
    name={OSLE 9.6.h},
    description={describe the energy transformations (transfers) occurring in a parallel circuit}
}

\newglossaryentry{OSLE 9.6.i}{
    name={OSLE 9.6.i},
    description={determine the (i) current through, voltage drop across, and power of each component, and (ii) total current of circuit, when given a series circuit diagram}
}


\newglossaryentry{OSLE 9.7.a}{
    name={OSLE 9.7.a},
    description={determine whether elements of a combination circuit have the same current or voltage}
}

\newglossaryentry{OSLE 9.7.b}{
    name={OSLE 9.7.b},
    description={predict which bulbs will light if switches are open or closed}
}







% \setlength{\parindent}{0pt}
% \theoremstyle{definition}
% \newtheorem{example}{Example}

% \setrandomizerseed{1}

% \firstpageheader{Physics}{Unit 5: Force Analysis}{}
% \runningheader{Physics}{}{Unit 5}


\documentclass[../main-physics-problems.tex]{subfiles}
\begin{document}


\subsection{Newton’s Law of Universal Gravitation}
\subsubsection{The Law of Gravitation}
\subsubsection{Comparing Acceleration of Two Masses using Gravitation}
\subsubsection{Ranking the Gravitational Force at Various Distances}
\subsubsection{Ranking the Gravitational Force of Various Masses}
\subsubsection{The Gravitational Field}
\subsubsection{Calculating Force using the Law of Gravitation}
\subsubsection{The Gravitational Field Strength of Earth, g}
\subsubsection{Mass and Weight}
\subsubsection{Calculating Force using Mass and Gravitational Field Strength}
\subsection{Forces from Surfaces}
\subsubsection{Normal Force and Frictional Force}
\subsubsection{Why the Normal and Gravitational Forces are not Force Pairs}
\subsubsection{How Motion, Surfaces, and the Normal Force affect the Frictional Force}
\subsection{Applied Forces}
\subsubsection{Tension Force}
\subsubsection{Identifying other Applied Forces}
\subsection{Analyzing Forces using Motion}
\subsubsection{Free Body Diagrams from Word Problems}
\subsubsection{Net Resultant Force using Vector Addition}
\subsubsection{Solving for Unknown Force, Mass, or Acceleration}
\subsubsection{Solving for Other Unknowns}

\begin{questions}
\question 
The table below is based on the equation for Newton's universal law of gravitation. Some information is omitted from the table. Complete the table by filling in the missing information.

\begin{center}
    \begin{tabular}{cl|cl}
    \hline
    \textbf{Symbol} & \textbf{Quantity} & \textbf{Base SI Unit} & \textbf{Unit Symbol}  \\
    \hline\hline
    \rule{0pt}{2.5ex}
        $F_g$ & \ifprintanswers \color{red} \else \color{white} \fi gravitational force between two objects & newton & N\\
        $G$ & \ifprintanswers \color{red} \else \color{white} \fi gravitational constant & (\textit{see} $\rightarrow$) & \SI{}{N \cdot m^2/kg^2}\\
        $m_1$ & mass of object 1 & \ifprintanswers \color{red} \else \color{white} \fi kilogram & \ifprintanswers \color{red} \else \color{white} \fi kg\\
        $m_2$ & \ifprintanswers \color{red} \else \color{white} \fi mass of object 2 & kilogram & kg\\
        $r$ & distance between centers of mass & \ifprintanswers \color{red} \else \color{white} \fi meter & \ifprintanswers \color{red} \else \color{white} \fi m\\
    \hline
    \end{tabular}
\end{center}

\question
A \SI{90.0}{kg} person stands \SI{1.00}{m} from a \SI{60.0}{kg} person sitting on a bench nearby. What is the magnitude of the gravitational force between them?

\begin{randomizechoices}
    \correctchoice \SI{3.6e-7}{N}
    \choice \SI{5.4e-7}{N}
    \choice \SI{6.0e-9}{N}
    \choice \SI{4.0e-9}{N}
\end{randomizechoices}

\question
A planet of mass $M$ has a satellite of mass $m$, located a distance $d$ away, attracted by a force $F$. If the mass of the satellite is doubled while $d$ and $M$ remain unchanged, by what factor will $F$ change? 

\begin{randomizechoices}
    \correctchoice 2
    \choice $\frac{1}{2}$
    \choice 4
    \choice $\frac{1}{4}$
\end{randomizechoices}

\question
A planet of mass $M$ has a satellite of mass $m$, located a distance $d$ away, attracted by a force $F$. If the distance to the satellite is cut in half while $M$ and $m$ remain unchanged, by what factor will $F$ change?

\begin{randomizechoices}
    \choice 2
    \choice $\frac{1}{2}$
    \correctchoice 4
    \choice $\frac{1}{4}$
\end{randomizechoices}

\question
A planet of mass $M$ has a satellite of mass $m$, located a distance $d$ away, attracted by a force $F$. The mass of the satellite and the planet are both doubled while distance apart is quadrupled, and the new force of attraction is $F^\prime$. What is the ratio of force $F^\prime$ to the old force $F$?

\begin{randomizechoices}
    \choice 2
    \choice $\frac{1}{2}$
    \choice 4
    \correctchoice $\frac{1}{4}$
\end{randomizechoices}

\question
The masses of Earth and the moon are \SI{5.98e24}{kg} and \SI{7.35e22}{kg}, and the distance separating their centers of mass is \SI{3.84e8}{m}. Calculate the gravitational force of attraction between the Moon and the Earth.

\begin{randomizechoices}
    \correctchoice \SI{1.99e20}{N}
    \choice \SI{7.63e28}{N}
    \choice \SI{2.98e30}{N}
    \choice \SI{1.14e39}{N}
\end{randomizechoices}

\question
The gravitational force of attraction between two objects is \SI{18}{N}. If the mass of one object doubles, the new force of attraction is

\begin{randomizechoices}
    \correctchoice \SI{36}{N}
    \choice \SI{9}{N}
    \choice \SI{2}{N}
    \choice \SI{24}{N}
\end{randomizechoices}

\question
The gravitational force of attraction between two objects is \SI{520}{N}. If the distance between the two objects doubles, the new force of attraction is

\begin{randomizechoices}
    \correctchoice \SI{130}{N}
    \choice \SI{260}{N}
    \choice \SI{2080}{N}
    \choice \SI{1040}{N}
\end{randomizechoices}

\question
The gravitational force of attraction between two objects is \SI{26}{N}. If the mass of one object is halved, the new force of attraction is

\begin{randomizechoices}
    \correctchoice \SI{13}{N}
    \choice \SI{52}{N}
    \choice \SI{2}{N}
    \choice \SI{4}{N}
\end{randomizechoices}

\question
The gravitational force of attraction between two objects is \SI{72}{N}. If the distance between the objects triples, the new force of attraction is

\begin{randomizechoices}
    \correctchoice \SI{8}{N}
    \choice \SI{24}{N}
    \choice \SI{216}{N}
    \choice \SI{648}{N}
\end{randomizechoices}

\question
The gravitational force of attraction between two objects is \SI{10}{N}. If the distance between the objects is halved, the new force of attraction is

\begin{randomizechoices}
    \correctchoice \SI{40}{N}
    \choice \SI{20}{N}
    \choice \SI{5}{N}
    \choice \SI{2.5}{N}
\end{randomizechoices}

\question
What is the force of attraction between two friends Phineas, whose mass is \SI{100.0}{kg}, and Ferb, whose mass is 80.0 kg, when they are \SI{2.00}{m} apart? 

\begin{randomizechoices}
    \correctchoice \SI{1.33e-7}{N}
    \choice \SI{2.00e3}{N}
    \choice \SI{4.00e3}{N}
    \choice \SI{2.67e-7}{N}
\end{randomizechoices}

\question
If the magnitude of the gravitational force of Po, from Kung Fu Panda, on Mantis is \SI{8.92e-11}{N} when they are 0.25 meters apart while sparring, then the magnitude of the gravitational force of Mantis on Po is

\begin{randomizechoices}[keeplast]
    \correctchoice equal to \SI{8.92e-11}{N}
    \choice greater than \SI{8.92e-11}{N}
    \choice less than \SI{8.92e-11}{N}
    \choice no possible to determine without the masses of Mantis and Po
\end{randomizechoices}    
\end{questions}


\clearpage


\subsubsection*{Exercises}
\begin{questions}
\question
The same experiment that you just conducted in your classroom was done on several other planets. 

\begin{center}
\begin{tikzpicture}
    \begin{axis}[height=6cm,width=9cm,
        axis lines=left,
        ylabel={Weight (N)},
        xlabel={Mass (kg)},
        ymin=0,ymax=140,
        xmin=0,xmax=6,
        ytick={0,20,...,140},
        xtick={0,1,...,6},
        grid=both,
        minor x tick num=1,
    ]
        \addplot[very thick,mark=*] coordinates{(0.5,11.6)(2,46.2)(2.75,63.5)(3,69.3)(3.5,80.9)(5,115.5)} node[above] {Jupiter};
        \addplot[very thick,mark=*] coordinates{(0.5,5.5)(2,22)(2.75,30.3)(3,33)(3.5,38.5)(5,55)} node[above] {Neptune};
        \addplot[very thick,mark=*] coordinates{(0.5,1.9)(2,7.4)(2.75,10.2)(3,11.1)(3.5,13)(5,18.5)} node[above] {Mars};
    \end{axis}
\end{tikzpicture}
\end{center}

\begin{parts}
\part Use the results on the graph to find the strength of the gravitational field near the surface of each planet.

\begin{solution}
Answers will vary due uncertainty in the coordinates.

The gravitational field strength is the slope on a weight vs mass graph.

For Jupiter,

\begin{equation*}
    g_\mathrm{J} = \frac{\SI{115}{N} - \SI{70}{N}}{\SI{5}{kg} - \SI{3}{kg}} = \SI{22.5}{N/kg}
\end{equation*}

For Neptune,

\begin{equation*}
    g_\mathrm{N} =  \frac{\SI{40}{N} - \SI{20}{N}}{\SI{3.5}{kg} - \SI{2}{kg}} = \SI{13.3}{N/kg}
\end{equation*}

For Mars,

\begin{equation*}
     g_\mathrm{M} = \frac{\SI{19}{N} - \SI{8}{N}}{\SI{5}{kg} - \SI{2}{kg}} = \SI{3.7}{N/kg}
\end{equation*}

Note: The \href{https://nssdc.gsfc.nasa.gov/planetary/factsheet/}{actual values} are $g_\mathrm{J} = \SI{23.1}{N/kg}$, $g_\mathrm{N} = \SI{11.0}{N/kg}$, $g_\mathrm{M} = \SI{3.7}{N/kg}$.
\end{solution}

\part Rank the strength of the gravitational field of these planets from least to greatest. Explain your ranking.
\end{parts}
 

\question
Match the following characteristics to either mass or weight.

\begin{parts}
    \part The amount of matter in an object: \fillin[mass]
    \part Measured in Newtons: \fillin[weight]
    \part The strength of the gravitational force acting on an object: \fillin[weight]
    \part Changes based on the strength of the gravitational field: \fillin[weight]
    \part Remains constant regardless of location: \fillin[mass]
    \part Measured in kilograms: \fillin[mass]
\end{parts}

\question
One of the objects from the previous question was a bowling ball with a mass of \SI{3}{kg}. If the gravitational field strength is \SI{11}{N/kg} on Neptune, what is the mass of the bowling ball on Neptune? Explain your answer.

\begin{solution}
Mass is constant regardless of location. So, the mass is still \SI{3}{kg}.
\end{solution}

\question
An object of unknown mass is on Earth and then brought to Jupiter. If Jupiter's gravitational field strength is 2.5 times greater than Earth's, how will the weight of the object compare on Jupiter to on Earth? Explain your answer.

\begin{solution}
    Weight is proportional to the stength of the gravitational field. Therefore, the object's weight will be 2.5 times greater on Jupiter.
\end{solution}


\question
Using Earth's mass $M = \SI{5.97e24}{kg}$ and radius $R = \SI{6378}{km}$, calculate the acceleration due to gravity on Earth's surface, $g$.

\begin{solution}
\begin{equation*}
    g = \frac{GM}{R^2} = \frac{(\SI{6.67e-11}{N\cdot m^2/kg^2})(\SI{5.97e24}{kg})}{\left(\SI{6.378e6}{m}\right)^2} = \boxed{\SI{9.78}{N/kg}}
\end{equation*}
\end{solution}
\end{questions}

\clearpage




\clearpage
\subsubsection*{Free Body Diagram Whiteboarding}

Draw a free-body diagram for each scenario below. Each diagram should include:

\begin{itemize}[itemsep=0pt,topsep=0pt]
    \item system dot
    \item vector arrows representing magnitude of all forces involved on system object
    \item force type labels on each vector
\end{itemize}

\begin{questions}

% \ifprintanswers
% \else
\setlength\columnsep{50pt}
\begin{multicols*}{2}
% \fi

\question
An object lies motionless.

\begin{center}
\begin{tikzpicture}[x=7mm,y=7mm]
    \draw (0,0) -- (4,0);
    \draw[fill=black!10] (1.5,0) rectangle ++(1,1);
    \ifprintanswers
    \color{red}
    \else
    \color{white}
    \fi
    \begin{scope}[xshift=4cm,x=1cm,y=1cm]
        \draw[thick,->] (0,0) -- (0,+1) node[above] {$F_\mathrm{N}$};
        \draw[thick,->] (0,0) -- (0,-1) node[below] {$F_\mathrm{g}$};
        \fill[black] (0,0) circle (3pt);
    \end{scope}
\end{tikzpicture}
\end{center}

\question
Object slides at constant speed without friction.

\begin{center}
\begin{tikzpicture}[x=7mm,y=7mm]
    \draw (0,0) -- (4,0);
    \draw[fill=black!10] (1.5,0) rectangle ++(1,1);
    \draw[thick,->,yshift=2mm] (1.5,1) -- ++(1,0) node[above,pos=0.5] {$v$};
    \ifprintanswers
    \color{red}
    \else
    \color{white}
    \fi
    \begin{scope}[xshift=4cm,x=1cm,y=1cm]
        \draw[thick,->] (0,0) -- (0,+1) node[above] {$F_\mathrm{N}$};
        \draw[thick,->] (0,0) -- (0,-1) node[below] {$F_\mathrm{g}$};
        \fill[black] (0,0) circle (3pt);
    \end{scope}
\end{tikzpicture}
\end{center}


\question
Object slows due to kinetic friction.

\begin{center}
\begin{tikzpicture}[x=7mm,y=7mm]
    \draw (0,0) -- (4,0);
    \draw[fill=black!10] (1.5,0) rectangle ++(1,1);
    \draw[thick,->,yshift=2mm] (1.5,1) -- ++(1,0) node[above,pos=0.5] {$v$};
    \ifprintanswers
    \color{red}
    \else
    \color{white}
    \fi
    \begin{scope}[xshift=5cm,x=1cm,y=1cm]
        \draw[thick,->] (0,0) -- (0,+1) node[above] {$F_\mathrm{N}$};
        \draw[thick,->] (0,0) -- (0,-1) node[below] {$F_\mathrm{g}$};
        \draw[thick,->] (0,0) -- (-1,0) node[left] {$F_f$};
        \fill[black] (0,0) circle (3pt);
    \end{scope}
\end{tikzpicture}
\end{center}

\question
The object is pulled upward at constant speed.

\begin{center}
\begin{tikzpicture}[x=7mm,y=7mm]
    \draw[fill=black!10] (0,0) rectangle ++(1,1);
    \draw[very thick] (0.5,1) -- ++(0,1.2);
    \draw[thick,->,xshift=2mm] (1,0) -- ++(0,1) node[right,pos=0.5] {$v$};
    \draw[white,thick,->,xshift=-2mm] (0,0) -- ++(0,1) node[left,pos=0.5] {$v$};
    \ifprintanswers
    \color{red}
    \else
    \color{white}
    \fi
    \begin{scope}[xshift=4cm,x=1cm,y=1cm]
        \draw[thick,->] (0,0) -- (0,+1) node[above] {$F_\mathrm{T}$};
        \draw[thick,->] (0,0) -- (0,-1) node[below] {$F_\mathrm{g}$};
        \fill[black] (0,0) circle (3pt);
    \end{scope}
\end{tikzpicture}
\end{center}

\columnbreak

\question
Static friction prevents sliding.

\begin{center}
\begin{tikzpicture}[x=7mm,y=7mm,rotate=-37]
    \draw (0,0) -- (5,0);
    \draw[fill=black!10] (2,0) rectangle ++(1,1);
    \ifprintanswers
    \color{red}
    \else
    \color{white}
    \fi
    \begin{scope}[xshift=5cm,yshift=2cm,x=1.5cm,y=1.5cm,rotate=37]
            \draw[thick,->] (0,0) -- (0,-1) node[below] {$F_g$};
            \begin{scope}[rotate=-37]
                \draw[thick,->] (0,0) -- (0,{cos(37)}) node[above] {$F_\mathrm{N}$};
                \draw[thick,->] (0,0) -- ({-sin(37)},0) node[left] {$F_f$};
            \end{scope}
            \fill[black] (0,0) circle (3pt);
    \end{scope}
\end{tikzpicture}
\end{center}

\question
An object is suspended from the ceiling.

\begin{center}
\begin{tikzpicture}[x=7mm,y=7mm]
    \draw[fill=black!10] (2,0) rectangle ++(1,1);
    \draw[very thick] (2.5,1) -- ++(0,1.2);
    \draw[pattern=north west lines, pattern color=black] (0,2.2) rectangle ++(5,0.2);
    \ifprintanswers
    \color{red}
    \else
    \color{white}
    \fi
    \begin{scope}[xshift=5cm,yshift=5mm,x=1cm,y=1cm]
        \draw[thick,->] (0,0) -- (0,+1) node[above] {$F_\mathrm{T}$};
        \draw[thick,->] (0,0) -- (0,-1) node[below] {$F_\mathrm{g}$};
        \fill[black] (0,0) circle (3pt);
    \end{scope}
\end{tikzpicture}
\end{center}


\question
An object is suspended from the ceiling.

\begin{center}
\begin{tikzpicture}[x=7mm,y=7mm]
    \draw[fill=black!10] (0,0) rectangle ++(1,1);
    \draw (0.5,1) -- ++({-2},{2*tan(30)});
    \draw (0.5,1) -- ++({+2},{2*tan(30)});
    \draw[pattern=north west lines, pattern color=black] (-2,2.18) rectangle ++(5,0.2);
    \ifprintanswers
    \color{red}
    \else
    \color{white}
    \fi
    \begin{scope}[xshift=4cm,yshift=5mm,x=1.2cm,y=1.2cm]
        \draw[thick,->] (0,0) -- (0,-1) node[below] {$F_\mathrm{g}$};
        \draw[thick,->] (0,0) -- ({-cos(30)},{sin(30)}) node[pos=1.3] {$F_\mathrm{T_1}$};
        \draw[thick,->] (0,0) -- ({+cos(30)},{sin(30)}) node[pos=1.3] {$F_\mathrm{T_2}$};
        \fill[black] (0,0) circle (3pt);
    \end{scope}
\end{tikzpicture}
\end{center}


\question
The object is motionless.

\begin{center}
\begin{tikzpicture}[x=7mm,y=7mm]
    \draw[fill=black!10] (0,0) rectangle ++(1,1);
    \draw (0.5,1) -- ++({-1},{tan(50)});
    \draw (1,0.5) -- ++(1,0);
    \draw[pattern=north west lines, pattern color=black] ({-tan(50)},{1+tan(50)}) rectangle ++(2,0.2);
    \draw[pattern=north west lines, pattern color=black] (2,1.2) rectangle ++(0.2,-1.4);
    \ifprintanswers
    \color{red}
    \else
    \color{white}
    \fi
    \begin{scope}[xshift=4cm,x=1.5cm,y=1.5cm]
        \draw[thick,->] (0,0) -- (0,-1) node[below] {$F_\mathrm{g}$};
        \draw[thick,->] (0,0) -- ({-cos(50)/sin(50)},1) node[pos=1.2] {$F_\mathrm{T_1}$};
        \draw[thick,->] (0,0) -- ({+cos(50)/sin(50)},0) node[right] {$F_\mathrm{T_2}$};
        \fill[black] (0,0) circle (3pt);
    \end{scope}
\end{tikzpicture}
\end{center}

\clearpage

\question
The object is accelerated rightward by an applied force parallel to the rough surface.

\begin{center}
\begin{tikzpicture}[x=7mm,y=7mm,remember picture]
    \draw (-1.5,0) -- ++(4,0);
    \draw[fill=black!10] (0,0) rectangle ++(1,1);
    \draw[thick,->] (1,0.5) -- ++(1.2,0) node[right] {$F$};
    \ifprintanswers
    \color{red}
    \else
    \color{white}
    \fi
    \begin{scope}[xshift=4cm,x=1cm,y=1cm]
        \draw[thick,->] (0,0) -- (0,+1) node[above] {$F_\mathrm{N}$};
        \draw[thick,->] (0,0) -- (0,-1) node[below] {$F_g$};
        \draw[thick,->] (0,0) -- (1.5,0) node[right] {$F$};
        \draw[thick,->] (0,0) -- (-0.7,0) node[left] {$F_f$};
        \fill[black] (0,0) circle (3pt);  
    \end{scope}
\end{tikzpicture}
\end{center}


\question
The object is pulled at constant velocity by a force at an angle to the surface with friction.

\begin{center}
\begin{tikzpicture}[x=7mm,y=7mm]
    \draw (-1.5,0) -- ++(4,0);
    \draw[fill=black!10] (0,0) rectangle ++(1,1);
    \draw[thick,->] (1,1) -- ++({1.4*cos(45)},{1.4*sin(45)}) node[right] {$F$};
    \draw[dashed] (1,1) -- ++({1.4*cos(45)},0);
    \node[xshift=4.4mm,yshift=2mm] at (1,1) {\small $\theta$};
    \ifprintanswers
    \color{red}
    \else
    \color{white}
    \fi
    \begin{scope}[xshift=4cm,x=1.5cm,y=1.5cm]
        \draw[thick,->] (0,0) -- (0,{1-0.8*sin(45)}) node[above] {$F_\mathrm{N}$};
        \draw[thick,->] (0,0) -- (0,-1) node[below] {$F_g$};
        \draw[thick,->] (0,0) -- ({0.8*cos(45)},{0.8*sin(45)}) node[right] {$F$};
        \draw[thick,->] (0,0) -- ({-0.8*cos(45)},0) node[left] {$F_f$};
        \fill[black] (0,0) circle (3pt);
    \end{scope}
\end{tikzpicture}
\end{center}


\question
The ball is rising in a parabolic trajectory with air resistance.

\begin{center}
\begin{tikzpicture}
    \begin{axis}[height=4cm,width=4cm,
        axis lines=none,
        ymin=0,ymax=15,
        xmin=0,xmax=15,]
        \addplot[domain=0:5.73,dashed] (x,{trajectoryequation(x,15,75)});
        \draw[fill=lightgray] (2.33,6.93) circle (4pt);
    \end{axis}
    \ifprintanswers
    \color{red}
    \else
    \color{white}
    \fi
    \begin{scope}[xshift=3cm,yshift=1.5cm,x=1.5cm,y=1.5cm]
        \draw[thick,->] (0,0) -- ({-0.6*cos(70)},{-0.6*sin(70)}) node[below left=-3pt] {$F_d$};
        \draw[thick,->] (0,0) -- (0,-1) node[below] {$F_\mathrm{g}$};
        \fill[black] (0,0) circle (3pt);
    \end{scope}
\end{tikzpicture}
\end{center}

\question
The ball is at the top of a parabolic trajectory with air resistance.

\begin{center}
\begin{tikzpicture}
    \begin{axis}[height=4cm,width=4cm,
        axis lines=none,
        ymin=0,ymax=15,
        xmin=0,xmax=15,]
        \addplot[domain=0:5.73,dashed] (x,{trajectoryequation(x,15,75)});
        \draw[fill=lightgray] (5.73,10.7) circle (4pt);
    \end{axis}
    \ifprintanswers
    \color{red}
    \else
    \color{white}
    \fi
    \begin{scope}[xshift=4cm,yshift=1.5cm,x=1.5cm,y=1.5cm]
        \draw[thick,->] (0,0) -- ({-0.5},{0}) node[below left=-3pt] {$F_d$};
        \draw[thick,->] (0,0) -- (0,-1) node[below] {$F_\mathrm{g}$};
        \fill[black] (0,0) circle (3pt);
    \end{scope}
\end{tikzpicture}
\end{center}

\question
The object is falling with no air resistance.

\begin{center}
\begin{tikzpicture}[x=7mm,y=7mm]
    \draw[fill=black!10] (2,0) rectangle ++(1,1);
    \ifprintanswers
    \color{red}
    \else
    \color{white}
    \fi
    \begin{scope}[xshift=4cm,,yshift=5mm,x=1.5cm,y=1.5cm]
        \draw[thick,->] (0,0) -- (0,-1) node[below] {$F_g$};
        \fill[black] (0,0) circle (3pt);
    \end{scope}
\end{tikzpicture}
\end{center}


\columnbreak

\question
The object is falling at constant (terminal) velocity.

\begin{center}
\begin{tikzpicture}[x=7mm,y=7mm]
    \draw[fill=black!10] (2,0) rectangle ++(1,1);
    \ifprintanswers
    \color{red}
    \else
    \color{white}
    \fi
    \begin{scope}[xshift=3.5cm,yshift=5mm,x=1.5cm,y=1.5cm]
        \draw[thick,->] (0,0) -- (0,-1) node[below] {$F_g$};
        \draw[thick,->] (0,0) -- (0,+1) node[above] {$F_d$};
        \fill[black] (0,0) circle (3pt);
    \end{scope}
\end{tikzpicture}
\end{center}


% \ifprintanswers
% \else
% \clearpage
% \fi

\question
(Bonus) The object is accelerated leftward by a force applied downward at an angle with friction.

\begin{center}
\begin{tikzpicture}[x=7mm,y=7mm]
    \draw (-1.5,0) -- ++(4,0);
    \draw[fill=black!10] (0,0) rectangle ++(1,1);
    \draw[thick,<-] (1,1) -- ++({1.4*cos(45)},{1.4*sin(45)}) node[right] {$F$};
    \draw[dashed] (1,1) -- ++({1.4*cos(45)},0);
    \node[xshift=4.4mm,yshift=2mm] at (1,1) {\small $\theta$};
    \ifprintanswers
    \color{red}
    \else
    \color{white}
    \fi
    \begin{scope}[xshift=4cm,x=1.2cm,y=1.2cm]
        \draw[thick,->] (0,0) -- (0,{1+0.8*sin(45)}) node[above] {$F_\mathrm{N}$};
        \draw[thick,->] (0,0) -- (0,-1) node[below] {$F_g$};
        \draw[thick,->] (0,0) -- ({-0.8*cos(45)},{-0.8*sin(45)}) node[left] {$F$};
        \draw[thick,->] (0,0) -- ({0.8*cos(45)},0) node[right] {$F_f$};
        \fill[black] (0,0) circle (3pt);
    \end{scope}
\end{tikzpicture}
\end{center}


\question
(Bonus) The object slides without friction.

\begin{center}
\begin{tikzpicture}[x=7mm,y=7mm,rotate=-37]
    \draw (0,0) -- (4,0);
    \draw[fill=black!10] (1.5,0) rectangle ++(1,1);
    \ifprintanswers
    \color{red}
    \else
    \color{white}
    \fi
    \begin{scope}[xshift=4cm,yshift=1.5cm,x=1.5cm,y=1.5cm,rotate=37]
        \draw[thick,->] (0,0) -- (0,-1) node[below] {$F_g$};
        \draw[thick,->,rotate=-37] (0,0) -- (0,{cos(37)}) node[above] {$F_\mathrm{N}$};
        \fill[black] (0,0) circle (3pt);
    \end{scope}
\end{tikzpicture}
\end{center}

\end{multicols*}
% \fi
\end{questions}

\clearpage

\subsubsection*{Exercises}
Instructions: You will be given a scenario, free-body diagram, and net force equations. Find the mistake, explain why it is wrong, and correct it. There will only be one mistake in each scenario.

\begin{questions}
\question
A person lifts a bag at a constant speed.

\begin{center}
\fbox{
\begin{minipage}[c][3.5cm][c]{0.2\textwidth}
\centering
\begin{tikzpicture}[x=1.2cm,y=1.2cm]
    \draw[thick,->] (0,0) -- (0,+1) node[above] {$F_a$};
    \draw[thick,->] (0,0) -- (0,+0.5) node[right] {$F_\mathrm{N}$};
    \draw[thick,->] (0,0) -- (0,-1) node[below] {$F_g$};
    \fill (0,0) circle (3pt);
\end{tikzpicture}
\end{minipage}}%
\fbox{
\begin{minipage}[c][3.5cm][c]{0.2\textwidth}
    \begin{equation*}
        F_a - F_g = 0
    \end{equation*}
\end{minipage}}%
\fbox{
\begin{minipage}[c][3.5cm][t]{0.2\textwidth}
\centering
    Why is it wrong?
\end{minipage}}%
\fbox{
\begin{minipage}[c][3.5cm][t]{0.2\textwidth}
\centering
    Corrected.
\end{minipage}}

\end{center}
\end{questions}

\clearpage

\subsubsection*{Linking Force to Motion}

\textbf{Scenario}: A string pulls horizontally on a cart so that it moves at increasing speed along a smooth, frictionless, horizontal surface. When the cart is moving medium-fast, the pulling is stopped abruptly.

\begin{questions}
\question
Describe in words what happens to the cart's motion when the pulling stops. Use proper physics vocabulary.

\ifprintanswers
\else
\fillwithlines{3cm}
\fi

\begin{solution}
    Due to inertia, the cart will continue its motion. It will be moving at a constant velocity due to the fact that the forces on it are balanced.
\end{solution}

\question
Using the cart as your system, draw a force schema diagram for the scenario before the pulling stops. Highlight the forces that are relevant to our system before the pulling stops. 

\begin{solutionorbox}[3cm]
    
\end{solutionorbox}


\question
Using the cart as your system, draw force diagrams for the following parts of the scenario. Write balanced or unbalanced next to each diagram.

\begin{parts}
\part while the string is pulling

\begin{solution}
\begin{center}
\begin{minipage}{0.3\textwidth}
\centering
\begin{tikzpicture}[x=8mm,y=8mm]
    \fill (0,0) circle (3pt);
    \draw[thick,->] (0,0) -- (2,0) node[right] {$F_\mathrm{T}$};
    \draw[thick,->] (0,0) -- (0,1) node[above] {$F_\mathrm{N}$};
    \draw[thick,->] (0,0) -- (0,-1) node[below] {$F_g$};
\end{tikzpicture}
\end{minipage}%
\begin{minipage}{0.3\textwidth}
\begin{align*}
    F_{\mathrm{net},x} &= F_\mathrm{T} = ma \\[1ex]
    F_{\mathrm{net},y} &= F_\mathrm{N} - F_g = 0
\end{align*}
\end{minipage}
\end{center}
\end{solution}

\part after the pulling stops

\begin{solution}
\begin{center}
\begin{minipage}{0.3\textwidth}
\centering
\begin{tikzpicture}[x=8mm,y=8mm]
    \draw[thick,->,white] (0,0) -- (2,0) node[right] {$F_\mathrm{T}$};
    \fill (0,0) circle (3pt);
    \draw[thick,->] (0,0) -- (0,1) node[above] {$F_\mathrm{N}$};
    \draw[thick,->] (0,0) -- (0,-1) node[below] {$F_g$};
\end{tikzpicture}
\end{minipage}%
\begin{minipage}{0.3\textwidth}
\begin{align*}
    F_{\mathrm{net},x} &= 0 \\[1ex]
    F_{\mathrm{net},y} &= F_\mathrm{N} - F_g = 0
\end{align*}
\end{minipage}
\end{center}
\end{solution}
\end{parts}
				
\question
Draw a motion map for the scenario. (This will just represent the motion, do not worry about using numbers or trying to calculate anything. Label where the pulling stops on your motion map.)

\begin{solutionorbox}[1cm]
\begin{center}
    \begin{tikzpicture}[x=5mm,y=5mm]
        \draw[domain=0:10,mark=*,only marks, samples=11] plot({0.1*\x^2},0);
        \draw[domain=0:5,mark=*,only marks, samples=6] plot({0.1*(10^2-9^2)*\x+10},0);
        \draw[<-,thick] (10,0.4) -- ++(0,2) node[above] {pulling stops};
    \end{tikzpicture}
\end{center}
\end{solutionorbox}

\question
On the given axes complete the graphs for the scenario. (Label where the pulling stops on your graphs.)

\begin{parts}
    \part Highlight on the position graph where kinetic energy is changing
    \part Highlight on the velocity graph where kinetic energy is the greatest.
    \part In the box under each graph, write a short description about what unbalanced forces would look like on that graph.    
\end{parts}

\begin{center}
\begin{tikzpicture}
    \begin{axis}[height=5cm,width=5.5cm,
        axis lines=left,
        ylabel={Position (m)},
        xlabel={Time (s)},
        ymin=0,ymax=10,
        xmin=0,xmax=10,
        grid=both,
        ytick={0,1,...,10},
        xtick={0,1,...,10},
        yticklabel=\empty,
        xticklabel=\empty,
    ]
        \ifprintanswers
        \addplot[red,very thick,domain=0:6] {0.1*x^2};
        \addplot[red,very thick,domain=6:10] {0.2*6*(x-6) + 0.1*6^2}; %..Calculus, derivative at t=6
        \draw[red,thick,<-,xshift=-2pt,yshift=2pt] (6,{0.1*6^2}) -- ++(-2,2) node[above] {pulling stops};
        \fi
    \end{axis}
\end{tikzpicture}
\hspace{1mm}
\begin{tikzpicture}
    \begin{axis}[height=5cm,width=5.5cm,
        axis y line=left,
        axis x line=center,
        ylabel={Velocity (m/s)},
        xlabel={Time (s)},
        ymin=-5,ymax=5,
        xmin=0,xmax=10,
        grid=both,
        ytick={-5,-4,...,5},
        xtick={0,1,...,10},
        yticklabel=\empty,
        xticklabel=\empty,
        x label style={at={(axis description cs: 0.5,-0.1)},anchor=north}
    ]
        \ifprintanswers
        \addplot[very thick,red] coordinates{(0,0)(6,3)(10,3)};
        \draw[red,thick,<-,xshift=2pt,yshift=-2pt] (6,3) -- (8,1);
        \fi
    \end{axis}
\end{tikzpicture}
\hspace{1mm}
\begin{tikzpicture}
    \begin{axis}[height=5cm,width=5.5cm,
        axis y line=left,
        axis x line=center,
        ylabel={Acceleration (\SI{}{m/s^2})},
        xlabel={Time (s)},
        ymin=-5,ymax=5,
        xmin=0,xmax=10,
        grid=both,
        ytick={-5,-4,...,5},
        xtick={0,1,...,10},
        yticklabel=\empty,
        xticklabel=\empty,
        x label style={at={(axis description cs: 0.5,-0.1)},anchor=north}
    ]
    \ifprintanswers
    \addplot[red,very thick] coordinates{(0,3)(6,3)};
    \addplot[red,dashed] coordinates{(6,3)(6,0)};
    \addplot[red,very thick] coordinates{(6,0)(10,0)};
    \draw[red,thick,<-,xshift=2pt,yshift=-2pt] (6,3) -- (8,1);
    \fi
    \end{axis}
\end{tikzpicture}
\end{center}
\end{questions}

\clearpage

\subsubsection*{Writing Net Force Equations}

\begin{questions}
\question
An object lies motionless.

\begin{center}
\begin{minipage}[c][4cm][c]{0.3\textwidth}
\begin{tikzpicture}[x=7mm,y=7mm]
    \draw (0,0) -- (4,0);
    \draw[fill=black!10] (1.5,0) rectangle ++(1,1);
    \ifprintanswers
    \color{red}
    \else
    \color{white}
    \fi
    \begin{scope}[xshift=4cm,x=1cm,y=1cm]
        \draw[thick,->] (0,0) -- (0,+1) node[above] {$F_\mathrm{N}$};
        \draw[thick,->] (0,0) -- (0,-1) node[below] {$F_\mathrm{g}$};
        \fill[black] (0,0) circle (3pt);
    \end{scope}
\end{tikzpicture}
\end{minipage}%
\begin{minipage}[c][4cm][c]{0.35\textwidth}
    \large
    \begin{align*}
        F_{\mathrm{net},x} &= {\ifprintanswers \color{red} 0 \fi} \\[1ex]
        F_{\mathrm{net},y} &= {\ifprintanswers \color{red} F_\mathrm{N} - F_g = 0 \fi}
    \end{align*}
\end{minipage}
\end{center}

\question
Object slides at constant speed without friction.

\begin{center}
\begin{minipage}[c][4cm][c]{0.3\textwidth}
\begin{tikzpicture}[x=7mm,y=7mm]
    \draw (0,0) -- (4,0);
    \draw[fill=black!10] (1.5,0) rectangle ++(1,1);
    \draw[thick,->,yshift=2mm] (1.5,1) -- ++(1,0) node[above,pos=0.5] {$v$};
    \ifprintanswers
    \color{red}
    \else
    \color{white}
    \fi
    \begin{scope}[xshift=4cm,x=1cm,y=1cm]
        \draw[thick,->] (0,0) -- (0,+1) node[above] {$F_\mathrm{N}$};
        \draw[thick,->] (0,0) -- (0,-1) node[below] {$F_\mathrm{g}$};
        \fill[black] (0,0) circle (3pt);
    \end{scope}
\end{tikzpicture}
\end{minipage}%
\begin{minipage}[c][4cm][c]{0.35\textwidth}
    \large
    \begin{align*}
        F_{\mathrm{net},x} &= {\ifprintanswers \color{red} 0 \fi} \\[1ex]
        F_{\mathrm{net},y} &= {\ifprintanswers \color{red} F_\mathrm{N} - F_g = 0 \fi}
    \end{align*}
\end{minipage}
\end{center}



\question
Object slows due to kinetic friction.

\begin{center}
\begin{minipage}[c][4cm][c]{0.35\textwidth}
\begin{tikzpicture}[x=7mm,y=7mm]
    \draw (0,0) -- (4,0);
    \draw[fill=black!10] (1.5,0) rectangle ++(1,1);
    \draw[thick,->,yshift=2mm] (1.5,1) -- ++(1,0) node[above,pos=0.5] {$v$};
    \ifprintanswers
    \color{red}
    \else
    \color{white}
    \fi
    \begin{scope}[xshift=5cm,x=1cm,y=1cm]
        \draw[thick,->] (0,0) -- (0,+1) node[above] {$F_\mathrm{N}$};
        \draw[thick,->] (0,0) -- (0,-1) node[below] {$F_\mathrm{g}$};
        \draw[thick,->] (0,0) -- (-1,0) node[left] {$F_f$};
        \fill[black] (0,0) circle (3pt);
    \end{scope}
\end{tikzpicture}
\end{minipage}%
\begin{minipage}[c][4cm][c]{0.35\textwidth}
    \large
    \begin{align*}
        F_{\mathrm{net},x} &= {\ifprintanswers \color{red} - F_f = m a_x \fi} \\[1ex]
        F_{\mathrm{net},y} &= {\ifprintanswers \color{red} F_\mathrm{N} - F_g = 0 \fi}
    \end{align*}
\end{minipage}
\end{center}

\question
The object is pulled upward at constant speed.

\begin{center}
\begin{minipage}[c][4cm][c]{0.3\textwidth}
\begin{tikzpicture}[x=7mm,y=7mm]
    \draw[fill=black!10] (0,0) rectangle ++(1,1);
    \draw[very thick] (0.5,1) -- ++(0,1.2);
    \draw[thick,->,xshift=2mm] (1,0) -- ++(0,1) node[right,pos=0.5] {$v$};
    \draw[white,thick,->,xshift=-2mm] (0,0) -- ++(0,1) node[left,pos=0.5] {$v$};
    \ifprintanswers
    \color{red}
    \else
    \color{white}
    \fi
    \begin{scope}[xshift=3cm,x=1cm,y=1cm]
        \draw[thick,->] (0,0) -- (0,+1) node[above] {$F_\mathrm{T}$};
        \draw[thick,->] (0,0) -- (0,-1) node[below] {$F_\mathrm{g}$};
        \fill[black] (0,0) circle (3pt);
    \end{scope}
\end{tikzpicture}
\end{minipage}%
\begin{minipage}[c][4cm][c]{0.35\textwidth}
    \large
    \begin{align*}
        F_{\mathrm{net},x} &= {\ifprintanswers \color{red} 0 \fi} \\[1ex]
        F_{\mathrm{net},y} &= {\ifprintanswers \color{red} F_\mathrm{T} - F_g = 0 \fi}
    \end{align*}
\end{minipage}
\end{center}

\clearpage

\question
Static friction prevents sliding.

\bigskip

\begin{center}
\begin{minipage}[c][4cm][c]{0.45\textwidth}
\begin{tikzpicture}[x=7mm,y=7mm,rotate=-37]
    \draw (0,0) -- (5,0);
    \draw[fill=black!10] (2,0) rectangle ++(1,1);
    \begin{scope}[xshift=0cm,yshift=0.5cm,transform shape]
        \draw[->] (0,0) -- (0,1) node[above] {\small $\perp$};
        \draw[->] (0,0) -- (1,0) node[right] {\small $\parallel$};
    \end{scope}
    \ifprintanswers
    \color{red}
    \else
    \color{white}
    \fi
    \begin{scope}[xshift=5cm,yshift=2cm,x=1.5cm,y=1.5cm,rotate=37]
        \draw[thick,->,opacity=0.3] (0,0) -- (0,-1) node[below] {$F_g$};
        \begin{scope}[rotate=-37]
            \draw[thick,->] (0,0) -- (0,{cos(37)}) node[above] {$F_\mathrm{N}$};
            \draw[thick,->] (0,0) -- ({-sin(37)},0) node[left] {$F_f$};
            \draw[thick,->] (0,0) -- (0,{-cos(37)}) node[below] {$F_{g,\perp}$};
            \draw[thick,->] (0,0) -- ({sin(37)},0) node[right] {$F_{g,\parallel}$};
        \end{scope}
        \fill[black] (0,0) circle (3pt);
    \end{scope}
\end{tikzpicture}
\end{minipage}%
\begin{minipage}[c][4cm][c]{0.35\textwidth}
    \large
    \begin{align*}
        F_{\mathrm{net},\parallel} &= {\ifprintanswers \color{red} F_{g,\parallel} - F_f = 0 \fi} \\[1ex]
        F_{\mathrm{net},\perp} &= {\ifprintanswers \color{red} F_\mathrm{N} - F_{g,\perp} = 0 \fi}
    \end{align*}
\end{minipage}
\end{center}

\question
An object is suspended from the ceiling.

\begin{center}
\begin{minipage}[c][4cm][c]{0.35\textwidth}
\begin{tikzpicture}[x=7mm,y=7mm]
    \draw[fill=black!10] (2,0) rectangle ++(1,1);
    \draw[very thick] (2.5,1) -- ++(0,1.2);
    \draw[pattern=north west lines, pattern color=black] (0,2.2) rectangle ++(5,0.2);
    \ifprintanswers
    \color{red}
    \else
    \color{white}
    \fi
    \begin{scope}[xshift=5cm,yshift=5mm,x=1cm,y=1cm]
        \draw[thick,->] (0,0) -- (0,+1) node[above] {$F_\mathrm{T}$};
        \draw[thick,->] (0,0) -- (0,-1) node[below] {$F_\mathrm{g}$};
        \fill[black] (0,0) circle (3pt);
    \end{scope}
\end{tikzpicture}
\end{minipage}%
\begin{minipage}[c][4cm][c]{0.35\textwidth}
    \large
    \begin{align*}
        F_{\mathrm{net},x} &= {\ifprintanswers \color{red} 0 \fi} \\[1ex]
        F_{\mathrm{net},y} &= {\ifprintanswers \color{red} F_\mathrm{T} - F_g = 0 \fi}
    \end{align*}
\end{minipage}
\end{center}


\question
An object is suspended from the ceiling.

\begin{center}
\begin{minipage}[c][4cm][c]{0.5\textwidth}
\begin{tikzpicture}[x=7mm,y=7mm]
    \draw[fill=black!10] (0,0) rectangle ++(1,1);
    \draw (0.5,1) -- ++({-2},{2*tan(30)});
    \draw (0.5,1) -- ++({+2},{2*tan(30)});
    \draw[pattern=north west lines, pattern color=black] (-2,2.18) rectangle ++(5,0.2);
    \ifprintanswers
    \color{red}
    \else
    \color{white}
    \fi
    \begin{scope}[xshift=4cm,yshift=5mm,x=1.2cm,y=1.2cm]
        \draw[thick,->] (0,0) -- (0,-1) node[below] {$F_\mathrm{g}$};
        \draw[thick,->,opacity=0.3] (0,0) -- ({-cos(30)},{sin(30)});
        \draw[thick,->] (0,0) -- ({-cos(30)},0) node[left] {$F_\mathrm{T_{1x}}$};
        \draw[thick,->] (0,0) -- (0,{sin(30)}) node[left] {$F_\mathrm{T_{1y}}$};
        \draw[thick,->,opacity=0.3] (0,0) -- ({+cos(30)},{sin(30)});
        \draw[thick,->] (0,0) -- ({+cos(30)},0) node[right] {$F_\mathrm{T_{2x}}$};
        \draw[thick,->] (0,{sin(30)}) -- ++(0,{sin(30)}) node[above] {$F_\mathrm{T_{2y}}$};
        \fill[black] (0,0) circle (3pt);
    \end{scope}
\end{tikzpicture}
\end{minipage}%
\begin{minipage}[c][4cm][c]{0.35\textwidth}
    \large
    \begin{align*}
        F_{\mathrm{net},x} &= {\ifprintanswers \color{red} F_\mathrm{T_{2x}} - F_\mathrm{T_{1x}} = 0 \fi} \\[1ex]
        F_{\mathrm{net},y} &= {\ifprintanswers \color{red} F_\mathrm{T_{1y}} + F_\mathrm{T_{2y}} - F_g = 0 \fi}
    \end{align*}
\end{minipage}
\end{center}


\question
The object is motionless.

\begin{center}
\begin{minipage}[c][4cm][c]{0.5\textwidth}
\begin{tikzpicture}[x=7mm,y=7mm]
    \draw[fill=black!10] (0,0) rectangle ++(1,1);
    \draw (0.5,1) -- ++({-1},{tan(50)});
    \draw (1,0.5) -- ++(1,0);
    \draw[pattern=north west lines, pattern color=black] ({-tan(50)},{1+tan(50)}) rectangle ++(2,0.2);
    \draw[pattern=north west lines, pattern color=black] (2,1.2) rectangle ++(0.2,-1.4);
    \ifprintanswers
    \color{red}
    \else
    \color{white}
    \fi
    \begin{scope}[xshift=5cm,x=1.5cm,y=1.5cm]
        \draw[thick,->] (0,0) -- (0,-1) node[below] {$F_\mathrm{g}$};
        \draw[thick,->,opacity=0.3] (0,0) -- ({-cos(50)/sin(50)},1) node[pos=1.2] {$F_\mathrm{T_1}$};
        \draw[thick,->] (0,0) -- ({-cos(50)/sin(50)},0) node[left] {$F_\mathrm{T_{1x}}$};
        \draw[thick,->] (0,0) -- (0,1) node[above] {$F_\mathrm{T_{1y}}$};
        \draw[thick,->] (0,0) -- ({+cos(50)/sin(50)},0) node[right] {$F_\mathrm{T_2}$};
        \fill[black] (0,0) circle (3pt);
    \end{scope}
\end{tikzpicture}
\end{minipage}%
\begin{minipage}[c][4cm][c]{0.35\textwidth}
    \large
    \begin{align*}
        F_{\mathrm{net},x} &= {\ifprintanswers \color{red} F_\mathrm{T_2} - F_\mathrm{T_{1x}} = 0 \fi} \\[1ex]
        F_{\mathrm{net},y} &= {\ifprintanswers \color{red} F_\mathrm{T_{1y}}  - F_g = 0 \fi}
    \end{align*}
\end{minipage}
\end{center}

\clearpage

\question
The object is accelerated rightward by an applied force parallel to the rough surface.

\begin{center}
\begin{minipage}[c][4cm][c]{0.45\textwidth}
\begin{tikzpicture}[x=7mm,y=7mm,remember picture]
    \draw (-1.5,0) -- ++(4,0);
    \draw[fill=black!10] (0,0) rectangle ++(1,1);
    \draw[thick,->] (1,0.5) -- ++(1.2,0) node[right] {$F$};
    \ifprintanswers
    \color{red}
    \else
    \color{white}
    \fi
    \begin{scope}[xshift=4cm,x=1cm,y=1cm]
        \draw[thick,->] (0,0) -- (0,+1) node[above] {$F_\mathrm{N}$};
        \draw[thick,->] (0,0) -- (0,-1) node[below] {$F_g$};
        \draw[thick,->] (0,0) -- (1.5,0) node[right] {$F$};
        \draw[thick,->] (0,0) -- (-0.7,0) node[left] {$F_f$};
        \fill[black] (0,0) circle (3pt);  
    \end{scope}
\end{tikzpicture}
\end{minipage}%
\begin{minipage}[c][4cm][c]{0.35\textwidth}
    \large
    \begin{align*}
        F_{\mathrm{net},x} &= {\ifprintanswers \color{red} F - F_f = m a_x \fi} \\[1ex]
        F_{\mathrm{net},y} &= {\ifprintanswers \color{red} F_\mathrm{N}  - F_g = 0 \fi}
    \end{align*}
\end{minipage}
\end{center}


\question
The object is pulled at constant velocity by a force at an angle to the surface with friction.

\begin{center}
\begin{minipage}[c][4cm][c]{0.45\textwidth}
\begin{tikzpicture}[x=7mm,y=7mm]
    \draw (-1.5,0) -- ++(4,0);
    \draw[fill=black!10] (0,0) rectangle ++(1,1);
    \draw[thick,->] (1,1) -- ++({1.4*cos(45)},{1.4*sin(45)}) node[right] {$F$};
    \draw[dashed] (1,1) -- ++({1.4*cos(45)},0);
    \node[xshift=4.4mm,yshift=2mm] at (1,1) {\small $\theta$};
    \ifprintanswers
    \color{red}
    \else
    \color{white}
    \fi
    \begin{scope}[xshift=4cm,x=1.5cm,y=1.5cm]
        \draw[thick,->] (0,0) -- (0,{1-0.8*sin(45)}) node[left] {$F_\mathrm{N}$};
        \draw[thick,->] (0,0) -- (0,-1) node[below] {$F_g$};
        \draw[thick,->,opacity=0.3] (0,0) -- ({0.8*cos(45)},{0.8*sin(45)}) node[right] {$F$};
        \draw[thick,->] (0,{1-0.8*sin(45)}) -- ++(0,{0.8*sin(45)}) node[above] {$F_y$};
        \draw[thick,->] (0,0) -- ({0.8*cos(45)},0) node[right] {$F_x$};
        \draw[thick,->] (0,0) -- ({-0.8*cos(45)},0) node[left] {$F_f$};
        \fill[black] (0,0) circle (3pt);
    \end{scope}
\end{tikzpicture}
\end{minipage}%
\begin{minipage}[c][4cm][c]{0.35\textwidth}
    \large
    \begin{align*}
        F_{\mathrm{net},x} &= {\ifprintanswers \color{red} F_x - F_f = 0 \fi} \\[1ex]
        F_{\mathrm{net},y} &= {\ifprintanswers \color{red} F_\mathrm{N} + F_y  - F_g = 0 \fi}
    \end{align*}
\end{minipage}
\end{center}



\question
The object is falling with no air resistance.

\begin{center}
\begin{minipage}[c][4cm][c]{0.3\textwidth}
\begin{tikzpicture}[x=7mm,y=7mm]
    \draw[fill=black!10] (2,0) rectangle ++(1,1);
    \ifprintanswers
    \color{red}
    \else
    \color{white}
    \fi
    \begin{scope}[xshift=4cm,,yshift=5mm,x=1.5cm,y=1.5cm]
        \draw[thick,->] (0,0) -- (0,-1) node[below] {$F_g$};
        \fill[black] (0,0) circle (3pt);
    \end{scope}
\end{tikzpicture}
\end{minipage}%
\begin{minipage}[c][4cm][c]{0.3\textwidth}
    \large
    \begin{align*}
        F_{\mathrm{net},x} &= {\ifprintanswers \color{red}  0 \fi} \\[1ex]
        F_{\mathrm{net},y} &= {\ifprintanswers \color{red} - F_g = - m a_y \fi}
    \end{align*}
\end{minipage}
\end{center}


\question
The object is falling at constant (terminal) velocity.

\begin{center}
\begin{minipage}[c][4cm][c]{0.3\textwidth}
\begin{tikzpicture}[x=7mm,y=7mm]
    \draw[fill=black!10] (2,0) rectangle ++(1,1);
    \ifprintanswers
    \color{red}
    \else
    \color{white}
    \fi
    \begin{scope}[xshift=3.5cm,yshift=5mm,x=1.5cm,y=1.5cm]
        \draw[thick,->] (0,0) -- (0,-1) node[below] {$F_g$};
        \draw[thick,->] (0,0) -- (0,+1) node[above] {$F_d$};
        \fill[black] (0,0) circle (3pt);
    \end{scope}
\end{tikzpicture}
\end{minipage}%
\begin{minipage}[c][4cm][c]{0.3\textwidth}
    \large
    \begin{align*}
        F_{\mathrm{net},x} &= {\ifprintanswers \color{red}  0 \fi} \\[1ex]
        F_{\mathrm{net},y} &= {\ifprintanswers \color{red} F_d - F_g = 0 \fi}
    \end{align*}
\end{minipage}
\end{center}


% \ifprintanswers
% \else
% \clearpage
% \fi


\clearpage

\question
(Bonus) The object is accelerated leftward by a force applied downward at an angle with friction.

\begin{center}
\begin{minipage}[c][6cm][c]{0.45\textwidth}
\begin{tikzpicture}[x=7mm,y=7mm]
    \draw (-1.5,0) -- ++(4,0);
    \draw[fill=black!10] (0,0) rectangle ++(1,1);
    \draw[thick,<-] (1,1) -- ++({1.4*cos(45)},{1.4*sin(45)}) node[right] {$F$};
    \draw[dashed] (1,1) -- ++({1.4*cos(45)},0);
    \node[xshift=4.4mm,yshift=2mm] at (1,1) {\small $\theta$};
    \ifprintanswers
    \color{red}
    \else
    \color{white}
    \fi
    \begin{scope}[xshift=4cm,x=1.2cm,y=1.2cm]
        \draw[thick,->] (0,0) -- (0,{1+1.4*sin(45)}) node[above] {$F_\mathrm{N}$};
        \draw[thick,->,opacity=0.3] (0,0) -- ({-1.4*cos(45)},{-1.4*sin(45)});
        \draw[thick,->] (0,0) -- (0,{-1.4*sin(45)}) node[right] {$F_y$};
        \draw[thick,->] (0,{-1.4*sin(45)}) -- ++(0,-1) node[below] {$F_g$};
        \draw[thick,->] (0,0) -- ({-1.4*cos(45)},0) node[left] {$F_x$};
        \draw[thick,->] (0,0) -- ({0.8*cos(45)},0) node[right] {$F_f$};
        \fill[black] (0,0) circle (3pt);
    \end{scope}
\end{tikzpicture}
\end{minipage}%
\begin{minipage}[c][4cm][c]{0.3\textwidth}
    \large
    \begin{align*}
        F_{\mathrm{net},x} &= {\ifprintanswers \color{red}  -F_x + F_f = - m a_x \fi} \\[1ex]
        F_{\mathrm{net},y} &= {\ifprintanswers \color{red} F_\mathrm{N} - F_y - F_g = 0 \fi}
    \end{align*}
\end{minipage}
\end{center}


\question
(Bonus) The object slides without friction.

\begin{center}
\begin{minipage}[c][6cm][c]{0.45\textwidth}
\begin{tikzpicture}[x=7mm,y=7mm,rotate=-37]
    \draw (0,0) -- (4,0);
    \draw[fill=black!10] (1.5,0) rectangle ++(1,1);
    \begin{scope}[xshift=-0.3cm,yshift=0.5cm,transform shape]
        \draw[->] (0,0) -- (0,1) node[above] {\small $\perp$};
        \draw[->] (0,0) -- (1,0) node[right] {\small $\parallel$};
    \end{scope}
    \ifprintanswers
    \color{red}
    \else
    \color{white}
    \fi
    \begin{scope}[xshift=5cm,yshift=2cm,x=1.5cm,y=1.5cm,rotate=37]
        \draw[thick,->,opacity=0.3] (0,0) -- (0,-1) node[below] {$F_g$};
        \begin{scope}[rotate=-37]
            \draw[thick,->] (0,0) -- (0,{cos(37)}) node[above] {$F_\mathrm{N}$};
            \draw[thick,->] (0,0) -- (0,{-cos(37)}) node[below] {$F_{g,\perp}$};
            \draw[thick,->] (0,0) -- ({sin(37)},0) node[right] {$F_{g,\parallel}$};
        \end{scope}
        \fill[black] (0,0) circle (3pt);
    \end{scope}
\end{tikzpicture}
\end{minipage}%
\begin{minipage}[c][4cm][c]{0.35\textwidth}
    \large
    \begin{align*}
        F_{\mathrm{net},\parallel} &= {\ifprintanswers \color{red} F_{g,\parallel} = m a_\parallel \fi} \\[1ex]
        F_{\mathrm{net},\perp} &= {\ifprintanswers \color{red} F_\mathrm{N} - F_{g,\perp} = 0 \fi}
    \end{align*}
\end{minipage}
\end{center}


\end{questions}





\end{document}