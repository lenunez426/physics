\documentclass[../main-physics-problems.tex]{subfiles}
\begin{document}



\subsection{Charges and Matter}
\subsubsection{Particles that Contribute Positive, Negative, or No Charge}
\subsubsection{Neutral Objects}
\subsubsection{Determining Charge}
\subsubsection{Two Objects Attract, Repel, or Have No Interaction}
\subsubsection{Drawing the Electric Field}

\clearpage

\subsection{Transfer of Charge}
\subsubsection{Charge is Conserved}
\subsubsection{Only Electrons are Transferred}
\subsubsection{Charging by Induction and Conduction}
\subsubsection{Polarization}
\subsubsection{The Electroscope}
\subsubsection{Known and Unknown Charges}
\subsubsection{Distributions of Charge}

\clearpage

\subsection{Coulomb’s Law}
\subsubsection{Free Body Diagrams for Two Charged Objects}
\subsubsection{Electric Force, Charges, and Distance}
\subsubsection{Electric Force and Gravitational Force}
\subsubsection{How Changing Charge or Distance Affects the Electric Force}

\clearpage

\subsection{Simple Circuits}
\subsubsection{Components for a Simple Circuit}
\subsubsection{Tracing the Conducting Path}
\subsubsection{Ohm’s Law: Current, Resistance, and Voltage}
\subsubsection{Measurements using a Multimeter, Ammeter, and Current Probe}
\subsubsection{Calculating Voltage Drop}
\subsubsection{Changes in Current with Changes in Voltage or Resistance}
\subsubsection{Electrical Power}
\subsubsection{Light Bulb Brightness}

\clearpage

\subsection{Series Circuits}
\subsubsection{Measurements in Series using a Multimeter, Ammeter, and Current Probe}
\subsubsection{Current Flow through a Series Circuit}
\subsubsection{Equivalent Resistance in Series}
\subsubsection{Equivalent Voltage in Series}
\subsubsection{Total Voltage}
\subsubsection{Energy Transformations in Series}
\subsubsection{Determining Current, Voltage, and Power in Series}

\clearpage

\subsection{Parallel Circuits}
\subsubsection{Measurements in Parallel using a Multimeter, Ammeter, and Current Probe}
\subsubsection{Junctions}
\subsubsection{Current Flow through a Parallel Circuit}
\subsubsection{Voltage Drops in Parallel}
\subsubsection{Parallel Circuits Compared to Series Circuits}
\subsubsection{Equivalent Resistance in Parallel}
\subsubsection{Equivalent Voltage in Parallel}
\subsubsection{Energy Transformations in Parallel}
\subsubsection{Determining Current, Voltage, and Power in Parallel}

\clearpage

\subsection{Combination Circuits}
\subsubsection{Voltage and Current in Combination Circuits}
\subsubsection{Predicting which Bulbs will Light with Switches}

    
\begin{questions}

\question
Watch ``Charge of an Electron: Millikan's Oil Drop Experiment'' by Tyler DeWitt on YouTube 
%(\href{https://youtu.be/2HhaQtvICe8}{click here})
and answer the exercises below.

\question
In what year did Millikan and Fletcher conduct the oil drop experiment?

\begin{randomizechoices}
    \correctchoice 1913
    \choice 1900
    \choice 1897
    \choice 1850
\end{randomizechoices}

\question
J.~J.~Thompson discovered the electron, the fundamental particle of negative charge, in the year \fillin[][3cm]

\begin{randomizechoices}
    \choice 1913
    \choice 1900
    \correctchoice 1897
    \choice 1850
\end{randomizechoices}

\question
The Millikan's oil drop experiment involves balancing tiny drops of oil by using gravity and \fillin[][3cm].

\begin{randomizechoices}
    \correctchoice electricity
    \choice magic
    \choice magnetism
    \choice tension
\end{randomizechoices}

\question
Two parts of the equipment used by Millikan include an atomizer and 

\begin{randomizechoices}
    \correctchoice a microscope
    \choice a power supply
    \choice a telescope
    \choice an oil dispenser
\end{randomizechoices}

\question
Which force acts downward on the oil drop?

\begin{randomizechoices}
    \correctchoice gravity force
    \choice electric force
    \choice magnetic force
    \choice tension force
\end{randomizechoices}

\question
Which plate--the positively or the negatively charged---will the positive oil drop be attracted to? Explain.

\begin{randomizechoices}
    \correctchoice negative plate
    \choice positive plate
    \choice both the positive and negative plate
    \choice neither the positive nor negative plate
\end{randomizechoices}

\question
What force pushes the oil drop in the opposite direction of gravity?

\begin{randomizechoices}
    \choice gravity force
    \correctchoice electric force
    \choice magnetic force
    \choice tension force
\end{randomizechoices}

\question
When the voltage is too \textbf{high}, the oil drop moves

\begin{randomizechoices}
    \correctchoice up
    \choice down
    \choice left
    \choice right
\end{randomizechoices}

\question
When the voltage is too \textbf{low}, the oil drop moves

\begin{randomizechoices}
    \choice up
    \correctchoice down
    \choice left
    \choice right
\end{randomizechoices}

\question
What happens to the oil drop when the voltage through the apparatus is just right?

\begin{randomizechoices}
    \correctchoice The net force is zero, and the drop will float at rest.
    \choice It accelerates toward the negative plate.
    \choice It accelerates toward the positive plate.
    \choice It evaporates, disappearing from view.
\end{randomizechoices}

\question
What does the force of gravity on the drop depend on?

\begin{randomizechoices}
    \correctchoice mass of drop
    \choice voltage on plates
    \choice charge of drop
    \choice distance between the plates
\end{randomizechoices}

\question
What 2 factors does the force of electricity on the drop depend on?

\begin{randomizechoices}[keeplast]
    \correctchoice charge of drop and voltage on plates
    \choice mass of drop and charge of drop
    \choice voltage on plates and mass of drop
    \choice none of the above
\end{randomizechoices}

\question 
What is the unit used to measure charge?

\begin{randomizechoices}
    \correctchoice coulomb (C)
    \choice voltage (V)
    \choice newton (N)
    \choice ampere (A)
\end{randomizechoices}

\question
What is the charge of 1 electron?

\begin{randomizechoices}
    \correctchoice \SI{-1.60e-19}{C}
    \choice \SI{1.60e-19}{C}
    \choice \SI{-3.20e-19}{C}
    \choice \SI{3.20e-19}{C}
\end{randomizechoices}

\end{questions}


\clearpage
\begin{questions}
\question
Like charges repel, and opposite charges attract.

\begin{randomizechoices}[norandomize]
    \correctchoice True
    \choice False
\end{randomizechoices}

\question
An atom that loses electrons becomes negatively charged.

\begin{randomizechoices}[norandomize]
    \choice True
    \correctchoice False
\end{randomizechoices}

\question
An atom that gains electrons becomes negatively charged.

\begin{randomizechoices}[norandomize]
    \correctchoice True
    \choice False
\end{randomizechoices}

\question
Charging a neutral object by touching that object with a charged object is referred to as

\begin{randomizechoices}
    \correctchoice conduction
    \choice induction
    \choice friction
\end{randomizechoices}

\question
Charging an object without direct contact by bringing a charged object close to it is known as

\begin{randomizechoices}
    \choice conduction
    \correctchoice induction
    \choice friction
\end{randomizechoices}

\question
Charging an object through transfer of electrons by rubbing is referred to as charging by

\begin{randomizechoices}
    \choice conduction
    \choice induction
    \correctchoice friction
\end{randomizechoices}



\question
Two positive charged particles will attract while two opposite charges will repel each other.

\begin{randomizechoices}[norandomize]
    \choice True
    \correctchoice False
\end{randomizechoices}


\end{questions}



\end{document}