\documentclass{exam}
\input{../preamble}
\usepackage{circuitikz}
\usepackage{exam-randomizechoices}

\def\one{1}
\def\two{2}
\def\three{3}
\def\four{4}

\def\testVersion{\two} % Edit Version number here only.

\setrandomizerseed{\testVersion}

\pagestyle{headandfoot}
\firstpageheader{Physics}{Test on Unit 9: Electromagnetism}{Version \testVersion}
\CorrectChoiceEmphasis{\color{red}\bfseries}
\SolutionEmphasis{\color{red}}

%\printanswers

\begin{document}

\begin{center}
\fbox{\fbox{\parbox{5.5in}{
\vspace{-1em}
\begin{center}
    \textbf{EQUATIONS}
\end{center}

\vspace{-1.5em}

\begin{equation*}
    B =\frac{\mymu_0 I}{2\pi r} \qquad
    \mymu_0 = 4 \pi \times 10^{-7}  \qquad
    \pi \approx 3.14
\end{equation*}

\begin{center}
    $\bigodot$ means ``out of page'' \hspace{2em}
    $\bigotimes$ means ``into page''
\end{center}
}}}
\end{center}


\begin{questions}

\question
Magnets are categorized as magnetic dipoles. What is meant by \textit{dipole}?

\begin{randomizechoices}
    \correctchoice That a bar magnet has two poles.
    \choice That a bar magnet has one pole.
    \choice That a bar magnet has no poles.
    \choice That a bar magnet has infinitely many poles.
\end{randomizechoices}

\question
What are the labels of the poles of a bar magnet?

\begin{randomizechoices}
\choice North only
\choice North, south, east, and west
\choice East and west
\CorrectChoice North and south
\end{randomizechoices}

\question
What do you get when you cut a bar magnet in half?

\begin{randomizechoices}
\choice One magnet with two south poles, and one magnet with two north poles.
\choice Two objects that are not magnetism anymore.
\choice It’s impossible to cut a magnet in half.
\CorrectChoice Two magnets, each with a north and south pole.
\end{randomizechoices}

\question
True or False? Earth generates a magnetic field which surrounds it.

\begin{randomizechoices}
    \correctchoice True
    \choice False
\end{randomizechoices}

\question
The south geographic pole of Earth (in Antarctica) is near what type of \textit{magnetic} pole?

\begin{randomizechoices}
    \correctchoice north
    \choice south
    \choice east
    \choice west
\end{randomizechoices}


\question
What type of \textit{magnetic} pole is close to Earth's north \textit{geographic} pole (near Alaska)? 

\begin{randomizechoices}
\choice north
\CorrectChoice south
\choice east
\choice west
\end{randomizechoices}

% \question
% The \textit{magnetic} pole of Earth that is close to the \textit{geographic} South Pole is a\\ magnetic \fillin Pole.

% \begin{randomizechoices}
% \CorrectChoice North
% \choice South
% \choice East
% \choice West
% \end{randomizechoices}

% \question
% The \textit{magnetic} North Pole of Earth is close to the \textit{geographic} \fillin Pole.

% \begin{randomizechoices}
% \choice North
% \CorrectChoice South
% \choice East
% \choice West
% \end{randomizechoices}

% \question
% The \textit{magnetic} South Pole of Earth is close to the \textit{geographic} \fillin Pole.

% \begin{randomizechoices}
% \CorrectChoice North
% \choice South
% \choice East
% \choice West
% \end{randomizechoices}

\question
One way to visualize magnetic field lines around a magnet is by\ldots 

\begin{randomizechoices}
\choice using certified electromagnetic goggles.
\choice attracting or repelling the magnet with a second magnet.
\CorrectChoice sprinkling iron filings around the magnet.
\choice brushing the magnet with glow-in-the-dark liquid and turning off the lights.
\end{randomizechoices}

\question
Besides magnets, what other thing creates a magnetic field?

\begin{randomizechoices}
\choice Water.
\choice The gravitational force.
\CorrectChoice electric current (i.e., moving charges)
\choice Nothing. Only magnets can create magnetic fields.
\end{randomizechoices}

%\clearpage

% \question
% Where were the first known magnets discovered?

% \begin{randomizechoices}
% \choice Houston, TX
% \CorrectChoice Magnesia (present-day western Turkey)
% \choice London, England
% \choice The North Pole
% \end{randomizechoices}

% \question
% Suppose that equal parts of sand and iron powder are mixed together in a glass petri dish. What happens when a magnet is placed next to one side of the dish?

% \begin{randomizechoices}
% \CorrectChoice Magnet attracts the iron only
% \choice Magnet attracts the sand only
% \choice Magnet attracts both the sand and the iron
% \choice The magnet attracts the iron and repels the sand
% \end{randomizechoices}

\question
In which direction do magnetic field lines from a bar magnet flow?

\begin{randomizechoices}
\choice South to North
\choice North to East
\CorrectChoice North to South
\choice South to West
\end{randomizechoices}

\question
What is electric current?

\begin{randomizechoices}
\choice a current of water
\CorrectChoice electric charges that are moving
\choice directional field lines around a magnet
\choice the atoms that make up a metal wire
\end{randomizechoices}

% \question
% The SI units for electric current, amperes (A), are equivalent to which other units?

% \begin{randomizechoices}
% \CorrectChoice C/s
% \choice e/s
% \choice $-$e/s
% \choice C/$\mathrm{s^2}$
% \end{randomizechoices}

\question
The electrons in a straight wire flow downwards. In which direction does electric current flow?

\begin{randomizechoices}
\CorrectChoice up
\choice down
\choice $\bigodot$
\choice $\bigotimes$
\end{randomizechoices}

% \question
% The current in a straight wire flows to the right. In which direction do electrons flow?

\clearpage

\question
What is the purpose of the right-hand rule?

\begin{center}
\begin{minipage}{0.3\textwidth}
\centering
    \begin{tikzpicture}
        \draw[ultra thick,->] (0,-2) -- ++(0,3.5) node[above] {$I$};
        \begin{scope}[rotate around x=50]
            \draw[red,very thick,->] (0,0) arc (90:280:1);
        \end{scope}
        \begin{scope}[yshift=2mm,rotate around x=50]
            \draw[red,very thick,->] (0,0) arc (90:283:1.6);
        \end{scope}
        \begin{scope}[yshift=4mm,rotate around x=50]
            \draw[red,very thick,->] (0,0) arc (90:286:2.2);
        \end{scope}
        \node[red] at (1.3,-0.5) {$\vec{B}$};
    \end{tikzpicture}
\end{minipage}%
\hspace{1em}
\begin{minipage}{0.3\textwidth}
    \centering
        \includegraphics[width=3.6cm]{Figures/Unit10_RightHandRule.jpeg}
\end{minipage}
\end{center}

\begin{randomizechoices}
\choice to show the direction of gravity
\CorrectChoice to show the direction of magnetic field around a current-carrying wire
\choice to show the direction of current between two magnets
\choice to show the direction of magnetic field between two magnets
\end{randomizechoices}

% \question
% What is the direction of the magnetic field when the current is

% \begin{center}
%     $\bigodot I$
% \end{center}

% \begin{randomizechoices}
% \choice {$\circlearrowright$}
% \CorrectChoice {$\circlearrowleft$}
% \choice $\bigodot$
% \choice $\bigotimes$
% \end{randomizechoices}

% \clearpage
% \question
% What is the direction of the magnetic field when the current is

% \begin{center}
%     $\bigotimes I$
% \end{center}

% \begin{randomizechoices}
% \CorrectChoice {$\circlearrowright$}
% \choice {$\circlearrowleft$}
% \choice $\bigodot$
% \choice $\bigotimes$
% \end{randomizechoices}

\question Current in a wire flows to the left, as shown by the figure. What is the direction of the magnetic field in the region \textit{above} the wire?

\begin{center}
\begin{tikzpicture}
\draw[ultra thick,<-] (0,0) node[left]{$I$} -- ++(5,0);
\begin{scope}[yshift=3mm]
    \draw[dashed] (0,0) rectangle (5,1) node[pos=0.5]{Above};    
\end{scope}
\end{tikzpicture}
\end{center}

\begin{randomizechoices}
\choice left
\choice right
\choice out of the page %$\bigodot$
\CorrectChoice into the page %$\bigotimes$
\end{randomizechoices}

\question Current in a wire flows to the left, as shown by the figure. What is the direction of the magnetic field in the region \textit{below} the wire?

\begin{center}
\begin{tikzpicture}
\draw[ultra thick,<-] (0,0) node[left]{$I$} -- ++(5,0);
\begin{scope}[yshift=-13mm]
    \draw[dashed] (0,0) rectangle (5,1) node[pos=0.5]{Below};    
\end{scope}
\end{tikzpicture}
\end{center}

\begin{randomizechoices}
\choice left
\choice right
\CorrectChoice out of the page %$\bigodot$
\choice into the page %$\bigotimes$
\end{randomizechoices}

\clearpage

\question Electric current flows through a vertical wire. The surrounding magnetic field goes into and out of the page as shown below. In what direction do \textbf{ELECTRONS} through the wire flow?

\begin{center}
\begin{tikzpicture}
    \draw[ultra thick] (0,0) node[below] {down} -- ++(0,3) node[above] {up} ; 
    \foreach \i in {0,0.5,...,1.5}
        \foreach \j in {0.5,1.0,...,2.5}
      {
        \begin{scope}[xshift=-2cm]
            \node at (\i,\j) {$\otimes$};
        \end{scope}
        \begin{scope}[xshift=0.5cm]
            \node at (\i,\j) {$\odot$};
        \end{scope}
      }
\end{tikzpicture}
\end{center}

\begin{randomizechoices}
    \correctchoice up
    \choice down
    \choice out of the page
    \choice into the page
\end{randomizechoices}

\vspace{1em}

\cyanhrule

\begin{EnvUplevel}
\textbf{Questions \ref{JAlbal} through \ref{JxMpCM} are OPTIONAL and are worth EXTRA CREDIT.} To get full credit, show all your work on a sheet of paper, express your answer with correct units, and box your answer.
\end{EnvUplevel}

\question \label{JAlbal}
A wire carries 7.0 amperes of electric current. What is the magnetic field strength 4.5 centimeters outside the current-carrying wire? %Answer: 

\begin{solution}
    \SI{3.1e-5}{T}
\end{solution}

\question
The magnitude of the magnetic field at a point outside a current-carrying wire is shown below. Calculate the distance, in meters, of the point above the wire.

\begin{center}
    \begin{tikzpicture}
        \draw[ultra thick,->] (0,0) -- (3,0) node[right] {$\SI{12}{A}$};
        \draw[dashed] (1.5,0) -- ++(0,0.9);
        \node[red] at (1.5,1) {$\bigodot$}; \node[black,right=4pt] at (1.5,1) {\SI{8e-7}{T}};
    \end{tikzpicture}
\end{center}

\begin{solution}
    3.0 meters
\end{solution}

\question
The magnetic field strength 3 millimeters outside a current-carrying wire is \num{5e-4} tesla. Calculate the electric current through the wire.

\begin{solution}
    \SI{7.5}{A}
\end{solution}

\begin{EnvUplevel}
    \textbf{Questions \ref{LhsqrR} and \ref{JxMpCM} are in reference to the magnetic field equation:}
\end{EnvUplevel}

\begin{equation*}
    B = \frac{\mymu_0 I}{2\pi r}
\end{equation*}

\question \label{LhsqrR}
Solve the equation algebraically for distance. Show all your steps.

\question \label{JxMpCM}
Solve the equation algebraically for electric current. Show all your steps.

\clearpage
\keylistkeyname{Key Version \testVersion}
\printkeytable





% \question
% What is the magnetic field strength at a distance of 3 m from a wire carrying a current of 2 A?

% \begin{randomizechoices}
% \CorrectChoice $1.3 \times 10^{-7}$ T
% \choice $6.7 \times 10^{-6}$ T
% \choice 6 T
% \choice 5 T
% \end{randomizechoices}

% \question
% What is the magnetic field strength at a distance of 0.05 m from a wire carrying a current of 7.3 A?

% \begin{randomizechoices}
% \choice $9.1 \times 10^{-5}$ T
% \CorrectChoice $2.9 \times 10^{-5}$ T
% \choice 0.05 T
% \choice 7.3 T
% \end{randomizechoices}


\end{questions}
\end{document}

\question
What is a magnetic field?

\begin{randomizechoices}
\choice a field of magnets
\CorrectChoice directional lines around a magnetic material that indicate the direction and magnitude of the magnetic force
\choice a material or object that produces a magnetism
\choice part of a magnet that exerts the strongest force on other magnets or magnetic material
\end{randomizechoices}






