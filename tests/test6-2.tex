\documentclass[dvipsnames]{exam}
\input{../preamble}
\usepackage{exam-randomizechoices}

\setrandomizerseed{12345}

\printanswers
\CorrectChoiceEmphasis{\color{red}\bfseries}

\begin{document}

\header{Physics}{Unit 6 Summative Assessment}{}
\begin{questions}

\question
Memo, the goalkeeper, wants to kick a soccer ball as far down the field as possible. Which of the following launch angles will produce the largest horizontal displacement?

\begin{minipage}{6cm}
\centering
    \begin{randomizechoices}
        \correctchoice \ang{47}
        \choice \ang{86}
        \choice \ang{15}
        \choice \ang{39}
    \end{randomizechoices}
\end{minipage}%
\begin{minipage}{6cm}
    \begin{center}
        \begin{tikzpicture} 
        \tikzmath{
            \vi = 18.0;
            \yi = 0;
            \thetai = 70.0;
        }
        \tikzset{declare function={f(\x)=\yi + tan(\thetai)*\x - 9.8*\x^2/(2*(\vi*cos(\thetai))^2);}} %equation of path
        \pgfplotsset{compat=1.11}
            \begin{axis}[width=7cm,height=6cm,ticks=none,
            axis lines = center,
            clip=false,
            ylabel = $y$,
            xlabel = $x$,
            xmin=0, xmax=20,
            ymin=0, ymax=18,
            % axis line style={draw=none},
            ]
            \draw[dashed,domain=0:16,variable=\x,samples=200] plot ({\x},{f(\x)}) node {\faSoccerBallO};
            \draw (1.5,0) arc (0:60:1.5) node[right=2pt,pos=0.8] {$\theta$};
            \end{axis}
        \end{tikzpicture}
        \label{i1Y7vV}
    \end{center}
\end{minipage}

\question
What is the angle between the $x$ and $y$ components of a vector?

\begin{randomizechoices}
    \correctchoice \ang{90}
    \choice \ang{45}
    \choice \ang{0}
    \choice \ang{30}
\end{randomizechoices}

\question
After a projectile is launched in the air, in which direction does it experience constant, non-zero acceleration? Ignore air resistance.

\begin{randomizechoices}[keeplast]
    \correctchoice The $y$ direction
    \choice The $x$ direction
    \choice Both the $x$ and $y$ direction
    \choice Neither direction
\end{randomizechoices}

\question
True or False? Range is defined as the maximum vertical distance travelled by a projectile. 

\begin{choices}
    \choice True
    \correctchoice False
\end{choices}

\question
For what angle of a projectile is its range equal to zero?

\begin{randomizechoices}
    \correctchoice \ang{90}
    \choice \ang{45}
    \choice \ang{10}
    \choice \ang{1}
\end{randomizechoices}

\clearpage
\question
On November 5, 2022, Yordan Alvarez of the Houston Astros hit a game-winning home run at Minute Maid Park. The baseball was hit with an exit velocity of \SI{50.3}{m/s} at a launch angle of \ang{27}. When the ball's horizontal displacement from the plate was 110 meters, its height above ground was 26.5 meters. Calculate the ball's displacement from home plate at this instant.

\vspace{1em}

\begin{minipage}{6cm}
    \centering 
    \begin{randomizechoices}
        \correctchoice \SI{113}{m}
        \choice \SI{110}{m}
        \choice \SI{136}{m}
        \choice \SI{12803}{m}
    \end{randomizechoices}
\end{minipage}%
\begin{minipage}{6cm}
    \centering 
    \begin{center}
        \begin{tikzpicture} 
        \tikzmath{
            \vi = 50.3;
            \yi = 0;
            \thetai = 27.0;
        }
        \tikzset{declare function={f(\x)=\yi + tan(\thetai)*\x - 9.8*\x^2/(2*(\vi*cos(\thetai))^2);}} %equation of path
        \pgfplotsset{compat=1.11}
        \begin{axis}[width=6cm,height=6cm,ticks=none,
            axis lines = center,
            clip=false,
            axis equal image,
            xmin=0, xmax=120,
            ymin=0, ymax=40,
            axis line style={draw=none},
            ticks=none,
            ]
            \draw[lightgray,domain=0:110,variable=\x,samples=200] plot ({\x},{f(\x)});
            \draw[->,thick] (0,0) -- ++(110,{f(110)}) node[pos=0.75,below]{$d$};
            \draw[->,dashed] (0,0) -- ++(110,0) node[pos=0.5,below] {\SI{110}{m}};
            \draw[->,dashed] (110,0) -- ++(0,{f(110)}) node[pos=0.5,right] {\SI{26.5}{m}};
            \draw[fill=black] (110,{f(110)}) circle (2pt);
            \end{axis}
        \end{tikzpicture}
    \end{center}    
\end{minipage}%

\question
The Harbinger Willow Tail is a comet-like firework shell that makes for an awesome spectacle. It's launched with horizontal and vertical velocity components of \SI{20.6}{m/s} and \SI{51.0}{m/s}, respectively. How much time after launch will it take the Harbinger to reach its maximum height?

\begin{randomizechoices}
    \correctchoice \SI{5.2}{s}
    \choice \SI{6.9}{s}
    \choice \SI{4.8}{s}
    \choice \SI{7.3}{s}
\end{randomizechoices}

\question
During archery practice, Aloy launches an arrow at 18 meters per second, as shown in the figure below. What horizontal distance does the arrow travel when it strikes the ground?  

\vspace{1em}

\begin{minipage}{6cm}
    \centering
    \begin{randomizechoices}
        \correctchoice \SI{19.4}{m}
        \choice \SI{18}{m}
        \choice \SI{14.9}{m}
        \choice \SI{1.1}{m}
    \end{randomizechoices}
\end{minipage}%
\begin{minipage}{6cm}
    \centering
    \begin{center}
        \begin{tikzpicture}[
            declare function={f(\x,\yi,\vi,\thetai)=\yi + tan(\thetai)*\x - \grav*\x^2/(2*(\vi*cos(\thetai))^2);}, %equation of path
            declare function ={R(\vi,\thetai)= \vi^2*sin(2*\thetai)/\grav;}, %range
            declare function={h(\vi,\thetai)=(\vi*sin(\thetai))^2/(2*\grav);}, %maximum height
        ]
        \tikzmath{
            \grav = 9.8;
            \sf = 0.5; %scale factor for vector components
        }
        \pgfplotsset{compat=1.11}
            \begin{axis}[width=6cm,height=6cm,ticks=none,
                axis lines = center,
                clip=false,
                ylabel = $y$,
                xlabel = $x$,
                xmin=0, xmax={R(18,72)*1.2},
                ymin=0, ymax={h(18,72)*1.1},
                ]
                \draw[dashed,domain=0:{R(18,72)},variable=\x,samples=250] plot ({\x},{f(\x,0,18,72)});
                \draw (1.5,0) arc (0:72:1.5) node[right=2pt,pos=0.8] {$\ang{72}$};
                \draw[violet,ultra thick,->] (0,0) -- ++({0.3*18*cos(72)},{0.3*18*sin(72)});
            \end{axis}
        \end{tikzpicture}
    \end{center}
\end{minipage}

\clearpage

\question \label{EJzjXh}
The JPMorgan Chase Tower is the tallest building in Houston, at a height of 305 meters. If Tiger Woods horizontally launches a golf ball at 12 meters per second off the roof, how long will it take the ball to strike the ground? Ignore air resistance.

\begin{minipage}{6cm}
    \centering
    \begin{randomizechoices}
        \correctchoice \SI{7.9}{s}
        \choice \SI{5.6}{s}
        \choice \SI{1.6}{s}
        \choice \SI{3.9}{s}
    \end{randomizechoices}
\end{minipage}%
\begin{minipage}{6cm}
    \centering
    \begin{center}
        \begin{tikzpicture}[
                declare function={f(\x,\yi,\vi,\thetai)=\yi + tan(\thetai)*\x - \grav*\x^2/(2*(\vi*cos(\thetai))^2);}, %equation of path
                declare function ={R(\vi,\thetai)= \vi^2*sin(2*\thetai)/\grav;}, %range
                declare function={h(\vi,\thetai)=(\vi*sin(\thetai))^2/(2*\grav);}, %maximum height
            ]
            \tikzmath{
                \grav = 9.8;
                \sf = 0.5; %scale factor for vector components
            }
        \pgfplotsset{compat=1.11}
            \begin{axis}[width=6cm,height=6cm,ticks=none,
            axis lines = center,
            clip=false,
            xmin=0, xmax=150,
            ymin=0, ymax=350,
            axis line style={draw=none}
            ]
            \draw[dashed,domain=0:95,variable=\x,samples=100] plot ({\x},{f(\x,305,12,0)});
            \begin{scope}[shift={(-20,0)}]
                \draw[fill=black!50] (0,0) rectangle (20,305);
            \end{scope}
            \end{axis}
        \end{tikzpicture}
    \end{center}
\end{minipage}

\question
How does the elapsed time from Question \ref{EJzjXh} change if the ball is dropped vertically from rest instead of launched horizontally?

\begin{randomizechoices}[keeplast]
    \correctchoice The new elapsed time remains the same.
    \choice The new elapsed time is longer.
    \choice The new elapsed time is shorter.
    \choice It's impossible to determine.
\end{randomizechoices}

\question
What does \textit{centripetal} mean?

\begin{randomizechoices}
    \correctchoice center-seeking
    \choice center of petal
    \choice circular motion
    \choice acceleration due to gravity
\end{randomizechoices}

\question
In uniform circular motion, what is the angle between \textbf{tangential velocity} and \textbf{centripetal acceleration}?

\begin{randomizechoices}
    \correctchoice \ang{90}
    \choice \ang{45}
    \choice \ang{0}
    \choice \ang{180}
\end{randomizechoices}

\question
In uniform circular motion, what is the angle between \textbf{centripetal acceleration} and \textbf{centripetal force}?

\begin{randomizechoices}
    \correctchoice \ang{0}
    \choice \ang{90}
    \choice \ang{45}
    \choice \ang{180}
\end{randomizechoices}

\clearpage

\begin{EnvUplevel}
    \textbf{\ref{Q3KcTI}--\ref{kbFrGW} A fly tethered by a spider web is undergoing uniform circular motion.}
\end{EnvUplevel}

\question \label{Q3KcTI}
 At the instant the fly's velocity vector points to the right, what is the direction of the centripetal acceleration?

\begin{minipage}{6cm}
    \centering
    \begin{randomizechoices}
        \correctchoice up
        \choice down
        \choice left
        \choice right
    \end{randomizechoices}
\end{minipage}%
\begin{minipage}{6cm}
    \centering
\begin{center}
    \begin{tikzpicture}
        \begin{axis}[
            width=6cm,height=6cm,
            axis line style={draw=none},
            ticks=none,
            axis lines=middle,
            xmin=-1,xmax=1,
            ymin=-1,ymax=1,
            clip=false,
        ]
        \draw[gray] (0,0) circle (1);
        \fill (0,0) circle (1pt);
        \fill ({cos(270)},{sin(270)}) circle (3pt);
        \draw[<->] (0,-0.2) node[below] {down} -- ++(axis direction cs: 0,0.4) node[above] {up};
        \draw[<->] (-0.2,0) node[left] {left} -- ++(axis direction cs: 0.4,0) node[right] {right};
        \draw[red,very thick,->] ({cos(270)},{sin(270)}) -- ++(axis direction cs: 0.7,0) node[black,pos=1.1] {$v$};
    \end{axis}
    \end{tikzpicture}
    \end{center}    
\end{minipage}


\question \label{kbFrGW}
Some time later, the centripetal acceleration vector on the fly points to the right. What is the direction of the fly's velocity now?

\begin{randomizechoices}
    \correctchoice down
    \choice up
    \choice left    
    \choice right
\end{randomizechoices}

\vspace{1em}

\hrule

\question \label{OXyBFH}
Cody banks a curve on a road in his Smart car at \SI{35}{mph} (\SI{15.6}{m/s}), as shown in the figure below. Calculate the magnitude of centripetal acceleration on his car.

\begin{minipage}{6cm}
\centering
    \begin{randomizechoices}
        \correctchoice \SI{3.7}{m/s^2}
        \choice \SI{0.24}{m/s^2}
        \choice \SI{270}{m/s^2}
        \choice \SI{4.2}{m/s^2}
    \end{randomizechoices}
\end{minipage}%
\begin{minipage}{6cm}
    \centering
\begin{center}
    \begin{tikzpicture}
        \begin{axis}[
            width=7cm,height=7cm,
            axis line style={draw=none},
            ticks=none,
            axis lines=middle,
            xmin=-1,xmax=1,
            ymin=-1,ymax=1,
            clip=false,
        ]
        \draw[gray] (0,0) circle (1);
        \fill (0,0) circle (1pt);
        \fill ({cos(45)},{sin(45)}) circle (3pt) node[above right] {$m$};
        \begin{scope}[shift={(axis direction cs: 0.04,-0.04)}]
            \draw[dashed,<->] (0,0) -- ({cos(45)},{sin(45)}) node[pos=0.5,below right=-1pt] {$\SI{65}{m}$};
        \end{scope}
        \draw[Green,very thick,->] ({cos(45)},{sin(45)}) -- ++(axis direction cs: -0.3,-0.3) node[black,above=5pt,pos=1.2] {$a_{\text{c}}$};
        \draw[red,very thick,->] ({cos(45)},{sin(45)}) -- ++(axis direction cs: -0.6,0.6) node[black,pos=1.1] {$\SI{15.6}{m/s}$};
    \end{axis}
    \end{tikzpicture}
    \end{center}    
\end{minipage}

\clearpage

\question
The combined mass of Cody and his car, from Question \ref{OXyBFH}, is \SI{740}{kg}. Calculate the centripetal force on the car.

\begin{randomizechoices}
    \correctchoice \SI{2770}{N}
    \choice \SI{178}{N}
    \choice \SI{3083}{N}
    \choice \SI{200400}{N}
\end{randomizechoices}

\question \label{N3JMlC}
In athletics, a ``hammer'' is not construction tool but a 4-kilogram metal ball connected to a handle by a steel wire. Suppose Anita spins the hammer in a circular path, supplying a centripetal force of \SI{320}{N} and giving the hammer a tangential speed of \SI{12}{m/s}. Calculate the radius of curvature of the hammer's path.

\begin{minipage}{6cm}
    \centering
    \begin{randomizechoices}
        \correctchoice \SI{1.8}{m}
        \choice \SI{26.7}{m}
        \choice \SI{1.1}{m}
        \choice \SI{2.5}{m}
    \end{randomizechoices}
\end{minipage}%
\begin{minipage}{6cm}
    \centering
\begin{center}
    \begin{tikzpicture}
        \begin{axis}[
            width=8cm,height=6cm,
            axis line style={draw=none},
            axis equal image,
            ticks=none,
            axis lines=middle,
            xmin=-1.1,xmax=1.6,
            ymin=-0.2,ymax=1.1,
            clip=true,
        ]
        % \clip (-1.1,-0.2) rectangle ++(2.2,1.4);
        \draw[gray,dashed] (0,0) circle (1);
        \fill (0,0) circle (1pt);
        \fill ({cos(0)},{sin(0)}) circle (3pt) node[right] {$\SI{4.0}{kg}$};
        \begin{scope}[shift={(axis direction cs: 0,-0.05)}]
            \draw[gray,<->] (0,0) -- ({cos(0)},{sin(0)}) node[pos=0.5,below] {$r$};
        \end{scope}
        \draw[Green,very thick,->] ({cos(0)},{sin(0)}) -- ++(axis direction cs: -0.4,0) node[black,above=3pt] {$a_{\text{c}}$};
        \draw[red,very thick,->] ({cos(0)},{sin(0)}) -- ++(axis direction cs: 0,0.7) node[black,pos=1.1] {$\SI{12}{m/s}$};
    \end{axis}
    \end{tikzpicture}
    \end{center}
\end{minipage}

\question
Suppose your car experiences a centripetal acceleration of $a_c$ as it turns on a circular road. If the radius of curvature is doubled, and the car's speed remains constant, what does the centripetal acceleration change to?

\begin{randomizechoices}
    \correctchoice $\frac{1}{2} a_c$
    \choice $2 a_c$
    \choice $\frac{1}{4} a_c$
    \choice $4 a_c$
\end{randomizechoices}

\question Anita, from Question \ref{N3JMlC}, is experimenting with a modified hammer that has twice the mass of the standard one. If she originally applied a centripetal force of $F_c$ on the standard hammer, what centripetal force should she apply to maintain the modified hammer's speed at \SI{12}{m/s}?

\begin{randomizechoices}
    \correctchoice $2 F_c$
    \choice $\frac{F_c}{2}$
    \choice $4 F_c$
    \choice $\frac{F_c}{4}$
\end{randomizechoices}

    





\end{questions}
\end{document}