\documentclass[main.tex]{subfiles}

\begin{document}

\subsection{Lesson Plans}

\textbf{Week 2}
\vspace{1ex}

\begin{tabular}{|m{0.25\textwidth}|m{0.7\textwidth}|}
    \hline
    \cellcolor{black!20}\textbf{Date} & \cellcolor{black!20}\textbf{Thursday, 1/12/2023} \\
    \hline
    Learning Intention (TPO) & We will introduce the law of conservation of momentum and model the solution to a worked example of an ice skater. \\
    \hline
    Hook/Warm Up/Opening & Students (Ss) watch \href{https://youtu.be/UqLJjqESlJY}{this video} of a scene from Harry Potter. Ss follow these instructions: \texttt[red]{Pay attention to the cannon in the video. What does the cannon do after firing the shot? Write your response in your notebook.} Ss think-pair-share for 1 minute. Students learn that this is an example of recoil velocity.\\
    %Students read the first two paragraphs of Section \ref{HnP5Pm}, on the transfer of momentum between two colliding cars. They copy the first figure in their notebook.\\
    \hline
    Lesson/Learning Activities & Whiteboard lecture. Ss are introduced to Equations (\ref{LgDXpb}) and (\ref{eVkwoQ}). We model how to use Eq.~(\ref{eVkwoQ}) to solve Example~\ref{W5BnBq} on Nancy, the ice skater, who experiences recoil velocity (see Hook).\\
    \hline
    Graded Activities & In their notebook, Ss work on Exercise \ref{Fl5xyW}, which closely relates to the worked example covered in lecture.\\
    \hline
    Closure & Discuss solution of Exercise \ref{Fl5xyW}.\\
    \hline
\end{tabular}
\vspace{1ex}

\textbf{Week 3}
\vspace{1ex}

\begin{tabular}{|m{0.25\textwidth}|m{0.7\textwidth}|}
    \hline
    \cellcolor{black!20}\textbf{Date} & \cellcolor{black!20}\textbf{Tuesday, 1/17/2023} \\
    \hline
    Learning Intention (TPO) & We introduce and learn about elastic collisions, Section \ref{vecM50}.\\
    \hline
    Hook/Warm Up/Opening & Ss explore elastic collisions using the \href{https://phet.colorado.edu/en/simulations/collision-lab}{PhET Simulation: Collision Lab}.\\
    \hline
    Lesson/Learning Activities & \textit{Whiteboard lecture}. We revisit the law of conservation of momentum, Equation \ref{eVkwoQ}. We model how to apply this law to solve Example \ref{wXY1rT}. Ss carefully copy notes from whiteboard as we finish the example. \\
    \hline
    Graded Activities & Ss complete Exercises \ref{rhrGbz} through \ref{jhvCZ1}, showing all work and steps in notebook. \\
    \hline
    Closure & Towards end of class, teacher shows students the answers to graded exercises, takes questions, and (if necessary) shows solutions to any desired exercise.\\
    \hline
    \hline

    \cellcolor{black!20}\textbf{Date} & \cellcolor{black!20}\textbf{Wednesday, 1/18/2023} \\
    \hline
    Learning Intention (TPO) & We introduce and learn about inelastic collisions, Section \ref{v90mza}.\\
    \hline
    Hook/Warm Up/Opening & Ss explore inelastic collisions using the \href{https://phet.colorado.edu/en/simulations/collision-lab}{PhET Simulation: Collision Lab}.\\
    \hline
    Lesson/Learning Activities & We introduce a variant of the law of conservation of momentum (Eq.~\ref{eVkwoQ}): Equation~(\ref{6HFSQ5}). We model how to apply this equation to solve Example \ref{usQeew}. Ss carefully copy notes from whiteboard as we finish the example. \\
    \hline
    Graded Activities & Exercises \ref{Hjk6xk} through \ref{jhvCZ1}.\\
    \hline
    Closure & \textit{Demo}. Two students volunteer, S1 to throw a ball, S2 to catch it. When S2 catches the ball, we ask the class, ``Where is the ball?'' Class responds, ``{\color{red}\textit{In}} the hand of S2.'' The key word {\color{red}\textit{in}} indicates an {\color{red}\textit{in}}elastic collision. \\
    \hline
    \hline

    \cellcolor{black!20}\textbf{Date} & \cellcolor{black!20}\textbf{Thursday, 1/19/2023} \\
    \hline
    Learning Intention (TPO) & We are reviewing for Monday's test on Unit 6: momentum.\\
    \hline
    Hook/Warm Up/Opening & Review\\
    \hline
    Lesson/Learning Activities & Review\\
    \hline
    Graded Activities & Review\\
    \hline
    Closure & Review\\
    \hline
    \hline

    \cellcolor{black!20}\textbf{Date} & \cellcolor{black!20}\textbf{Friday, 1/20/2023} \\
    \hline
    Learning Intention (TPO) & We are reviewing for Monday's test on Unit 6: momentum.\\
    \hline
    Hook/Warm Up/Opening & Review\\
    \hline
    Lesson/Learning Activities & Review\\
    \hline
    Graded Activities & Review\\
    \hline
    Closure & Review\\
    \hline
    \hline
\end{tabular}

\clearpage

\subsection{Teacher Lesson Plans}

\begin{tabular}{|m{0.25\textwidth}|m{0.7\textwidth}|}
%%%%    
    \hline  
    \cellcolor{black!20}\textbf{Date} &
    \cellcolor{black!20}\textbf{Wednesday, January 25, 2023} \\
    \hline
    Learning Intention (TPO) & We introduce \gls{work} (Sec.~\ref{obMgG8}), a concept that is scientifically defined.\\
    \hline
    Hook/Warm Up/Opening & Ss read the opening paragraphs of Section \ref{obMgG8}, up until Example \ref{HKavsZ}.\\
    \hline
    Lesson/Learning Activities &  Introduce Equation (\ref{y0Z6Cw}). Do Example \ref{HKavsZ}.  \\
    \hline
    Graded Activities & \ref{obMgG8} Exercises \#\ref{S3zmkv}--\ref{KU49Fc} \\
    \hline
    Closure & We provide answers and show solutions to select Exercises.\\   
    \hline
\end{tabular}

\begin{tabular}{|m{0.25\textwidth}|m{0.7\textwidth}|}
    \hline  
    \cellcolor{black!20}\textbf{Date} &
    \cellcolor{black!20}\textbf{Thursday, January 26, 2023} \\
    \hline
    Learning Intention (TPO) &  We reinforce conceptual and mathematical understanding of work.\\
    \hline
    Hook/Warm Up/Opening & Cold call to review select Exercises \ref{S3zmkv}--\ref{KU49Fc}\\
    \hline
    Lesson/Learning Activities & Volunteer student reminds class \textcolor{red}{What is a force?} from previous unit. (\href{https://phet.colorado.edu/en/simulations/forces-and-motion-basics}{Click here} for a visual reminder.) Student reminds class \textcolor{red}{What is the equation for an object's weight?} from previous unit. Do Example \ref{O9RxyJ}. Volunteer Ss model on whiteboard how to solve warm up Exercises.\\
    \hline
    Graded Activities & \ref{obMgG8} Exercises \ref{VMEVDL}--\ref{sRogOc} \\
    \hline
    Closure & We provide answers and show solutions to select Exercises.\\  
    \hline
\end{tabular} 

\begin{tabular}{|m{0.25\textwidth}|m{0.7\textwidth}|}
    \hline  
    \cellcolor{black!20}\textbf{Date} &
    \cellcolor{black!20}\textbf{Friday, January 27, 2023} \\
    \hline
    Learning Intention (TPO) & We introduce \gls{power} (Sec.~\ref{bxZysw}). \\
    \hline
    Hook/Warm Up/Opening & Watch \href{https://youtu.be/zhL5DCizj5c}{this video} on Watt's role in the industrial revolution. Ss may have heard of horsepower in cars. Show them that this is equivalent to the metric unit of power: watts (W). \\
    \hline
    Lesson/Learning Activities & Introduce Equation (\ref{k6uV1p}). Students learn that power is simply work divided by time. While no examples are provided, the exercises will help students apply the equation.\\
    \hline
    Graded Activities & \ref{bxZysw} Exercises \#\ref{AlYFHO}--\ref{b9UIEt} \\
    \hline
    Closure & We provide answers and show solutions to select Exercises.\\  
    \hline
\end{tabular}



\begin{tabular}{|m{0.25\textwidth}|m{0.7\textwidth}|}
    \hline  
    \cellcolor{black!20}\textbf{Date} &
    \cellcolor{black!20}\textbf{Monday, Jan 30, 2023} \\
    \hline
    Learning Intention (TPO) & We introduce \gls{energy} (Sec.~\ref{iHOPFl}). \\
    \hline
    Hook/Warm Up/Opening & Watch \href{https://youtu.be/snj1wBtn6I8}{this video} by Nick Lucid. Students answer the question, \textcolor{red}{What are the definitions of work and energy provided in the video? Write them down.} Note that Lucid cleverly defines energy as ``the amount of stuff that could happen.'' This is a more conceptual and intuitive definition than the textbook definition.\\
    \hline
    Lesson/Learning Activities & Ss will click \href{https://phet.colorado.edu/sims/html/energy-forms-and-changes/latest/energy-forms-and-changes_en.html}{this link} to the \texttt{PhET Simulation} \textit{Energy Forms and Changes: Systems}. Ss see how chemical energy gets converted to mechanical energy and ultimately to light energy. In small groups, students create a story about how the energy is transferred but never destroyed (for shadowing the \gls{law of conservation of energy}.\\
    \hline
    Graded Activities & \ref{iHOPFl} Exercises \#\ref{UVgdHj}--\ref{IPOKOY} \\
    \hline
    Closure & We provide answers and show solutions to select Exercises.\\
    \hline
\end{tabular}

\begin{tabular}{|m{0.25\textwidth}|m{0.7\textwidth}|}
    \hline  
    \cellcolor{black!20}\textbf{Date} &
    \cellcolor{black!20}\textbf{Tuesday, January 31, 2023} \\
    \hline
    Learning Intention (TPO) & We introduce \gls{gravitational potential energy} (Sec.~\ref{vRflRn}). \\
    \hline
    Hook/Warm Up/Opening & \textit{Demo}: Introducing gravitational potential energy. I bring a giant bowl and some eggs to school. I remind students that energy is ``the amount of stuff that could happen.'' Let's suppose the egg breaking open is the ``stuff that could happen.'' If I drop the egg from a height of \SI{1}{cm}, it might crack but it won't break open. We need more energy to make the breaking happen. When I drop the egg from a height of 1 meter, it breaks because their is enough potential energy.\\
    \hline
    Lesson/Learning Activities & Introduce Equation (\ref{RlpJa5}). Do Example \ref{J1Qehp}.\\
    \hline
    Graded Activities & \ref{vRflRn} Exercises \#\ref{xxSTEr} -- \ref{JangCo} \\
    \hline
    Closure & Ss check answers and correct mistakes.\\  
    \hline
\end{tabular}    

\begin{tabular}{|m{0.25\textwidth}|m{0.7\textwidth}|}
    \hline
    \cellcolor{black!20}\textbf{Date} &
    \cellcolor{black!20}\textbf{Thursday, February 2, 2023} \\
    \hline
    Learning Intention (TPO) & We will introduce \gls{kinetic energy} (Sec.~\ref{aOaKHZ}). \\
    \hline
    Hook/Warm Up/Opening & Watch \href{https://youtu.be/_z9kdqDwA80}{this video} about the Chris Farley's appearance on the David Letterman show. On a sticky note, students complete the sentence \textcolor{red}{Chris Farley has a lot of \rule{2cm}{0.15mm}.}\\
    \hline
    Lesson/Learning Activities & \textit{Whiteboard}. Define \gls{kinetic energy} and introduce Equation (\ref{s57crz}). Do Example \ref{5qVtmV}, while letting Ss solve it on their own first; then show solution. By now, most students should be able to solve new problems independently given the correct equation. Next, guide Ss on how to do Example \ref{PUlJG6}, which requires many more steps.\\
    \hline
    Graded Activities & \ref{aOaKHZ} Exercises \#\ref{jnYiIT} -- \ref{X7RPxf}\\
    \hline
    Closure & Ss correct answers and correct mistakes.\\
    \hline
    \hline
\end{tabular}  

\begin{tabular}{|m{0.25\textwidth}|m{0.7\textwidth}|}
    \hline  
    \cellcolor{black!20}\textbf{Date} &
    \cellcolor{black!20}\textbf{Monday, February 6, 2023} \\
    \hline
    Learning Intention (TPO) & We hypothesis a law relatic kinetic and potential energies by sharing ideas through a lab on the energy of a tossed ball.\\
    \hline
    Hook/Warm Up/Opening & \\
    \hline
    Lesson/Learning Activities & \\
    \hline
    Graded Activities & \\
    \hline
    Closure & \\  
    \hline
\end{tabular} 



\begin{tabular}{|m{0.25\textwidth}|m{0.7\textwidth}|}
    \hline  
    \cellcolor{black!20}\textbf{Date} &
    \cellcolor{black!20}\textbf{Tuesday, February 7, 2023} \\
    \hline
    Learning Intention (TPO) &  We will explain mechanical energy and the law of conservation of energy and share ideas through a close reading of the text.\\
    \hline
    Hook/Warm Up/Opening & Ss access the \href{https://phet.colorado.edu/en/simulations/energy-skate-park}{PhET Simulation: Energy Skate Park} to see what the transformation of potential to kinetic energy, and vice-versa, looks like. We introduce the \gls{law of conservation of energy} (Sec.~\ref{KRxiCV}). \\
    \hline
    Lesson/Learning Activities & Introduce Equations (\ref{QyLUh5}), (\ref{yOUj22}), and (\ref{ZSmSin}). Do Examples \ref{mUfgIz} and \ref{nQVUKM}.\\
    \hline
    Graded Activities & \ref{KRxiCV} Exercises \#\ref{2EkX0c}--\ref{c4eP75}\\
    \hline
    Closure & Ss check answers and correct mistakes.\\  
    \hline
\end{tabular} 

\begin{tabular}{|m{0.25\textwidth}|m{0.7\textwidth}|}
    \hline  
    \cellcolor{black!20}\textbf{Date} &
    \cellcolor{black!20}\textbf{Wednesday, February 8, 2023} \\
    \hline
    Learning Intention (TPO) &  \\
    \hline
    Hook/Warm Up/Opening & \\
    \hline
    Lesson/Learning Activities & \\
    \hline
    Graded Activities & \\
    \hline
    Closure & \\  
    \hline
\end{tabular} 

\begin{tabular}{|m{0.25\textwidth}|m{0.7\textwidth}|}
    \hline  
    \cellcolor{black!20}\textbf{Date} &
    \cellcolor{black!20}\textbf{Thursday, February 9, 2023} \\
    \hline
    Learning Intention (TPO) & \\
    \hline
    Hook/Warm Up/Opening & \\
    \hline
    Lesson/Learning Activities & \\
    \hline
    Graded Activities &  \\
    \hline
    Closure & \\  
    \hline
\end{tabular}


\begin{tabular}{|m{0.25\textwidth}|m{0.7\textwidth}|}
    \hline  
    \cellcolor{black!20}\textbf{Date} &
    \cellcolor{black!20}\textbf{Friday, February 10, 2023} \\
    \hline
    Learning Intention (TPO) & We review the law of conservation of momentum.\\
    \hline
    Hook/Warm Up/Opening & \\
    \hline
    Lesson/Learning Activities & \\
    \hline
    Graded Activities & \\
    \hline
    Closure & \\  
    \hline
\end{tabular} 

\begin{tabular}{|m{0.25\textwidth}|m{0.7\textwidth}|}
    \hline  
    \cellcolor{black!20}\textbf{Date} &
    \cellcolor{black!20}\textbf{Friday, February 10, 2023} \\
    \hline
    Learning Intention (TPO) &  We review for next Tuesday's Test on Unit 7: Energy.\\
    \hline
    Hook/Warm Up/Opening & \\
    \hline
    Lesson/Learning Activities & \\
    \hline
    Graded Activities & \\
    \hline
    Closure & \\  
    \hline
\end{tabular} 

\begin{tabular}{|m{0.25\textwidth}|m{0.7\textwidth}|}
    \hline  
    \cellcolor{black!20}\textbf{Date} &
    \cellcolor{black!20}\textbf{Monday, February 13, 2023} \\
    \hline
    Learning Intention (TPO) & We review for tomorrow's Test on Unit 7: Energy.\\
    \hline
    Hook/Warm Up/Opening & \\
    \hline
    Lesson/Learning Activities & \\
    \hline
    Graded Activities & \\
    \hline
    Closure & \\  
    \hline
\end{tabular} 

\begin{tabular}{|m{0.25\textwidth}|m{0.7\textwidth}|}
    \hline  
    \cellcolor{black!20}\textbf{Date} &
    \cellcolor{black!20}\textbf{Tuesday, February 14, 2023} \\
    \hline
    Learning Intention (TPO) & We take the Test on Unit 7: Energy.\\
    \hline
    Hook/Warm Up/Opening & Clear desk of everything except a pencil, calculator, scratch paper, and laptop.\\
    \hline
    Lesson/Learning Activities & Take the Test on Unit 7: Energy in Schoology.\\
    \hline
    Graded Activities & Test on Unit 7: Energy.\\
    \hline
    Closure & Throw away scratch paper and prepare to go to next class.\\  
    \hline
\end{tabular} 

\begin{tabular}{|m{0.25\textwidth}|m{0.7\textwidth}|}
    \hline  
    \cellcolor{black!20}\textbf{Date} &
    \cellcolor{black!20}\textbf{Wednesday February 15, 2023} \\
    \hline
    Learning Intention (TPO) & Ss not successful on yesterday's test complete test corrections.\\
    \hline
    Hook/Warm Up/Opening & \\
    \hline
    Lesson/Learning Activities & \\
    \hline
    Graded Activities & \\
    \hline
    Closure & \\  
    \hline
\end{tabular} 



\end{document}